% ------------------------------------------------------------------------------------------
\section{Arithmetic expression evaluation}
% ------------------------------------------------------------------------------------------

The \texttt{mo\_expression} module evaluates basic arithmetic
expressions specified by character-strings.
%
It is possible to include mathematical functions, operators, and
constants. 
An application of this module is the evaluation of arithmetic
expressions povided as namelist parameters.

Besides, Fortran variables can be linked to the expression and used in
the evaluation.
The implementation supports scalar input variables as well as 2D and
3D fields.

From a users' point of view, the basic usage of this module is
described in Section~\ref{section:examples_for_arithmetic_expressions}
below.
Technically, infix expressions are processed based on a Finite State
Machine (FSM) and Dijkstra's shunting yard algorithm.
A more detailed described of the Fortran interface is given in
Section~\ref{section:usage_with_fortran}.


% ------------------------------------------------------------------------------------------
\subsection{Examples for arithmetic expressions}
\label{section:examples_for_arithmetic_expressions}
% ------------------------------------------------------------------------------------------

Basic examples:
\begin{itemize}
  \item \texttt{ "sqrt(2.0)" }
  \item \texttt{ "sin(45*pi/180.) * 10 + 5" }
  \item \texttt{ "if(1. > 2, 99, -1.*pi)" }
  \item \texttt{ "min(1,2)" }
\end{itemize}

Variables are used with a bracket notation:
\begin{itemize}
 \item \verb;"sqrt([u]^2 + [v]^2)";
\end{itemize}
Note that the use of variables requires that these are enabled
("linked") by the Fortran routine that calls the
\texttt{mo\_expression} module.


% ------------------------------------------------------------------------------------------
\subsection{Expression syntax}
% ------------------------------------------------------------------------------------------

% ------------------------------------------------------------------------------------------
\subsubsection{List of functions}
% ------------------------------------------------------------------------------------------

\begin{tabular}{|l|c|p{7cm}|}
  \hline
  name                           & \#args & description \\
  \hline\hline
  \texttt{log()}, \texttt{exp()} & 1                & natural logarithm and its inverse function. \\
  \texttt{sin()}, \texttt{cos()} & 1                & trigonometric functions                     \\
  \texttt{sqrt()}                & 1                & square root                                 \\
  \texttt{erf()}                 & 1                & Gauss error function                        \\
  \texttt{min()}, \texttt{max()} & 2                & minimum and maximum of two values           \\
  \texttt{if(\textit{value},\textit{then},\textit{else})} & 3 & conditional expression 
                                                      (\texttt{\textit{value} > 0.})              \\
  \hline
\end{tabular}


% ------------------------------------------------------------------------------------------
\subsubsection{List of operators}
% ------------------------------------------------------------------------------------------

\begin{tabular}{|p{3cm}|p{7cm}|}
  \hline
  name                              & evaluates to \\
  \hline\hline
  \raggedright%
  \texttt{\textit{a}~+~\textit{b}},
  \texttt{\textit{a}~-~\textit{b}},
  \texttt{\textit{a}~*~\textit{b}},
  \texttt{\textit{a}~/~\textit{b}}  & $(a+b)$, $(a-b)$, $(a*b)$, $(a/b)$ \\[0.5em]
  \texttt{\textit{a}~\^~\textit{b}} & $a^b$                              \\[0.5em]
  \texttt{\textit{a}~>~\textit{b}}  & $\begin{cases}
                                        ~  1, & \text{if } a > b, \\
                                        ~  0, & \text{otherwise}.
                                       \end{cases}$                      \\[1.5em]
  \texttt{\textit{a}~<~\textit{b}}  & $\begin{cases}
                                        ~  1, & \text{if } a < b, \\
                                        ~  0, & \text{otherwise}.
                                       \end{cases}$                      \\[0.5em]
  \hline
\end{tabular}



% ------------------------------------------------------------------------------------------
\subsubsection{List of available constants}
% ------------------------------------------------------------------------------------------

\begin{tabular}{|l|l|p{7cm}|}
  \hline
  name of constant & assigned value & description \\
  \hline\hline
  \texttt{pi} & $4 \, \atan(1)$     & mathematical constant equal to a circle's circumference 
                                      divided by its diameter  \\
  \texttt{r}  & $6.371229\cdot10^6$ & Earth's radius\footnotemark[1]  \\
  \hline
\end{tabular}


% ------------------------------------------------------------------------------------------
\subsection{Usage with Fortran}
\label{section:usage_with_fortran}
% ------------------------------------------------------------------------------------------

The minimal Fortran interface is as follows:
\begin{enumerate}
\item The \texttt{TYPE expression} which is initialized with the
  character-string that specifies the arithmetic expression.
\item The type-bound procedure \texttt{evaluate()}, which returns the
  result (scalar or array-shaped) as a \texttt{POINTER}.
\item The type-bound procedure \texttt{link()} connecting a variable
 to a name in the character-string expression.
\end{enumerate}


% ------------------------------------------------------------------------------------------
\subsubsection{Fortran examples}
% ------------------------------------------------------------------------------------------

The following examples illustrate the arithmetic expression parser.
The calls to \texttt{DEALLOCATE} the data structures have been
ommitted for the sake of brevity:

\begin{enumerate}
  \item Scalar arithmetic expression:
   \begin{verbatim}
  formula = expression("sin(45*pi/180.) * 10 + 5")
  CALL formula%evaluate(val)
    ... use "val" for some purpose ...
   \end{verbatim}
  \item Masking of a 2D array as an example for the \texttt{link} procedure:
    \begin{verbatim}
  formula = expression("if([z_sfc] > 2., [z_sfc], 0. )")
  CALL formula%link("z_sfc", z_sfc)
  CALL formula%evaluate(val_2D)
    ... use "val_2D(:,:)" for some purpose ...
   \end{verbatim}
\end{enumerate}
  





% ------------------------------------------------------------------------------------------
\subsubsection{Error handling}
% ------------------------------------------------------------------------------------------

Invalid arithmetic expressions yield "empty" expression objects. When these
are evaluated, a \texttt{NULL()} pointer is returned.

A successful expression evaluation can be tested with the \texttt{err\_no} variable:
\begin{verbatim}
  IF (formula%err_no == ERR_NONE) THEN
    ...
  END IF
\end{verbatim}
In case of error, the \texttt{err\_no} variable also provides the
reason for the aborted evaluation process.


% ------------------------------------------------------------------------------------------
\subsection{Remarks}
% ------------------------------------------------------------------------------------------

\begin{itemize}
  \item Variable names are treated case-sensitive!
  \item For 3D array input it is implicitly assumed that 2D fields are
    embedded in 3D fields as "3D(:,level,:) = 2D(:,:)".
\end{itemize}


\footnotetext[1]{This number seems to be based on Hayford's 1910 estimate
  of the Earth. It is used in ICON as well as MPAS and was almost
  certainly taken from the Jablonowski and Williamson test case
  (QJRMS, 2006).}

