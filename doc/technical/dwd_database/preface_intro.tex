% --------------------------------------------------------------------------------
\chapter{Introduction}
% --------------------------------------------------------------------------------

The \textbf{ICO}sahedral \textbf{N}onhydrostatic model ICON is the new global numerical 
weather prediction model at DWD. It became operational at 2015-01-20, replacing the former  
operational global model GME. 
In June 2015 a refined subregion (\emph{nest}) over Europe was activated, which is going to 
replace the current regional model COSMO-EU during the course of the year 2016.
%Later in 2015, it will further replace DWD's regional model COSMO-EU  
%through the addition of a refined subregion (\emph{nest}) over Europe. 
The ICON modelling system as a whole is developed jointly by DWD and the Max-Planck Institute 
for Meteorology in Hamburg (MPI-M). While ICON is the new working horse for short and medium range 
global weather forecast at DWD, it will serve as the core of a new climate modelling system at MPI-M.

Since 2015-01-20, ICON analysis and forecast fields at $13\,\mathrm{km}$ horizontal resolution 
serve as initial and boundary data for
\begin{itemize}
 \item the regional model COSMO-EU
 \item RLMs (\textbf{R}elocatable \textbf{L}ocal \textbf{M}odel) of the German armed forces
 \item DWD's wave models
\end{itemize}

Since 2015-07-21, the ICON-EU nest provides forecast fields at $6.5\,\mathrm{km}$ horizontal resolution
which will serve as initial and boundary data for
\begin{itemize}
 \item the regional model COSMO-DE
\end{itemize}
For the time being, initial and boundary data for COSMO-DE are still provided by COSMO-EU. The 
latter will be operated in parallel to the ICON-EU nest until the end of the 1st quarter of 2016, 
at least.

This document provides some basic information about ICON's horizontal and vertical grid structure, 
numerical algorithms (see also \cite{Zaengl15}) and physical parameterizations (the latter two are 
planned but not yet available). Furthermore, it provides an overview about the available ICON analysis 
and forecast fields stored in the database SKY at DWD. Some examples on how to read these data from 
the database are given as well.

\vfill
If you encounter bugs or inconsistencies, or if you have suggestions for improving this document, 
please contact one of the following colleagues:

\begin{note}
\begin{minipage}{\textwidth}
\centering
\begin{minipage}{0.32\textwidth}
 \textbf{Daniel Reinert}, FE13 \\
 Tel: +49 (69) 8062-2060 \\ 
 Mail: daniel.reinert@dwd.de
\end{minipage}
\begin{minipage}{0.32\textwidth}
 \textbf{Helmut Frank}, FE13\\
 Tel: +49 (69) 8062-2742 \\ 
 Mail: helmut.frank@dwd.de
\end{minipage}
\begin{minipage}{0.32\textwidth}
 \textbf{Florian Prill}, FE13 \\
 Tel: +49 (69) 8062-2727 \\ 
 Mail: florian.prill@dwd.de
\end{minipage}
\end{minipage}
\end{note}  