% --------------------------------------------------------------------------------
\chapter{History of model changes}
% --------------------------------------------------------------------------------

The forecasting environment, which is composed of the ICON model and the data assimilation system, 
is subject to continuous improvements and modifications. The most important ones in terms of forecast 
quality and output products are depicted below. For additional information, the reader is referred to the 
official change notifications which are available from
\begin{note}
\url{http://www.dwd.de/DE/fachnutzer/forschung_lehre/numerische_wettervorhersage/nwv_aenderungen/nwv_aenderungen_node.html}
\end{note}
Alternatively you can click on the timeline-date to see the corresponding change notification.

% Timeline:
%
% The macro "\colorA" corresponds to ICON model changes
%           "\colorB" corresponds to data assimilation changes.
%
\begin{vtimeline}[description={text width=0.78\textwidth}, 
 row sep=3ex,
 line offset=10pt,
 timeline color=colorBlue,
 timeline color2=colorRed]
\colorA  \href{http://www.dwd.de/DE/fachnutzer/forschung_lehre/numerische_wettervorhersage/nwv_aenderungen/_functions/DownloadBox_modellaenderungen/icon/pdf_2015/pdf_icon_17_12_2014.pdf?__blob=publicationFile&v=6}
         {\textbf{2015-01-20}} & First operational ICON forecast run at a horizontal resolution of $13\,\mathrm{km}$ and $90$ vertical levels. Model top at $75\,\mathrm{km}$\endlr
\colorA  \href{http://www.dwd.de/DE/fachnutzer/forschung_lehre/numerische_wettervorhersage/nwv_aenderungen/_functions/DownloadBox_modellaenderungen/icon/pdf_2015/pdf_icon_10_02_2015.pdf?__blob=publicationFile&v=7}
         {\textbf{2015-02-25}} & $4$ additional forecast runs 03, 09, 15, 21\,UTC up to $30\,\mathrm{h}$ lead time. Maximum forecast lead times for 06 and 18\,UTC runs extended to $120\,\mathrm{h}$\endlr
\colorA  \href{http://www.dwd.de/DE/fachnutzer/forschung_lehre/numerische_wettervorhersage/nwv_aenderungen/_functions/DownloadBox_modellaenderungen/icon/pdf_2015/pdf_icon_04_03_2015.pdf?__blob=publicationFile&v=4}
         {\textbf{2015-03-04}} & Improved wind gust diagnostic for mountainous regions \endlr
%        2015-06-17 & Introduction of a surface wind dependent Charnock-Parameter.
\colorA  \href{http://www.dwd.de/DE/fachnutzer/forschung_lehre/numerische_wettervorhersage/nwv_aenderungen/_functions/DownloadBox_modellaenderungen/icon/pdf_2015/pdf_icon_07_07_2015.pdf?__blob=publicationFile&v=2}
         {\textbf{2015-07-07}} & Time interval over which max/min temperatures \texttt{TMAX\_2M}, \texttt{TMIN\_2M} are collected changed to $6\,\mathrm{h}$ (formerly $3\,\mathrm{h}$). 
                      Time interval over which maximum wind gusts \texttt{VMAX\_10M} are collected changed to $1\,\mathrm{h}$ (formerly $3\,\mathrm{h}$).\endlr  
\colorA  \href{http://www.dwd.de/DE/fachnutzer/forschung_lehre/numerische_wettervorhersage/nwv_aenderungen/_functions/DownloadBox_modellaenderungen/icon/pdf_2015/pdf_icon_02_07_2015.pdf?__blob=publicationFile&v=2}
         {\textbf{2015-07-21}} & Launch of the ICON-EU nest with a horizontal resolution of $6.5\,\mathrm{km}$ and $50$ vertical levels. Model top at $50\,\mathrm{km}$.\endlr
\colorB  \href{http://www.dwd.de/DE/fachnutzer/forschung_lehre/numerische_wettervorhersage/nwv_aenderungen/_functions/DownloadBox_modellaenderungen/icon/pdf_2015/pdf_icon_01_09_2015.pdf?__blob=publicationFile&v=4}
         {\textbf{2015-09-02}} & Assimilation of selected microwave satellite radiance channels over land (AMSU-A and ATMS), which have so far been assimilated only over the oceans.\endlr
\colorA  \href{http://www.dwd.de/DE/fachnutzer/forschung_lehre/numerische_wettervorhersage/nwv_aenderungen/_functions/DownloadBox_modellaenderungen/icon/pdf_2015/pdf_icon_27_10_2015.pdf?__blob=publicationFile&v=3}
         {\textbf{2015-12-01}} & Surface tile approach for land, sea-ice and lake points, including snow-tiles. Each grid point can have up to $5$ tiles consisting of 3 land tiles 
                      (dominant land-use types), a lake tile or a sea-water plus sea-ice tile. Land tiles may have additional snow tiles.\endlr
\colorB  \href{http://www.dwd.de/DE/fachnutzer/forschung_lehre/numerische_wettervorhersage/nwv_aenderungen/_functions/DownloadBox_modellaenderungen/icon/pdf_2016/pdf_icon_20_01_2016.pdf?__blob=publicationFile&v=2}
         {\textbf{2016-01-20}} & Launch of the ensemble data assimilation system (LETKF, Localized Ensemble Transform Kalman Filter), providing a $40$ member analysis ensemble at R2B6N7. 
                      For deterministic forecasts the 3D-Var assimilation system is replaced by En-Var (Ensemble Variational analysis system) which makes use 
                      of the ensemble-based model error covariances.\endlr
\end{vtimeline}

\begin{vtimeline}[description={text width=0.78\textwidth}, 
 row sep=3ex, 
 add bottom line,
 line offset=10pt,
 timeline color=colorBlue,
 timeline color2=colorRed]
\colorA  \href{http://www.dwd.de/DE/fachnutzer/forschung_lehre/numerische_wettervorhersage/nwv_aenderungen/_functions/DownloadBox_modellaenderungen/icon/pdf_2016/pdf_icon_13_04_2016.pdf?__blob=publicationFile&v=3}
         {\textbf{2016-04-13}} & Improved version of the ICON model which
                      \small
                      \begin{itemize}
                       \item accounts for oceanic salt content in the saturation vapor pressure computation.
                       \item contains modifications to the detrainment tendencies from the convection parameterization in order to reduce spurious drizzle.
                       \item makes use of the aerosol climatology for the computation of cloud droplet number concentrations in the radiation parameterization.
                      \end{itemize}
                      \endlr
\colorA  \href{http://www.dwd.de/DE/fachnutzer/forschung_lehre/numerische_wettervorhersage/nwv_aenderungen/_functions/DownloadBox_modellaenderungen/icon/pdf_2016/pdf_icon_20_04_2016.pdf?__blob=publicationFile&v=2}
         {\textbf{2016-04-20}} & The pressure levels for regular grid output have been revised. 
                      Newly available levels: $650$, $550$, $450$, $350$, $275$, $225$, $175$, $125\,\mathrm{hPa}$. 
                      Deprecated levels: $725$, $20$, $7$, $3$, $0.3\,\mathrm{hPa}$.
                      \endlr
\colorA  \href{http://www.dwd.de/DE/fachnutzer/forschung_lehre/numerische_wettervorhersage/nwv_aenderungen/_functions/DownloadBox_modellaenderungen/icon/pdf_2016/pdf_icon_28_09_2016.pdf?__blob=publicationFile&v=3}
         {\textbf{2016-09-28}} & New version of the ICON model in which 
                      \small
                      \begin{itemize}
                       \item the convection parameterization has been improved with respect to mixed-phase cloud processes
                       \item the process of rain water evaporation below cloud base has been updated. 
                      \end{itemize}
                      \endlr        
\colorA  \href{http://www.dwd.de/DE/fachnutzer/forschung_lehre/numerische_wettervorhersage/nwv_aenderungen/_functions/DownloadBox_modellaenderungen/icon/pdf_2016/pdf_icon_29_11_2016.pdf?__blob=publicationFile&v=4}
         {\textbf{2016-11-30}} & New version of the ICON model in which 
                      \small
                      \begin{itemize}
                       \item the incremental analysis update (IAU) scheme has been improved
                       \item the diagnostic relative humidity at $2\,\mathrm{m}$ above ground is used in the assimilation system instead of the prognostic one at $10\,\mathrm{m}$ above ground.  
                      \end{itemize}
                      \endlr
\colorB  \href{http://www.dwd.de/DE/fachnutzer/forschung_lehre/numerische_wettervorhersage/nwv_aenderungen/_functions/DownloadBox_modellaenderungen/icon/pdf_2017/pdf_icon_31_01_2017.pdf?__blob=publicationFile&v=1}
         {\textbf{2017-01-31}} & Modifications to the ICON-internal lat-lon interpolation for temperature and specific humidity. Changes to the data assimilation system, i.e.
                      \small
                      \begin{itemize}
                       \item bias correction for scatterometer data
                       \item improvements regarding the usage of wind measurements from buoys
                      \end{itemize}
                      \endlr
\end{vtimeline}
