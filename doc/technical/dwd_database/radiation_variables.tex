\documentclass[11pt,notitlepage]{article}  %a4paper,landscape,
\usepackage[colorlinks,bookmarksopen,bookmarksnumbered,citecolor=red,urlcolor=red]{hyperref}  %dvips
\usepackage[a4paper,landscape]{geometry}

\title{Radiation Variables in GRIB2 and ICON}
\author{Martin K\"ohler}
\begin{document}  
\maketitle

%-------------------------------------------------------------------------

Radiative fluxes are stored for solar (diffuse, direct, total) and thermal bands (also called short-wave and long-wave, respectively).  
They are available as upward, downward and net (``budget'') and at the levels top of atmosphere (TOA) and the surface.  
Accumulation (ACC), average (A) or instantaneous ("") values can be archived.  There statistics are valid from the beginning of the forecast
to the output time.  
The short-names in the DWD GRIB2 description convention are then produced by the following components.


\vspace{1cm}
\begin{tabular}{l l l l l}  
\textbf{Statistic}      & \textbf{Band}         & \textbf{Direction} & \textbf{Text} & \textbf{Level} \\
\hline
A (average)             & TH (thermal/lw)       & U (up)             &  \_           &  T             \\
ACC (accumulated)       & SO (solar/sw)         & D (down)           &               &  S             \\
nothing (instantaneous) & SODIF (solar diffuse) & B (net)            &               &                \\
                        & SODIF (solar direct)  &                    &               &                \\
\hline
\end{tabular}
\vspace{1cm}


An example is ASOB\_T, that is the net solar flux at TOA.  Not all fluxes exist.  For example there is no downward thermal flux at TOA.  The following table 
lists the existing fluxes with the associated DWD shortnames and the GRIB2 descriptors. 


\vspace{1cm}
\hspace{-2cm}
\begin{centering}
\begin{tabular}{l p{3cm} p{3cm} p{3cm} l l l l}  

\textbf{Description}           & \textbf{DWD \linebreak ShortName} 
                               & \textbf{ICON \linebreak ShortName} 
                               & \textbf{ECMWF \linebreak ShortName} 
                               & \textbf{Discipline} & \textbf{Category} & \textbf{Number} & \textbf{levType} \\
\hline                                                                                                                                                                          

Top net solar radiation        &  ASOB\_T     &  asob\_t     &  tsr        &  0        &  4      &  9       &  8      \\
Top up solar radiation         &  ASOU\_T     &              &  ---        &  0        &  4      &  8       &  8      \\
Top down solar radiation       &  ASOD\_T     &              &  tisr       &  0        &  4      &  7       &  8      \\
Surface net solar radiation    &  ASOB\_S     &  asob\_s     &  ssr        &  0        &  4      &  9       &  1      \\
Surface up solar radiation     &  ASOU\_S     &              &  ---        &  0        &  4      &  8       &  1      \\
Surface down solar radiation   &  ASOD\_S     &              &  ssrd       &  0        &  4      &  7       &  1      \\

Surface down solar diff. rad.  &  ASODIFD\_S  &              &             &  ?        &  ?      &  ?       &  1      \\
Surface up solar diff. rad.    &  ASODIFU\_S  &              &             &  ?        &  ?      &  ?       &  1      \\
Surface down solar direct rad. &  ASODIRU\_S  &              &  dsrp       &  ?        &  ?      &  ?       &  1      \\


Top net thermal radiation      &  ATHB\_T     &  athb\_t     &  ttr        &  0        &  5      &  5       &  8      \\
Surface net thermal radiation  &  ATHB\_S     &  athb\_s     &  str        &  0        &  5      &  5       &  1      \\
Surface up thermal radiation   &  ATHU\_S     &              &  ---        &  0        &  5      &  4       &  1      \\
Surface down thermal radiation &  ATHD\_S     &              &  strd       &  0        &  5      &  3       &  1      \\

\hline
\end{tabular}
\end{centering}
\vspace{1cm}

Note that ATHD\_T = 0 and therefore ATHU\_T = ATHB\_T.  Also, ASODIRU\_S = 0.


\end{document}
