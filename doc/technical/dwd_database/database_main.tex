\documentclass[a4paper,twoside,10pt]{book}

\usepackage[english]{babel}
\usepackage{float}
\usepackage[utf8]{inputenc}
\usepackage{amsmath}
\usepackage{fancyhdr}
\usepackage{geometry}
\usepackage{amsthm}        %--- Theorem-Package

\usepackage{pgfplots,pgfplotstable}
\usepackage{colortbl}

\usepackage[nottoc]{tocbibind}     %--- adds bibliography to toc 
% --- but without toc itself [nottoc] 
\usepackage[round,sort,comma,authoryear]{natbib} %--- creates bibliography; author-year with curly brackets within the text
% \usepackage[  %--- no numbering of positions 
% toc,           %--- let it appear in the document's table of contents
% style=mylong3colheader %--- see glossaries Documentation for a list of available styles
% ]{glossaries}  %--- use the glossary package
\geometry{a4paper,textwidth=15.8cm,textheight=23.0cm,twoside,rmargin=0.96in} %--- Variation Randbereiche
\usepackage{caption}
\usepackage{subcaption}  %--- simplifies putting pictures side by side
\usepackage{longtable}
%\usepackage{setspace}
\usepackage{multirow}
\usepackage{tabularx}
\usepackage{booktabs}    % to give tables a more professional look
\usepackage{rotating}    % rotate column labels
\usepackage{xcolor}
\usepackage{listings}
\usepackage{hyperref}    % ability to create hyperlinks within the document
\usepackage{vhistory}
\usepackage{tikz}
\usepackage{xargs}
\usepackage{xstring}
\usepackage{enumitem}
\usepackage[strict]{changepage} % for adjustwidth environment
\usepackage{framed}
\usepackage{array}

% \usepackage{setspace}
% \usepackage{pdftricks}   %--- for drawing arrows, frames, etc in Latex
% \usepackage{pstricks}    %--- for drawing arrows, frames etc in Latex
% \usepackage{layout}      %--- overview about boundaries and widths
                           % usage: write \layout somewhere in the document



% left-aligned fixed-width table column
\newcolumntype{P}[1]{>{\raggedright\arraybackslash}p{#1}}

% define some colors
\definecolor{darkred}{rgb}{0.5,0,0}
\definecolor{darkblue}{rgb}{0,0,0.5}
%\definecolor{darkgreen}{rgb}{0,0.5,0}

% customize hyperref
\hypersetup{colorlinks, linkcolor=darkblue, citecolor=blue, urlcolor=darkred}


% Fortran environment
\lstnewenvironment{fortran}%
{\lstset{language=[95]Fortran,%
    basicstyle=\ttfamily\footnotesize\color{darkgreen},%
    commentstyle=\ttfamily\color{blue},%
    emptylines=0,%
    keywordstyle=\color{black}\ttfamily\bfseries,%
    backgroundcolor=\color{darkgrey!5},
    framexleftmargin=4mm,%
    frame=shadowbox,%
    rulesepcolor=\color{darkgreen}}}
{}




\newcolumntype{x}[1]{%
  >{\centering\hspace{0pt}}p{#1}}%


% ---------- Glossary: All the stuff and more ----------%
% \newglossary{symbolslist}{sym}{sbl}{Symbolverzeichnis} % define new glossary i.e. a list of symbols
% \makeglossaries

% \input{./text/listofsymbols}  %--- include defined symbols






% ---------- Definition des Styles von Theoremen ----------%
\theoremstyle{plain}
\newtheorem*{thrm}{Theorem}  %--- define new theorem-style






% ---------------bold caption for figures-------------------------%
\makeatletter
\long\def\@makecaption#1#2{%
  \fontfamily{cmr}
  \fontseries{m}
  \fontshape{sl}
  \fontsize{10pt}{11pt}
  \selectfont
  \vskip 10\p@
  \setbox\@tempboxa\hbox{{\bf#1:} #2}%
  \ifdim \wd\@tempboxa >\hsize
  {\bf #1:} #2\par
  \else
  \hbox to\hsize{\hfil\box\@tempboxa\hfil}%
  \fi}
\makeatother
% ----------------------------------------------------------------%

\newcommand{\markRed}{\tikz \fill[red!60] (0,0) rectangle (0.08,0.8em);}
\newcommand{\markBlue}{\tikz \fill[blue!80] (0,0) rectangle (0.08,0.8em);}
\newcommand{\markWhite}{\tikz \fill[white!100] (0,0) rectangle (0.08,0.8em);}

% Macro for indicating whether field is available on native grid (tri), lat-lon grid (ll), or both
\newcommandx{\groups}[2][1=none,2=none,usedefault]{
  \IfEqCase{#1}{%
      {tri}{%
           \IfEqCase{#2}{%
           {ll}{\markRed\markBlue}      %group (tri,ll)
           {none}{\markRed\markWhite}   %group (tri)
           }%
      }%
  }%
  \IfEqCase{#1}{%
      {none}{%
           \IfEqCase{#2}{%
           {ll}{\markWhite\markBlue}   %group (ll)
           }%
      }%
  }%
}%




% environment derived from framed.sty: see leftbar environment definition
\definecolor{formalshade}{rgb}{0.95,0.95,1}
\definecolor{darkblue}{rgb}{0.,0.,0.6}

\newenvironment{note}{%
  \def\FrameCommand{%
    \hspace{1pt}%
    {\color{darkblue}\vrule width 2pt}%
    {\color{formalshade}\vrule width 4pt}%
    \colorbox{formalshade}%
  }%
  \MakeFramed{\advance\hsize-\width\FrameRestore}%
  \noindent\hspace{-4.55pt}% disable indenting first paragraph
  \begin{adjustwidth}{}{7pt}%
%  \vspace{-2pt}\vspace{-2pt}%
}
{%
  %\vspace{2pt}%
  \end{adjustwidth}\endMakeFramed%
}


%\parskip0.4ex plus0.2ex minus0.3ex
\setlength{\parskip}{0.2cm plus0.1cm minus0.1cm}
\setlength{\parindent}{0.0em}

\includeonly{./title_toc,
             ./versionhistory,
             ./analysis,
             ./input,
             ./grid_geometry,
             ./GRIB2_output_tables,
             ./icon_in_sky}



% ------------------ start of document -----------------------%
\begin{document}

\setcounter{tocdepth}{3}     %--- maximum toc depth

\pagestyle{fancy}
\fancyhead{}
\fancyfoot{}
% --- Seitenzahl bei geraden/linken Seiten nach links/aussen
% --- Seitenzahl bei ungeraden/rechten Seiten nach rechts/aussen
\fancyhead[EL,OR]{\thepage}
% --- Kapitel/Abschnitt bei geraden/linken Seiten rechts/aussen
\fancyhead[ER]{\leftmark}
% --- Unterkapitel/Unterabschnitt bei ungeraden/rechten Seiten links/aussen
\fancyhead[OL]{\rightmark}

\renewcommand{\chaptermark}[1]{%
  \markboth{\chaptername
    \ \thechapter.\ #1}{}}

\renewcommand{\sectionmark}[1]{
  \markright{\thesection.\ #1}
}


% --- Erhoehung des vertikalen Abstandes zwischen den Zeilen eines Arrays
\renewcommand{\arraystretch}{1.5}

% --- Auswahl der Formatierung des Literaturverzeichnisses (deutsche Version), erstellt mit custom-bib package
% \bibliographystyle{plain}
% \bibliographystyle{diss_bibstyle.bst}
\bibliographystyle{jas99}

% --- search path for figures
\graphicspath{{./pics/}}
% ----------------------------------------------------------------%

\frontmatter     %--- aendert Zahlen von Arabisch auf roemisch

% ---------------- Generate title page ---------------------%
\begin{titlepage}
  \begin{picture}(50,50)
  \put(0,0){\includegraphics[width=0.08\textwidth]{DWD_logo.png}}
\end{picture}
\vspace*{-1.5cm}
\begin{center}
  \Huge
  \textbf{ICON Database}\\
  \vspace{0.3cm}
  \Huge
  \textbf{Reference Manual}\\
  \vspace{2.cm}
  \Large
  \textbf{D.\ Reinert, F.\ Prill, H.\ Frank and G.\ Z\"angl}\\[1em]
  Deutscher Wetterdienst\\
  Research and development (FE13)\\
  \vspace{1.0cm}
  \begin{figure}[H]
    \centering
    \includegraphics[width=0.75\textwidth]{icon_with_nest.png}
  \end{figure}
  \vspace{0.8cm}
  \textcolor{red}{\textbf{Version: \vhCurrentVersion}}\\
  \vspace{0.5cm}
  \textbf{Last changes: \today}\\
  \vspace{2.2cm}
  Offenbach am Main, Germany\\

  \newpage

\end{center}
\end{titlepage}
% ----------------------------------------------------------------%

%---- generate version history page -----------------%
\begin{versionhistory}
  \vhEntry{0.1.0}{10.01.13}{DR|FP}{generated preliminary list of available GRIB2 output fields}
  \vhEntry{0.2.0}{12.07.13}{DR|FP}{added a short section describing the horizontal ICON grid. AUMFL\_S, AVMFL\_S added to the list of available output fields}
\end{versionhistory}

\tableofcontents              %--- generate table of contents

% remove chapter number from header for abstracts
\renewcommand{\chaptermark}[1] {
  \markboth{#1}{}
}

% sign function
\newcommand{\sgn}{\operatorname{sgn}}

% color
\newcommand{\tred}{\textcolor{red}}
\newcommand{\tblu}{\textcolor{blue}}

% shifting table captions...
\newcommand{\rb}[1]{\raisebox{4.0ex}[0pt]{#1}}

% use default again
\renewcommand{\chaptermark}[1]{%
  \markboth{\chaptername
    \ \thechapter.\ #1}{}}

\mainmatter                   %--- Zurcksetzen der Nummerierung auf arabisch


% ---------- Einbinden der verschiedenen Teildokumente -----------%

% --------------------------------------------------------------------------------
\chapter{Grid geometry}
% --------------------------------------------------------------------------------


% --------------------------------------------------------------------------------
\section{Horizontal grid}
% --------------------------------------------------------------------------------

The horizontal ICON grid consists of a set of spherical triangles that seamlessly span the entire sphere. The grid is constructed from an icosahedron (see Figure 
\ref{fig_ico_a}) which is projected onto a sphere. The spherical icosahedron (Figure \ref{fig_ico_b}) consists of $20$ equilateral spherical triangles. The edges of each triangle 
are bisected into equal halves or more generally into $n$ equal sections. Connecting the new edge points by great circle arcs yields $4$ or more generally $n^2$ spherical triangles 
within the original triangle (Figure \ref{fig_bisect}, \ref{fig_nsect}). 

\begin{figure}[h]
  \begin{minipage}[b]{0.4\textwidth}
    \centering
    \includegraphics[width=0.69\textwidth]{icosahedron.png}
    \subcaption{}\label{fig_ico_a}
  \end{minipage}\hfill
  \begin{minipage}[b]{0.4\textwidth}
    \centering
    \includegraphics[width=0.69\textwidth]{icosahedron_spherical.png}
    \subcaption{}\label{fig_ico_b}
  \end{minipage}\hfill
  \caption{Icosahedron before (a) and after (b) projection onto a sphere }

\hfill

  \begin{minipage}[b]{0.4\textwidth}
    \centering
    \includegraphics[width=0.69\textwidth]{bisection.png}
    \subcaption{}\label{fig_bisect}
  \end{minipage}\hfill
  \begin{minipage}[b]{0.4\textwidth}
    \centering
    \includegraphics[width=0.78\textwidth]{nsection.png}
    \subcaption{}\label{fig_nsect}
  \end{minipage}\hfill
  \caption{(a) Bisection of the original triangle edges (b) More general division into $n$ equal sections}
\end{figure}

ICON grids are constructed by an initial root division into $n$ sections (\textbf{R}n) followed by $k$ bisection steps (\textbf{B}k), 
resulting in a \textbf{R}n\textbf{B}k grid. Figures \ref{fig_R2B00} and \ref{fig_R2B02} show \textbf{R}2\textbf{B}00 and 
\textbf{R}2\textbf{B}02 ICON grids. Such grids avoid polar singularities of latitude-longitude grids (Figure \ref{fig_lonlat}) 
and allow a high uniformity in resolution over the whole sphere.

\begin{figure}[h]
  \begin{minipage}[b]{0.3\textwidth}
    \centering
    \includegraphics[width=0.9\textwidth]{icon_grid_R2B00.png}
    \subcaption{}\label{fig_R2B00}
  \end{minipage}\hfill
  \begin{minipage}[b]{0.3\textwidth}
    \centering
    \includegraphics[width=0.95\textwidth]{icon_grid_R2B02.png}
    \subcaption{}\label{fig_R2B02}
  \end{minipage}\hfill
  \begin{minipage}[b]{0.3\textwidth}
    \centering
    \includegraphics[width=0.95\textwidth]{lon-lat-grid.png}
    \subcaption{}\label{fig_lonlat}
  \end{minipage}\hfill
  \caption{(a) R2B00 grid. (b) R2B02 grid. (c) traditional regular latitude-longitude grid with polar singularities}
\end{figure}

Throughout this document, the grid is referred to as the ``\textbf{R}n\textbf{B}k grid'' or ``\textbf{R}n\textbf{B}k resolution''. For a given resolution \textbf{R}n\textbf{B}k, 
the total number of cells, edges, and vertices can be computed from
\begin{eqnarray*}
 n_{c} &=& 20\,n^{2}\,4^{k} \\
 n_{e} &=& 30\,n^{2}\,4^{k} \\
 n_{v} &=& 10\,n^{2}\,4^{k} + 2
\end{eqnarray*}
The average cell area $\overline{\Delta A}$ can be computed from
\begin{align*}
 \overline{\Delta A} = \frac{4\pi\,r_{e}^{2}}{n_{c}}\,,
\end{align*}
with the earth radius $r_{e}$, and $n_{c}$ the total number of cells. 
%
ICON uses an earth radius of
\begin{align*}
 r_{e} =  6.371229 \cdot 10^6 \, \m.
\end{align*}
% \footnote{This number seems to be based on Hayford's 1910 estimate
%  of the Earth. It is used in ICON as well as MPAS and was almost
%  certainly taken from the Jablonowski and Williamson test case
%  (QJRMS, 2006).}
%
Based on $\overline{\Delta A}$ one can derive an estimate of the average grid resolution 
$\overline{\Delta x}$:
\begin{align*}
 \overline{\Delta x} = \sqrt{\overline{\Delta A}} = \sqrt{\frac{\pi}{5}} \frac{r_{e}}{n\,2^{k}}
\end{align*}
Visually speaking, $\overline{\Delta x}$ is the edge length of a square which has the same area as our triangular cell.


In Table \ref{tab_res}, some characteristics of frequently used ICON grids are given. The table contains information about the total number of triangles ($n_{c}$), the average 
resolution $\overline{\Delta x}$, and the maximum/minimum cell area. The latter may be interpreted as the area for which the prognosed meteorological quantities (like temperature, 
pressure, \dots) are representative. Some additional information about ICON's horizontal grid can be found in \citet{Wan13}.

\begin{table}[H]
  \caption{Characteristics of frequently used ICON grids. $\Delta A_{max}$ and $\Delta A_{min}$ refer to the maximum and minimum area of the grid cells, respectively.}\label{tab_res}
  \begin{center}
    \begin{tabular}{p{2.0cm}>{\raggedleft\arraybackslash}p{3.5cm}>{\centering\arraybackslash}p{3.5cm}>{\raggedleft\arraybackslash}p{2.5cm}>{\raggedleft\arraybackslash}p{2.5cm}}
    \toprule
    \textbf{Grid} & \textbf{number of cells ($n_{c}$)} & \textbf{avg.\ resolution [km]} & $\mathbf{\Delta A_{max}\,[km^{2}]}$ & $\mathbf{\Delta A_{min}\,[km^{2}]}$\\
    \midrule
    R2B04         &    20480                           &  157.8                         &  25974.2                  &  18777.3 \\
    R2B05         &    81920                           &   78.9                         &  6480.8                   & 4507.5\\
    R2B06         &   327680                           &   39.5                         &  1618.4                   & 1089.6 \\
    R2B07         &  1310720                           &   19.7                         &  404.4                    & 265.1 \\
    R3B07         &  2949120                           &   13.2                         &  179.7                    & 116.3 \\
    \bottomrule
    \end{tabular}
  \end{center}
\end{table}

\begin{note}
  \textbf{\textcolor{red}{The first operational version of ICON is
      based on the R3B07 grid, thus, having a horizontal resolution of
      about $13\,\mathrm{km}$!}}
\end{note}


% --------------------------------------------------------------------------------
\section{Vertical grid}
% --------------------------------------------------------------------------------

The vertical grid consists of a set of vertical layers with height-based vertical coordinates.
Each of these layers carries the horizontal $2D$ grid structure, thus forming the $3D$ structure of the grid.
The ICON grid employs a Lorenz-type staggering with the vertical velocity defined at the boundaries of layers (half levels) 
and the other prognostic variables in the center of the layer (full levels).

To improve simulations of flow past complex topography, the ICON model employs a smooth level vertical (SLEVE) coordinate~\citep{Leuenberger2010}. 
It allows for a faster transition to smooth levels in the upper troposphere and lower stratosphere, as compared to the classical height-based Gal-Chen 
coordinate. In the operational setup, the transition from terrain following levels in the lower atmosphere to constant height levels is completed 
at $z=16\,\mathrm{km}$. Model levels above are flat. The required smooth large-scale contribution of the model topography is generated by 
digital filtering with a $\nabla^2$-diffusion operator. Figure~\ref{fig:vertical_levels} shows the (half) levels of the operational 
ICON setup with 90 vertical levels. The table to the right shows the height above ground of selected half levels (for zero height topography) 
and the corresponding pressure, assuming the US standard atmosphere. Standard heights for all $91$ half levels are given in Table 
\ref{tab:half_level_heights}.

\textbf{Please note that for grid cells with non-zero topography these values only represent rough estimates of the true level height. 
Actual heights and layer thicknesses may vary considerably from location to location, due to grid level stretching/compression over 
non-zero topography.}
  
%In the operational setup, the transition from terrain following levels in the lower atmosphere to constant height levels is completed at 
%$z=16\,\mathrm{km}$. Model levels above are flat.



% ---------------------------------------------------------------------------------------
% ICON vertical levels -- SLEVE coordinates
% 
% author: 07/2013: F. Prill, DWD
% created with a Matlab script and the following settings:
%
%   min_lay_thckn   = 20.;        % Layer thickness of lowermost layer
%   top_height      = 75000.;     % Height of model top
%   stretch_fac     = 0.9;        % Scaling factor for stretching/squeezing 
%                                 % the model layer distribution
%   
%   decay_scale_1   = 4000.;      % Decay scale of large-scale topography component
%   decay_scale_2   = 2500.;      % Decay scale of small-scale topography component
%   decay_exp       = 1.2;        % Exponent for decay function
%   flat_height     = 16000.;     % Height above which the coordinate surfaces are flat
%   topo            = 0.;
%   topo_smt        = 0.;  
% 
% see also ICON routines "init_sleve_coord", "init_vert_coord"        
%
% Input files for this figure are created with the following commands:
%
%  octave -q --eval 'nlev=90; [z_m, z_i, pres_m, pres_i] = icon_levels(nlev); printf("k z p\n"); for k=1:nlev printf("%d %5.3f %5.3f\n",k,z_i(k),z_i(k)-z_i(k+1));end' > vertical_levels_i.txt 
%
%  octave -q --eval 'nlev=90; [z_m, z_i, pres_m, pres_i] = icon_levels(nlev); printf("k z p\n"); for k=[1:5:nlev,nlev] printf("%d %d %5.1f\n",k,z_i(k),pres_i(k));end' > vertical_levels_i_small.txt 
%
% (Alternatively, for a pressure plot:)
%  octave -q --eval 'nlev=90; [z_m, z_i, pres_m, pres_i] = icon_levels(nlev); printf("k z p\n"); for k=1:nlev printf("%d %5.3f %5.3f\n",k,z_i(k),pres_i(k));end' > vertical_levels_i.txt 
%
%
% ---------------------------------------------------------------------------------------
\begin{figure}[hbt]
\begin{tabbing}
  \begin{minipage}[t]{0.65\linewidth} \vspace*{0pt}%
    \pgfplotstableread{level_tables/vertical_levels_i.txt}{\loadedtable}
    \pgfplotsset{
      tick label style={font=\small},
    }
  \begin{tikzpicture}[scale=0.7, transform shape]
    \begin{axis}[ minor tick num=1, axis x line=bottom, axis y line=left, 
                  every inner x axis line/.append style= {|->},
                  every inner y axis line/.append style= {|->}, 
                  width=10cm,height=14.5cm, ymajorgrids,
                  xlabel=level, 
                  scaled y ticks = false,
                  ylabel=\textbf{\color{blue}$z~\text{[m]}$}, enlargelimits=false,
                  every axis y label/.style={at={(current axis.north west)},
                                             yshift=3em,anchor=north east}
                ]
      \addplot[blue,mark=*] table[x={k},y={z}] \loadedtable;
    \end{axis}
    \begin{axis}[ minor tick num=1, axis x line=bottom, axis y line=right, 
                  every inner x axis line/.append style= {|->},
                  every inner y axis line/.append style= {|->}, 
                  width=10cm,height=14.5cm, 
                  scaled y ticks = false,
                  ylabel=\textbf{\color{red}$\Delta z~\text{[m]}$}, enlargelimits=false ,
                  every axis y label/.style={at={(current axis.north east)},
                                             yshift=3em,anchor=north west}
                 ]
      \addplot[red,only marks,mark=diamond] table[x={k},y={p}] \loadedtable;
    \end{axis}
  \end{tikzpicture}
  \end{minipage}
%
%
\=
%
%
  \begin{minipage}[t]{0.35\linewidth}\vspace*{0em}%
   \renewcommand{\baselinestretch}{0.95}\normalsize%
   \pgfkeys{/pgf/number format/set thousands separator={\,}}
   \pgfplotstableread{level_tables/vertical_levels_i_small.txt}{\loadedtablesmall}\vspace*{0pt}%
   \pgfplotstabletypeset[ columns={k,z,p},every  head row/.style={after row={\hline}},
          font=\footnotesize,
          columns/k/.style={column name=$level$, column type=r,column type/.add={>{\columncolor[gray]{.8}}}{}},
          columns/z/.style={column name=$[m]$,   fixed,dec sep align},
          columns/p/.style={column name=$[Pa]$, fixed,dec sep align, zerofill,precision=1},
                        ] {\loadedtablesmall}
    \vspace*{1em}%
  \end{minipage}
\end{tabbing}
%
  \caption{Vertical half levels (blue) and layer thickness (red) of the ICON operational setup.
    The table of selected pressure values (for zero height) is based on the 1976 US standard atmosphere.}
  \label{fig:vertical_levels}%
\end{figure}


% --------------------------------------------------------------------------------
\section{Refined subregion over Europe (``local nest'')}
% --------------------------------------------------------------------------------

ICON has the capability for running global simulations with refined
domains (so called \emph{nests}).
%
The triangular mesh of the refined area is generated by bisection of
triangles in the global ``parent'' grid, see
Fig.~\ref{fig:icon_grid_refinement_zoom_view}.
In the vertical the global grid extends into the mesosphere (which
greatly facilitates the assimilation of satellite data) whereas the
nested domains extend only into the lower stratosphere in order to
save computing time. For the same orography the heights of levels $1$--$60$ of
the Europe nest are the same as those of levels $31$--$90$ of the global grid. 
In practice, however, near surface level heights of nests and the global 
domain differ due to the fact that the underlying orography differs, 
with deeper slopes and higher summits in the high resolution nests.

For each nesting level, the time step is automatically divided by a
factor of two.
%
Note that the grid nests are computed in a concurrent fashion:  
\begin{itemize}
\item Points that are covered by the refined subdomain additionally
  contain data for the global grid state.
\item The data points on the triangular grid are the cell
  circumcenters. Therefore the global grid data points are closely
  located to nest data sites, but they \emph{do not coincide} exactly 
  (see Fig.~\ref{fig:icon_grid_refinement_zoom_view}).
\end{itemize}


\begin{figure}[hbt]
  \centering
  \includegraphics[width=0.90\textwidth]{pics/grid_refinement.png}
  \caption{ICON grid refinement (zoom view). Blue and red dots indicate the cell circumcenters for the global (``parent'') and the refined (``child'') domain, respectively.}
  \label{fig:icon_grid_refinement_zoom_view}
\end{figure}

ICON's refined subregion over Europe is comparable to the COSMO-EU region 
of DWD's COSMO model. Key figures like edge coordinates and mesh size of the COSMO-EU region and the ICON-EU nest 
are given in Table ~\ref{tab:COSMO_ICON_nest_extent}. The geographical location of the nest is visualized in 
Fig.~\ref{fig:EU_nest} and Fig.~\ref{fig:EU_nest_polar}.

\begin{table}
\centering
\captionabove{Key figures of the ICON-EU nest and the COSMO-EU region.}\label{tab:COSMO_ICON_nest_extent}

\begin{tabular}{|p{5cm}|l|l|}\hline
\rowcolor{Gray}
                           &    {\centering\textbf{ICON-EU nest}}                 &     {\centering\textbf{COSMO-EU}} \\ \hline\hline
% ----------------------------------------------------------------------------------------------------------------------------------------
geogr. coordinates         &    $23.5^\degree~\text{W}$ -- $62.5^\degree~\text{E}$    &     $\lambda_{\text{N}} = 170^\degree~\text{W}$, 
                                                                                          $\phi_{\text{N}}    =  40^\degree~\text{N}$,     \\
                           &    $29.5^\degree~\text{N}$ -- $70.5^\degree~\text{N}$   &      $18.0^\degree~\text{W} - 23.5^\degree~\text{E}$ \\
                           &                                                      &     $20.0^\degree~\text{S} - 21.0^\degree~\text{N}$  \\[0.5em]
mesh size                  &    $\approx 6.5~\km$ (R3B08)                         &     $0.0625^\degree$ ($\approx 7~\km$) \\
                           &    659156~triangles                                  &     $665 \times 657 = 436905$ grid points \\
vertical levels            &    60 levels                                         &     40 levels      \\
upper boundary             &    $22.5~\km$                                        &     $22.5~\km$ \\ \hline

\end{tabular}
\end{table}

\begin{note}
     \textbf{Model simulations including the nesting region over
      Europe are running regularly, starting from 
      \begin{center}
        2015-07-21, 06~UTC (\texttt{roma}).
      \end{center}
      Main forecasts starting at $00$, $06$, $12$, $18\,\mathrm{UTC}$ reach out to 
      $120\,\mathrm{h}$, while additional short-range forecasts starting at 
      $03$, $09$, $15$, $21\,\mathrm{UTC}$ provide data until $+30\,\mathrm{h}$.
      }     
\end{note}


Simulation on the global grid and the regional (Europe) domain are tightly coupled
(\emph{two-way nesting}):
Boundary data for the nest area is updated every time step ($120\,\mathrm{s}$).
%
Feedback of atmospheric prognostic variables (except precipitation) is
computed via relaxation on a $3\,\mathrm{h}$ time scale.

  
\begin{figure}[h]
  \begin{minipage}[b]{\textwidth}
    \centering
    \includegraphics[width=0.90\textwidth]{ICON_EU_nest_noCEU_1.png}
    \subcaption{}\label{fig:EU_nest}
  \end{minipage}
  \hfill
  \par
  \begin{minipage}[b]{\textwidth}
    \centering
    \includegraphics[width=0.80\textwidth]{ICON_EU_nest_noCEU_2.png}
    \subcaption{}\label{fig:EU_nest_polar}
  \end{minipage}\hfill
  \caption{\ref{fig:EU_nest}: Horizontal extent of the ICON-EU nest
    (orange shaded area) in a cylindrical equidistant projection. 
    \ref{fig:EU_nest_polar}:  Same as \ref{fig:EU_nest} but in a polar
    stereographic projection.}
\end{figure}





\chapter{Analysis fields}

The 3-hourly first guess output of ICON contains the following fields:

\begin{longtable}{p{4.0cm}P{7.0cm}}
\caption[]{Available 3h first guess output fields}\\
  \toprule
\multicolumn{1}{c}{\textbf{Type}}  &  \multicolumn{1}{c}{\textbf{GRIB shortName}}\\
\midrule
\endfirsthead
\caption[]{\emph{continued}}\\
\midrule
\endhead
\hline \multicolumn{2}{r}{\textit{Continued on next page}} \\
\endfoot
\endlastfoot
Atmosphere                             &  VN, U, V, W, DEN, THETA\_V, T, QV, QC, QI, QR, QS, TKE, P                     \\
Surface (general)                      &  T\_G, T\_SO(0), QV\_S, T\_2M, TD\_2M, U\_10M, V\_10M, PS, Z0                       \\
Land specific                          &  W\_SNOW, T\_SNOW, RHO\_SNOW, H\_SNOW, FRESHSNW, W\_I, T\_SO(1:nlev\_soil), W\_SO, W\_SO\_ICE \\
Lake/sea ice specific                  &  T\_MNW\_LK, T\_WML\_LK, H\_ML\_LK, T\_BOT\_LK, C\_T\_LK, T\_B1\_LK, H\_B1\_LK, T\_ICE, H\_ICE, FR\_ICE\\
Time invariant                         &  FR\_LAND, HHL, CLON, CLAT, ELON, ELAT, VLON, VLAT \\
  \bottomrule
\end{longtable}

Atmospheric analysis fields are computed every 3 hours ($00$, $03$, $06$,$\dots$ $21$ UTC) with the 3DVar data assimilation system. Sea surface temperature (T\_SO(0)) and 
sea ice cover (FR\_ICE) are provided once per day (00 UTC) by the SST-Analysis. A snow analysis is conducted every 3 hours. In addition a soil moisture analysis (SMA) 
is conducted once per day (00 UTC). It basically modifies the soil moisture content (W\_SO), in order to improve the $2\,\mathrm{m}$ temperature forecast. 

 
For the 3-hourly analysis cycle, ICON must be provided with $2$ input files, containing First Guess~(FG) and analysis~(AN) fields, respectively. Variables for which no analysis 
is available are always read from the first guess file (e.g.\ TKE). Other variables may be either read from the first guess or the analysis file, depending on the 
starting time. E.g.\ for T\_SO(0) the first guess is read at 03, 06, 09, 12, 15, 18, 21 UTC, however, the analyis is read at 00 UTC. In Table~\ref{tbl_analysis} the available 
and employed first guess and analysis fields are listed as a function of starting time.

\begin{longtable}{p{3.3cm}>{\centering\arraybackslash}p{2.5cm}p{0.7cm}p{0.7cm}p{0.7cm}p{0.7cm}p{0.7cm}p{0.7cm}p{0.7cm}p{0.7cm}}
\caption[]{The leftmost column shows variables that are mandatory for the assimilation cycle and forecast runs.  Column 2 indicates, whether or not an analysis is performed 
for these variables. Columns 3 to 10 show the origin of these variables (analysis or first guess), depending on the starting time.}\label{tbl_analysis}\\
  \toprule
\textbf{ShortName}  &  \textbf{Analysis}  & \textbf{00} & \textbf{03} & \textbf{06} & \textbf{09} & \textbf{12} & \textbf{15} & \textbf{18} &  \textbf{21} \\
\midrule
\endhead
\hline \multicolumn{10}{r}{\textit{Continued on next page}} \\
\endfoot
\endlastfoot
\hline \multicolumn{10}{l}{\textbf{Atmosphere}} \\
VN                  &     --              &   FG         &     FG      &     FG      &     FG      &     FG      &     FG      &     FG      &    FG         \\
THETA\_V            &     --              &   FG         &     FG      &     FG      &     FG      &     FG      &     FG      &     FG      &    FG         \\
DEN                 &     --              &   FG         &     FG      &     FG      &     FG      &     FG      &     FG      &     FG      &    FG         \\
W                   &     --              &   FG         &     FG      &     FG      &     FG      &     FG      &     FG      &     FG      &    FG         \\
TKE                 &     --              &   FG         &     FG      &     FG      &     FG      &     FG      &     FG      &     FG      &    FG         \\
T                   &     3DVar           &   \tred{AN}  &  \tred{AN}  &  \tred{AN}  &   \tred{AN} &   \tred{AN} &  \tred{AN}  &  \tred{AN}  &  \tred{AN}    \\
P                   &     3DVar           &   \tred{AN}  &  \tred{AN}  &  \tred{AN}  &   \tred{AN} &   \tred{AN} &  \tred{AN}  &  \tred{AN}  &  \tred{AN}    \\
U, V                &     3DVar           &   \tred{AN}  &  \tred{AN}  &  \tred{AN}  &   \tred{AN} &   \tred{AN} &  \tred{AN}  &  \tred{AN}  &  \tred{AN}    \\
QV, QC, QI, QR, QS  &     3DVar           &   \tred{AN}  &  \tred{AN}  &  \tred{AN}  &   \tred{AN} &   \tred{AN} &  \tred{AN}  &  \tred{AN}  &  \tred{AN}    \\
\hline \multicolumn{10}{l}{\textbf{Surface}} \\
Z0                  &     --              &   FG         &     FG      &     FG      &     FG      &     FG      &     FG      &     FG      &    FG         \\
T\_G                &     --              &   FG         &     FG      &     FG      &     FG      &     FG      &     FG      &     FG      &    FG         \\
QV\_S               &     --              &   FG         &     FG      &     FG      &     FG      &     FG      &     FG      &     FG      &    FG         \\
T\_SO(0)            &    Ana\_SST         &   \tred{AN}  &     FG      &     FG      &     FG      &     FG      &     FG      &     FG      &    FG         \\
T\_SO(1:nlevsoil)   &     --              &   FG         &     FG      &     FG      &     FG      &     FG      &     FG      &     FG      &    FG         \\
W\_SO\_ICE          &     --              &   FG         &     FG      &     FG      &     FG      &     FG      &     FG      &     FG      &    FG         \\
W\_SO               &      SMA            &   \tred{AN}  &  \tred{AN}  &  \tred{AN}  &   \tred{AN} &   \tred{AN} &  \tred{AN}  &  \tred{AN}  &  \tred{AN}    \\
W\_I                &    Ana\_SNOW        &   \tred{AN}  &  \tred{AN}  &  \tred{AN}  &   \tred{AN} &   \tred{AN} &  \tred{AN}  &  \tred{AN}  &  \tred{AN}    \\
W\_SNOW             &    Ana\_SNOW        &   \tred{AN}  &  \tred{AN}  &  \tred{AN}  &   \tred{AN} &   \tred{AN} &  \tred{AN}  &  \tred{AN}  &  \tred{AN}    \\
T\_SNOW             &    Ana\_SNOW        &   \tred{AN}  &  \tred{AN}  &  \tred{AN}  &   \tred{AN} &   \tred{AN} &  \tred{AN}  &  \tred{AN}  &  \tred{AN}    \\
RHO\_SNOW           &    Ana\_SNOW        &   \tred{AN}  &  \tred{AN}  &  \tred{AN}  &   \tred{AN} &   \tred{AN} &  \tred{AN}  &  \tred{AN}  &  \tred{AN}    \\
H\_SNOW             &    Ana\_SNOW        &   \tred{AN}  &  \tred{AN}  &  \tred{AN}  &   \tred{AN} &   \tred{AN} &  \tred{AN}  &  \tred{AN}  &  \tred{AN}    \\
FRESHSNW            &    Ana\_SNOW        &   \tred{AN}  &  \tred{AN}  &  \tred{AN}  &   \tred{AN} &   \tred{AN} &  \tred{AN}  &  \tred{AN}  &  \tred{AN}    \\
\hline \multicolumn{10}{l}{\textbf{Sea ice/Lake}} \\
T\_ICE              &     Ana\_SST        &   \tred{AN}  &     FG      &     FG      &     FG      &     FG      &     FG      &     FG      &    FG         \\
H\_ICE              &     Ana\_SST        &   \tred{AN}  &     FG      &     FG      &     FG      &     FG      &     FG      &     FG      &    FG         \\
FR\_ICE             &     Ana\_SST        &   \tred{AN}  &     FG      &     FG      &     FG      &     FG      &     FG      &     FG      &    FG         \\
T\_MNW\_LK          &      --             &   FG         &     FG      &     FG      &     FG      &     FG      &     FG      &     FG      &    FG         \\
T\_WML\_LK          &      --             &   FG         &     FG      &     FG      &     FG      &     FG      &     FG      &     FG      &    FG         \\
H\_ML\_LK           &      --             &   FG         &     FG      &     FG      &     FG      &     FG      &     FG      &     FG      &    FG         \\
T\_BOT\_LK          &      --             &   FG         &     FG      &     FG      &     FG      &     FG      &     FG      &     FG      &    FG         \\
C\_T\_LK            &      --             &   FG         &     FG      &     FG      &     FG      &     FG      &     FG      &     FG      &    FG         \\
T\_B1\_LK           &      --             &   FG         &     FG      &     FG      &     FG      &     FG      &     FG      &     FG      &    FG         \\
H\_B1\_LK           &      --             &   FG         &     FG      &     FG      &     FG      &     FG      &     FG      &     FG      &    FG         \\
  \bottomrule
\end{longtable}

\chapter{Mandatory input fields}

Several input files are needed to perform runs of the ICON model. 
%
These can be divided into three classes:
%
Grid files, external parameters, and initialization (analysis) files. The latter 
will be described in Chapter \ref{sec_analysis}.


%%%%%%%%%%%%%%%%%%%%%%%%%%%%%%%%%%%%%%%%%%%%%%%%%%%%%%%%%%%%%%%%%%%%%%%%%%%%%%%%%%%
\section{Grid Files}
\label{section:grid_files}


% TODO[FP,DR]: Please add here information
% on the
% - UUID, numberOfGridUsed
% - additional neighbor relationship info (neighbor cell)
% - general remark: What is NetCDF; mention tools for processing NetCDF


In order to run ICON, it is necessary to load the horizontal grid
information as an input para\-meter. 
This information is stored within so-called grid files. For an ICON 
run, at least one global grid file is required.
For model runs with nested grids, additional files of the nested
domains are necessary. Optionally, a reduced radiation grid for
the global domain may be used.

%
The unstructured triangular ICON grid resulting from the grid
generation process is represented in NetCDF format.  The most
important data entries are

\begin{itemize}
 \item \texttt{cell} (INTEGER dimension) \\
        number of (triangular) cells
 \item \texttt{vertex} (INTEGER dimension) \\
        number of triangle vertices
 \item \texttt{edge} (INTEGER dimension) \\
        number of triangle edges
 \item \texttt{clon}, \texttt{clat} (double array, dimension: \#triangles, given in radians) \\
        longitude/latitude of the triangle circumcenters
 \item \texttt{vlon}, \texttt{vlat} (double array, dimension: \#triangle vertices, given in radians) \\
        longitude/latitude of the triangle vertices
 \item \texttt{elon}, \texttt{elat} (double array, dimension: \#triangle edges, given in radians) \\
       longitude/latitude of the edge midpoints
 \item \texttt{cell\_area} (double array, dimension: \#triangles) \\
       triangle areas
 \item \texttt{vertex\_of\_cell} (INTEGER array, dimensions: [3, \#triangles]) \\
       The indices \texttt{vertex\_of\_cell(:,i)} denote the triangle vertices that belong 
       to the triangle~\texttt{i}.
 \item \texttt{edge\_of\_cell} (INTEGER array, dimensions: [2, \#triangles]) \\
       The indices \texttt{edge\_of\_cell(:,i)} denote the triangle edges that belong
       to the triangle~\texttt{i}.
\end{itemize}

%For fixed domain sizes and resolutions a list of grid files has been pre-built for the ICON model.
%These are publically available via the download server
%{\normalsize%
%\begin{verbatim}
%   http://icon-downloads.zmaw.de
%\end{verbatim}
%}%

%%%%%%%%%%%%%%%%%%%%%%%%%%%%%%%%%%%%%%%%%%%%%%%%%%%%%%%%%%%%%%%%%%%%%%%%%%%%%%%%%%%

\section{External parameters}
\label{section:extpar}
External parameters are used to describe the properties of the earth's surface. 
These data include e.g.\ the orography, the land-sea-mask as well as parameters describing 
soil and surface properties, like the soiltype or the plant cover fraction.

The ExtPar software (ExtPar -- External parameter for Numerical Weather Prediction and Climate Application) 
is able to generate external parameters for the ICON model. The generation is based on a set of 
raw datafields which are listed in Table \ref{table_extpar_raw}. For a more detailed overview of ExtPar, 
the reader is referred to the \emph{User and Implementation Guide} of Extpar.

\begin{longtable}{p{6.5cm}p{6cm}p{1.8cm}}
\captionabove[]{Raw datasets from which the ICON external parameter fields are derived.}\label{table_extpar_raw}\\
  \toprule
\textbf{Dataset} &\textbf{Source} &\textbf{Resolution} \\
\midrule
\endfirsthead
\caption[]{\emph{continued}}\\
\midrule
\endhead
\hline \multicolumn{3}{r}{\textit{Continued on next page}} \\
\endfoot
\endlastfoot
GLOBE orography                                        &  NOAA/NGDC                  &  30'' \\
%ASTER orography \newline (limited domain: 60 N - 60 S) &  METI/NASA                  &  1''   \\
GlobCover 2009                                         &  ESA                        &  10''  \\
%GLC2000 land use                                       &  JRC Ispra                  &  30''  \\
GLCC land use                                          &  USGS                       &  30''  \\
DSMW Digital Soil Map of the World                     &  FAO                        &  5'    \\
%HWSD Harmonized World Soil Database                    &  FAO/IIASA/ISRIC/ISSCAS/JRC &  30''  \\
NDVI Climatotology, SeaWiFS                            &  NASA/GSFC                  &  2.5'  \\
CRU near surface climatology                           &  CRU University of East Anglia & $0.5^{\circ}$  \\
GACP Aerosol Optical thickness                         &  NASA/GISS \newline (Global Aerosol Climatology Project)   &  $4x5^{\circ}$ \\
GLDB Global lake database                              &  DWD/RSHU/MeteoFrance       &  30''  \\
MODIS albedo                                           &  NASA                       &  5'    \\
\bottomrule
\end{longtable}

\emph{GlobCover 2009} is a land cover database covering the whole globe, except for Antarctica. Therefore, we make use of 
\emph{GlobCover 2009} for $90^{\circ} > \phi > -56^{\circ}$ (with $\phi$ denoting latitude) and switch to the coarser, 
however globally available dataset \emph{GLCC} for $ -56^{\circ} \geq \psi > -90^{\circ}$.

The products generated by the ExtPar software package are listed in Table \ref{table_extpar_products} together with the underlying 
raw dataset. These are mandatory input fields for assimilation- and forecast runs.

\begin{longtable}{p{2.5cm}p{8.5cm}p{3.3cm}}
\captionabove[]{External parameter fields for ICON, produced by the ExtPar software package (in alphabetical order)}\label{table_extpar_products}\\
% \begin{tabular}{p{2.5cm}p{8.5cm}p{3.3cm}}
  \toprule
\multicolumn{1}{c}{\textbf{ShortName}}  &  \multicolumn{1}{c}{\textbf{Description}}  &  \multicolumn{1}{c}{\textbf{Raw dataset}}\\
\midrule
\endfirsthead
\caption[]{\emph{continued}}\\
\midrule
\endhead
\hline \multicolumn{3}{r}{\textit{Continued on next page}} \\
\endfoot
\endlastfoot
  AER\_SS12                             & Sea salt aerosol climatology (monthly fields)   &       GACP                \\
  AER\_DUST12                           & Total soil dust aerosol climatology (monthly fields) &  GACP                \\
  AER\_ORG12                            & Organic aerosol climatology (monthly fields)       &    GACP                \\
  AER\_SO412                            & Total sulfate aerosol climatology (monthly fields) &    GACP                \\
  AER\_BC12                             & Black carbon aerosol climatology (monthly fields)  &    GACP                \\
  ALB\_DIF12                            & Shortwave ($0.3 - 5.0\, \mathrm{\mu m}$) albedo for diffuse radiation (monthly fields)&  MODIS    \\
  ALB\_UV12                             & UV-visible ($0.3 - 0.7\, \mathrm{\mu m}$) albedo for diffuse radiation (monthly fields)& MODIS     \\
  ALB\_NI12                             & Near infrared ($0.7 - 5.0\, \mathrm{\mu m}$) albedo for diffuse radiation (monthly fields)& MODIS     \\
  DEPTH\_LK                             & Lake depth                                      &        GLDB               \\
  EMIS\_RAD                             & Surface longwave (thermal) emissivity           &        GlobCover 2009     \\               
  FOR\_D  (*)                           & Fraction of deciduous forest                    &        GlobCover 2009     \\
  FOR\_E  (*)                           & Fraction of evergreen forest                    &        GlobCover 2009     \\
  FR\_LAKE                              & Lake fraction (fresh water)                     &        GLDB               \\                     
  FR\_LAND                              & Land fraction (excluding lake fraction but including glacier fraction) & GlobCover2009   \\
  FR\_LUC                               & Landuse class fraction                          &                           \\
  HSURF                                 & Orography height at cell centres                &        GLOBE              \\
  LAI\_MX  (*)                          & Leaf area index in the vegetation phase         &        GlobCover 2009     \\
  NDVI\_MAX                             & Normalized differential vegetation index        &        SeaWiFS            \\
  NDVI\_MRAT                            & proportion of monthly mean NDVI to yearly maximum (monthly fields)&  SeaWiFS \\
  PLCOV\_MX  (*)                        & Plant covering degree in the vegetation phase   &        GlobCover 2009     \\
  ROOTDP (*)                            & Root depth                                      &        GlobCover 2009     \\
  RSMIN  (*)                            & Minimum stomatal resistance                     &        GlobCover 2009     \\
  SOILTYP                               & Soil type                                       &        DSMW               \\
  SSO\_STDH                             & Standard deviation of sub-grid scale orographic height  &   GLOBE           \\
  SSO\_THETA                            & Principal axis-angle of sub-grid scale orography &          GLOBE           \\
  SSO\_GAMMA                            & Horizontal anisotropy of sub-grid scale orography &         GLOBE           \\
  SSO\_SIGMA                            & Average slope of sub-grid scale orography       &           GLOBE           \\
  T\_2M\_CL                             & Climatological 2m temperature (serves as lower boundary condition for soil model)  &  CRU \\
  Z0 (*)                                & Surface roughness length (over land), containing a contribution from subgrid-scale orography  & GlobCover 2009    \\                        
  \bottomrule
% \end{tabular}
\end{longtable}

Note that fields marked with (*) are not required in operational model runs. I.e.\ the surface roughness \texttt{Z0} is only required, 
if the additional contribution from sub-grid scale orography shall be taken into account (i.e.\ for \nml{itype\_z0=1}). In operational runs 
this is not the case. Instead, land-cover class specific roughness lengths are taken from a GlobCover-based lookup table. 
\texttt{FOR\_D}, \texttt{FOR\_E}, \texttt{LAI\_MX}, \texttt{PLCOV\_MX}, 
\texttt{RSMIN}, and \texttt{ROOTDP} became obsolete with the activation of the surface tile approach (2015-03-04). The latter $4$ fields 
are replaced by land-cover class specific values taken from lookup tables.



\subsubsection*{Remarks on post-processing}
Some of the external parameter fields are further modified by ICON. The following fields are affected: 
\begin{tabbing}
\hspace{0.20\textwidth} \= \hspace{0.20\textwidth} \= \hspace{0.20\textwidth} \= \hspace{0.20\textwidth} \= \hspace{0.20\textwidth} \kill
\texttt{DEPTH\_LK}  \>
\texttt{HSURF} \>
\texttt{FR\_LAND} \>
\texttt{FR\_LAKE} \>
\texttt{Z0}
%\texttt{SSO\_STDH}
%DR Note that sso_stdh is affected as well, however we do not output it!
\end{tabbing}
Thus, for consistency, the modified fields should be used for post-processing tasks rather than the original external parameter fields. 
See Section \ref{sec_const_outfields} for more details.

\chapter{Available output fields in GRIB2-format}

% shifting table captions...
\newcommand{\rb}[1]{\raisebox{4.0ex}[0pt]{#1}}

In GRIB2, a variable is uniquely defined by the following set of metadata:
\begin{itemize}
 \item \emph{Discipline} (see GRIB2 code table 4.2)
 \item \emph{ParameterCategory} (see GRIB2 code table 4.2)
 \item \emph{ParameterNumber} (see GRIB2 code table 4.2)
 \item \emph{typeOfFirstfixedSurface} and \emph{typeOfSecondFixedSurface} (see GRIB2 code table 4.5)
 \item \emph{stepType} (instant, accum, avg, max, min, diff, rms, sd, cov, \dots)
\end{itemize}
In the following, \emph{typeOfFirstFixedSurface} and \emph{typeOfSecondFixedSurface} will be abbreviated by \emph{Lev-Typ~1/2}.

\section{Deprecated output fields}
With the launch of ICON, the following output fields will no longer be available:

\begin{itemize}
 \item \textbf{OMEGA} [Pa/s]: Vertical velocity in pressure coordinates $\omega=\frac{\mathrm{d}p}{\mathrm{d}t}$
 \item \textbf{BAS\_CON} [\textendash]: Level index of convective cloud base
 \item \textbf{TOP\_CON} [\textendash]: Level index of convective cloud top
 \item \textbf{T\_S} [K]: Temperature at the soil-atmosphere-, or soil-snow-interface. Note that $\mathrm{T\_S} = \mathrm{T\_SO(0)}$, thus $\mathrm{T\_S}$ is redundant.
 \item \textbf{W\_G1} [mm H2O]: Soil water content in upper layer ($0$ to $10\,\mathrm{cm}$) 
 \item \textbf{W\_G2} [mm H2O]: Soil water content in middle layer ($10$ to $100\,\mathrm{cm}$)
\end{itemize}


\section{New output fields}
New output fields that will become available with the launch of ICON are:
\begin{itemize}
 \item \textbf{W} $[m/s]$: vertical velocity in height coordinates $w=\frac{\mathrm{d}z}{\mathrm{d}t}$
 \item \textbf{DEN} $[kg/m^{3}]$: density of moist air
\end{itemize}


\section{Available output fields listed in tabular form}

\begin{table}[H]
\caption{Hybrid multi-layer forecast ($VV>0$) and initialised analysis ($VV=0$) products}
 \begin{tabular}{p{2.0cm}p{5.0cm}p{0.8cm}p{0.8cm}p{0.8cm}p{0.9cm}p{1cm}p{1cm}}
  \toprule
\multicolumn{1}{c}{\begin{sideways}\textbf{ShortName}\end{sideways}}    &      \multicolumn{1}{c}{\rb{\textbf{Description}}}    & \begin{sideways}\textbf{Discipline}\end{sideways} & \begin{sideways}\bf{Category}\end{sideways} & \begin{sideways}\bf{Number}\end{sideways}  & \begin{sideways}\bf{Lev-Typ}\end{sideways}  & \begin{sideways}\bf{stepType}\end{sideways} &\begin{sideways}\bf{Unit}\end{sideways}\\
\midrule
U                          &  Zonal wind                       &               0                                   &                     2                       &                    2                       &                 105                         &                      inst                   &        $\mathrm{m\,s^{-1}}$   \\ 
V                          &  Meridional wind                  &               0                                   &                     2                       &                    3                       &                 105                         &                      inst                   &        $\mathrm{m\,s^{-1}}$   \\
W                          &  Vertical wind                    &               0                                   &                     2                       &                    9                       &                 105                         &                      inst                   &        $\mathrm{m\,s^{-1}}$   \\
T                          &  Temperature                      &               0                                   &                     0                       &                    0                       &                 105                         &                      inst                   &        $\mathrm{K}$          \\
DEN                        &  Density of moist air             &               0                                   &                     3                       &                    10                      &                 105                         &                      inst                   &        $\mathrm{kg\,m^{-3}}$ \\
QV                         &  Specific humidity                &               0                                   &                     1                       &                    0                       &                 105                         &                      inst                   &        $\mathrm{kg\,kg^{-1}}$ \\
QC                         &  \textcolor{red}{Cloud mixing ratio}\footnotemark[1]&             0                   &                     1                       &                    22                      &                 105                         &                      inst                   &        $\mathrm{kg\,kg^{-1}}$ \\
QI                         &  \textcolor{red}{Cloud ice mixing ratio}\footnotemark[1]&         0                   &                     1                       &                    82                      &                 105                         &                      inst                   &        $\mathrm{kg\,kg^{-1}}$ \\
QR                         &  \textcolor{red}{Rain mixing ratio}\footnotemark[1]&              0                   &                     1                       &                    24                      &                 105                         &                      inst                   &        $\mathrm{kg\,kg^{-1}}$ \\
QS                         &  \textcolor{red}{Snow mixing ratio}\footnotemark[1]&              0                   &                     1                       &                    25                      &                 105                         &                      inst                   &        $\mathrm{kg\,kg^{-1}}$ \\
CLC                        &  Cloud cover                       &              0                                   &                     6                       &                    22                      &                 105                         &                      inst                   &        $\mathrm{\%}$ \\
\textcolor{gray}{O3}       &  \textcolor{gray}{Ozone mixing ratio}\footnotemark[2] &              \textcolor{gray}{0}              &        \textcolor{gray}{14}                 &      \textcolor{gray}{1}                   &     \textcolor{gray}{105}                   &     \textcolor{gray}{inst}                  &        \textcolor{gray}{$\mathrm{kg\,kg^{-1}}$} \\                                             
  \bottomrule
 \end{tabular}
\end{table}
\footnotetext[1]{for the time being, erroneously encoded as mixing ratios instead of specific quantities}
\footnotetext[2]{not clear yet, whether ozone will be provided as output field}



\begin{longtable}{p{2.0cm}p{5.0cm}p{0.8cm}p{0.8cm}p{0.8cm}p{0.9cm}p{1cm}p{1cm}}
\caption[]{Single-layer forecast ($VV>0$) and initialised analysis ($VV=0$) products}\\
  \toprule
\multicolumn{1}{c}{\begin{sideways}\textbf{ShortName}\end{sideways}}  &  \multicolumn{1}{c}{\rb{\textbf{Description}}}  & \begin{sideways}\textbf{Discipline}\end{sideways} & \begin{sideways}\bf{Category}\end{sideways} & \begin{sideways}\bf{Number}\end{sideways}  & \begin{sideways}\bf{Lev-Typ}\end{sideways}  & \begin{sideways}\bf{stepType}\end{sideways} &\begin{sideways}\bf{Unit}\end{sideways}\\
\midrule
\endfirsthead
\caption[]{\emph{continued}}\\
\midrule
\endhead
\hline \multicolumn{8}{r}{\textit{Continued on next page}} \\
\endfoot
\endlastfoot
PS                             &  Surface pressure (not reduced)                                                        &               0                                   &                     3                       &                    1                       &                 1                           &                      inst                   &        $\mathrm{Pa}$   \\ 
T\_SNOW                        &  Temperature of the snow surface                                                       &               0                                   &                     0                       &                    18                      &                 1                           &                      inst                   &        $\mathrm{K}$    \\
T\_G                           &  Ground temperature (temperature at sfc-atm interface)                                 &               0                                   &                     0                       &                    0                       &                 1                           &                      inst                   &        $\mathrm{K}$    \\
QV\_S                          &  Surface specific humidity                                                             &               0                                   &                     1                       &                    0                       &                 1                           &                      inst                   &        $\mathrm{kg\,kg^{-1}}$    \\
W\_SNOW                        &  Snow depth water equivalent                                                           &               0                                   &                     1                       &                    60                      &                 1                           &                      inst                   &        $\mathrm{kg\,m^{-2}}$    \\
W\_I                           &  Plant canopy surface water                                                            &               2                                   &                     0                       &                    13                      &                 1                           &                      inst                   &        $\mathrm{kg\,m^{-2}}$    \\
TCM                            &  Turbulent transfer coefficient for momentum (surface)                                 &               0                                   &                     2                       &                    29                      &                 1                           &                      inst                   &        $\mathrm{--}$    \\ 
TCH                            &  Turbulent transfer coefficient for heat and moisture (surface)                        &               0                                   &                     0                       &                    19                      &                 1                           &                      inst                   &        $\mathrm{--}$    \\
ASOB\_S                        &  Net short-wave radiation flux at surface (average since model start)                  &               0                                   &                     4                       &                     9                      &                 1                           &                      avg                    &        $\mathrm{W\,m^{-2}}$    \\
ATHB\_S                        &  Net long-wave radiation flux at surface (average since model start)                   &               0                                   &                     5                       &                     5                      &                 1                           &                      avg                    &        $\mathrm{W\,m^{-2}}$    \\
ASOB\_T                        &  Net short-wave radiation flux at TOA (average since model start)                      &               0                                   &                     4                       &                     9                      &                 8                           &                      avg                    &        $\mathrm{W\,m^{-2}}$    \\
ATHB\_T                        &  Net long-wave radiation flux at TOA (average since model start)                       &               0                                   &                     5                       &                     5                      &                 8                           &                      avg                    &        $\mathrm{W\,m^{-2}}$    \\ 
ALB\_RAD                       &  Surface albedo for visible range, diffuse                                             &               0                                   &                    19                       &                     1                      &                 1                           &                      inst                   &        $\mathrm{\%}$    \\
RAIN\_GSP                      &  Large scale rain (accumulated since model start)                                      &               0                                   &                     1                       &                    77                      &                 1                           &                      accu                   &        $\mathrm{kg\,m^{-2}}$    \\
SNOW\_GSP                      &  Large snowfall water equivalent (accumulated since model start)                       &               0                                   &                     1                       &                    56                      &                 1                           &                      accu                   &        $\mathrm{kg\,m^{-2}}$    \\
RAIN\_CON                      &  Convective rain (accumulated since model start)                                       &               0                                   &                     1                       &                    76                      &                 1                           &                      accu                   &        $\mathrm{kg\,m^{-2}}$    \\
SNOW\_CON                      &  Convective snowfall water equivalent (accumulated since model start)                  &               0                                   &                     1                       &                    55                      &                 1                           &                      accu                   &        $\mathrm{kg\,m^{-2}}$    \\
TOT\_PREC                      &  Total precipitation (accumulated since model start)                                   &               0                                   &                     1                       &                    52                      &                 1                           &                      accu                   &        $\mathrm{kg\,m^{-2}}$  \\
\textcolor{gray}{RUNOFF\_S}    &  \textcolor{gray}{Surface water runoff (accumulated since model start)}\footnotemark[3]&               \textcolor{gray}{2}                 &                     \textcolor{gray}{0}     &                     \textcolor{gray}{5}    &                 \textcolor{gray}{1}         &                      \textcolor{gray}{accu} &        \textcolor{gray}{$\mathrm{kg\,m^{-2}}$}  \\
\textcolor{gray}{RUNOFF\_G}    &  \textcolor{gray}{Soil water runoff (accumulated since model start)}\footnotemark[3]   &               \textcolor{gray}{2}                 &                     \textcolor{gray}{0}     &                     \textcolor{gray}{5}    &                 \textcolor{gray}{1}         &                      \textcolor{gray}{accu} &        \textcolor{gray}{$\mathrm{kg\,m^{-2}}$}  \\                                      
U\_10M                         &  Zonal wind at 10m above ground                                                        &               0                                   &                     2                       &                     2                      &               103                           &                      inst                   &        $\mathrm{m\,s^{-1}}$  \\
V\_10M                         &  Meridional wind at 10m above ground                                                   &               0                                   &                     2                       &                     3                      &               103                           &                      inst                   &        $\mathrm{m\,s^{-1}}$  \\
T\_2M                          &  Temperature at 2m above ground                                                        &               0                                   &                     0                       &                     0                      &               103                           &                      inst                   &        $\mathrm{K}$          \\
TD\_2M                         &  Dew point temperature at 2m above ground                                              &               0                                   &                     0                       &                     6                      &               103                           &                      inst                   &        $\mathrm{K}$          \\
Z0                             &  Surface roughness (above land and water)                                              &               2                                   &                     0                       &                     1                      &                 1                           &                      inst                   &        $\mathrm{m}$          \\
CLCT                           &  Total cloud cover                                                                     &               0                                   &                     6                       &                     1                      &                 1                           &                      inst                   &        $\mathrm{\%}$          \\
\textcolor{gray}{CLCH}         &  \textcolor{gray}{High level clouds}\footnotemark[3]                                   &               \textcolor{gray}{0}                 &                     \textcolor{gray}{6}     &                    \textcolor{gray}{1}     &                 \textcolor{gray}{100}       &                      inst                   &        $\mathrm{\%}$          \\
\textcolor{gray}{CLCM}         &  \textcolor{gray}{Mid level clouds}\footnotemark[3]                                    &               \textcolor{gray}{0}                 &                     \textcolor{gray}{6}     &                    \textcolor{gray}{1}     &                 \textcolor{gray}{100}       &                      inst                   &        $\mathrm{\%}$          \\
\textcolor{gray}{CLCL}         &  \textcolor{gray}{Low level clouds}\footnotemark[3]                                    &               \textcolor{gray}{0}                 &                     \textcolor{gray}{6}     &                    \textcolor{gray}{1}     &                 \textcolor{gray}{100}       &                      inst                   &        $\mathrm{\%}$          \\
TQV                            &  Total column integrated water vapour                                                  &               0                                   &                     1                       &                    64                      &                 1                           &                      inst                   &        $\mathrm{kg\,m^{-2}}$  \\
TQC                            &  Total column integrated cloud water                                                   &               0                                   &                     1                       &                    69                      &                 1                           &                      inst                   &        $\mathrm{kg\,m^{-2}}$  \\
TQI                            &  Total column integrated cloud ice                                                     &               0                                   &                     1                       &                    70                      &                 1                           &                      inst                   &        $\mathrm{kg\,m^{-2}}$  \\
\textcolor{gray}{TQR}          &  \textcolor{gray}{Total column integrated rain}\footnotemark[3]                        &               \textcolor{gray}{0}                 &                     \textcolor{gray}{1}     &                    \textcolor{gray}{45}    &                 \textcolor{gray}{1}         &                      \textcolor{gray}{inst} &        \textcolor{gray}{$\mathrm{kg\,m^{-2}}$}  \\
\textcolor{gray}{TQS}          &  \textcolor{gray}{Total column integrated snow}\footnotemark[3]                        &               \textcolor{gray}{0}                 &                     \textcolor{gray}{1}     &                    \textcolor{gray}{46}    &                 \textcolor{gray}{1}         &                      \textcolor{gray}{inst} &        \textcolor{gray}{$\mathrm{kg\,m^{-2}}$}  \\
HBAS\_CON                      &  Height of convective cloud base above msl                                             &               0                                   &                     6                       &                    26                      &                 2                           &                      inst                   &        $\mathrm{m}$  \\
HTOP\_CON                      &  Height of convective cloud top above msl                                              &               0                                   &                     6                       &                    27                      &                 3                           &                      inst                   &        $\mathrm{m}$  \\
HZEROCL                        &  Height of 0 degree Celsius isotherm above msl                                         &               0                                   &                     3                       &                     6                      &                 4                           &                      inst                   &        $\mathrm{m}$  \\
ASHFL\_S                       &  Sensible heat net flux at surface (average since model start)                         &               0                                   &                     0                       &                    11                      &                 1                           &                      avg                    &        $\mathrm{W\,m^{-2}}$  \\
ALHFL\_S                       &  Latent heat net flux at surface (average since model start)                           &               0                                   &                     0                       &                    10                      &                 1                           &                      avg                    &        $\mathrm{W\,m^{-2}}$  \\
FR\_ICE                        &  Sea ice cover  (possible range: $[0,1]$)                                              &              10                                   &                     2                       &                     0                      &                 1                           &                      inst                   &        $\mathrm{\textendash}$  \\
T\_ICE                         &  Sea ice temperature (at ice-atm interface)                                            &              10                                   &                     2                       &                     8                      &                 1                           &                      inst                   &        $\mathrm{K}$  \\
H\_ICE                         &  Sea ice thickness (Max: $3\,\mathrm{m}$)                                              &              10                                   &                     2                       &                     1                      &                 1                           &                      inst                   &        $\mathrm{m}$  \\
FRESHSNW                       &  Fresh snow factor (weighting function for albedo indicating freshness of snow)        &               0                                   &                     1                       &                   203                      &                 1                           &                      inst                   &        $\mathrm{\textendash}$  \\
RHO\_SNOW                      &  Snow density                                                                          &               0                                   &                     1                       &                    61                      &                 1                           &                      inst                   &        $\mathrm{kg\,m^{-3}}$  \\
H\_SNOW                        &  Snow depth                                                                            &               0                                   &                     1                       &                    11                      &                 1                           &                      inst                   &        $\mathrm{m}$  \\
  \bottomrule
\end{longtable}

\footnotetext[3]{Output fields not yet available, but planned.}



\begin{longtable}{p{2.0cm}p{5.0cm}p{0.8cm}p{0.8cm}p{0.8cm}p{0.9cm}p{1cm}p{1cm}}
\caption[]{Multi-layer forecast ($VV>0$) and initialised analysis ($VV=0$) products of the soil model}\\
  \toprule
\multicolumn{1}{c}{\begin{sideways}\textbf{ShortName}\end{sideways}}  &  \multicolumn{1}{c}{\rb{\textbf{Description}}}  & \begin{sideways}\textbf{Discipline}\end{sideways} & \begin{sideways}\bf{Category}\end{sideways} & \begin{sideways}\bf{Number}\end{sideways}  & \begin{sideways}\bf{Lev-Typ}\end{sideways}  & \begin{sideways}\bf{stepType}\end{sideways} &\begin{sideways}\bf{Unit}\end{sideways}\\
\midrule
%\endfirsthead
%\caption[]{\emph{continued}}\\
%\midrule
\endhead
\hline \multicolumn{8}{r}{\textit{Continued on next page}} \\
\endfoot
\endlastfoot
T\_SO                          &  Soil temperature                                                                      &               2                                   &                     3                       &                    18                       &               106                           &                      inst                   &        $\mathrm{K}$   \\
W\_SO                          &  Soil moisture integrated over individual soil layers  (ice + liquid)                  &               2                                   &                     3                       &                    20                       &               106                           &                      inst                   &        $\mathrm{kg\,m^{-2}}$   \\
W\_SO\_ICE                     &  Soil ice content integrated over individual soil layers                               &               2                                   &                     3                       &                    22                       &               106                           &                      inst                   &        $\mathrm{kg\,m^{-2}}$   \\
  \bottomrule
\end{longtable}

Soil temperature is defined at the soil depths given in Table \ref{tab_soillayer} (column 2). Levels $1$ to $8$ define the full levels of the soil model. A zero gradient 
condition is assumed between levels $0$ and $1$, meaning that temperatures at the surface-atmosphere interface are set equal to the temperature at the first full level depth.
($0.5\,\mathrm{cm}$). Temperatures are prognosed for levels $1$ to $7$. At the lowermost level ($1458\,\mathrm{cm}$) the temperature is fixed to the climatological 
average $2\,\mathrm{m}$-temperature.

Soil moisture $\mathrm{W\_SO}$ is prognosed for layers $1$ to $6$. In the two lowermost layers $\mathrm{W\_SO}$ is time constant.

\begin{table}
\center
\caption{Soil model: vertical distribution of levels and layers}\label{tab_soillayer}
 \begin{tabular}{>{\centering\arraybackslash}p{2.0cm}>{\centering\arraybackslash}p{2.5cm}|>{\centering\arraybackslash}p{2.5cm}>{\centering\arraybackslash}p{5.0cm}}
 \toprule
  \bf{level no.}       &  \bf{depth [cm]}        &   \bf{layer no.}        & \bf{upper/lower bounds [cm]} \\
 \midrule
         0             &     $0.0$               &                         &                                     \\
         1             &     $0.5$               &         1               &     $0.0$\, \textemdash\, $1.0$     \\
         2             &     $2.0$               &         2               &     $1.0$\, \textemdash\, $3.0$     \\
         3             &     $6.0$               &         3               &     $3.0$\, \textemdash\, $9.0$     \\
         4             &     $18.0$              &         4               &     $9.0$\, \textemdash\, $27.0$    \\
         5             &     $54.0$              &         5               &    $27.0$\, \textemdash\, $81.0$    \\
         6             &     $162.0$             &         6               &    $81.0$\, \textemdash\, $243.0$   \\
         7             &     $486.0$             &         7               &   $243.0$\, \textemdash\, $729.0$   \\
         8             &     $1458.0$            &         8               &   $729.0$\, \textemdash\, $2187.0$  \\
 \bottomrule
 \end{tabular}
\end{table}


\svnInfo $Id$
% --------------------------------------------------------------------------------
\chapter[ICON data in the SKY data bases of DWD]{ICON data in the\\SKY data bases of DWD}
\label{section:icon_in_sky}
% --------------------------------------------------------------------------------


GRIB data of the numerical weather prediction models are stored in the data base
SKY at DWD.
Documentation on the SKY system is available in the intranet of DWD at
%\url{IT/Messnetz/Technik $\rightarrow$ Datenmanagement (technisch) $\rightarrow$
%Management der DWD Fachdaten -Dokumentation $\rightarrow$ SKY}.
%\url{IT/Messnetz/Technik -> Datenmanagement (technisch) ->
%Management der DWD Fachdaten - Dokumentation -> SKY}.
{\tt IT/Messnetz/Technik $\rightarrow$ Datenmanagement (technisch) $\rightarrow$
Management der DWD Fachdaten -Dokumentation $\rightarrow$ SKY}.
Here, some remarks are given on the SKY categories for ICON data, and some
examples are given how to retrieve data from the data base.

\section{SKY categories for ICON}\label{sec_skycat}

In SKY the data is stored in different categories and data base subsystems.
These are identified by the cat=CAT\_NAME parameter.
The name of a category is made up of 4 parts:

\begin{center}
 \begin{minipage}{0.25\textwidth}
  \centering
  \textbf{\${model}\_\${run}\_\${type}\_\${suite}}
 \end{minipage}
\end{center}
 
run, type, and suite are general for all forecast models of DWD. They can have the 
following values:
\begin{itemize}
 \item \textbf{run}: \textbf{main} for main forecast runs,
                     \textbf{ass} for assimilation runs,
                     \textbf{pre} for pre-assimilation runs,
                     \textbf{const} for invariant data.
 \item \textbf{type}: \textbf{an} for analysis data,
                      \textbf{fc} for forecast data.
 \item \textbf{suite}: 
       \begin{itemize}
         \item \textbf{rout} for operational data in \emph{db=roma,}
         \item \textbf{para} or \textbf{para1} for pre-operational data in \emph{db=parma},
                       The category extension para1 denotes the data with EU nest
                       (starting at 12 UTC, 2015032612).
         \item \textbf{exp} or \textbf{exp1} for data from experiments in
                       \emph{db=numex}. The category extension exp1 is used for
                       experiments of the NUMEX wizard, a special NUMEX user.

                       Data from experiments is additionally
                       identified by the parameter \emph{exp=}$NUM$ where $NUM$ is
                       the experiment number.
       \end{itemize}
\end{itemize}

The ICON categories start with the string \textbf{ico} for ICON data on 
the native ICON grid, or with \textbf{icr} for data on a regular lat-lon grid.
Next follows a two-letter string to identify the domain of ICON; \textbf{gl} for the
global domain, \textbf{eu} for the nest over Europe. Further particulars of the category 
name differ for the global and nested domain. In case of the global domain, \textbf{gl} 
is followed by the mesh width of the model in units of 100 m, and then the number of levels 
preceeded by the letter l. As an example \textbf{icogl130l90} is on the native grid from a global 
model with a mesh width $13\,\mathrm{km}$ (grid R3B07) and 90 levels. \textbf{icrgl400l90} is data 
on a regular grid from a global model with mesh width $40\,\mathrm{km}$ (R2B06) and 90 levels. 
For the nested domain, the specification of the mesh width and number of levels is omitted. As an 
example, \textbf{icreu} is the ICON nest over Europe (with a mesh width of $6.5\,\mathrm{km}$ and 60 
levels), interpolated to a regular lat-lon grid.

For ensemble forecasts or ensemble analyses the first part of the
category is extended by an \textbf{e} (for instance \textbf{icogle}
or \textbf{icrgle}). Ensemble members or ranges of ensemble members
are specified by the parameter \emph{enum=}$NUM$
or \emph{enum=}$NUM1-NUM2$ where $NUM$ is the member id.

Hence, the full category name for data from a global operational forecast run of ICON on a
regular grid will be \textbf{icrgl130l90\_main\_fc\_rout}. The initial analysis for this
run is in category\\ \textbf{icogl130l90\_main\_an\_rout}.

\begin{note}
Since 2014-08-12 12 UTC ICON is running pre-operationally at DWD. Hence, forecast data
is available in the sky database \textbf{db=parma} in categories
\textbf{icogl130l90\_main\_fc\_para} and \textbf{icrgl130l90\_main\_fc\_para}.
\end{note}

\begin{note}
Since 2015-01-20 06 UTC the \emph{global} ICON model is running operationally at DWD.
Forecast data is available in the sky database \textbf{db=roma} in categories
\textbf{icogl130l90\_main\_fc\_rout} and \textbf{icrgl130l90\_main\_fc\_rout}.
\end{note}

\section{Retrieving ICON data from SKY}\label{sec_example}

Here we shall give several examples how to retrieve ICON data from SKY.
The parameter d specifies the reference or initial date, s is the forecast step, p the parameter,
and f the name of the GRIB data file.

\begin{itemize}
\item Retrieve the 2m temperature and dew point temperature 
 for forecast hours 3 to 78 every 3 hours of today's run at 00 UTC
 on the global domain from an ICON run on a R3B07 grid with 90 levels to file icon2mdat
\begin{skydb}
 read db=roma cat=icrgl130l90_main_fc_rout d=t00 s[h]=3/to/78/by/3 p=t_2m,td_2m bin f=icon2mdat
\end{skydb}

\item Retrieve the analysis of T on the native grid from yesterday 18 UTC:
\begin{skydb}
read db=roma cat=icogl130l90_main_an_rout d=t18-1d p=T gptype=0 bin f=t_icon_ana
\end{skydb}

\item Retrieve the 6, 12, 18, and 24 hour forecast of the 2m temperature from a forecast in experiment
10096 on 2015-09-05 at 00 UTC on the global domain from an ICON run on a R3B07 grid with 90 levels:

\begin{skydb}
read db=numex cat=icrgl130l90_main_fc_exp exp=10096 d=2015090500 s[h]=6,12,18,24 p=t_2m bin f=t_2m_fc.grb
\end{skydb}

\item Retrieve accumulated precipiation of the ICON-EU nest on the regular grid every 6 hours to 72 hours
from the yesterday's operational run at 12 UTC:

\begin{skydb}
read db=roma cat=icreu_main_fc_rout d=t12-1d s[h]=6/to/72/by/6 p=tot_prec bin f=tot_prec_ieu
\end{skydb}



\item List the data on pressure levels of the 18 hours forecast from 06 UTC of ICON-EU nest on the
regular grid. Write reference date (d), forecast step (s), level type (lv), value of first level (lv1),
decoding date (dedat), and store date (stdat) in information file icr.info.
\begin{skydb}
read db=roma cat=icreu_main_fc_rout d=06 step[h]=18 lv=P info=metaData metaArray=d,s,p,lv,lv1,dedat,stdat sort=d,s,p,lv,lv1 infof=icr.info
\end{skydb}

\item Retrieve temperature in 850 hPa from the first guess of the 40 ensemble members of the EDA
on the 40 km grid in the parallel suite yesterday at 21 UTC. Sort the data by ensemble member.

\begin{skydb}
read db=parma cat=icrgle_ass_fc_para1 enum=1/to/ d=t21-1d s=3 p=T lv=P lv1=85000 info=epsInfo sort=enum
\end{skydb}

\item Retrieve all available time-invariant (constant) fields on the native grid and store them in the file const\_icongl. 
Write reference date (d), forecast step (s), level type (lv), value of first level (lv1), decoding date (dedat), 
store date (stdat), and validity date (valdat) in information file icr.info. It is important to set \textbf{invar=true}.

\begin{skydb}
read db=roma cat=icogl130l90_const_an_rout invar=true info=metaData metaArray=d,s,p,lv,lv1,dedat,stdat bin infof=icr.info f=const_icongl
\end{skydb}

\end{itemize}

% ----------------------------------------------------------------%

\appendix
% \chapter{ICON standard half level heights}
\label{appendix_levelheights}

ICON standard half level heights $z^{h}$ are listed in Table \ref{tab:half_level_heights}. If full level heights $z^{f}$ are required, 
these can be deduced as follows: Let $i$ denote the full level index for which the height is wanted. Then the full level 
height $z^{f}_{i}$ is given by
\begin{align}
 z^{f}_{i} = 0.5 \left(z^{h}_{i} + z^{h}_{i+1}\right)
\end{align}


\begin{table}[hb]
  \caption{Standard heights (i.e.\ for zero topography height) for all $91$ vertical half levels.}
  \label{tab:half_level_heights}%

   \renewcommand{\baselinestretch}{1.00}\normalsize%
   \pgfkeys{/pgf/number format/set thousands separator={\,}}
   \pgfplotstableread{vertical_levels_i.txt}{\loadedtable}\vspace*{0pt}%
%   \pgfplotstabletypeset[ columns={k,z,p},every  head row/.style={after row={\hline}},
   \pgfplotstabletypeset[ 
          begin table=\begin{longtable}, 
          end table=\end{longtable},
          columns={k,z,k,z,k,z},
          every  head row/.style={after row={\hline}},
          precision=2,
          font=\normalsize,
%          columns/k/.style={column name=level index, column type=c,column type/.add={>{\columncolor[gray]{.8}}}{}},
%          columns/z/.style={column name=height $[m]$,   fixed,dec sep align},
%          columns/p/.style={column name=$[Pa]$, fixed,dec sep align, zerofill,precision=1},
          columns/k/.style={column name=level index, column type=c, column type/.add={>{\columncolor[gray]{.8}}}{}},
          columns/z/.style={column name=height $[m]$, fixed,dec, zerofill,precision=1},
          display columns/0/.style={select equal part entry of={0}{3},string type},
          display columns/1/.style={select equal part entry of={0}{3},string type},
          display columns/2/.style={select equal part entry of={1}{3},string type},
          display columns/3/.style={select equal part entry of={1}{3},string type},
          display columns/4/.style={select equal part entry of={2}{3},string type},
          display columns/5/.style={select equal part entry of={2}{3},string type},
                        ] {\loadedtable}

\end{table}

  %--- Einbinden der Appendix

% \listoffigures                    %--- Erstellung des Figurenverzeichnisses

% \glsaddall
% \printglossary[type=symbolslist]  %--- Erstellung des Symbolverzeichnisses

\backmatter
% \bibliography{/home/dreinert/Documents/bibliography/litera.bib,/home/dreinert/Documents/bibliography/litera_notprinted.bib} %--- Erstellen der Liste aller Referenzen
\bibliography{litera.bib} %--- Create list of references

% remove chapter number from header for Acknowledgements
\renewcommand{\chaptermark}[1] {
  \markboth{#1}{}
}
% --- Danksagung ---
% \include{./text/danksagung}

% use default again
\renewcommand{\chaptermark}[1]{%
  \markboth{\chaptername
    \ \thechapter.\ #1}{}}


\end{document}
