\documentclass[a4paper,twoside,10pt]{book}

\usepackage{a4,amssymb,amsxtra,alltt}
\usepackage[english]{babel}
\usepackage{float}
\usepackage[utf8]{inputenc}
\usepackage{graphics,graphicx}
\usepackage{amsmath,amsbsy,bm}
\usepackage{fancyhdr}
\usepackage{geometry}
\usepackage{amsthm}        %--- Theorem-Package
% \usepackage{pdftricks}    %--- for drawing arrows, frames, etc in Latex
% \begin{psinputs}
\usepackage{pstricks}      %--- for drawing arrows, frames etc in Latex
% \end{psinputs}
\usepackage[nottoc]{tocbibind}     %--- adds bibliography to toc 
% --- but without toc itself [nottoc] 
\usepackage[numbers,round,authoryear]{natbib} %--- creates bibliography; author-year with curly brackets within the text
% \usepackage[  %--- no numbering of positions 
% toc,           %--- let it appear in the document's table of contents
% style=mylong3colheader %--- see glossaries Documentation for a list of available styles
% ]{glossaries}  %--- use the glossary package
\geometry{a4paper,textwidth=15.8cm,textheight=23.0cm,twoside,rmargin=0.96in} %--- Variation Randbereiche
\usepackage{epsfig}
% \usepackage{subfigure}  %--- simplifies putting pictures side by side
\usepackage{layout}      %--- overview about boundaries and widths
                         % usage: write \layout somewhere in the document
\usepackage{longtable, lscape}
\usepackage{setspace}
\usepackage{multirow}
\usepackage{tabularx}
\usepackage{colortbl}
\usepackage{booktabs}    % to give tables a more professional look
\usepackage{rotating}    % rotate column labels
\usepackage{xcolor}
\usepackage{listings}

% Fortran environment
\lstnewenvironment{fortran}%
{\lstset{language=[95]Fortran,%
    basicstyle=\ttfamily\footnotesize\color{darkgreen},%
    commentstyle=\ttfamily\color{blue},%
    emptylines=0,%
    keywordstyle=\color{black}\ttfamily\bfseries,%
    backgroundcolor=\color{darkgrey!5},
    framexleftmargin=4mm,%
    frame=shadowbox,%
    rulesepcolor=\color{darkgreen}}}
{}



\newcolumntype{x}[1]{%
  >{\centering\hspace{0pt}}p{#1}}%


% ---------- Glossary: All the stuff and more ----------%
% \newglossary{symbolslist}{sym}{sbl}{Symbolverzeichnis} % define new glossary i.e. a list of symbols
% \makeglossaries

% \input{./text/listofsymbols}  %--- include defined symbols






% ---------- Definition des Styles von Theoremen ----------%
\theoremstyle{plain}
\newtheorem*{thrm}{Theorem}  %--- define new theorem-style






% ---------------bold caption for figures-------------------------%
\makeatletter
\long\def\@makecaption#1#2{%
  \fontfamily{cmr}
  \fontseries{m}
  \fontshape{sl}
  \fontsize{10pt}{11pt}
  \selectfont
  \vskip 10\p@
  \setbox\@tempboxa\hbox{{\bf#1:} #2}%
  \ifdim \wd\@tempboxa >\hsize
  {\bf #1:} #2\par
  \else
  \hbox to\hsize{\hfil\box\@tempboxa\hfil}%
  \fi}
\makeatother
% ----------------------------------------------------------------%




\includeonly{./title_toc,
             ./GRIB2_output_tables}



% ------------------ start of document -----------------------%
\begin{document}

\setcounter{tocdepth}{3}     %--- maximum toc depth

\pagestyle{fancy}
\fancyhead{}
\fancyfoot{}
% --- Seitenzahl bei geraden/linken Seiten nach links/aussen
% --- Seitenzahl bei ungeraden/rechten Seiten nach rechts/aussen
\fancyhead[EL,OR]{\thepage}
% --- Kapitel/Abschnitt bei geraden/linken Seiten rechts/aussen
\fancyhead[ER]{\leftmark}
% --- Unterkapitel/Unterabschnitt bei ungeraden/rechten Seiten links/aussen
\fancyhead[OL]{\rightmark}

\renewcommand{\chaptermark}[1]{%
  \markboth{\chaptername
    \ \thechapter.\ #1}{}}

\renewcommand{\sectionmark}[1]{
  \markright{\thesection.\ #1}
}


% --- Erhoehung des vertikalen Abstandes zwischen den Zeilen eines Arrays
\renewcommand{\arraystretch}{1.5}

% --- Auswahl der Formatierung des Literaturverzeichnisses (deutsche Version), erstellt mit custom-bib package
% \bibliographystyle{plain}
% \bibliographystyle{diss_bibstyle.bst}
% \bibliographystyle{jas99}

% --- search path for figures
\graphicspath{{./pics/}}
% ----------------------------------------------------------------%

\frontmatter     %--- aendert Zahlen von Arabisch auf roemisch

% ---------------- Generate title page ---------------------%
\begin{titlepage}
  \begin{picture}(50,50)
  \put(0,0){\includegraphics[width=0.08\textwidth]{DWD_logo.png}}
\end{picture}
\vspace*{-1.5cm}
\begin{center}
  \Huge
  \textbf{ICON Database}\\
  \vspace{0.3cm}
  \Huge
  \textbf{Reference Manual}\\
  \vspace{2.cm}
  \Large
  \textbf{D.\ Reinert, F.\ Prill, H.\ Frank and G.\ Z\"angl}\\[1em]
  Deutscher Wetterdienst\\
  Research and development (FE13)\\
  \vspace{1.0cm}
  \begin{figure}[H]
    \centering
    \includegraphics[width=0.75\textwidth]{icon_with_nest.png}
  \end{figure}
  \vspace{0.8cm}
  \textcolor{red}{\textbf{Version: \vhCurrentVersion}}\\
  \vspace{0.5cm}
  \textbf{Last changes: \today}\\
  \vspace{2.2cm}
  Offenbach am Main, Germany\\

  \newpage

\end{center}
\end{titlepage}
% ----------------------------------------------------------------%

\tableofcontents              %--- generate table of contents

% remove chapter number from header for abstracts
\renewcommand{\chaptermark}[1] {
  \markboth{#1}{}
}

% sign function
\newcommand{\sgn}{\operatorname{sgn}}


% use default again
\renewcommand{\chaptermark}[1]{%
  \markboth{\chaptername
    \ \thechapter.\ #1}{}}

\mainmatter                   %--- Zurcksetzen der Nummerierung auf arabisch


% ---------- Einbinden der verschiedenen Teildokumente -----------%
\chapter{Available output fields in GRIB2-format}

% shifting table captions...
\newcommand{\rb}[1]{\raisebox{4.0ex}[0pt]{#1}}

In GRIB2, a variable is uniquely defined by the following set of metadata:
\begin{itemize}
 \item \emph{Discipline} (see GRIB2 code table 4.2)
 \item \emph{ParameterCategory} (see GRIB2 code table 4.2)
 \item \emph{ParameterNumber} (see GRIB2 code table 4.2)
 \item \emph{typeOfFirstfixedSurface} and \emph{typeOfSecondFixedSurface} (see GRIB2 code table 4.5)
 \item \emph{stepType} (instant, accum, avg, max, min, diff, rms, sd, cov, \dots)
\end{itemize}
In the following, \emph{typeOfFirstFixedSurface} and \emph{typeOfSecondFixedSurface} will be abbreviated by \emph{Lev-Typ~1/2}.

\section{Deprecated output fields}
With the launch of ICON, the following output fields will no longer be available:

\begin{itemize}
 \item \textbf{OMEGA} [Pa/s]: Vertical velocity in pressure coordinates $\omega=\frac{\mathrm{d}p}{\mathrm{d}t}$
 \item \textbf{BAS\_CON} [\textendash]: Level index of convective cloud base
 \item \textbf{TOP\_CON} [\textendash]: Level index of convective cloud top
 \item \textbf{T\_S} [K]: Temperature at the soil-atmosphere-, or soil-snow-interface. Note that $\mathrm{T\_S} = \mathrm{T\_SO(0)}$, thus $\mathrm{T\_S}$ is redundant.
 \item \textbf{W\_G1} [mm H2O]: Soil water content in upper layer ($0$ to $10\,\mathrm{cm}$) 
 \item \textbf{W\_G2} [mm H2O]: Soil water content in middle layer ($10$ to $100\,\mathrm{cm}$)
\end{itemize}


\section{New output fields}
New output fields that will become available with the launch of ICON are:
\begin{itemize}
 \item \textbf{W} $[m/s]$: vertical velocity in height coordinates $w=\frac{\mathrm{d}z}{\mathrm{d}t}$
 \item \textbf{DEN} $[kg/m^{3}]$: density of moist air
\end{itemize}


\section{Available output fields listed in tabular form}

\begin{table}[H]
\caption{Hybrid multi-layer forecast ($VV>0$) and initialised analysis ($VV=0$) products}
 \begin{tabular}{p{2.0cm}p{5.0cm}p{0.8cm}p{0.8cm}p{0.8cm}p{0.9cm}p{1cm}p{1cm}}
  \toprule
\multicolumn{1}{c}{\begin{sideways}\textbf{ShortName}\end{sideways}}    &      \multicolumn{1}{c}{\rb{\textbf{Description}}}    & \begin{sideways}\textbf{Discipline}\end{sideways} & \begin{sideways}\bf{Category}\end{sideways} & \begin{sideways}\bf{Number}\end{sideways}  & \begin{sideways}\bf{Lev-Typ}\end{sideways}  & \begin{sideways}\bf{stepType}\end{sideways} &\begin{sideways}\bf{Unit}\end{sideways}\\
\midrule
U                          &  Zonal wind                       &               0                                   &                     2                       &                    2                       &                 105                         &                      inst                   &        $\mathrm{m\,s^{-1}}$   \\ 
V                          &  Meridional wind                  &               0                                   &                     2                       &                    3                       &                 105                         &                      inst                   &        $\mathrm{m\,s^{-1}}$   \\
W                          &  Vertical wind                    &               0                                   &                     2                       &                    9                       &                 105                         &                      inst                   &        $\mathrm{m\,s^{-1}}$   \\
T                          &  Temperature                      &               0                                   &                     0                       &                    0                       &                 105                         &                      inst                   &        $\mathrm{K}$          \\
DEN                        &  Density of moist air             &               0                                   &                     3                       &                    10                      &                 105                         &                      inst                   &        $\mathrm{kg\,m^{-3}}$ \\
QV                         &  Specific humidity                &               0                                   &                     1                       &                    0                       &                 105                         &                      inst                   &        $\mathrm{kg\,kg^{-1}}$ \\
QC                         &  \textcolor{red}{Cloud mixing ratio}\footnotemark[1]&             0                   &                     1                       &                    22                      &                 105                         &                      inst                   &        $\mathrm{kg\,kg^{-1}}$ \\
QI                         &  \textcolor{red}{Cloud ice mixing ratio}\footnotemark[1]&         0                   &                     1                       &                    82                      &                 105                         &                      inst                   &        $\mathrm{kg\,kg^{-1}}$ \\
QR                         &  \textcolor{red}{Rain mixing ratio}\footnotemark[1]&              0                   &                     1                       &                    24                      &                 105                         &                      inst                   &        $\mathrm{kg\,kg^{-1}}$ \\
QS                         &  \textcolor{red}{Snow mixing ratio}\footnotemark[1]&              0                   &                     1                       &                    25                      &                 105                         &                      inst                   &        $\mathrm{kg\,kg^{-1}}$ \\
CLC                        &  Cloud cover                       &              0                                   &                     6                       &                    22                      &                 105                         &                      inst                   &        $\mathrm{\%}$ \\
\textcolor{gray}{O3}       &  \textcolor{gray}{Ozone mixing ratio}\footnotemark[2] &              \textcolor{gray}{0}              &        \textcolor{gray}{14}                 &      \textcolor{gray}{1}                   &     \textcolor{gray}{105}                   &     \textcolor{gray}{inst}                  &        \textcolor{gray}{$\mathrm{kg\,kg^{-1}}$} \\                                             
  \bottomrule
 \end{tabular}
\end{table}
\footnotetext[1]{for the time being, erroneously encoded as mixing ratios instead of specific quantities}
\footnotetext[2]{not clear yet, whether ozone will be provided as output field}



\begin{longtable}{p{2.0cm}p{5.0cm}p{0.8cm}p{0.8cm}p{0.8cm}p{0.9cm}p{1cm}p{1cm}}
\caption[]{Single-layer forecast ($VV>0$) and initialised analysis ($VV=0$) products}\\
  \toprule
\multicolumn{1}{c}{\begin{sideways}\textbf{ShortName}\end{sideways}}  &  \multicolumn{1}{c}{\rb{\textbf{Description}}}  & \begin{sideways}\textbf{Discipline}\end{sideways} & \begin{sideways}\bf{Category}\end{sideways} & \begin{sideways}\bf{Number}\end{sideways}  & \begin{sideways}\bf{Lev-Typ}\end{sideways}  & \begin{sideways}\bf{stepType}\end{sideways} &\begin{sideways}\bf{Unit}\end{sideways}\\
\midrule
\endfirsthead
\caption[]{\emph{continued}}\\
\midrule
\endhead
\hline \multicolumn{8}{r}{\textit{Continued on next page}} \\
\endfoot
\endlastfoot
PS                             &  Surface pressure (not reduced)                                                        &               0                                   &                     3                       &                    1                       &                 1                           &                      inst                   &        $\mathrm{Pa}$   \\ 
T\_SNOW                        &  Temperature of the snow surface                                                       &               0                                   &                     0                       &                    18                      &                 1                           &                      inst                   &        $\mathrm{K}$    \\
T\_G                           &  Ground temperature (temperature at sfc-atm interface)                                 &               0                                   &                     0                       &                    0                       &                 1                           &                      inst                   &        $\mathrm{K}$    \\
QV\_S                          &  Surface specific humidity                                                             &               0                                   &                     1                       &                    0                       &                 1                           &                      inst                   &        $\mathrm{kg\,kg^{-1}}$    \\
W\_SNOW                        &  Snow depth water equivalent                                                           &               0                                   &                     1                       &                    60                      &                 1                           &                      inst                   &        $\mathrm{kg\,m^{-2}}$    \\
W\_I                           &  Plant canopy surface water                                                            &               2                                   &                     0                       &                    13                      &                 1                           &                      inst                   &        $\mathrm{kg\,m^{-2}}$    \\
TCM                            &  Turbulent transfer coefficient for momentum (surface)                                 &               0                                   &                     2                       &                    29                      &                 1                           &                      inst                   &        $\mathrm{--}$    \\ 
TCH                            &  Turbulent transfer coefficient for heat and moisture (surface)                        &               0                                   &                     0                       &                    19                      &                 1                           &                      inst                   &        $\mathrm{--}$    \\
ASOB\_S                        &  Net short-wave radiation flux at surface (average since model start)                  &               0                                   &                     4                       &                     9                      &                 1                           &                      avg                    &        $\mathrm{W\,m^{-2}}$    \\
ATHB\_S                        &  Net long-wave radiation flux at surface (average since model start)                   &               0                                   &                     5                       &                     5                      &                 1                           &                      avg                    &        $\mathrm{W\,m^{-2}}$    \\
ASOB\_T                        &  Net short-wave radiation flux at TOA (average since model start)                      &               0                                   &                     4                       &                     9                      &                 8                           &                      avg                    &        $\mathrm{W\,m^{-2}}$    \\
ATHB\_T                        &  Net long-wave radiation flux at TOA (average since model start)                       &               0                                   &                     5                       &                     5                      &                 8                           &                      avg                    &        $\mathrm{W\,m^{-2}}$    \\ 
ALB\_RAD                       &  Surface albedo for visible range, diffuse                                             &               0                                   &                    19                       &                     1                      &                 1                           &                      inst                   &        $\mathrm{\%}$    \\
RAIN\_GSP                      &  Large scale rain (accumulated since model start)                                      &               0                                   &                     1                       &                    77                      &                 1                           &                      accu                   &        $\mathrm{kg\,m^{-2}}$    \\
SNOW\_GSP                      &  Large snowfall water equivalent (accumulated since model start)                       &               0                                   &                     1                       &                    56                      &                 1                           &                      accu                   &        $\mathrm{kg\,m^{-2}}$    \\
RAIN\_CON                      &  Convective rain (accumulated since model start)                                       &               0                                   &                     1                       &                    76                      &                 1                           &                      accu                   &        $\mathrm{kg\,m^{-2}}$    \\
SNOW\_CON                      &  Convective snowfall water equivalent (accumulated since model start)                  &               0                                   &                     1                       &                    55                      &                 1                           &                      accu                   &        $\mathrm{kg\,m^{-2}}$    \\
TOT\_PREC                      &  Total precipitation (accumulated since model start)                                   &               0                                   &                     1                       &                    52                      &                 1                           &                      accu                   &        $\mathrm{kg\,m^{-2}}$  \\
\textcolor{gray}{RUNOFF\_S}    &  \textcolor{gray}{Surface water runoff (accumulated since model start)}\footnotemark[3]&               \textcolor{gray}{2}                 &                     \textcolor{gray}{0}     &                     \textcolor{gray}{5}    &                 \textcolor{gray}{1}         &                      \textcolor{gray}{accu} &        \textcolor{gray}{$\mathrm{kg\,m^{-2}}$}  \\
\textcolor{gray}{RUNOFF\_G}    &  \textcolor{gray}{Soil water runoff (accumulated since model start)}\footnotemark[3]   &               \textcolor{gray}{2}                 &                     \textcolor{gray}{0}     &                     \textcolor{gray}{5}    &                 \textcolor{gray}{1}         &                      \textcolor{gray}{accu} &        \textcolor{gray}{$\mathrm{kg\,m^{-2}}$}  \\                                      
U\_10M                         &  Zonal wind at 10m above ground                                                        &               0                                   &                     2                       &                     2                      &               103                           &                      inst                   &        $\mathrm{m\,s^{-1}}$  \\
V\_10M                         &  Meridional wind at 10m above ground                                                   &               0                                   &                     2                       &                     3                      &               103                           &                      inst                   &        $\mathrm{m\,s^{-1}}$  \\
T\_2M                          &  Temperature at 2m above ground                                                        &               0                                   &                     0                       &                     0                      &               103                           &                      inst                   &        $\mathrm{K}$          \\
TD\_2M                         &  Dew point temperature at 2m above ground                                              &               0                                   &                     0                       &                     6                      &               103                           &                      inst                   &        $\mathrm{K}$          \\
Z0                             &  Surface roughness (above land and water)                                              &               2                                   &                     0                       &                     1                      &                 1                           &                      inst                   &        $\mathrm{m}$          \\
CLCT                           &  Total cloud cover                                                                     &               0                                   &                     6                       &                     1                      &                 1                           &                      inst                   &        $\mathrm{\%}$          \\
\textcolor{gray}{CLCH}         &  \textcolor{gray}{High level clouds}\footnotemark[3]                                   &               \textcolor{gray}{0}                 &                     \textcolor{gray}{6}     &                    \textcolor{gray}{1}     &                 \textcolor{gray}{100}       &                      inst                   &        $\mathrm{\%}$          \\
\textcolor{gray}{CLCM}         &  \textcolor{gray}{Mid level clouds}\footnotemark[3]                                    &               \textcolor{gray}{0}                 &                     \textcolor{gray}{6}     &                    \textcolor{gray}{1}     &                 \textcolor{gray}{100}       &                      inst                   &        $\mathrm{\%}$          \\
\textcolor{gray}{CLCL}         &  \textcolor{gray}{Low level clouds}\footnotemark[3]                                    &               \textcolor{gray}{0}                 &                     \textcolor{gray}{6}     &                    \textcolor{gray}{1}     &                 \textcolor{gray}{100}       &                      inst                   &        $\mathrm{\%}$          \\
TQV                            &  Total column integrated water vapour                                                  &               0                                   &                     1                       &                    64                      &                 1                           &                      inst                   &        $\mathrm{kg\,m^{-2}}$  \\
TQC                            &  Total column integrated cloud water                                                   &               0                                   &                     1                       &                    69                      &                 1                           &                      inst                   &        $\mathrm{kg\,m^{-2}}$  \\
TQI                            &  Total column integrated cloud ice                                                     &               0                                   &                     1                       &                    70                      &                 1                           &                      inst                   &        $\mathrm{kg\,m^{-2}}$  \\
\textcolor{gray}{TQR}          &  \textcolor{gray}{Total column integrated rain}\footnotemark[3]                        &               \textcolor{gray}{0}                 &                     \textcolor{gray}{1}     &                    \textcolor{gray}{45}    &                 \textcolor{gray}{1}         &                      \textcolor{gray}{inst} &        \textcolor{gray}{$\mathrm{kg\,m^{-2}}$}  \\
\textcolor{gray}{TQS}          &  \textcolor{gray}{Total column integrated snow}\footnotemark[3]                        &               \textcolor{gray}{0}                 &                     \textcolor{gray}{1}     &                    \textcolor{gray}{46}    &                 \textcolor{gray}{1}         &                      \textcolor{gray}{inst} &        \textcolor{gray}{$\mathrm{kg\,m^{-2}}$}  \\
HBAS\_CON                      &  Height of convective cloud base above msl                                             &               0                                   &                     6                       &                    26                      &                 2                           &                      inst                   &        $\mathrm{m}$  \\
HTOP\_CON                      &  Height of convective cloud top above msl                                              &               0                                   &                     6                       &                    27                      &                 3                           &                      inst                   &        $\mathrm{m}$  \\
HZEROCL                        &  Height of 0 degree Celsius isotherm above msl                                         &               0                                   &                     3                       &                     6                      &                 4                           &                      inst                   &        $\mathrm{m}$  \\
ASHFL\_S                       &  Sensible heat net flux at surface (average since model start)                         &               0                                   &                     0                       &                    11                      &                 1                           &                      avg                    &        $\mathrm{W\,m^{-2}}$  \\
ALHFL\_S                       &  Latent heat net flux at surface (average since model start)                           &               0                                   &                     0                       &                    10                      &                 1                           &                      avg                    &        $\mathrm{W\,m^{-2}}$  \\
FR\_ICE                        &  Sea ice cover  (possible range: $[0,1]$)                                              &              10                                   &                     2                       &                     0                      &                 1                           &                      inst                   &        $\mathrm{\textendash}$  \\
T\_ICE                         &  Sea ice temperature (at ice-atm interface)                                            &              10                                   &                     2                       &                     8                      &                 1                           &                      inst                   &        $\mathrm{K}$  \\
H\_ICE                         &  Sea ice thickness (Max: $3\,\mathrm{m}$)                                              &              10                                   &                     2                       &                     1                      &                 1                           &                      inst                   &        $\mathrm{m}$  \\
FRESHSNW                       &  Fresh snow factor (weighting function for albedo indicating freshness of snow)        &               0                                   &                     1                       &                   203                      &                 1                           &                      inst                   &        $\mathrm{\textendash}$  \\
RHO\_SNOW                      &  Snow density                                                                          &               0                                   &                     1                       &                    61                      &                 1                           &                      inst                   &        $\mathrm{kg\,m^{-3}}$  \\
H\_SNOW                        &  Snow depth                                                                            &               0                                   &                     1                       &                    11                      &                 1                           &                      inst                   &        $\mathrm{m}$  \\
  \bottomrule
\end{longtable}

\footnotetext[3]{Output fields not yet available, but planned.}



\begin{longtable}{p{2.0cm}p{5.0cm}p{0.8cm}p{0.8cm}p{0.8cm}p{0.9cm}p{1cm}p{1cm}}
\caption[]{Multi-layer forecast ($VV>0$) and initialised analysis ($VV=0$) products of the soil model}\\
  \toprule
\multicolumn{1}{c}{\begin{sideways}\textbf{ShortName}\end{sideways}}  &  \multicolumn{1}{c}{\rb{\textbf{Description}}}  & \begin{sideways}\textbf{Discipline}\end{sideways} & \begin{sideways}\bf{Category}\end{sideways} & \begin{sideways}\bf{Number}\end{sideways}  & \begin{sideways}\bf{Lev-Typ}\end{sideways}  & \begin{sideways}\bf{stepType}\end{sideways} &\begin{sideways}\bf{Unit}\end{sideways}\\
\midrule
%\endfirsthead
%\caption[]{\emph{continued}}\\
%\midrule
\endhead
\hline \multicolumn{8}{r}{\textit{Continued on next page}} \\
\endfoot
\endlastfoot
T\_SO                          &  Soil temperature                                                                      &               2                                   &                     3                       &                    18                       &               106                           &                      inst                   &        $\mathrm{K}$   \\
W\_SO                          &  Soil moisture integrated over individual soil layers  (ice + liquid)                  &               2                                   &                     3                       &                    20                       &               106                           &                      inst                   &        $\mathrm{kg\,m^{-2}}$   \\
W\_SO\_ICE                     &  Soil ice content integrated over individual soil layers                               &               2                                   &                     3                       &                    22                       &               106                           &                      inst                   &        $\mathrm{kg\,m^{-2}}$   \\
  \bottomrule
\end{longtable}

Soil temperature is defined at the soil depths given in Table \ref{tab_soillayer} (column 2). Levels $1$ to $8$ define the full levels of the soil model. A zero gradient 
condition is assumed between levels $0$ and $1$, meaning that temperatures at the surface-atmosphere interface are set equal to the temperature at the first full level depth.
($0.5\,\mathrm{cm}$). Temperatures are prognosed for levels $1$ to $7$. At the lowermost level ($1458\,\mathrm{cm}$) the temperature is fixed to the climatological 
average $2\,\mathrm{m}$-temperature.

Soil moisture $\mathrm{W\_SO}$ is prognosed for layers $1$ to $6$. In the two lowermost layers $\mathrm{W\_SO}$ is time constant.

\begin{table}
\center
\caption{Soil model: vertical distribution of levels and layers}\label{tab_soillayer}
 \begin{tabular}{>{\centering\arraybackslash}p{2.0cm}>{\centering\arraybackslash}p{2.5cm}|>{\centering\arraybackslash}p{2.5cm}>{\centering\arraybackslash}p{5.0cm}}
 \toprule
  \bf{level no.}       &  \bf{depth [cm]}        &   \bf{layer no.}        & \bf{upper/lower bounds [cm]} \\
 \midrule
         0             &     $0.0$               &                         &                                     \\
         1             &     $0.5$               &         1               &     $0.0$\, \textemdash\, $1.0$     \\
         2             &     $2.0$               &         2               &     $1.0$\, \textemdash\, $3.0$     \\
         3             &     $6.0$               &         3               &     $3.0$\, \textemdash\, $9.0$     \\
         4             &     $18.0$              &         4               &     $9.0$\, \textemdash\, $27.0$    \\
         5             &     $54.0$              &         5               &    $27.0$\, \textemdash\, $81.0$    \\
         6             &     $162.0$             &         6               &    $81.0$\, \textemdash\, $243.0$   \\
         7             &     $486.0$             &         7               &   $243.0$\, \textemdash\, $729.0$   \\
         8             &     $1458.0$            &         8               &   $729.0$\, \textemdash\, $2187.0$  \\
 \bottomrule
 \end{tabular}
\end{table}


% ----------------------------------------------------------------%

\appendix
% \chapter{ICON standard half level heights}
\label{appendix_levelheights}

ICON standard half level heights $z^{h}$ are listed in Table \ref{tab:half_level_heights}. If full level heights $z^{f}$ are required, 
these can be deduced as follows: Let $i$ denote the full level index for which the height is wanted. Then the full level 
height $z^{f}_{i}$ is given by
\begin{align}
 z^{f}_{i} = 0.5 \left(z^{h}_{i} + z^{h}_{i+1}\right)
\end{align}


\begin{table}[hb]
  \caption{Standard heights (i.e.\ for zero topography height) for all $91$ vertical half levels.}
  \label{tab:half_level_heights}%

   \renewcommand{\baselinestretch}{1.00}\normalsize%
   \pgfkeys{/pgf/number format/set thousands separator={\,}}
   \pgfplotstableread{vertical_levels_i.txt}{\loadedtable}\vspace*{0pt}%
%   \pgfplotstabletypeset[ columns={k,z,p},every  head row/.style={after row={\hline}},
   \pgfplotstabletypeset[ 
          begin table=\begin{longtable}, 
          end table=\end{longtable},
          columns={k,z,k,z,k,z},
          every  head row/.style={after row={\hline}},
          precision=2,
          font=\normalsize,
%          columns/k/.style={column name=level index, column type=c,column type/.add={>{\columncolor[gray]{.8}}}{}},
%          columns/z/.style={column name=height $[m]$,   fixed,dec sep align},
%          columns/p/.style={column name=$[Pa]$, fixed,dec sep align, zerofill,precision=1},
          columns/k/.style={column name=level index, column type=c, column type/.add={>{\columncolor[gray]{.8}}}{}},
          columns/z/.style={column name=height $[m]$, fixed,dec, zerofill,precision=1},
          display columns/0/.style={select equal part entry of={0}{3},string type},
          display columns/1/.style={select equal part entry of={0}{3},string type},
          display columns/2/.style={select equal part entry of={1}{3},string type},
          display columns/3/.style={select equal part entry of={1}{3},string type},
          display columns/4/.style={select equal part entry of={2}{3},string type},
          display columns/5/.style={select equal part entry of={2}{3},string type},
                        ] {\loadedtable}

\end{table}

  %--- Einbinden der Appendix

% \listoffigures                    %--- Erstellung des Figurenverzeichnisses

% \glsaddall
% \printglossary[type=symbolslist]  %--- Erstellung des Symbolverzeichnisses

% \backmatter
% \bibliography{/home/dreinert/Documents/bibliography/litera.bib,/home/dreinert/Documents/bibliography/litera_notprinted.bib} %--- Erstellen der Liste aller Referenzen
\bibliography{litera.bib} %--- Create list of references

% remove chapter number from header for Acknowledgements
\renewcommand{\chaptermark}[1] {
  \markboth{#1}{}
}
% --- Danksagung ---
% \include{./text/danksagung}

% use default again
\renewcommand{\chaptermark}[1]{%
  \markboth{\chaptername
    \ \thechapter.\ #1}{}}


\end{document}
