\chapter{Mandatory input fields}

Several input files are needed to perform runs of the ICON Model. 
%
These can be divided into three classes:
%
Grid files, external parameters, and initialization (analysis). The latter 
will be described in Chapter \ref{sec_analysis}.


%%%%%%%%%%%%%%%%%%%%%%%%%%%%%%%%%%%%%%%%%%%%%%%%%%%%%%%%%%%%%%%%%%%%%%%%%%%%%%%%%%%
\section{Grid Files}
\label{section:grid_files}

In order to run ICON, it is necessary to load the horizontal grid
information as an input para\-meter. 
This information is stored within so-called grid files. For an ICON 
run, at least one global grid file is required.
For model runs with nested grids, additional files of the nested
domains are necessary. Optionally, a reduced radiation grid for
the global domain may be used.

%
The unstructured triangular ICON grid resulting from the grid
generation process is represented in NetCDF format.  The most
important data entries are

\begin{itemize}
 \item \texttt{cell} (INTEGER dimension) \\
        number of (triangular) cells
 \item \texttt{vertex} (INTEGER dimension) \\
        number of triangle vertices
 \item \texttt{edge} (INTEGER dimension) \\
        number of triangle edges
 \item \texttt{clon}, \texttt{clat} (double array, dimension: \#triangles, given in radians) \\
        longitude/latitude of the triangle circumcenters
 \item \texttt{vlon}, \texttt{vlat} (double array, dimension: \#triangle vertices, given in radians) \\
        longitude/latitude of the triangle vertices
 \item \texttt{elon}, \texttt{elat} (double array, dimension: \#triangle edges, given in radians) \\
       longitude/latitude of the edge midpoints
 \item \texttt{cell\_area} (double array, dimension: \#triangles) \\
       triangle areas
 \item \texttt{vertex\_of\_cell} (INTEGER array, dimensions: [3, \#triangles]) \\
       The indices \texttt{vertex\_of\_cell(:,i)} denote the triangle vertices that belong 
       to the triangle~\texttt{i}.
 \item \texttt{edge\_of\_cell} (INTEGER array, dimensions: [2, \#triangles]) \\
       The indices \texttt{edge\_of\_cell(:,i)} denote the triangle edges that belong
       to the triangle~\texttt{i}.
\end{itemize}


%%%%%%%%%%%%%%%%%%%%%%%%%%%%%%%%%%%%%%%%%%%%%%%%%%%%%%%%%%%%%%%%%%%%%%%%%%%%%%%%%%%

\section{External parameter}
\label{section:extpar}
External parameters are used to describe the properties of the earth's surface. 
These data include the orography and the land-sea-mask. Also, several parameters
are needed to specify the dominant land use of a grid box like the soiltype
or the plant cover fraction.

The ExtPar software (ExtPar -- External parameter for Numerical Weather Prediction and Climate Application) 
is able to generate external parameters for the ICON model. The generation is based on a set of 
raw-datafields which are listed in Table \ref{table_extpar_raw}. For a more detailed overview of ExtPar, 
the reader is referred to the \emph{User and Implementation Guide} of Extpar.

\begin{longtable}{p{6.5cm}p{6cm}p{1.8cm}}
\caption[]{Raw datasets from which the ICON external parameter fields are derived.}\label{table_extpar_raw}\\
  \toprule
\textbf{Dataset} &\textbf{Source} &\textbf{Resolution} \\
\midrule
\endfirsthead
\caption[]{\emph{continued}}\\
\midrule
\endhead
\hline \multicolumn{3}{r}{\textit{Continued on next page}} \\
\endfoot
\endlastfoot
GLOBE orography                                        &  NOAA/NGDC                  &  30'' \\
%ASTER orography \newline (limited domain: 60 N - 60 S) &  METI/NASA                  &  1''   \\
GlobCover 2009                                         &  ESA                        &  10''  \\
%GLC2000 land use                                       &  JRC Ispra                  &  30''  \\
GLCC land use                                          &  USGS                       &  30''  \\
%DSMW Digital Soil Map of the World                     &  FAO                        &  5'    \\
HWSD Harmonized World Soil Database                    &  FAO/IIASA/ISRIC/ISSCAS/JRC &  30''  \\
NDVI Climatotology, SeaWiFS                            &  NASA/GSFC                  &  2.5'  \\
CRU near surface climatology                           &  CRU University of East Anglia & $0.5^{\circ}$  \\
GACP Aerosol Optical thickness                         &  NASA/GISS \newline (Global Aerosol Climatology Project)   &  $4x5^{\circ}$ \\
GLDB Global lake database                              &  DWD/RSHU/MeteoFrance       &  30''  \\
MODIS albedo                                           &  NASA                       &  5'    \\
\bottomrule
\end{longtable}

\emph{GlobCover 2009} is a land cover database covering the whole globe, except for Antarctica. Therefore, we make use of 
\emph{GlobCover 2009} for $90^{\circ} > \phi > -56^{\circ}$ (with $\phi$ denoting latitude) and switch to the coarser, 
however globally available dataset \emph{GLCC} for $ -56^{\circ} > \psi > -90^{\circ}$.

The products generated by the ExtPar software package are listed in Table \ref{table_extpar_products} together with the underlying 
raw dataset. Note that these are mandatory input fields for assimilation- and forecast runs.

\begin{longtable}{p{2.5cm}p{8.5cm}p{3.3cm}}
\caption[]{External parameter fields for ICON, produced by the ExtPar software package (in alphabetical order)}\label{table_extpar_products}\\
% \begin{tabular}{p{2.5cm}p{8.5cm}p{3.3cm}}
  \toprule
\multicolumn{1}{c}{\textbf{ShortName}}  &  \multicolumn{1}{c}{\textbf{Description}}  &  \multicolumn{1}{c}{\textbf{Raw dataset}}\\
\midrule
\endfirsthead
\caption[]{\emph{continued}}\\
\midrule
\endhead
\hline \multicolumn{3}{r}{\textit{Continued on next page}} \\
\endfoot
\endlastfoot
  AER\_SS12                             & Sea salt aerosol climatology (monthly fields)   &       GACP                \\
  AER\_DUST12                           & Total soil dust aerosol climatology (monthly fields) &  GACP                \\
  AER\_ORG12                            & Organic aerosol climatology (monthly fields)       &    GACP                \\
  AER\_SO412                            & Total sulfate aerosol climatology (monthly fields) &    GACP                \\
  AER\_BC12                             & Black carbon aerosol climatology (monthly fields)  &    GACP                \\
  ALB\_DIF12                            & Shortwave ($0.3 - 5.0\, \mathrm{\mu m}$) albedo for diffuse radiation (monthly fields)&  MODIS    \\
  ALB\_UV12                             & UV-visible ($0.3 - 0.7\, \mathrm{\mu m}$) albedo for diffuse radiation (monthly fields)& MODIS     \\
  ALB\_NI12                             & Near infrared ($0.7 - 5.0\, \mathrm{\mu m}$) albedo for diffuse radiation (monthly fields)& MODIS     \\
  DEPTH\_LK                             & Lake depth                                      &        GLDB               \\
  EMIS\_RAD                             & Surface longwave (thermal) emissivity           &        GlobCover 2009     \\               
  FOR\_D                                & Fraction of deciduous forest                    &        GlobCover 2009     \\
  FOR\_E                                & Fraction of evergreen forest                    &        GlobCover 2009     \\
  FR\_LAKE                              & Lake fraction (fresh water)                     &        GLDB               \\                     
  FR\_LAND                              & Land fraction (excluding lake fraction but including glacier fraction) & GlobCover2009   \\
  FR\_LUC                               & Landuse class fraction                          &                           \\
  HSURF                                 & Orography height at cell centres                &        GLOBE              \\
  LAI\_MX                               & Leaf area index in the vegetation phase         &        GlobCover 2009     \\
  NDVI\_MAX                             & Normalized differential vegetation index        &        SeaWiFS            \\
  NDVI\_MRAT                            & proportion of monthly mean NDVI to yearly maximum (monthly fields)&  SeaWiFS \\
  PLCOV\_MX                             & Plant covering degree in the vegetation phase   &        GlobCover 2009     \\
  ROOTDP                                & Root depth                                      &        GlobCover 2009     \\
  RSMIN                                 & Minimum stomatal resistance                     &        GlobCover 2009     \\
  SOILTYP                               & Soil type                                       &        HWSD               \\
  SSO\_STDH                             & Standard deviation of sub-grid scale orographic height  &   GLOBE           \\
  SSO\_THETA                            & Principal axis-angle of sub-grid scale orography &          GLOBE           \\
  SSO\_GAMMA                            & Horizontal anisotropy of sub-grid scale orography &         GLOBE           \\
  SSO\_SIGMA                            & Average slope of sub-grid scale orography       &           GLOBE           \\
  T\_2M\_CL                             & Climatological 2m temperature (serves as lower boundary condition for soil model)  &  CRU \\
  Z0 (*)                                & Surface roughness length (over land), containing a contribution from subgrid-scale orography  & GlobCover 2009    \\                        
  \bottomrule
% \end{tabular}
\end{longtable}

Note that fields marked with (*) are not required in operational model runs. I.e.\ the surface roughness \texttt{Z0} is only needed, 
if the additional contribution from sub-grid scale orography is taken into account (i.e.\ for \nml{itype\_z0=1}). In operational runs, land-use 
specific roughness lengths are taken from a GlobCover-based lookup table. \texttt{FOR\_D} and \texttt{FOR\_E} will become obsolete, as soon as the 
surface tile approach (which is currently under development) is activated. However, due to technical reasons, all the above fields must be provided 
as input, irrespective of the options chosen.



\subsubsection*{Remarks on post-processing}
Some of the external parameter fields produced by ExtPar are modified by ICON. The following fields are affected: \texttt{HSURF}, 
\texttt{FR\_LAND}, \texttt{FR\_LAKE}, \texttt{Z0}. Thus, for consistency reasons, the modified fields should be used for post-processing tasks 
rather than the original external parameter fields.
