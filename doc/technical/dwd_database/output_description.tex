\svnInfo $Id$


In order to facilitate the selection and interpretation of fields and to guard against possible mis-interpretation or mis-usage, the following section provides a more thorough 
description of the available output fields.

%\subsection{External parameter}

% uses package enumitem
%\begin{description}[font=\sffamily\bfseries, leftmargin=3.0cm,style=sameline]
%\begin{description}[leftmargin=3.0cm,style=sameline]
% \item [AER\_SS] description of Hello
% \item [AER\_DUST12] another description
%\end{description}


\section{Cloud products}
% uses package enumitem
%\begin{description}[font=\sffamily\bfseries, leftmargin=3.0cm,style=sameline]
\begin{description}[leftmargin=3.0cm,style=sameline]
  \item [CLCT\_MOD] Modified total cloud cover ($0 \leq$ \texttt{CLCT\_MOD} $\leq 1$). Used for visualization purpose 
                    (i.e.\ gray-scale figures) in the media. It is derived from \texttt{CLC}, neglecting cirrus clouds if 
                    there are only high clouds present at a given grid point. The reason for this treatment is that 
                    the general public does not regard transparent cirrus clouds as `real' clouds.
  \item [CLDEPTH]   Modified cloud depth ($0 \leq$ \texttt{CLDEPTH} $\leq 1$). Used for visualization purpose (i.e.\ gray-scale figures) 
                    in the media. A cloud reaching a vertical extent of $700\,\mathrm{hPa}$ or more, has \texttt{CLDEPTH}= $1$.
  \item [HBAS\_CON] Height of the convective cloud base in m above msl. \texttt{HBAS\_CON} is initialized with $-500\,\mathrm{m}$ 
                    at points where no convection is diagnosed.
  \item [HTOP\_CON] Same, but for cloud top.
\end{description}


\section{Atmospheric products}
\begin{description}[leftmargin=3.0cm,style=sameline]
 \item [HZEROCL] Height of the $0^{\circ}\,\mathrm{C}$ isotherm above MSL. At points where the temperature is below 
                 $0^{\circ}\,\mathrm{C}$ within the entire atmospheric column, \texttt{HZEROCL} is undefined 
                 and set to $-999$.
 \item [SNOWLMT] Height of snow fall limit above MSL. It is defined as the height where the wet bulb 
                 temperature $T_{w}$ first exceeds $1.3^{\circ}\mathrm{C}$ (scanning mode from top to bottom). 
                 If this threshold is never reached within the entire atmospheric column, \texttt{SNOWLMT} 
                 is undefined (GRIB2 bitmap).
\end{description}


\section{Near surface products}
% uses package enumitem
%\begin{description}[font=\sffamily\bfseries, leftmargin=3.0cm,style=sameline]
\begin{description}[leftmargin=3.0cm,style=sameline]
  \item [TMIN\_2M]  Minimum temperature at $2\,\mathrm{m}$ above ground. Minima are collected over $3$-hourly intervals 
                    on the global grid and over $1$-hourly intervals for the EU-nest. For forecast times larger than  
                    $vv=78\,\mathrm{h}$ the interval is prolonged to $3$ hours for the EU-nest as well. 
  \item [TMAX\_2M]  Same, but for maximum $2\,\mathrm{m}$ temperature.
  \item [VMAX\_10M] Maximum wind gust at $10\,\mathrm{m}$ above ground. It is diagnosed from the turbulence state 
                    in the atmospheric boundary layer, including a potential enhancement by the SSO parameterization 
                    over mountainous terrain. In the presence of deep convection, it contains an additional contribution 
                    due to convective gusts. 
                    
                    Maxima are collected over $3$-hourly intervals on the global grid and over hourly intervals 
                    for the EU-nest. For forecast times larger than $vv=78\,\mathrm{h}$ the interval is prolonged 
                    to $3$ hours for the EU-nest as well.
\end{description}



\subsection{General comment on statistically processed fields}

In GRIB2, the overall time interval over which a statistical process (like averaging, computation of maximum/minimum) has taken place is 
encoded as follows:

The beginning of the overall time interval is defined by \texttt{referenceTime + forecastTime}, whereas the end of the 
overall time interval is given by \texttt{referenceTime + forecastTime + lengthOfTimeRange}. See Section 
\ref{sec_statproc_fields} for more details on statistically processed fields.

%\begin{note}%
%\textbf{Note:} Fields for which the beginning of the time interval differs from \texttt{referenceTime} are currently 
%encoded incorrectly. The begining of the time interval is erroneously set to \texttt{referenceTime}. I.e.\ this is currently the case 
%for \texttt{TMAX\_2M}, \texttt{TMIN\_2M}, \texttt{VMAX\_10M}.
%\end{note}

%\subsection{Atmospheric products}


\section{Surface products}
% uses package enumitem
%\begin{description}[font=\sffamily\bfseries, leftmargin=3.0cm,style=sameline]
\begin{description}[leftmargin=3.0cm,style=sameline]
% \item [APAB\_S] Photosynthetically active radiation flux at the surface, i.e.\ that part of the net short-wave radiation flux 
%                 that photosynthetic organisms are able to use in the process of photosynthesis. This is approximately the case for 
%                 solar radiation in the spectrakl range from $400$ to $700\,\mathrm{nm}$. Average over forecast time.
 \item [ASWDIFD\_S] Downward solar diffuse radiation flux at the surface, averaged over forecast time. See Section 
                    \ref{sec_statproc_fields} for more information on time averaging. 
 
 \item [ASWDIR\_S] Downward solar direct radiation flux at the surface. See Section \ref{sec_statproc_fields} for more information 
                   on time averaging.
                   
 \item [ALB\_RAD] Ratio of upwelling to downwelling diffuse radiative flux for wavelength interval [$0.3\,\mathrm{\mu m},\,5.0\,\mathrm{\mu m}$]. 
                  Values over snow-free land points are based on a monthly mean MODIS climatology. MODIS values have been limited to a minimum value 
                  of $2\,\%$.  
\end{description}

From \texttt{ASWDIFD\_S} and \texttt{ASWDIR\_S} the time averaged global radiation at the surface \texttt{GLOB} can easily be computed as follows:
\begin{align*}
 \texttt{GLOB} = \texttt{ASWDIFD\_S} + \texttt{ASWDIR\_S}
\end{align*}
An estimate of \texttt{GLOB} can also be derived from the net solar radiation flux at the surface \texttt{ASOB\_S} 
and the albedo \texttt{ALB\_RAD}:
\begin{align*}
 \texttt{GLOB} = \frac{\texttt{ASOB\_S}}{1-0.01\,\texttt{ALB\_RAD}} 
\end{align*}
However be aware that this is only approximately true, because \texttt{ALB\_RAD} is an instantaneous field, and it  
only constitutes the albedo for the diffuse component of the incoming solar radiation (``white sky'' albedo). However, 
\texttt{ASOB\_S} contains both diffuse and direct components. As a consequence, the reflection of the incoming direct 
radiation, which is dependent on the solar zenith angle (and described by the so called ``black sky'' albedo), 
is not correctly taken into account. 



\begin{description}[leftmargin=3.0cm,style=sameline]
 \item [FR\_ICE] Sea and lake ice cover. Currently, the only possible values are $0$ (no ice cover) and $1$ (ice covered grid point). For 
                 lake points, \texttt{FR\_ICE} is synchronized with \texttt{H\_ICE} meaning that \texttt{FR\_ICE} is set to $1$ ($0$), 
                 where the lake model indicates $\texttt{H\_ICE}>0$ ($\texttt{H\_ICE}=0$).

 \item [H\_ICE] Ice thickness over sea and frozen fresh water lakes. The maximum allowable ice thickness is limited to $3\,\mathrm{m}$. 
                New sea-ice points generated by the analysis are initialized with $\texttt{H\_ICE}=0.5\,\mathrm{m}$.

 \item [T\_ICE] Ice temperature over sea-ice and frozen lake points. Melting ice has a temperature of $273.15\,\mathrm{K}$. Ice-free 
                points over land, sea, and lakes are set to \texttt{T\_SO(0)}.

 \item [T\_G]   Temperature at the atmosphere-surface interface. It is the temperature that is crucial for the computation of surface fluxes. 
                \texttt{T\_G} is equal to \texttt{T\_SO(0)} over open water and snow-free land. At other grid points one has
                \begin{itemize} 
                  \item $\texttt{T\_G}=\texttt{T\_SNOW} + (1 - \mathrm{f\_snow})*(\texttt{T\_SO(0)} - \texttt{T\_SNOW})$ over (partially) 
                        snow covered grid points. $\mathrm{f\_snow}$ is the grid point fraction that is snow covered.
                  \item $\texttt{T\_G}=\texttt{T\_ICE}$ over frozen sea and fresh water lakes
                \end{itemize}

 \item [TOT\_PREC] Total precipitation accumulated since model start.\\
                \texttt{TOT\_PREC = RAIN\_GSP + SNOW\_GSP + RAIN\_CON + SNOW\_CON}
                
 \item [W\_I]   Water content of interception layer, i.e.\ the amount of precipitation intercepted by vegetation canopies. The maximum 
                capacity of the interception reservoir is currently limited to $6.0E-3\,\mathrm{kg\,m^{-2}}$ due to numerical reasons 
                and thus almost negligible. Over water points, \texttt{W\_I} is set to 0.

 \item [Z0]     Surface roughness length. Constant over land, where it depends only on the type of land cover. I.e.\ it does 
                not contain any contribution from subgrid-scale orography. Over water, the roughness length usually varies 
                with time. It is computed by the so called Charnock-formula, which parameterizes the impact of waves on the 
                roughness length. Note that this field differs significantly from the extermal parameter field \texttt{Z0} 
                (see Table \ref{table_extpar_products} or \ref{table_constdb}).  
\end{description}


\section{Soil products}
% uses package enumitem
%\begin{description}[font=\sffamily\bfseries, leftmargin=3.0cm,style=sameline]
\begin{description}[leftmargin=3.0cm,style=sameline]
 \item [RUNOFF\_G] Water runoff from soil layers. Sum over forecast.

 \item [RUNOFF\_S] Surface water runoff from interception and snow reservoir and from limited infiltration rate. Sum over forecast.

 \item [T\_SO] Temperature of the soil and earth surface (uppermost level). The soil full level depths at which the 
               the soil temperature is defined are given in Table \ref{tab_soillayer}. The temperature at the uppermost 
               level \texttt{T\_SO(0)} is not prognostic. It is rather set equal to the temperature at the first prognostic 
               level \texttt{T\_SO(1)}. The temperature at the lowermost level \texttt{T\_SO(8)} is set to the climatological 
               $2\,\mathrm{m}$ temperature \texttt{T\_2M\_CL}. At sea-points, \texttt{T\_SO(0:7)} is filled with the sea-surface 
               temperature. Note that \texttt{T\_SO(0)} does not necessarily represent the temperature at the interface soil-atmosphere. 
               I.e.\ over snow/ice covered surfaces, \texttt{T\_SO(0)} represents the temperature below snow/ice.
\end{description}


\section{Vertical Integrals}
% uses package enumitem
%\begin{description}[font=\sffamily\bfseries, leftmargin=3.0cm,style=sameline]
\begin{description}[leftmargin=3.0cm,style=sameline]
 \item [TQX]      Column integrated water species \texttt{X}, derived from the 3D grid-scale prognostic quantities \texttt{QX}, 
                  with $\texttt{X}\in \{\texttt{V}, \texttt{C}, \texttt{I}, \texttt{R}, \texttt{S}\}$. \texttt{TQX} is based 
                  on the assumption that there would be no sub-grid-scale variability. That assumption is particularly problematic 
                  for precipitation generation, moist turbulence and radiation.

 \item [TQX\_DIA] Total column integrated water species \texttt{X}, with $\texttt{X}\in \{\texttt{C}, \texttt{I}\}$. 
                  Takes into account the sub-grid-scale variability that includes simple treatments of turbulent motion and convective 
                  detrainment. These cloud variables attempt to represent all model included physical processes. They are also 
                  consistent with the cloud cover variables \texttt{CLC}, \texttt{CLCT}, \texttt{CLCH}, \texttt{CLCM} and \texttt{CLCL}. 
\end{description}

