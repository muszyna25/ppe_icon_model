% ------------------------------------------------------------------------------------------
\chapter{ICON standard level heights}
\label{appendix_levelheights}
% ------------------------------------------------------------------------------------------


% ------------------------------------------------------------------------------------------
\section{Level heights for zero topography height}
% ------------------------------------------------------------------------------------------

ICON standard \emph{half level} heights $z^{h0}$ are listed in
Table~\ref{tab:half_level_heights}.
%
Please note that these values correspond to the actual level heights only
at grid points with zero topography height, e.\,g.\ at ocean grid points.

If \emph{full level} heights $z^{f0}$ are required, these can be deduced as
follows:
%
Let $i$ denote the full level index for which the height is wanted. Then the full level 
height $z^{f0}_{i}$ is given by
\begin{align*}
 z^{f0}_{i} = \frac{ z^{h0}_{i} + z^{h0}_{i+1} }{2}.
\end{align*}
See Table~\ref{tab:full_level_heights} for a list of all full level heights of the operational
setup.


\begin{table}[p]
  \caption{Standard heights (i.e.\ for zero topography height) for all $91$ vertical 
           \underline{\emph{half levels}}.}
  \label{tab:half_level_heights}%
  \addtocounter{table}{-1} % we need this because we have a nested table-longtable environment

   \renewcommand{\baselinestretch}{1.00}\normalsize%
   \pgfkeys{/pgf/number format/set thousands separator={\,}}
   \pgfplotstableread{vertical_levels_i.txt}{\loadedtable}\vspace*{0pt}%
   \pgfplotstabletypeset[ 
          begin table=\begin{longtable}, 
          end table=\end{longtable},
          columns={k,z,k,z,k,z},
          every  head row/.style={after row={\hline}},
          precision=2,
          font=\normalsize,
          columns/k/.style={column name=level index, column type=c, 
                            column type/.add={>{\columncolor[gray]{.8}}}{}},
          columns/z/.style={column name=height $[m]$, fixed,dec sep align, zerofill,precision=3},
          display columns/0/.style={select equal part entry of={0}{3},string type},
          display columns/1/.style={select equal part entry of={0}{3}},
          display columns/2/.style={select equal part entry of={1}{3},string type},
          display columns/3/.style={select equal part entry of={1}{3}},
          display columns/4/.style={select equal part entry of={2}{3},string type},
          display columns/5/.style={select equal part entry of={2}{3}},
                        ] {\loadedtable}

\end{table}

\begin{table}[p]
  \caption{Standard heights (i.e.\ for zero topography height) for all $90$ vertical 
           \underline{\emph{full levels}}.}
  \label{tab:full_level_heights}%
  \addtocounter{table}{-1} % we need this because we have a nested table-longtable environment

   \definecolor{maroon}{cmyk}{0,0.87,0.68,0.32}
   \renewcommand{\baselinestretch}{1.00}\normalsize%
   \pgfkeys{/pgf/number format/set thousands separator={\,}}
   \pgfplotstableread{vertical_full_levels_i.txt}{\loadedtable}\vspace*{0pt}%
   \pgfplotstabletypeset[ 
          begin table=\begin{longtable}, 
          end table=\end{longtable},
          columns={k,z,k,z,k,z},
          every  head row/.style={after row={\hline}},
          precision=2,
          font=\normalsize,
          columns/k/.style={column name=level index, column type=c, 
                            column type/.add={>{\columncolor{maroon!15}}}{}},
          columns/z/.style={column name=height $[m]$, fixed,dec sep align, zerofill,precision=3},
          display columns/0/.style={select equal part entry of={0}{3},string type},
          display columns/1/.style={select equal part entry of={0}{3}},
          display columns/2/.style={select equal part entry of={1}{3},string type},
          display columns/3/.style={select equal part entry of={1}{3}},
          display columns/4/.style={select equal part entry of={2}{3},string type},
          display columns/5/.style={select equal part entry of={2}{3}},
                        ] {\loadedtable}

\end{table}


% ------------------------------------------------------------------------------------------
\section{Non-zero topography heights}
% ------------------------------------------------------------------------------------------

The prerequisite ''zero topography height'' is seldom met in real
applications.
Instead the user has to compute the model level height for each grid
point separately.
To this end the invariant fields \texttt{HSURF} and \texttt{HHL} are
provided where \texttt{HHL} is the geometric height of model half
levels above sea level.
The level height can therefore be computed on full levels by the
following formula:
\begin{align*}
  z^{f}_{i}(x) = \frac{ \texttt{HHL}^{h}_{i}(x) + \texttt{HHL}^{h}_{i+1}(x) }{2} - \texttt{HSURF}(x) 
\end{align*}
