\chapter{Diagnostics and Debugging}


\section{Testcases and Diagnostics}

For tescase details the reader is referred to ... Here only some
special setups are described.


\subsection{Jablonowski Williamson Test}
This test can be run for dry dynamics only - as it is intended for-
but full physics can be also tested. For the latter two additional
namelist parameters are introduced in the  \texttt{testcase\_ctl} to control the initial moisture in
the atmosphere:
\begin{itemize}
\item here \texttt{rh\_at\_1000hpa} to be set between $0$ and $1$. The
default is set to $0.7$ which gives a quite smooth start. If you
really want to see early onsets of convection and microphysics just
tuned this parameter.
\item \texttt{qv\_max} is usually set to $2.e-3 kg/kg$ and refers to
the maximum value in the tropics
\end{itemize}

\subsection{Mountain Rossby wave}
In order to test the model dynamics in dry stage but with real or any
complex topography one can choose the mountain rossby wave test and
selet different types of topography. By setting this you might want to
have the turbulence scheme switched on while the rest of physics is
switched OFF. Simulating dry physics means to set the tracer fields to
zero. The transport is nto necessary but should be switched off via
the transport namelist, so the resulting namelist setting for this case is:
\begin{itemize}
\item \texttt{testcase\_ctl}
\begin{itemize}
\item  \texttt{nh\_test\_name  = 'mrw\_nh'}
\end{itemize}

\item 
\begin{itemize}
\item  \texttt{nh\_test\_name  = 'mrw\_nh'}
\end{itemize}

\end{itemize}

\section{Debugging}

\subsection{Message Levels}

\subsection{Extra output}


\begin{enumerate} 
\item In the namelist {\bf \texttt{run\_ctl}} set the number of fields with \texttt{inextra\_2d} or
  \texttt{inextra\_3d}. The logical variable for output
  \texttt{lwrite\_extra} then will be set automatically. Note, the
  number of extra fields is limited by $9$ each for 2D and 3D.
\item  \texttt{USE} these variables in the module needed.
\item Implement the storage of wished fields by using the
  nonhydrostatic diagnostic type with
  \texttt{p\_diag\%extra\_2d/3d}. 
\end{enumerate} 

Example for the use of  \texttt{p\_diag\%extra\_2d}:  

\begin{fortran}  
  USE mo_global_variables, ONLY: inextra_2d
...
  DO jc = i_startidx, i_endidx
       p_diag%extra_2d(jc,jb,1)= yxz(jc,jb)
  ENDDO
\end{fortran}  




\subsection{Diagnostics using tendencies}