\documentclass[11pt,notitlepage]{article}  %a4paper,landscape,
\usepackage[colorlinks,bookmarksopen,bookmarksnumbered,citecolor=red,urlcolor=red]{hyperref}  %dvips
\usepackage[a4paper,landscape]{geometry}

\title{Radiation Variables in GRIB2 and ICON}
\author{Martin K\"ohler}
\begin{document}  
\maketitle

%-------------------------------------------------------------------------

Radiative fluxes are stored for solar (diffuse, direct, total) and thermal bands (also called short-wave and long-wave, respectively).  
They are available as upward, downward and net and at the levels top of atmosphere (TOA) and the surface.  
Accumulation (preceding ``ACC''), average (preceding ``A'') or instantaneous (ending "\_RAD", remove other ``\_'') values can be archived.  
These statistics are valid from the beginning of the forecast to the output time.  
The short-names in the DWD GRIB2 description convention are then produced by the following components.


\vspace{1cm}
\hspace{-1cm}
\begin{tabular}{l l l l l l}  
\textbf{Statistic}      & \textbf{Band}         & \textbf{Direction} & \textbf{Text}            & \textbf{Level} & \textbf{Statistic}    \\
\hline   
A (average)             & TH (thermal/lw)       & U (up)             &  \_ (average)            &  T             & nothing (average)     \\
ACC (accumulated)       & SO (solar/sw)         & D (down)           &  \_ (accumulated)        &  S             & nothing (accumulated) \\
nothing (instantaneous) & SODIF (solar diffuse) & B (net or budget)  &  nothing (instantaneous) &                & \_RAD (instantaneous) \\
                        & SODIR (solar direct)  &                    &                          &                                        \\
\hline
\end{tabular}
\vspace{1cm}


An example is ASOB\_T, that is the net solar flux at TOA.  Not all fluxes exist.  For example there is no downward thermal flux at TOA.  The following table 
lists the existing fluxes with the associated DWD shortnames and the GRIB2 descriptors. 


\vspace{1cm}
\hspace{-2cm}
\begin{centering}
\begin{tabular}{l p{2.5cm} p{2.5cm} p{2.5cm} l l l l}  

\textbf{Description}           & \textbf{DWD \linebreak ShortName} 
                               & \textbf{ICON \linebreak ShortName} 
                               & \textbf{ECMWF \linebreak ShortName} 
                               & \textbf{Discipline} & \textbf{Category} & \textbf{Number} & \textbf{levType} \\
\hline                                                                                                                                                                          

Top net solar radiation        &  ASOB\_T     &  asob\_t     &  tsr        &  0        &  4      &  9       &  8      \\
Top up solar radiation         &  ASOU\_T     &              &  --- (red.) &  0        &  4      &  8       &  8      \\
Top down solar radiation       &  ASOD\_T     &              &  tisr       &  0        &  4      &  7       &  8      \\
Surface net solar radiation    &  ASOB\_S     &  asob\_s     &  ssr        &  0        &  4      &  9       &  1      \\
Surface up solar radiation     &  ASOU\_S     &              &  --- (red.) &  0        &  4      &  8       &  1      \\
Surface down solar radiation   &  ASOD\_S     &              &  ssrd       &  0        &  4      &  7       &  1      \\
\hline 
Surface down solar diff. rad.  &  ASODIFD\_S  &              &  --- (red.) &  0        &  4      &  199     &  1      \\
Surface up solar diff. rad.    &  ASODIFU\_S  &              &  --- (red.) &  0        &  4      &  8       &  1      \\
Surface down solar direct rad. &  ASODIRD\_S  &              &  dsrp       &  0        &  4      &  198     &  1      \\
\hline 
Top net thermal radiation      &  ATHB\_T     &  athb\_t     &  ttr        &  0        &  5      &  5       &  8      \\
Surface net thermal radiation  &  ATHB\_S     &  athb\_s     &  str        &  0        &  5      &  5       &  1      \\
Surface up thermal radiation   &  ATHU\_S     &              &  --- (red.) &  0        &  5      &  4       &  1      \\
Surface down thermal radiation &  ATHD\_S     &              &  strd       &  0        &  5      &  3       &  1      \\

\hline
\end{tabular}
\end{centering}
\vspace{1cm}

The above names are for the example ``average''.  See the first table how the other statistics ``accumulated'' and ``instantaneous'' are constructed.
Note that ATHD\_T = 0 and therefore ATHU\_T = ATHB\_T.  Also, ASODIRU\_S = 0, therefore ASODIRD\_S = ASODIRB\_S and ASODIFU\_S = ASOU\_S (both currently
have the same GRIB2 triple).

One set of variables that is sufficient to derive all others is the set defined by the ECMWF variables.  The others are marked 
as redundant (``red.'').


\end{document}
