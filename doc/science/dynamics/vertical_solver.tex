\documentclass[a4paper,10pt]{article}
\usepackage{a4wide}
\usepackage{amsmath}

%opening
\title{Vertical solver for the dynamical core}
\author{Almut Gassmann}

\begin{document}
\section{Vertical solver}
In this section, we assume the horizontal discretisation terms as well as the terrain-following terms to be known and to be discretized explicitly. Hence, we restrict ourselves to the vertical part of the equations. According to the findings of Gassmann and Herzog (2008), the symplectic time integration requires the following implicit weights to be used
\begin{itemize}
 \item mass flux term
	$$(\rho^{n+1}w^n+\rho^nw^{n+1})/2$$
 \item pressure gradient term
	$$-c_{pd}\theta_v \nabla_z \left(\frac{c_{vd}}{c_{pd}}\Pi^{n+1}+
	\frac{R_d}{c_{pd}}\Pi^n\right)$$
 \item kinetic energy gradient term
	$$-\nabla_z\frac{w^{n+1}w^n}{2}$$
\end{itemize}
Collecting now all explicitly known terms (index $exp$) on the RHS of the prognostic equations, the problem to solve in the vertical direction becomes
\begin{eqnarray}
w^{n+1}_l&=&w^{exp}_l-\Delta t c_{vd}\bar\theta^z_{v,l}\nabla_z\Pi^{n+1}-\Delta t \nabla_z \overline{\frac{w^{n+1}w^n}{2}}^z\label{w_prog}\\
\rho^{n+1}_k&=&\rho^{exp}_k-\Delta t \nabla_z \cdot \left(\frac{\rho^{n+1}w^n+\rho^nw^{n+1}}{2}\right)\label{rho_prog}\\
\Pi^{n+1}_k&=&\Pi^{exp}_k-\frac{\Delta t R_d \Pi^n_k}{c_{vd}\widetilde{\theta_v}^n_k}\nabla_z\cdot\left[\bar\theta^z_v\left(\frac{\rho^{n+1}w^n+\rho^nw^{n+1}}{2}\right)\right]
\label{Pi_prog}.
\end{eqnarray}
Inserting the Exner equation into the vertical velocity equation leaves us with a coupled problem in $w^{n+1}$ and $\rho^{n+1}$. To boil that down to a problem in one unknown instead of two unknowns, it is straightforward to pose the problem in terms of the unknown mass flux
\begin{equation}
M_l =(\bar{\rho}^{z,n+1}_lw^n_l+\bar{\rho}^{z,n}_lw^{n+1}_l)/2. \label{massfl_prog}
\end{equation}
Then, the unknown vertical velocity $w^{n+1}$ in the kinetic energy gradient term is replaced by
\begin{eqnarray}
w^{n+1}_l &=& \frac{1}{\bar\rho^{z,n}_l}(2M_l-w^n_l\bar\rho^{z,n+1}_l)\label{w_replace}\\
&=&\frac{1}{\bar\rho^{z,n}_l}[2M_l-w^n_l(\bar\rho^{z,exp}_l-\Delta t \overline{\nabla_z \cdot M}^z_l)]\nonumber\\
&=&\frac{2M_l}{\bar\rho^{z,n}_l}-\frac{w^n_l\bar\rho^{z,exp}_l}{\bar\rho^{z,n}_l}+
\frac{w^n_l\Delta t \overline{\nabla_z \cdot M}^z_l}{\bar\rho^{z,n}_l}\nonumber
\end{eqnarray}
Expressing equations (\ref{w_prog})-(\ref{Pi_prog}) in the sketched manner gives
\begin{eqnarray*}
w^{n+1}_l&=&w^{exp}_l-\Delta t c_{vd}\bar\theta^z_{v,l}\nabla_z\left( \Pi^{exp}-\frac{\Delta t R_d \Pi^n}{c_{vd}\widetilde{\theta_v}^n}\nabla_z\cdot(\bar\theta^z_vM)\right)\\
&&-\Delta t \nabla_z \left(
\overline{\frac{w^nM}{\bar\rho^{z,n}}}^z-\overline{\frac{w^{2,n}\bar\rho^{z,exp}}{2\bar\rho^{z,n}}}^z
+\overline{\frac{w^{2,n}\Delta t \overline{\nabla_z \cdot M}^z}{2\bar\rho^{z,n}}}^z\right)\\
\rho^{n+1}_k&=&\rho^{exp}_k-\Delta t \nabla_z \cdot (M)
\end{eqnarray*}
Using now (\ref{massfl_prog}) to combine those equations into one for the unknown mass flux vector $\vec{M}$ leaves us with
\begin{eqnarray*}
 M_l &=&
\frac{\bar{\rho}^{z,n}_lw^{exp}_l}{2}
-\frac{\bar{\rho}^{z,n}_l\Delta t c_{vd}\bar\theta^z_{v,l}}{2}\nabla_z\left( \Pi^{exp}-\frac{\Delta t R_d \Pi^n}{c_{vd}\widetilde{\theta_v}^n}\nabla_z\cdot(\bar\theta^z_vM)\right)\nonumber\\
&&
-\frac{\bar{\rho}^{z,n}_l\Delta t}{2} \nabla_z \left(
\overline{\frac{w^nM}{\bar\rho^{z,n}}}^z-\overline{\frac{w^{2,n}\bar\rho^{z,exp}}{2\bar\rho^{z,n}}}^z
+\overline{\frac{w^{2,n}\Delta t \overline{\nabla_z \cdot M}^z}{2\bar\rho^{z,n}}}^z\right)\nonumber\\
&&+\frac{w^n_l\overline{\rho^{exp}}^z_l}{2}-\frac{w^n_l\Delta t}{2}\overline{\nabla_z \cdot (M)}^z.
\end{eqnarray*}
Collecting all unknowns to the lhs and all known terms to the rhs gives the final equation to solve
\begin{gather}
 M_l
-\frac{\bar{\rho}^{z,n}_l\Delta t c_{vd}\bar\theta^z_{v,l}}{2}\nabla_z\left(\frac{\Delta t R_d\Pi^n}{c_{vd}\widetilde{\theta_v}^n}\nabla_z\cdot(\bar\theta^z_vM)\right)
+\frac{w^n_l\Delta t}{2}\overline{\nabla_z \cdot (M)}^z\nonumber\\
+\frac{\bar{\rho}^{z,n}_l\Delta t}{2} \nabla_z \left(\overline{\frac{w^n}{\bar\rho^{z,n}}M}^z\right)
+\frac{\bar{\rho}^{z,n}_l\Delta t}{2} \nabla_z \left(\overline{\frac{w^{2,n}\Delta t}{2\bar\rho^{z,n}}\overline{\nabla_z\cdot M}^z}^z\right)\nonumber\\
=
\frac{\bar{\rho}^{z,n}_lw^{exp}_l}{2}+\frac{w^n_l\overline{\rho^{exp}}^z_l}{2}
-\frac{\bar{\rho}^{z,n}_l\Delta t c_{vd}\bar\theta^z_{v,l}}{2}\nabla_z \Pi^{exp}
+\frac{\bar{\rho}^{z,n}_l\Delta t}{2} \nabla_z \left(
\overline{\frac{w^{2,n}_l\bar\rho^{z,exp}_l}{2\bar\rho^{z,n}_l}}^z\right)
.\label{matrix1}
\end{gather}
For convenience and better readability we define now the following abbreviations
\begin{gather}
 \alpha_l=\frac{\bar{\rho}^{z,n}_l\Delta t c_{vd}\bar\theta^z_{v,l}}{2\sqrt{g}_l}\qquad\qquad\qquad
 \beta_k =\frac{\Delta t R_d\Pi^n_k}{c_{vd}\widetilde{\theta_{v,k}}^n\sqrt{g}_k}\nonumber\\
 \gamma_l=\frac{w^n_l\Delta t}{4}\qquad\qquad\qquad
 \delta_l=\frac{\bar{\rho}^{z,n}_l\Delta t}{2\sqrt{g}_l}\qquad\qquad\qquad
 \varepsilon_l=\frac{w^n\sqrt{g}_l}{\bar\rho^{z,n}2}\nonumber\\
R_l=\frac{\bar{\rho}^{z,n}_lw^{exp}_l}{2}+\frac{w^n_l\overline{\rho^{exp}}^z_l}{2}
-\frac{\bar{\rho}^{z,n}_l\Delta t c_{vd}\bar\theta^z_{v,l}}{2}\nabla_z \Pi^{exp}
+\frac{\bar{\rho}^{z,n}_l\Delta t}{2} \nabla_z \left(
\overline{\frac{w^{2,n}_l\bar\rho^{z,exp}_l}{2\bar\rho^{z,n}_l}}^z\right).\nonumber
\end{gather}
Performing the spatial discretisation, (\ref{matrix1}) yields
\begin{gather}
 M_l
-\alpha_l\left(\beta_{k-1}(\bar{\theta}^z_{l-1}M_{l-1}-\bar{\theta}^z_lM_l)
               -\beta_k(\bar{\theta}^z_lM_l-\bar{\theta}^z_{l+1}M_{l+1})\right)\nonumber\\
+\gamma_l\left(\frac{1}{\sqrt{g}_{k-1}}(M_{l-1}-M_l)+\frac{1}{\sqrt{g}_{k}}(M_l-M_{l+1})\right)\nonumber\\
+\delta_l\left(\frac{1}{\sqrt{g}_{k-1}}\left(\varepsilon_{l-1}M_{l-1}+\varepsilon_lM_l\right)
              -\frac{1}{\sqrt{g}_k}\left(\varepsilon_lM_l+\varepsilon_{l+1}M_{l+1}\right)\right)\nonumber\\
+\delta_l\left\{
                \frac{1}{\sqrt{g}_{k-1}}\left[\varepsilon_{l-1}\gamma_{l-1}
                \left(\frac{1}{\sqrt{g}_{k-2}}(M_{l-2}-M_{l-1})+\frac{1}{\sqrt{g}_{k-1}}(M_{l-1}-M_{l})\right)\right.\right.\nonumber\\
\left.+\varepsilon_l\gamma_l                        \left(\frac{1}{\sqrt{g}_{k-1}}(M_{l-1}-M_l)+\frac{1}{\sqrt{g}_k}(M_l-M_{l+1})\right)\right]\nonumber\\
-\frac{1}{\sqrt{g}_k}\left[\varepsilon_l\gamma_l\left(\frac{1}{\sqrt{g}_{k-1}}(M_{l-1}-M_l)+\frac{1}{\sqrt{g}_k}(M_l-M_{l+1})\right)\right.\nonumber\\
\left.\left.+\varepsilon_{l+1}\gamma_{l+1}\left(\frac{1}{\sqrt{g}_k}(M_l-M_{l+1})+\frac{1}{\sqrt{g}_{k+1}}(M_{l+1}-M_{l+2})\right)\right]\right\}=R_l\nonumber
\end{gather}
Rearranging this equation gives 
$$a_lM_{l-2}+b_lM_{l-1}+c_lM_l+d_lM_{l+1}+e_lM_{l+2}=R_l$$
with the coefficients $a-e$ and the auxiliary variables $\chi$ and $\psi$
\begin{eqnarray*}
\chi_l&=&\left(\frac{1}{\sqrt{g}_{k-1}}-\frac{1}{\sqrt{g}_k}\right)\\
\psi_l&=&\frac{1}{\sqrt{g}_{k-1}\sqrt{g}_k}\\
a_l&=&\delta_l\varepsilon_{l-1}\gamma_{l-1}\psi_{l-1}\\
b_l&=&-\alpha_l\beta_{k-1}\bar{\theta}^z_{l-1}+\frac{1}{\sqrt{g}_{k-1}}\left(\gamma_l+
\delta_l\left(\varepsilon_{l-1}-\varepsilon_{l-1}\gamma_{l-1}\chi_{l-1}+\varepsilon_l\gamma_l\chi_l\right)\right)\\
c_l&=&1+\alpha_l\bar{\theta}^z_l(\beta_k+\beta_{k-1})-\gamma_l\chi_l+\delta_l\left(
\varepsilon_l\chi_l
-\frac{\varepsilon_{l-1}\gamma_{l-1}+\varepsilon_l\gamma_l}{\sqrt{g}_{k-1}^2}
-\frac{\varepsilon_{l+1}\gamma_{l+1}+\varepsilon_l\gamma_l}{\sqrt{g}_k^2}
+2\varepsilon_l\gamma_l\psi_l\right)\\
d_l&=&-\alpha_l\beta_k\bar{\theta}^z_{l+1}-\frac{1}{\sqrt{g}_k}\left(\gamma_l
+\delta_l\left(\varepsilon_{l+1}
    -\varepsilon_{l+1}\gamma_{l+1}\chi_{l+1}
    +\varepsilon_l\gamma_l\chi_l\right)\right)\\
e_l&=&\delta_l\varepsilon_{l+1}\gamma_{l+1}\psi_{l+1}
\end{eqnarray*}
The upper and lower boundary condition have to be specified additionally. At the upper boundary, a zero vertical velocity is specified. At the lower boundary, a free slip condition yields so that $w$ is determined explicitly by the horizonal wind and the slope of the terrain. For the parts that arise in the divergence term this can be achieved by setting $M_{L+1}$ and the contravariant metric correction term in the contravariant vertical velocity computation at the lowest half level to zero, which gives together the vanishing contravariant vertical velocity that accounts for a zero flux through the lower boundary. But for the kinetic energy, we have to use $w_{L+1}^{n+1}$ directly. This is achieved by setting $\varepsilon_{L+1}=0$ and the rhs is modified 
\begin{eqnarray*}
 R_L&=&\frac{\bar{\rho}^{z,n}_Lw^{exp}_L}{2}+\frac{w^n_L\overline{\rho^{exp}}^z_L}{2}
-\frac{\bar{\rho}^{z,n}_L\Delta t c_{vd}\bar\theta^z_{v,L}}{2}\nabla_z \Pi^{exp}\\
&&+\delta_L \left(\overline{\frac{w^{2,n}\bar\rho^{z,exp}}{2\bar\rho^{z,n}}}^z\right)_{K-1}
-\frac{\delta_L}{\sqrt{g}_K}\left(\frac{\sqrt{g}_L}{2}\frac{w_L^{2,n}\bar{\rho}^{z,exp}_L}{2\bar{\rho}^z_L}
-\frac{\sqrt{g}_{L+1}}{2}\frac{w_{L+1}^nw_{L+1}^{n+1}}{2}\right).
\end{eqnarray*}
For the solution of our 5-band matrix equation we use the LU decompostion. Assume that we have the coefficients $a-e$ as entries in the matrix $\mathsf{A}$. Then the LU decomposition seeks for two matrixes $\mathsf{L}$ and $\mathsf{U}$ which have lower or upper triangular structure, respectively, and one writes
\begin{displaymath}
 \mathsf{LU}\vec{M}=\mathsf{A}\vec{M}=\vec{R}=\mathsf{L}\vec{Q},
\end{displaymath}
where $\vec{Q}$ is an auxiliary solution vector. The structure of the mentioned matrices is
\begin{displaymath}
\mathsf{A}=\left(
\begin{matrix}
   c_2 & d_2 & e_2 & 0   & 0   & 0   & ... \\
   b_3 & c_3 & d_3 & e_3 & 0   & 0   & ... \\
   a_4 & b_4 & c_4 & d_4 & e_4 & 0   & ... \\
   0   & a_5 & b_5 & c_5 & d_5 & e_5 & 0   \\
   ... & ... & ... & ... & ... & ... & ... \\
   ... & 0   & 0   & a_{L-1} & b_{L-1} & c_{L-1} & d_{L-1}  \\
   ... & 0   & 0   &  0  & a_L & b_L & c_L 
  \end{matrix}
 \right),
\end{displaymath}
\begin{displaymath}
\mathsf{L}=\left(
\begin{matrix}
 1   & 0   & 0 & 0 & ... \\
 l_3 & 1   & 0 & 0 & ... \\
 k_4 & l_4 & 1 & 0 & ... \\
 0   & k_5 & l_5 & 1 & ...\\
 ... & ... & ...& ... & ...
\end{matrix}
\right)
\end{displaymath}
\begin{displaymath}
\mathsf{U}=\left(
\begin{matrix}
u_2 & v_2 & w_2 & 0   & 0 & ... \\
0   & u_3 & v_3 & w_3 & 0 & ... \\
0   & 0   & u_4 & v_4 & w_4 & 0 \\
... & ... & ... & ... & ... & ...
\end{matrix}
\right).
\end{displaymath}
By comparison of coefficients one immediately finds $w_l=e_l$. The other relations lead to
\begin{eqnarray*}
 u_2&=&c_2\\
 v_2&=&d_2
\end{eqnarray*}
\begin{eqnarray*}
 l_3&=&b_3/u_2\\
 u_3&=&c_3-l_3v_2\\
 v_3&=&d_3-l_3e_2
\end{eqnarray*}
\begin{eqnarray*}
 k_l&=&a_l/u_{l-2}\\
 l_l&=&(b_l-k_lv_{l-2})/u_{l-1}\\
 u_l&=&c_l-k_le_{l-2}-l_lv_{l-1}\\
 v_l&=&d_l-l_le_{l-1}.
\end{eqnarray*}
Using the structure of the lower triangular matrix, the equation $\mathsf{L}\vec{Q}=\vec{R}$ can be solved for the auxiliary vector $\vec{Q}$
\begin{eqnarray*}
 Q_2&=&R_2\\
 Q_3&=&R_3-l_3Q_2\\
 Q_l&=&R_l-l_lQ_{l-1}-k_lQ_{l-2}.
\end{eqnarray*}
As a second step, the solution for the desired mass flux vector $\vec{M}$ is obtained with the help of the upper triangular matrix equation $\mathsf{U}\vec{M}=\vec{Q}$
\begin{eqnarray*}
 M_L&=&Q_L/u_L\\
 M_{L-1}&=&(Q_{L-1}-v_{L-1}M_L)/u_{L-1}\\
 M_{l}&=&(Q_{l}-v_{l}M_{l+1}-e_{l}M_{l+2})/u_l.
\end{eqnarray*}
The such obtained mass flux is then reinserted into the equations (\ref{rho_prog}) and (\ref{Pi_prog}) in order to update the scalar variables. The vertical velocity for the next time step is computed using (\ref{w_replace}).




\end{document}
