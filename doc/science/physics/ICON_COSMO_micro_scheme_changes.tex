%\documentclass[11pt,dvips,twoside]{report}
%\documentclass[11pt,dvips,a4paper]{article}
\documentclass[a4paper,11pt]{article}

%\usepackage{ametsoc}
%\usepackage{a4}
\usepackage[english]{babel}
\usepackage{epsf}
\usepackage{epsfig}
\usepackage{graphicx}
\usepackage{amsmath,amssymb}
\usepackage{natbib}

\usepackage[a4paper,hmargin=1in,vmargin=1in]{geometry}

%\voffset-0.8cm
%\hoffset-1.0in
%\setlength{\oddsidemargin}{2.8cm}
%\setlength{\evensidemargin}{2.8cm}
%\setlength{\textwidth}{15.4cm}
%\setlength{\topmargin}{0.1cm}
%\setlength{\headheight}{0.5cm}
%\setlength{\headsep}{0.6cm}
\setlength{\parindent}{0.0cm}
%\setlength{\parskip}{0.3cm}
%\setlength{\textheight}{23.7cm}
%\setlength{\footskip}{1.4cm}
%%\addtolength{\skip\footins}{2.5mm}

%\input{makros.tex}
%\input{math_definitions.tex}

\newcommand{\clc}{\ensuremath{\mathcal N}}
\renewcommand{\l}{\ensuremath{\left}}
\renewcommand{\r}{\ensuremath{\right}}
\newcommand{\f}[2]{\ensuremath{\frac{#1}{#2}}}

\newcommand{\pd}[2]{\ensuremath{\frac{\partial #1}{\partial #2}}}
\newcommand{\pdt}[1]{\ensuremath{\pd{#1}{t}}}
\newcommand{\pdx}[1]{\ensuremath{\pd{#1}{x}}}
\newcommand{\pdy}[1]{\ensuremath{\pd{#1}{y}}}
\newcommand{\pdz}[1]{\ensuremath{\pd{#1}{z}}}

\newcommand{\kcc}{\ensuremath{k_{cc}}}

%\newfont{\elevenbfit}{cmbxsl10}
%\newfont{\twelvebfit}{cmbxsl11}

\begin{document}
 
%\pagestyle{empty}

\begin{center}
{\bfseries \LARGE Documentation of the Changes in the Microphysics Scheme (ICON/COSMO)} ~\\[5mm]

{\scshape Carmen K\"ohler} \\
{Deutscher Wetterdienst, Offenbach} \\
{email: carmen.koehler@dwd.de} \\[5mm]
{April 2014} \\
\end{center}

Different switches have been implemented in order to test the sensitivities of
the numerical weather prediction model behaviour. The default setting of the
switches are :
\begin{align} 
  lorig\_icon    &= .TRUE. \quad \text{switch for original ICON setup (only for hydci\_pp)}\\
  lsedi\_ice     &= .FALSE.\quad \text{switch for sedi of cloud ice (Heymsfield \& Donner 1990 *1/3)}\\
  lstickeff      &= .FALSE.\quad \text{switch for sticking coeff. (Guenther Zaengl)}\\
  lsuper\_coolw  &= .FALSE.\quad \text{switch for supercooled liquid water (Felix Rieper)}\\
\end{align}

The switch $lorig\_icon$ is only implemented in the two-category ice scheme
($hydci\_pp$). It ensures the original ICON model setup previous to the other
alterations. This original setup includes cloud ice sedimentation, the change
in the sticking efficiency and the use of the Cooper (1986) formulation for
the cloud ice crystal number (previously denoted by the switch
$lnew\_ni\_cooper$ which was now removed). The maximal number of ice crystals also differs
to the operational $hydci\_pp$ and is set to $znimax_{Thom} = 150
\,\text{L}^{-1} $ in the current ICON scheme. It is set to
$znimax_{Thom} = 250 \,\text{L}^{-1} $ when using the model change for supercooled liquid
water $lsuper\_coolw = .TRUE.$. 

Felix Rieper also changes $zasmel= 2.31e3 \rightarrow 2.43e3$ and  $ zcsdep =
3.2e-2 \rightarrow  3.367e-2$. These were bugfixes and are already operational
in the COSMO-DE. 

The three-category ice scheme is depicted in Fig.\ref{scheme}. Changes that
have been performed by, e.g., G\"unther Z\"angl and Felix Rieper, are
documented in the following. 

\begin{figure}[h]
\centering
\epsfxsize 12.7cm
\centerline{\epsfbox{graupelscheme.eps}}
\caption{\it Cloud microphysical processes from the three-category 
ice scheme} \label{scheme}
\end{figure}

\begin{tabbing}
\hspace*{0.5cm}\=\hspace{1.5cm}\= \kill
\> $S_c $ \> condensation and evaporation of cloud water\\
\> $S_{au}^c$ \> autoconversion of cloud water to form rain \\
\> $S_{ac}$ \> accretion of cloud water by raindrops \\
\> $S_{ev}$ \> evaporation of rain water \\
\> $S_{nuc}$ \> heterogeneous nucleation of cloud ice \\
\> $S_{frz}^c$ \> nucleation of cloud ice due to homogeneous freezing of
cloud water\\
\> $S_{dep}^i$ \> deposition growth and sublimation of cloud ice\\
\> $S_{melt}^i$ \> melting of cloud ice to form cloud water\\
\> $S_{au}^i$ \> autoconversion of cloud ice to form snow due to aggregation\\
\> $S_{aud}$ \> autoconversion of cloud ice to form snow due to deposition\\
\> $S_{agg}^s$ \> collection of cloud ice by snow (aggregation)\\
\> $S_{agg}^g$ \> collection of cloud ice by graupel (aggregation)\\
\> $S_{rim}^s$ \> collection of cloud water by snow (riming)\\
\> $S_{rim}^g$ \> collection of cloud water by graupel (riming)\\
\> $S_{shed}^s$ \> collection of cloud water by wet snow to form rain (shedding)\\
\> $S_{shed}^g$ \> collection of cloud water by wet graupel to form rain (shedding)\\
\> $S_{cri}^i$ \> collection of cloud ice by rain to form graupel\\
\> $S_{cri}^r$ \> freezing of rain due to collection of cloud ice to form graupel\\
\> $S_{frz}^r$ \> freezing of rain to form graupel\\
\> $S_{dep}^s$ \> deposition growth and sublimation of snow\\
\> $S_{dep}^g$ \> deposition growth and sublimation of graupel\\
\> $S_{melt}^s$ \> melting of snow to form rain water\\
\> $S_{melt}^g$ \> melting of graupel to form rain water\\
\> $S_{csg}$ \> conversion of snow to graupel due to riming\\
\end{tabbing}


\section{Cloud Ice Sedimentation}
Cloud ice sedimentation plays an important role in restructuring the high
clouds and counteracting overprediction or too long lifecyles of cirrus. Thus,
in the ICON cloud ice sedimentation was implemented (Felix Rieper) and tuned
(G\"unther Z\"angl). This sedimentation is now also implemented into the
graupel scheme and can be controlled by use of the switch
$\bold{lsedi\_ice}$. The equation used for the sedimentation velocity of cloud
ice is 
\begin{equation}
 zvzi(iv) = zvz0i * (0.5 q_i \rho)^{zbvi}
\end{equation}
 with the terminal fall velocity of ice based on Heymsfield and Donner (1990)
(multiplied by 1/3) $zvz0i  = 1.1$ and $zbvi   = 0.16$.

\section{Sticking Efficiency}
With cloud ice sedimentation change sticking efficieny as done in ICON
(implemented by Günther Z\"angl) switch is $\bold{lsedi\_ice}$. The sticking efficieny
influences the aggregation and ice autoconversion. It can be turned off with
the switch $\bold{lstickeff}$. In the following this change is documented.

Scaling factor $1/K$ for temperature-dependent cloud ice sticking efficiency
$zceff_{fac} = 3.5e-3$, the temperature at which cloud ice autoconversion starts
$Tmin_{iceautoconv} = 188.15$ and the minimum value for sticking efficiency
$zceff_min        = 0.02$.

\begin{align}
  stickeff &= min(\exp(0.09*(T-T_0)),1.0) \\
  stickeff &= max(stickeff, zceff_{min}, zceff_{fac}*(T-Tmin_{iceautoconv})) \\
  zsiau &= zciau * max( q_i - q_{i,0}, 0.0 ) * stickeff
\end{align} 

The sticking efficiency of cloud ice of the operational COSMO model code is
\begin{align}
  stickeff &= min(\exp(0.09*(T-T_0)),1.0) \\
  stickeff &= max(stickeff,0.2)\\
  zsiau &= zciau * max( q_i - q_{i,0}, 0.0) *stickeff.
\end{align}

The resulting equations for the source/sink terms are
\begin{align}
  sagg(iv)  & = stickeff \, q_i * zcagg(iv) * zcslam(iv)^{ccsaxp} \\
  sagg2(iv) & = stickeff \, q_i * zcagg_g * zelnrimexp_g(iv)\\
  siau(iv)  & = stickeff * zciau * max( q_i - q_{i,0}, 0.0).\\
\end{align}

\section{Evaporation}
G\"unther Z\"angl also implemented a limitation for maximum evaporation. This
is needed to provide numerical stability for large horizontal resolutions. The
limit for the evaporation rate is introduced in order to avoid overshoots
towards supersaturation. The pre-factor approximates
$(esat(T_{wb})-e)/(esat(T)-e)$ at temperatures between $0^{°}C$ and $30^{°}C$
\begin{align}
 T_c &=T - T_0 \\
 maxevap& =(0.61-0.0163*T_c+1.111e-4*T_c**2)*(zqvsw-q_v)/\Delta \,t \\
sev(iv) &= min(zcev*zx1*(zqvsw - q_v) * \exp(zcevxp * zlnqrk), maxevap)\\
\end{align}

\section{Supercooled Liquid Water Vapor}
The change for the supercooled liquid water based on the work done by Felix
Rieper can be switched on/off with the switch $\bold{lsuper\_coolw}$. The
supercooled liquid water approach reduces the depositional growth for
temperatures below $ztmix = 250\,\text{K}$ which is assumed to be the
threshhold for mixed-phase clouds (Forbes 2012). Also, a different
parameterization for the cloud ice number is required. In the operational COSMO model a modified version of
the Fletcher equation is used while Thompson (NCAR) proposed the use of the
Cooper (1986) 
\begin{equation}
fxna_{cooper}(T) = 5.0 * \exp(0.304 * (T_0 - T))   
\end{equation}
with the maximal number of ice crystals  $znimax_{Thom} = 250.0e3$ and $znimix = fxna_{cooper}(ztmix)$. 
The reduction coeff. for dep. growth of rain and ice is $reduce_{dep} = 1.0.$.

Calculation of reduction of depositional growth at cloud top (Forbes 2012)
\begin{align}
  znin & = min( fxna_{cooper}(T), znimax )\\
  fnuc & = min(znin/znimix, 1.0_ireals)\\
  zdh  & = 0.5 * (hhl(i,j,k-1)-hhl(i,j,k+1))\\
\end{align}
Then the distance from cloud top is calculated and the depositional growth
rate for cloud ice and snow reduced accordingly. 
\begin{equation}
  reduce_{dep}(i,j) = min(fnuc + (1.0 -fnuc)*(reduce_{dep,ref} + dist_{cldtop}(i,j)/dist_{cldtop,ref}), 1.0)
\end{equation}

\section{Reduced Freezing Rate}
Felix Rieper also implemented the reduction in freezing rate of in-cloud and
below cloud rainwater. This reduction takes effect for temperatures below the
threshhold for heterogeneous freezing of raindrops $ztrfrz=271.15 \,\text{K}$
according to Bigg (1953)
\begin{align}
  srfrz(i,j) = zcrfrz1*(EXP(zcrfrz2*(ztrfrz-T))-1.0) * \l(q_r \rho\r)^(7/4)
\end{align}

The changes performed by Felix Rieper where all tested within the COSMO-EU model scheme
($hydci\_pp$) and where chosen to be introduced in the COSMO Model V5.1 in order
to improve the forecast for aircraft icing (ADWICE) (decided in a Routine
Besprechung).

\bibliographystyle{ametsoc}
\bibliography{lmclouds}

\end{document}



