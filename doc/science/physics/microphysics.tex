
\chapter  A short introduction to microphysics in the NWP ICON model

\begin{center}
{\scshape Axel Seifert} \\
{Deutscher Wetterdienst, Offenbach} \\
{email: axel.seifert@dwd.de} \\[5mm]
{February 2010} \\
\end{center}

A short summary of the cloud physics is given for the non-hydrostatic branch of the ICON atmosphere model ICONAM. This parameterization is taken from the COSMO model, therefore the reader might also be refered to \citet{Doms-Schaettler-2004}.

\section{Grid-scale clouds}
ICONAM uses a two-category ice scheme which explicitly predicts the mass fractions of
cloud water $q_c$, rain water $q_r$, cloud ice $q_i$ and snow $q_s$ at every
grid point and includes the advection of all hydrometeors.  For the
non-precipitating categories we apply the budget equation including turbulent
fluxes ${\bf F}^{c,i}$, but neglecting sedimentation:

\underline{non-precipitating categories} (cloud water and cloud ice)
\begin{equation}
\frac{\partial q^{c,i}}{\partial t} + {\bf v} \cdot \nabla q^{c,i}
\, = \, S^{c,i}  - \frac{1}{\rho} \nabla \cdot {\bf F}^{c,i} \,, \label{bwc-3}
\end{equation}

While for precipitation-sized particles only sedimentation is taken into
account, since for larger particles the sedimentation fluxes are usually much
larger than the turbulent fluxes:

\underline{precipitating categories} (rain, snow and graupel)
\begin{equation}
\frac{\partial q^{s,r}}{\partial t} + {\bf v} \cdot \nabla q^{s,r}
 - \frac{1}{\rho}\frac{\partial \rho q^{s,r} v_T^{s,r} }{\partial z}
\, = \, S^{s,r}  \,, \label{bwc-4}
\end{equation}

Figure \ref{icescheme} gives an overview of the microphysical sources and
sinks  $S$ that are considered in this two-category ice scheme. The
individual microphysical processes are:

\begin{tabbing}
\hspace*{0.5cm}\=\hspace{1.5cm}\= \kill
\> $S_c $ \> condensation and evaporation of cloud water\\
\> $S_{au}^c$ \> autoconversion of cloud water to form rain \\
\> $S_{ac}$ \> accretion of cloud water by raindrops \\
\> $S_{ev}$ \> evaporation of rain water \\
\> $S_{nuc}$ \> heterogeneous nucleation of cloud ice \\
\> $S_{frz}^c$ \> nucleation of cloud ice due to homogeneous freezing of
cloud water\\
\> $S_{dep}^i$ \> deposition growth and sublimation of cloud ice\\
\> $S_{melt}^i$ \> melting of cloud ice to form cloud water\\
\> $S_{au}^i$ \> autoconversion of cloud ice to form snow due to aggregation\\
\> $S_{aud}$ \> autoconversion of cloud ice to form snow due to deposition\\
\> $S_{agg}$ \> collection of cloud ice by snow (aggregation)\\
\> $S_{rim}$ \> collection of cloud water by snow (riming)\\
\> $S_{shed}$ \> collection of cloud water by wet snow to form rain (shedding)\\
\> $S_{cri}^i$ \> collection of cloud ice by rain to form snow\\
\> $S_{cri}^r$ \> freezing of rain due to collection of cloud ice to form snow\\
\> $S_{frz}^r$ \> freezing of rain due heterogeneous nucleation to form snow\\
\> $S_{dep}^s$ \> deposition growth and sublimation of snow\\
\> $S_{melt}^s$ \> melting of snow to form rain water\\
\end{tabbing}

\begin{figure}[t]
\centering
\epsfxsize 11cm
\centerline{\epsfbox{PICTURES/icescheme.eps}}
\caption{\it Cloud microphysical processes considered in the two-category 
ice scheme} \label{icescheme}
\end{figure}

The following main assumption are made in the parameterization:
\begin{itemize}
\item The raindrops are assumed to be exponentially distributed with respect
  to drop diameter $D$:
  \begin{equation}
    f_r(D)\, = \,N^r_0 \exp ( - \lambda_r D ) \,,\label{7.3-2}
  \end{equation}
  where $N^r_0 = 8\times 10^6$ m$^{-4}$ is an empirically determined
  distribution parameter (Marshall-Palmer distribution). For the terminal fall
  velocities of raindrops as functions of size we use the following empirical
  relation
  \begin{equation}
    v^{rp}_T(D) \,=\, v^r_0 D^{1/2}
  \end{equation}
  where $v^r_0$ = 130 m$^{1/2}$s$^{-1}$.
\item The autoconversion scheme is parameterized using \citet{Seifert-Beheng-2001} which reads
  \begin{equation}
    \pdt{L_r} \biggr |_{au}
    =  \f{\kcc}{20 \, x^*}
    \f{(\nu+2)(\nu+4)}{(\nu+1)^2} \, 
    L_c^4 \, N_c^{-2} \,
    \l [ 1 + \f{\Phi_{au}(\tau)}{(1-\tau)^2} \r ]
  \end{equation}
  with $L_{c/r}$ cloud/rain water content, $N_c$ cloud droplet number
  concentration, $\nu$ shape parameter, $\kcc=9.44 \times 10^9$ s$^{-1}$
  kg$^{-2}$ m$^{3}$, $x^*=2.6\times 10^{-10}$ kg m$^{-3}$.  The function
  $\Phi_{au}(\tau)$ describes the aging (broadening) of the cloud droplet
  distribution as a function of the dimensionless internal time scale
  \begin{equation*}
    \tau = 1 - \frac{L_c}{L_c + L_r}
  \end{equation*}
  \citep[for details see][]{Seifert-Beheng-2001}. In the one-moment schemes of
  the ICON model we simplify the scheme by assuming a constant cloud droplet
  number concentration of $N_c = 5 \times 10^8$ m$^{-3}$ and a constant shape
  parameter $\nu=2$.
\item Snow particles are interpreted as unrimed or partly rimed
  aggregates. The equation
  \begin{equation}
    m \,=\, \alpha \,D_s^2 \label{7.4-4}
  \end{equation}
  with $\alpha=0.069$ is used to specify their mass-size
  relation and the terminal
  fall velocity is parameterized as $v=15 \; D^{1/2}$ (here $D$ in m, $m$ in kg
  and $v$ in m/s). The size distribution of snow is an inverse exponential
  \begin{equation*}
    f(D) = N_{0,s} \exp(-\lambda D).
  \end{equation*}
  The intercept parameter is parameterized
  as a function of temperature $T$ and snow mixing ratio $q_s$ by:
  \begin{equation*}
    N_{0,s} = \frac{27}{2} a(3,T) \left(\frac{q_s}{\alpha}\right)^{4-3b(3,T)}
  \end{equation*}
  The functions $a(3,T)$ and $b(3,T)$ are given by Table 2
  of \citet{Field-2005}. This parameterization is used instead of the constant
  $N_{0,s}=8\times10^5$ m$^{-4}$ which was used in the old version of
  the scheme. Especially at cold temperatures the new formulation leads to a much higher
  intercept parameter, this corresponds to smaller snowflakes at
  high levels which fall out much slower. The choices about the size
  distribution and fall speeds of snow are very important for wintertime
  orographic precipitation.
    
\item The rate of autoconversion from cloud ice to snow due to 
  cloud ice crystal aggregation ($S_{au}^i$) is parameterized by the simple 
  relations
  \begin{eqnarray}
    S_{au}^i &=& \max\{ c^i_{au}\,( q^i - q^i_0 )\,,\, 0 \,\}\,. \label{7.5-18}
  \end{eqnarray}
  Currently we do not use an autoconversion threshold value for cloud ice
  (hence, $q^i_0 = 0$).  The rate coefficient is set to $c^i_{au}= 10^{-3}\,\rm
  s^{-1}$.  We assume a monodispers size distribution for cloud ice with a
  mean crystal mass given by
  \begin{equation}
    m_i \,=\, \rho q^i N_i^{-1}\,,\label{7.5-1}
  \end{equation}
  where $N_i$ is the number of cloud ice particles per unit volume of air. The
  number density $N_i$ is parameterized as a function of temperature by
  \begin{equation}
    N_i(T)\,=\, N_0^i \exp \{ \,0.2 \,(T_0 -T)\}\,,\qquad N_0^i 
    \,=\,1.0\cdot 10^2 m^{-3}\,. \label{7.5-3} 
  \end{equation}
  This simple approximation is based on aircraft measurements of the
  concentration of pristine crystals in stratiform clouds using data of
  \citet{Hobbs-Rangno-1985} and \citet{Meyers-1992}.  A more physically based
  approach must involve a double-moment representation of cloud ice including
  a budget equation for the concentration of ice particles and maybe ice
  nuclei. Such schemes are not yet available for the ICON model.

  For the autoconversion of cloud ice and the aggregation of cloud ice by snow a
  temperature dependent sticking efficiency has been introduced similar to
  \citet{Lin-1983}:
  \begin{equation*}
    e_{i}(T) = \max(0.2,\min(\exp(0.09(T-T_0)),1.0))
  \end{equation*}
  with $T_0=273.15$ K. 

  Note that depositional growth is explicitly
  parameterized, thus the model predicts ice supersaturation. The change of
  the cloud ice mixing ratio by depostion is given by
  \begin{equation}
    \l.\pdt{q_i} \r|_{\text{dep}} 
    = \frac{4 D_v}{1 + H} \; \l ( \frac{m_i}{a_m} \r ) ^{1/3} 
    \, N_i \, (q_v - q_{i,sat})
  \end{equation}
  where $H$ is the so-called Howell factor (see Doms et al. 2004 for details).

\end{itemize}

%\include{p4c05}




%\end{document}



