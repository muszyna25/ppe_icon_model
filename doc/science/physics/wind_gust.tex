\documentclass[a4paper,11pt]{article}
%\documentclass[a4paper,11pt,twocolumn]{article}

\usepackage{natbib}
\setlength{\bibsep}{1pt}
%\newcommand{\bibfont}{\footnotesize}

\usepackage{enumitem}

\usepackage[dvips,colorlinks,bookmarksopen,bookmarksnumbered,citecolor=red,urlcolor=red]{hyperref}

\usepackage{moreverb}
\addtolength{\hoffset}{-2cm}
\addtolength{\textwidth}{4cm}
\addtolength{\voffset}{-3cm}
\addtolength{\textheight}{5cm}
\usepackage{array,amsmath,graphicx}
\author{Daniel Reinert}
\title{ICON Documentation: Parameterization of wind gusts}
%\date{ 2007} % optional  
\begin{document}  \maketitle

%-------------------------------------------------------------------------

%\chapter{Physical processes}

\section{Wind gusts}

The gust parameterization of ICON consists of two components: turbulent gusts and convective gusts. The diagnosis of turbulent gusts has been taken over from the COSMO 
model \citep{Schulz:2003,Schulz:2008}. Turbulent gusts at $10\,\mathrm{m}$ above ground are derived from the turbulence state in the atmospheric boundary layer, using the absolute speed 
of the mean wind at $10\,\mathrm{m}$ above ground $U_{10}$ and its standard deviation $\sigma$.
\begin{align}
 U_{10\,\mathrm{gust, turb}} = U_{10} + \alpha\, \sigma
\end{align}
Following \cite{Panofsky:1984}, the standard deviation $\sigma$ can be approximated as
\begin{align*}
 \sigma \approx 2.4\,u_{\ast}
\end{align*}
with $u_{\ast}$ denoting the friction velocity. Using the relationship $u_{\ast}=\sqrt{C_{D}}\,U_{10}$, where $C_{D}$ denotes the drag coefficient for momentum, we 
arrive at
\begin{align}
 U_{10\,\mathrm{gust, turb}} = U_{10} + \alpha\ 2.4 \sqrt{C_{D}}\,U_{10} \label{eq_turb_gust}
\end{align}
The tuning parameter $\alpha$ has been estimated to $\alpha=3$. Note that for computing $u_{\ast}$, the mean wind speed at the lowest model level is used rather 
than $U_{10}$.

If an SSO parameterization is used, the wind speed directly above the top of the envelope layer computed in the SSO
scheme is taken into account as well. Let $k_\mathrm{env}$ denote the level index the top of the envelope layer has at a given
grid point. If $k_\mathrm{env} < nlev$, the turbulent gust is enhanced by taking
\begin{align}
 U_{10\,\mathrm{gust, turb}} = \mathrm{MAX}( U_{10\,\mathrm{gust, turb}}, U(k_\mathrm{env} -1)) \,.
\end{align}
In practice, this exclusively affects mountainous regions where the SSO standard deviation has a significant magnitude.

Equation \eqref{eq_turb_gust} asesses the gustiness in the boundary layer, however it does not take into account gusts due to strong convective events. Therefore, a 
convective contribution is added to the turbulent wind gusts in the presence of deep convection. The parameterization follows \cite{Bechthold:2009} where the convective gusts 
are simply estimated as proportional to the low level wind shear:
\begin{align}
U_{10\,\mathrm{gust, conv}} = C_{conv} \max{(0,U_{850} - U_{950})} \label{eq_conv_gust}
\end{align}
with the convective mixing parameter $C_{conv}$ and $U_{850} - U_{950}$ the difference between $850\,\mathrm{hPa}$ and $950\,\mathrm{hPa}$ wind speeds. This contribution 
is computed only in regions where deep convection is active, as identified by the convection scheme (\emph{ktype=1}). Thus, the total gustiness 
$U_{10\,\mathrm{gust}}$ becomes:
\begin{align}
  U_{10\,\mathrm{gust}} = U_{10\,\mathrm{gust, turb}} + U_{10\,\mathrm{gust, conv}}
\end{align}
The parameter $U_{10\,\mathrm{gust}}$ is computed every fast-physics time step and its 6-hourly maximum (preliminary) is written to disk.


\subsection{Computation of $U_{850}$ and $U_{950}$}

The presence of topography and the fact that ICON uses a height based rather than a pressure based vertical coordinate, requires some care when implementing the convective 
contribution \eqref{eq_conv_gust}. In order to derive a sound estimate for $U_{850} - U_{950}$, we decided to
\begin{enumerate}
 \item convert the $850\,\mathrm{hPa}$ and $950\,\mathrm{hPa}$ pressure levels into heights (above sea) by using the US Standard Atmosphere
 \item compute the vertical level indices that matches best with those heights above ground, for each grid point. Wind speed is then evaluated at these level indices.
\end{enumerate}

For the US Standard Atmosphere below $11\,\mathrm{km}$ height, one can derive the following expression for height $z$ as a function of pressure 
$p$ \citep{USStandard:1976}:
\begin{align}
 z = \frac{T_{0}}{\Gamma}\left[\exp{\left(-\frac{R_{d}\Gamma}{g}\ln\frac{p}{p_{0}}\right)-1}\right]
\end{align}
with $\Gamma=-0.0065\, \mathrm{K/m}$ the tropospheric temperature gradient of the US standard atmosphere, $T_{0}=288.15\mathrm{K}$, $p_{0}=1013.25\,\mathrm{hPa}$ and the specific 
gas constant for dry air $R_{d}=287.04\,\mathrm{J/(kg\,K)}$. For $p=850\,\mathrm{hPa}$ and $p=950\,\mathrm{hPa}$, we get $z=1457.24\,\mathrm{m}$ and $z=540.31\,\mathrm{m}$, 
respectively. In the following, those heights are interpreted as heights above ground, and for each grid point we search for the full levels that are closest to those heights 
and store the corresponding level indices termed $k_{850}$ and $k_{950}$. For a particular horizontal grid point $i$, $U_{850} - U_{950}$ is then given by
\begin{align}
 U_{850} - U_{950} \approx \sqrt{u_{k_{850}}^2 + v_{k_{850}}^2} - \sqrt{u_{k_{950}}^2 + v_{k_{950}}^2}
\end{align}
 



%-------------------------------------------------------------------------

\bibliography {../references-icon-science}
\bibliographystyle {../wileyqj} %QJ


\end{document}
