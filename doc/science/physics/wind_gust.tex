\documentclass[a4paper,11pt]{article}
%\documentclass[a4paper,11pt,twocolumn]{article}

\usepackage{natbib}
\setlength{\bibsep}{1pt}
%\newcommand{\bibfont}{\footnotesize}

\usepackage{enumitem}

\usepackage[dvips,colorlinks,bookmarksopen,bookmarksnumbered,citecolor=red,urlcolor=red]{hyperref}

\usepackage{moreverb}
\addtolength{\hoffset}{-2cm}
\addtolength{\textwidth}{4cm}
\addtolength{\voffset}{-3cm}
\addtolength{\textheight}{5cm}
\usepackage{array,amsmath,graphicx}
\author{Daniel Reinert}
\title{ICON Documentation: Parameterization of wind gusts}
%\date{ 2007} % optional  
\begin{document}  \maketitle

%-------------------------------------------------------------------------

%\chapter{Physical processes}

\section{Wind gusts}

The gust parameterization of ICON consists of two components: turbulent gusts and convective gusts. The diagnosis of turbulent gusts has been taken over from the COSMO 
model \citep{Schulz:2003,Schulz:2008}. Turbulent gusts at $10\,\mathrm{m}$ above ground are derived from the turbulence state in the atmospheric boundary layer, using the absolute speed 
of the mean wind at $10\,\mathrm{m}$ above ground $U_{10}$ and its standard deviation $\sigma$.
\begin{align}
 U_{10\,\mathrm{gust, turb}} = U_{10} + \alpha\, \sigma
\end{align}
Following \cite{Panofsky:1984}, the standard deviation $\sigma$ can be approximated as
\begin{align*}
 \sigma \approx 2.4\,u_{\star}
\end{align*}
with $u_{\star}$ denoting the friction velocity. Using the relationship $u_{\star}=\sqrt{C_{D}}\,U_{10}$, where $C_{D}$ denotes the drag coefficient for momentum, we 
arrive at
\begin{align}
 U_{10\,\mathrm{gust, turb}} = U_{10} + \alpha\ 2.4 \sqrt{C_{D}}\,U_{10} \label{eq_turb_gust}
\end{align}
The tuning parameter $\alpha$ has been estimated to $\alpha=3$. Note that for computing $u_{\star}$, the wind speed at the lowest model level is used rather 
than $U_{10}$.

Equation \eqref{eq_turb_gust} asesses the gustiness in the boundary layer, however it does not take into account gusts due to strong convective events. Therefore, a 
convective contribution is added to the turbulent wind gusts in the presence of deep convection. The parameterization follows \cite{Bechthold:2009} where the convective gusts 
are simply estimated as proportional to the low level wind shear:
\begin{align}
U_{10\,\mathrm{gust, conv}} = C_{conv} \max{(0,U_{850} - U_{950})} 
\end{align}
with the convective mixing parameter $C_{conv}$ and $U_{850} - U_{950}$ the difference between $850\,\mathrm{hPa}$ and $950\,\mathrm{hPa}$ wind speeds. This contribution 
is computed only in regions where deep convection is active, as identified by the convection scheme (\emph{ktype=1}). Thus, the total gustiness 
$U_{10\,\mathrm{gust}}$ becomes:
\begin{align}
  U_{10\,\mathrm{gust}} = U_{10\,\mathrm{gust, turb}} + U_{10\,\mathrm{gust, conv}}
\end{align}

The parameter $U_{10\,\mathrm{gust}}$ is computed every fast-physics time step and its 6-hourly maximum (preliminary) is written to disk.

\subsection{Computation of $U_{850}$ and $U_{950}$}



%-------------------------------------------------------------------------

\bibliography {../references-icon-science}
\bibliographystyle {../wileyqj} %QJ


\end{document}
