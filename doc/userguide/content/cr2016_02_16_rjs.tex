\subsection[cr2016\_02\_16\_rjs: Aqua planet in ICON]{cr2016\_02\_16\_rjs:
  Aqua planet in ICON}\label{cr20140715rjs}

The Aqua Planet Experiment (APE) is incorporated in ICON as a set of 
testcases, all of them having in common that the bottom boundary
consists of water only, with all land surface and orography being
removed. Originally APEs where constructed as a test--bed for model
intercomparison regarding the interaction of dynamics and physical
parameterizations in atmospheric General Circulation Models (GCMs)
\cite{nea019,nea010}. However, this idealized setup is also well
suited for studies focusing on the interaction of basic atmospheric
phenomena and the ocean surface, e.g.~tropical convection, when a
realistic representation of the atmosphere is secondary. 

The APE is one of the test cases of ICON and can be switched on
by setting the respective {\tt nh\_test\_name} variable in the {\tt
  nh\_testcase\_nml} namelist, see Listing~\ref{cr20160216_testcasenml}.

Sea Surface Temperatures (SSTs) can either be prescribed by one of
several analytical functions and kept constant in time, or alternatively SSTs
may interact with the atmosphere by the use of a slab ocean layer. The 
slab ocean is not tested yet.
The following SSTs $T_{\rm surf}$ are available from {\tt
  mo\_ape\_params.f90} under the respective keywords ($T_{\rm
  m}=273.15\,{\rm K}$):  
\begin{description}
\item[keyword {\tt sst1}]
\begin{equation}
T_{\rm surf}=
\begin{cases}
T_{\rm m} + 27\,{\rm K} \left(1-\left(\sin(3\phi/2)\right)^2\right)
&\quad \text{for all} \quad -\pi/3<\phi<\pi/3\\
T_{\rm m} &\quad \text{otherwise}
\end{cases}
\end{equation}
\item[keyword {\tt sst2}]
\begin{equation}
T_{\rm surf}=
\begin{cases}
T_{\rm m} + 27\,{\rm K} \left(1-\frac{3}{\pi}|\phi|\right)
&\quad \text{for all} \quad -\pi/3<\phi<\pi/3\\
T_{\rm m} &\quad \text{otherwise}
\end{cases}
\end{equation}
\item[keyword {\tt sst3}]
\begin{equation}
T_{\rm surf}=
\begin{cases}
T_{\rm m} + 27\,{\rm K} \left(1-\left(\sin(3\phi/2)\right)^4\right)
&\quad \text{for all} \quad -\pi/3<\phi<\pi/3\\
T_{\rm m} &\quad \text{otherwise}
\end{cases}
\end{equation}
\item[keyword {\tt sst4}:]
This distribution has its temperature peak at $5^\circ$N:
\begin{equation}
T_{\rm surf}=
\begin{cases}
T_{\rm m} + 27\,{\rm K} \left(1-\left(\sin(1.6363(\phi-\frac{\pi}{36}))\right)^2\right)
&\quad \text{for all} \quad \pi/36<\phi<\pi/3\\
T_{\rm m} + 27\,{\rm K} \left(1-\left(\sin(1.3846(\phi-\frac{\pi}{36}))\right)^2\right)
&\quad \text{for all} \quad -\pi/3<\phi\le\pi/36\\
T_{\rm m} &\quad \text{otherwise}
\end{cases}
\end{equation}
\item[keyword {\tt sst\_qobs}]
\begin{equation}
T_{\rm surf}=
\begin{cases}
T_{\rm m} + \frac{27}{2}\,{\rm K} \left(2-\left(\sin(3\phi/2)\right)^2
\left(1+\left(\sin(3\phi/2)\right)^2
\right)\right)
&\quad \text{for all} \quad -\pi/3<\phi<\pi/3\\
T_{\rm m} &\quad \text{otherwise}
\end{cases}
\end{equation}
\item[keyword {\tt sst\_ice}]
\begin{equation}
T_{\rm surf}=
\begin{cases}
T_{\rm m} + 1.9\,{\rm K} + 27\,{\rm K} \left(1-\left(\sin(3\phi/2)\right)^2\right)
&\quad \text{for all} \quad -\pi/3<\phi<\pi/3\\
T_{\rm m}-1.9\,{\rm K} &\quad \text{otherwise}
\end{cases}
\end{equation}
\item[keyword {\tt sst\_const}]
\begin{equation}
T_{\rm surf}=T_{\rm m}+29\,{\rm K}
\end{equation}
 
\end{description}

%Further description of these SSTs can be found at {\tt http://www.met.reading.ac.uk/~mike/APE/ape\_spec\_sst\_2.pdf}
The various SSTs can be set in the namelist {\tt nh\_testcase\_nml} by
attributing the corresponding keyword to the variable {\tt
  ape\_sst\_case}. For an example see Listing~\ref{cr20160216_testcasenml}.

\begin{lstlisting}[caption=Testcase namelist for APE ({\tt
    exp.atm\_ape\_test}), label=cr20160216_testcasenml]
&nh_testcase_nml
 nh_test_name     = 'APE_echam'
 ape_sst_case     = 'sst_qobs'
/
\end{lstlisting}

The
APE can be conducted with the MPI/ECHAM physics only since 
the use of the PSRAD radiation is essential for the APE. A standard
example script {\tt exp.atm\_ape\_test} is provided.
In this standard version, a Kepler orbit with no eccentricity and
zero obliquity is used. The orbit parameters are set in the {\tt
  psrad\_orbit\_nml} namelist (Listing~\ref{cr20160216_orbitnml}).

\begin{lstlisting}[caption=Orbit namelist for APE ({\tt exp.atm\_ape\_test}),
label=cr20160216_orbitnml]
&psrad_orbit_nml
cecc        = 0.0      ! zero excentricity
cobld       = 0.0      ! zero obliquity
l_orbvsop87 = .FALSE.  ! use Kepler orbit instead of historic one
/
\end{lstlisting}

If a circular orbit with no obliquity is used, a year with months of a
different number of 
days does not make sense. Instead, a 360 day year should be used.
The calendar must be set by two variables at the beginning of the {\tt
  exp.atm\_ape\_test} script (Listing~\ref{cr20160216_calendar}).

\begin{lstlisting}[caption=Calendar for APE ({\tt
    exp.atm\_ape\_test}), label=cr20160216_calendar]
  calendar="'360 day year'"
  calendar_type=2
\end{lstlisting}

The solar irradiation is set to the preindustrial value, i.e.~to a
value of $1360.875\,{\rm W/m}^2$. The APE uses a composition of the
atmosphere where the O$_3$ concentration is read from a special file. Prognostic
water vapour and prognostic liquid and ice cloud water is used
together with constant CO$_2$ in space and time, and no
aerosols. All 
other greenhouse gases are set to zero (Listing~\ref{cr20160216_radiationnml}). 

\begin{lstlisting}[caption=Solar irradiation and composition of
  atmosphere ({\tt exp.atm\_ape\_test}), label=cr20160216_radiationnml] 
&radiation_nml
 isolrad    = 2 ! preindustrial solar constant of 1360.875 W/m^ at 1 AE
 irad_h2o   = 1 ! prognostic vapor, liquid and ice
 irad_co2   = 2 ! constant co2
 irad_o3    = 4 ! perpetual january of ozone file linked
 irad_ch4   = 0 ! no ch4
 irad_n2o   = 0 ! no n2o
 irad_o2    = 0 ! no o2
 irad_cfc11 = 0 ! no cfc11
 irad_cfc12 = 0 ! no cfc12
 irad_aero  = 0 ! no aerosol
/
\end{lstlisting}

The O$_3$ concentrations are read from the file {\tt
  ape\_o3\_iconR2B04-ocean\_aqua\_planet.nc} that is provided in the
directory {\tt data/external/ape\_ozone} of the ICON main directory.
Table~\ref{cr20160216_tabozone} lists possible ozone files.

\begin{table}[ht]\caption{Possible ozone file}\label{cr20160216_tabozone}
\begin{tabular*}{\textwidth}{l@{\extracolsep\fill}p{6.7cm}}\hline
file name & description\\\hline
{\tt ape\_o3\_iconR2B03-ocean\_aqua\_planet.nc} & contains one time
step of ozone values, interpolated from {\tt ape\_o3\_T42\_1Pa.nc} to
r2b3 resolution\\
{\tt ape\_o3\_iconR2B04-ocean\_aqua\_planet.nc} & contains one time
step of ozone values, interpolated from {\tt ape\_o3\_T42\_1Pa.nc} to
r2b4 resolution\\
{\tt ape\_o3\_R2B04\_1Pa\_spr0.90-cell.nc} & contains the same ozone
values as the file above\\
{\tt ape\_o3\_R2B04\_hex\_1Pa\_c.nc} & unclear, contains 10242 grid
points only instead of 20480 as the files for r2b4 resolution
(hexagons instead of triangles?)\\
{\tt tor120000s64\_ozone\_CMIP5\_aqua\_1870\_march.nc} & one time step
on 8192 cells, 36 levels, seems to be on a torus grid.\\
\hline
\end{tabular*}
\end{table}


