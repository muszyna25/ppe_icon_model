\subsection[cr2016\_06\_14\_rjs: RCE in ICON using ECHAM physics]{cr2016\_02\_16\_rjs:
  Radiative--Convective equilibrium in ICON using ECHAM physics}\label{cr20160614rjs}

The radiative--convective equilibrium (RCE) offers a possibility
to improve our fundamental understanding of processes in the
atmosphere and their impact on climate change
(e.g.~\cite{man641}). The idea behind this simplified modeling
of the atmosphere 
is that the basic 
atmospheric structure, especially in the tropics, is determined by the
balance between cooling of the atmosphere through radiative processes
and a commensurate heating through convection, mainly by the net
release of latent heat through precipitation. 

The RCE has been investigated in models of different complexity,
ranging from simple energy balance, 1--dimensional column models to
high resolution LES simulations. The RCE is also implemented into the 
general circulation model ICON by creating a model configuration,
where the resulting climate is given merely through the balance of
radiative processes and convection. Columns can interact with each
other and thus create a mean three--dimensional circulation which
develops interactively, although it is very different from the general
circulation we know from the real Earth. E.g., the RCE results in slowly moving
convective clusters of sometimes continental extension (\cite{pop131}).  

To inhibit net energy transport from the tropics to the poles,
homogeneous boundary conditions are specified, where every gridpoint
of the sphere receives the same incoming solar radiation
(e.g.~about $340\,{\rm W/m^2}$). A diurnal
cycle may be switched on, but is kept exactly the same for each column
representing a pulsating light source shining from all directions
equally. The Earth's rotation velocity is set to zero. In the standard
RCE configuration, land--sea contrasts are removed by specifiying an
underlying mixed--layer ocean with a constant ocean albedo, but can
easily be included in idealized form for land--sea contrast
studies~(\cite{bec14x}). 
This model version has not been tested
for possible equilibria dependence on the initial boundary conditions
yet, nor for complete isotropy of variables expected from the
homogeneous boundary conditions. 

\subsubsection{Setting initial and boundary conditions and parameters
  for the RCE}

We describe the settings in the standard example script 
{\tt exp.atm\_rce\_test} here. In that case, the mixed layer ocean is
switched on by

\begin{lstlisting}[caption=Mixed layer ocean switch in {\tt
    exp.atm\_rce\_test}, label=cr20160614_lmlonml]
&echam_phy_nml
...
lmlo = .TRUE.
...
/
\end{lstlisting}

For the initialization of the RCE configuration, we use the {\tt
  nh\_testcase\_nml} namelist. 

\begin{lstlisting}[caption=Testcase namelist for RCE in {\tt
    exp.atm\_ape\_test}, label=cr20160614_testcasenml]
&nh_testcase_nml
 nh_test_name     = 'RCE_glb'
 ape_sst_case     = 'sst_const'
 ape_sst_val      = 25.
 tpe_temp         = 298.15
 tpe_psfc         = 1013.25e2
/
\end{lstlisting}

In namelist {\tt nh\_testcase\_nml} (Listing~\ref{cr20160614_testcasenml}), 
the variable {\tt nh\_test\_name} set to {\tt
  'RCE\_glb'} has the effect to initialize a global model with all
values not depending on the geographical position. The primary initialization
is done in {\tt init\_nh\_state\_rce\_glb} of module {\tt
  mo\_nh\_rce\_exp.f90}. After this primary inizialization, a random
noise is added to the variables. The variable {\tt ape\_sst\_case} is used to
initialize the ocean surface temperature by a constant throughout the
globe whereas {\tt ape\_sst\_val} gives the value of the sea surface
temperature in degrees centigrade. The variable {\tt tpe\_temp} determines
the temperature of the atmosphere in Kelvin that is independent of altitude at
the beginning. The variable {\tt tpe\_psfc} gives the surface pressure
in Pascal.

The
RCE model can be run in a configuration with the MPI/ECHAM physics only, since 
the use of the PSRAD radiation is essential for the special radiation
settings of the RCE. 
A uniform irradiation of the globe that may undergo a daily cycle is
switched on by setting
{\tt l\_sph\_symm\_irr} to {\tt .TRUE.} in namelist {\tt
  psrad\_orbit\_nml} (see
Listing~\ref{cr20160614_psrad_orbit_nml}). However, a Kepler orbit
with 
no eccentricity has to be used to warrant a sun--earth distance that
is constant in time. This is assured by the three first variables in
the namelist {\tt psrad\_orbit\_nml} as shown in
Listing~\ref{cr20160614_psrad_orbit_nml}. The obliquity is set to zero
for convenience.

\begin{lstlisting}[caption=Orbit namelist {\tt psrad\_orbit\_nml} for
  RCE in {\tt exp.atm\_rce\_test}, label=cr20160614_psrad_orbit_nml]
&psrad_orbit_nml
cecc        = 0.0
cobld       = 0.0
l_orbvsop87 = .FALSE.
l_sph_symm_irr = .TRUE.
/
\end{lstlisting} 

The angular velocity of the earth is set to zero in order to exclude
any quantity that depends on latitude as the Coriolis force. To this
end, {\tt grid\_angular\_velocity} is set to zero in namelist {\tt
  grid\_nml} (see Listing~\ref{cr20160614_grid_nml}).

\begin{lstlisting}[caption=Grid namelist {\tt grid\_nml} for RCE in {\tt
    exp.atm\_rce\_test}, label=cr20160614_grid_nml]
&grid_nml
 grid_angular_velocity = 0.
/
\end{lstlisting}

The diurnal cycle can be switched on and off with the variable {\tt
  ldiurn} of namelist {\tt radiation\_nml} (see
Listing~\ref{cr20160614_radiation_nml}).

If a circular orbit with no obliquity is used, a year with months of a
different number of 
days does not make sense. Instead, a 360 day year should be used.
The calendar must be set by two variables at the beginning of the {\tt
  exp.atm\_ape\_test} script (Listing~\ref{cr20160216_calendar}).

\begin{lstlisting}[caption=Calendar for APE ({\tt
    exp.atm\_ape\_test}), label=cr20160216_calendar]
  calendar="'360 day year'"
  calendar_type=2
\end{lstlisting}

The solar irradiation has to be set to an average value that
corresponds to the actual energy flux into the atmosphere of the real
earth. If spherically symmetric irradiation without diurnal cycle is
chosen, {\tt isolrad} has to be set to 5, if the diurnal cycle is
switched on, {\tt isolrad} has to be set to 4, respectively (see
Listing~\ref{cr20160614_radiation_nml}). These settings result in a global mean
insolation of $340.3\,{\rm W/m}^2$. However, the sum of the 14 solar
wavelength bands is higher ($433.3371\,{\rm W/m}^2$ and
$1069.315\,{\rm W/m}^2$ for ${\tt isolrad}=5,4$, respectively), due to
the applied solar zenith angles and the eventual diurnal cycle. 

The
composition of the 
atmosphere used in the RCE model is described by the corresponding
switches in namelist {\tt radiation\_nml} (see
Listing~\ref{cr20160614_radiation_nml}). The O$_3$ concentration
is read from a special file {\tt r2b4\_ozone\_rce.nc} that is
distributed with the program code in the directory {\tt
  data/external/ape\_ozone} of the ICON main directory. Prognostic
water vapour and prognostic liquid and ice cloud water is used
together with constant CO$_2$ in space and time, and no
aerosols. All 
other greenhouse gases are set to zero
(see Listing~\ref{cr20160614_radiation_nml}). 

\begin{lstlisting}[caption=Radiation namelist {\tt radiation\_nml} for
  RCE in {\tt exp.atm\_rce\_test}, label=cr20160614_radiation_nml]
&radiation_nml
 irad_h2o   = 1 ! prognostic vapor, liquid and ice
 irad_co2   = 2 ! constant co2
 irad_ch4   = 0 ! no ch4
 irad_n2o   = 0 ! no n2o
 irad_o3    = 4 ! perpetual january of ozone file linked
 irad_o2    = 0 ! no o2
 irad_cfc11 = 0 ! no cfc11
 irad_cfc12 = 0 ! no cfc12
 irad_aero  = 0 ! no aerosol
 isolrad     = 5       ! 4 if a diurnal cycle is switched on
 ldiur       = .FALSE. ! no diurnal cycle, .TRUE. for diurnal cycle
/
\end{lstlisting}


