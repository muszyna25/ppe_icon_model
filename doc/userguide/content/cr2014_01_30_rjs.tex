\section{Running the model (Test scripts for \icon)}\label{sec_cr20140130rjs_test}

\subsection{Principles of testing
  \icon}\label{sec_cr2014_01_30_rjs_principles} 

The \icon{} developers use the buildbot tool in order to perform
automated tests on selected \icon{} experiments at regular time
intervals. The buildbot tool launches the respective test scripts on
various computer platforms and documents success or failure on a special web
site ({\tt https://buildbot.zmaw.de/icon/}). The automated tests are
performed on the newest model revision as available in the \icon{}
repository. Furthermore, tests can be ``forced'', i.e. started by
hand, at any time specifying a certain experiment and model revision
of any branch of the repository. However, it is impossible to test two
revisions against each other or to test a local revision.
Here, we present new tests for \icon{} into which various experiments
are integrated and which are designed
such that they can either be used in the framework of the buildbot
tool or be started manually without reference to buildbot (e.g.~on
a PC at MPI~Hamburg). The tests are designed to trap certain technical
errors and comprise the following experiments:
\begin{description}
\item[{base test:}] Just a simple base run over a short time period
  for a specific experiment. This run will be called
  simulation~A in the following.
\item[{update test:}] In addition to the model revision to test,
  the so--called ``test
  revision'', a ``reference revision'' can be specified. A short
  simulation~A of the test revision over one hour is 
  performed during which restart files are written. The same
  simulation is performed for the reference revision (simulation
  A'). The ``update 
  test'' is said to be passed if there are no differences in the
  output of simulation A and A' using the ``cdo diff'' command on the
  time steps in the output specified by the user.
\item[{restart test:}] In addition to a base simulation~A, a second
  simulation~B restarts \icon{} at some time after the initial date. 
  The ``restart test'' is
  said to be passed if there are no differences in the output for the
  time steps after the restart between the original and the restarted
  simulation using the ``cdo diff'' command. 
\item[{nproma test:}] The nproma test performs a base simulation~A
  and a
  simulation~C with a different value
  of {\tt nproma}. Instead of {\tt nproma} of simulation~A, a value
  of 17 is used 
  in simulation~C or, if ${\tt nproma}=17$ for simulation~A, ${\tt
    nproma}=19$ is used for simulation C. The nproma test is
  said to be passed if the ``cdo diff'' command does not find
  differences in the output.
\item[{mpi parallel test:}] 
  The mpi parallel test performs a base simulation~A and a simulation~D 
  with a reduced number of
  MPI threads compared to the base simulation~A. If more
  than one threads are used on each node, the number of threads on
  each node is reduced by one. If only one thread is used on each
  node, the number of nodes is reduced by one. If only one process is
  used, no mpi parallel test is performed. The parallel test is said
  to be passed if the ``cdo diff'' 
  command does not find differences between the output files.
\item[{openmp parallel test:}]
  The openmp parallel test performs a base simulation~A and a
  simulation~E with a reduced number of
  openmp threads compared to the base simulation~A.
  If only one openmp thread was used, no openmp test is performed.
  The openmp parallel test is said to be passed if the
  ``cdo diff'' 
  command does not find differences between the output files.
\end{description}

The testing procedure is such that tests can be
combined. Furthermore, the test script can be asked to re--use
existing runs without repeating these runs.

Only the following experiments are included into the test script:
\begin{description}
\item[{\bf atm\_amip\_test:}] Non--hydrostatic AMIP--like simulation but with
  transient solar irradiance using \echam{} physics.
\item[{\bf atm\_icoles\_nested:}] Nonhydrostatic atmosphere only
  simulation with a regional 
  grid refinement.
\item[{\bf atm\_jww\_hs\_test:}] Jablonowski Williamson baroclinic
  wave test for a hydrostatic atmosphere.
\item[{\bf oce\_omip\_0160km:}] Ocean only experiment with a 160km
  resolution.
\end{description}

\subsection{Description of test script}

The test script {\tt icon-dev.checksuite} is located in the {\tt
  run/checksuite.icon-dev} 
directory of \icon{} and uses the following run script of the {\tt
  run} directory for the
experiments: ({\it i}\/) {\tt exp.atm\_amip\_test} for the
AMIP--type experiment {\tt atm\_amip\_test}, ({\it ii}\/) {\tt
  exp.atm\_icoles\_nested} for the atmosphere experiment with a grid
refinement, 
({\it iii}\/) {\tt exp.atm\_jww\_hs\_test} for the Jablonowski
Williamson baroclinic wave test, and  ({\it iv}\/) {\tt
  exp.oce\_omip\_0160km} for the ocean only experiment {\tt
  oce\_omip\_0160km}.


These run scripts contain all
necessary namelist groups and links to files that contain the initial
and boundary conditions. By the standard {\tt make\_runscripts}
command invoked inside {\tt icon-dev.checksuite}, this 
script is transformed into the actually used form that contains an
additional suffix {\tt
  .run} at the end of its name. {\bf Attention:} {\tt
  icon-dev.checksuite} generates the 
specific run script by default and overwrites those that are
present. The script can be 
forced to use present runscripts.
For the various test runs for each experiment, these {\tt
  .run} files are copied and edited by {\tt sed}.

Here follows a more detailed description of the script:

\begin{description}
\item[{\tt icon-dev.checksuite}:]\begin{sloppypar} 
  This script uses the {\tt make\_runscripts} command to produce
  {\tt exp.<{\it exp\_name}>.run} from the basic run script {\tt
    exp.<{\it exp\_name}>}. The default is that any existing run script
  is overwritten but there is an option to keep existing
  runscripts. These run scripts  
  are then modifed by {\tt 
    sed} commands in order to perform the various test runs. Once the
  test runs are finished, the function {\tt 
    diff\_results} of {\tt icon-dev.checksuite} is called to
  determine the differences 
  between those runs.
  \end{sloppypar}
\item[{\tt exp.$<$experiment$>$}:] \begin{sloppypar} 
  These scripts contain all
  settings for the base simulation in one experiment. To date, the
  experiments {\tt $<$experiment$>$ $=$ atm\_amip\_test}, 
  {\tt atm\_icoles\_nested}, {\tt atm\_jww\_hs\_test}, and  {\tt
    oce\_omip\_160km} can be used in 
  the tests. The 
  base script {\tt exp.$<$experiment$>$}
  will be 
  transformed by {\tt 
    make\_runscripts} into a script that can actually run the \icon{}
  model. The resulting {\tt exp.$<$experiment$>$.run} scripts will then
  be copied to {\tt exp.$<$experiment$>$\_base.run}, {\tt
    exp.$<$experiment$>$\_restart.run}, {\tt
    exp.$<$experiment$>$\_nproma.run}, {\tt
    exp.$<$experiment$>$\_mpi.run}, and {\tt
    exp.$<$experiment$>$\_omp.run} 
  to perform simulations~A, B, C, D, and E, described in
  section~\ref{sec_cr2014_01_30_rjs_principles}, respectively. The
  latter scripts 
  are then modified by {\tt icon-dev.checksuite} using {\tt sed}
  according to the needs of the respective runs.
  \end{sloppypar}
\item[{\tt diff\_results}.] \begin{sloppypar} This function compares two
  simulations. The five arguments contain the base path of the model ({\tt
    .../icon-dev/} for example), and the name of the experiment to be compared
  (e.g.~{\tt $<$experiment$>$\_base}) for the two experiments,
  respectively. The path of
  the models can be identical (e.g.~for the restart or nproma tests
  that are performed on the same model revision). The fifth argument
  is the name of the test ({\tt update}, {\tt restart}, {\tt nproma},
  {\tt mpi}, {\tt omp})
  and is only used to produce more legible output. However, the {\tt
    diff\_results} function needs further information that is provided
  by variables that are set in the main script: ({\it i}\/) the
  respective infix
  of the output files in variable {\tt TYPES} (e.g. {\tt atm\_phy}),
  the output dates and time in variable {\tt DATES} (e.g. {\tt
    19780101T004000Z}) as they figure on the output filenames, and the
  restart 
  date in {\tt RESTART\_DATE}. These three variables can be set as
  arguments to the options {\tt -t}, {\tt -d}, and {\tt -s} in a call
  to {\tt icon-dev.checksuite}, respectively.
  The {\tt diff\_results} function checks for differences between two
  experiments by the {\tt cdo diff} command. If the variable {\tt
    SUB\_FILES} is set to {\tt 'yes'}, e.g.~by the use of the {\tt -u}
  option in the call of {\tt icon-dev.checksuite},
  the variables of the respective outputfiles are subtracted
  from each other resulting in
  difference files \newline
{\tt
    diff\_$<$EXP2$>$\_$<$TYPE$>$\_$<$DATE$>$-$<$EXP1$>$\_$<$TYPE$>$\_$<$DATE$>$.nc}
  \newline for {\tt
    $<$TYPE$>$} in {\tt TYPES} and {\tt $<$EXP[12]$>$} one of {\tt
    $<$experiment$>$\_base}, {\tt $<$experiment$>$\_restart}, 
    {\tt $<$experiment$>$\_nproma}, {\tt $<$experiment$>$\_mpi}, 
    {\tt $<$experiment$>$\_openmp}, or
  {\tt $<$experiment$>$\_update}. 
  The difference files are written to the path of experiment {\tt
    EXP2}.\end{sloppypar} 
\end{description}

\subsection{Usage}

There are three different ways to use the ``check suite'':
\begin{itemize}
\item[({\it i}\/)] \begin{sloppypar} 
Start on the command line: The test script {\tt icon-dev.checksuite}
can be called on 
the command line from the {\tt run/checksuite.icon-dev} directory. All
the below described options are available on the command line and the
full functionality can be used via the command line options easily.
\end{sloppypar} 
\item[({\it ii}\/)] \begin{sloppypar} 
    Submit to queue: Like buildbot does, it is possible to run {\tt
    make\_runscripts} on a respective test experiment script located
  in {\tt icon\_dev/run} and to submit the resulting run script to the
  respective queuing system. E.g.~from 
  {\tt exp.test\_atm\_amip}, the runscript
  {\tt exp.test\_atm\_amip.run} is generated and can be submitted. {\tt
    exp.test\_atm\_amip} is just a link to {\tt
    checksuite.icon-dev/check.atm\_amip}. In order to use
  the full functionality of {\tt 
    icon-dev.checksuite}, various environment variables have to be set
  in {\tt exp.test\_atm\_amip}. This way of calling {\tt
    run/checksuite.icon-dev} is good for testing on computers with a
  queuing system. To date, the {\tt exp.test\_atm\_amip} and {\tt
    exp.check\_oce\_omip\_160km} are the only test scripts that are
  available. \end{sloppypar}
\item[({\it iii}\/)] Buildbot: The script calling {\tt
    icon-dev.checksuit} can be 
  used by buidbot. In this case, it is important to check that the
  correct values of all the environment variables are set in the run
  scripts mentioned in paragraph ({\it ii}\/).
\end{itemize}

The calling syntax of {\tt icon-dev.checksuite} is:

%\begin{lstlisting}
\begin{Verbatim}[frame=single]
icon-dev.checksuite [-c] [-d <dates>] [-e <experiment>]  [-f yes|no] [-h] 
                    [-m b(ase)|u(pdate)|r(estart)|n(proma)|m(pi)|o(mp)
                    |ur|un|um|uo|rn|rm|ro|nm|no|mo|urn|urm|uro|unm|uno|umo
                    |rnm|rno|rmo|nmo|urnm|urno|urmo|unmo|rnmo|urnmo]
                    [-o yes|no ] [-r <reference model path>] 
                    [-s <restart_date>] [-t <files>] [-u]
\end{Verbatim}
%\end{lstlisting}

Description of options:
\begin{description}
\item[{\tt -c}:] colour line output. Colour output should not be used
  when the script is called
  by buildbot.
\item[{\tt -d}:] dates for which outputfiles exist. The default
  depends on the experiment.
\item[{\tt -e}:] experiment on which tests have to be
  performed. Currently, the non--hydrostatic amip--like experiment
    {\tt atm\_amip\_test}, the atmospheric experiment including a
    grid refinement {\tt atm\_icoles\_0160km}, the Jablonowski
    Williamson baroclinic wave test on a hydrostatic atmosphere {\tt
      atm\_jww\_hs\_test}, and the ocean only
    experiment {\tt oce\_omip\_0160km} are supported.
\item[{\tt -f}:] The argument {\tt yes} forces to create run scripts
  even if they 
  already exist (default), {\tt no} creates run scripts only if they are not yet
  present.
\item[{\tt -h}:] display help
\item[{\tt -m}:] \begin{sloppypar} 
  describes the test mode by its arguments that are one
  of {\tt b(ase)}, {\tt u(pdate)}, 
  {\tt r(estart)}, {\tt n(proma)}, {\tt m(pi)}, {\tt o(mp)}, {\tt ur}, {\tt
    un}, {\tt um}, {\tt uo}, {\tt rn}, {\tt rm}, {\tt ro},  {\tt nm},
  {\tt no}, {\tt mo}, {\tt
    urn}, {\tt urm}, {\tt uro}, {\tt unm}, {\tt uno}, {\tt umo}, {\tt
    rnm}, {\tt rno}, {\tt rmo}, {\tt nmo}, {\tt urnm}, {\tt urno},
  {\tt urmo}, {\tt unmo}, {\tt rnmo}, {\tt urnmo}. The first five
  tests modes describe the sole base run, or the update--, 
  restart--, nproma--, mpi--, and omp--tests, respectively. The last
  26~acronyms 
  describe combined tests 
  where each single test is represented by its initial letter.
  The default test mode is {\tt rnmo}.
  \end{sloppypar}
\item[{\tt -o}:] The argument of this option can be either {\tt yes}
  or {\tt no} depending on whether existing test simulations shall be
  overwritten ({\tt -o yes}) or will be re--used for the current tests ({\tt -o
    no}). The default is {\tt -o yes}, so all existing experiments are
  automatically overwritten if not specified otherwise. 
\item[{\tt -r}:] The argument of this option gives the absolute path
  to the reference model. If the test mode includes an update test, it
  is mandatory. No default.
\item[{\tt -s}:] Restart date for the restart test as given by the
  time settings in the respective experiment. Default depends on the
  experiment.
\item[{\tt -t}:] ``Types'' (infixes) of output files that have to be
  compared. The infixes depend on the experiment and are set by
  default accordingly.
\item[{\tt -u}:] If files in the various test runs differ, calculate
  the difference by {\tt cdo sub}.
\end{description}

Corresponding to the options on the command line, the following
environment variables can be set in the {\tt exp.test\_<{\it
    experiment}>}:

\begin{description}
\item[{\tt -c}:] {\tt COLOUR='yes'|'no'}. Not recommended in use with buildbot.
\item[{\tt -d}:] {\tt DATES=$<$date\_string$>$}. 
\item[{\tt -e}:] {\tt EXPERIMENT=$<$name$>$}.
\item[{\tt -f}:] {\tt FORCE\_MRS='yes'|'no'} 
\item[{\tt -m}:] {\tt MD=$<$test\_mode$>$}
\item[{\tt -o}:] {\tt OVERWRITE='yes'|'no'}
\item[{\tt -r}:] {\tt REFERENCE=$<$reference\_model\_path$>$}
\item[{\tt -s}:] {\tt RESTART\_DATE=$<$restart\_date$>$}
\item[{\tt -t}:] {\tt TYPES=$<$file\_type\_infixes$>$}
\item[{\tt -u}:] {\tt SUB\_FILES='yes'|'no'}
\end{description}

\subsection{Examples}

\begin{Verbatim}[frame=single]
icon-dev.checksuite -o no -c -u -e oce_omip_160km
\end{Verbatim}

This command runs the {\tt rnmo}, i.e.~the restart, nproma, mpi and
openmp test on the experiment {\tt oce\_omip\_160km}. Existing runs
are not overwritten ({\tt -o no}), there is colour output ({\tt -c}),
and the difference files are calculated between the various test
experiments and the base run ({\tt -u}).

\begin{Verbatim}[frame=single]
icon-dev.checksuite -c -f no -m ur -r <path>
\end{Verbatim}

This command performs the update and restart test ({\tt -m ur}) on the {\tt
  atm\_amip\_test} experiment, colour output ist switched on ({\tt
  -c}), the reference model is given in {\tt $<$path$>$}, the
runscripts are not newly generated if they are already there ({\tt
  -f no}).
