%  
%  ICON - Fortran 2003 programming guide  
%  
%  - Luis Kornblueh, 13 Jan 2000, original version   
%  - Martin Schultz, 17 Jan 2000, English language corrections  
%  - Luis Kornblueh, 12 Apr 2000, change for introducing the DOCTOR naming   
%                                 standard and labeled do loops in the physics
%                                 package  
%  - Luis Kornblueh, 26 Jul 2000, changes recommended by Marco Giorgetta,  
%                                 added, section about parallelization and  
%                                 libraries added   
%  - Andreas Rhodin,   
%    Luis Kornblueh, 17 Aug 2000, description of parallelization  
%  - Martin Schultz, 17 Aug 2000, netCDF and parallel I/O   
%  
%  - Luis Kornblueh, 24 Mar 2004, updated for use within the ICON development
%  - Thomas Heinze,  10 May 2004, further updated for use within the ICON 
%                                 development    
%  - Thomas Heinze,  11 Aug 2004, included several minor features discussed  
%                                 at ICON meeting at MPI-M Hamburg:  
%                                 authors, Fortran 2003, date, DOCTOR style 
%  - Thomas Heinze,  21 Sep 2004, updated templates in appendix  
%                                 replaced MPIM by MPI-M   
%                                 replaced mo_kind.f90 by current version   
%                                 included mo_math_constants.f90
%  - Thomas Heinze,  28 Sep 2004, discarded IMPLICIT NONE in templates for
%                                 functions and subroutines  
%  - Thomas Heinze,  18 Jul 2005, cleaning up DOCTOR style
%                                 include latest mo_XX_constants  
%                                 include Peter Korn among authors  
%  - Thomas Heinze,  25 Jul 2005, include gnames_*.tex  
%                                 updated rules for variable names suggested
%                                 by M.Giorgetta
%  - Thomas Heinze,  26 Jul 2005, updated gnames_*.tex after DWD suggestions  
%                                 updated DOCTOR standard
%  - Thomas Heinze,  16 Aug 2005, updated global variable names
%  - Thomas Heinze,  29 Aug 2005, updated global variable names
%  - Thomas Heinze,  16 Feb 2006, clarified naming conventions for m_modules
%                                 and mo_modules
%  - Thomas Heinze,  22 Feb 2006, included updated templates
%  - Thomas Heinze,  22 Nov 2006, added latest compilers
%  - Luis Kornblueh, 05 Dec 2008, updated and proposed doxygen instead of incoprotex
%  - Luis Kornblueh, 23 Jun 2009, preprocessor defines
%  - Luis Kornblueh, 20 Jul 2009, changed form ProTeX to doxygen 
%  - Luis KOrnblueh, 21 Jul 2009, updated with change requests so far available
%  - Daniel Reinert, 21 Sep 2009, included section about OpenMP parallelization
%  - Daniel Reinert, 22 Sep 2009, changed documentation of global variable names 
%                                 (appendix)
%  - Marco Giorgetta, 14 July 2010, general update, make main text more concise (this is not
%                                 an introduction to Fortran), replace outdated Fortran files in appendix by 
%                                 references to the file in the ICON repository.
%                                 
%                                 
  
\documentclass[a4paper,11pt,DIV16,BCOR1cm,titlepage]{scrartcl}  

\usepackage[blocks]{authblk}
\usepackage{xcolor}  
\usepackage{listings}  
\usepackage{amsmath,amssymb}  
\usepackage{graphicx}  
\usepackage{tabularx} 

\definecolor{darkgreen}{RGB}{0,119,112}  
\definecolor{darkgrey}{RGB}{209,206,198} 

\pagestyle{headings}  
  
\parindent 0pt  
\parskip 1ex plus 0.2ex minus 0.2ex  

\renewcommand{\ttdefault}{pcr}

\lstnewenvironment{fortran}%
{\lstset{language=[95]Fortran,%
basicstyle=\ttfamily\footnotesize\color{darkgreen},%
commentstyle=\ttfamily\color{blue},%
emptylines=0,%
keywordstyle=\color{black}\ttfamily\bfseries,%
backgroundcolor=\color{darkgrey!5},
framexleftmargin=4mm,%
frame=shadowbox,%
rulesepcolor=\color{darkgreen}}}
{}

\lstnewenvironment{lisp}%
{\lstset{language=lisp,%
basicstyle=\ttfamily\footnotesize\color{darkgreen},%
commentstyle=\ttfamily\color{blue},%
emptylines=0,%
keywordstyle=\color{black}\ttfamily\bfseries,%
backgroundcolor=\color{darkgrey!5},
framexleftmargin=4mm,%
frame=shadowbox,%
rulesepcolor=\color{darkgreen}}}
{}

\lstnewenvironment{csh}%
{\lstset{language=csh,%
basicstyle=\ttfamily\footnotesize\color{darkgreen},%
commentstyle=\ttfamily\color{blue},%
emptylines=0,%
keywordstyle=\ttfamily\footnotesize\color{darkgreen},%
backgroundcolor=\color{darkgrey!5},
framexleftmargin=4mm,%
frame=shadowbox,%
rulesepcolor=\color{darkgreen}}}
{}

\renewcommand\Affilfont{\itshape\normalsize}

\titlehead{  
\unitlength = 1mm  
\begin{picture}(80,0)  
\put(-10,-10){ \includegraphics[scale=0.3]{../resources/img/mpilogo.pdf} }  
\put(63,-20){\makebox(0,0)[rb]{\textbf{Max-Planck-Institut f\"ur Meteorologie}}}
\put(63,-24){\makebox(0,0)[rb]{Bundesstr. 53}}
\put(63,-29){\makebox(0,0)[rb]{D-20146 Hamburg}}
\end{picture}  
\hspace{1cm}
\begin{picture}(80,0)  
\put(50,-20){ \includegraphics[scale=0.6]{../resources/img/dwdlogo.pdf} }  
\put(5,-2){\makebox(0,0)[lb]{\textbf{Deutscher Wetterdienst}}}
\put(5,-7){\makebox(0,0)[lb]{Frankfurter Str. 135}}
\put(5,-11){\makebox(0,0)[lb]{D-63067 Offenbach}}
\end{picture}  
}  
  
\title{\vspace{6cm}ICON Programming Standard}  
 
\author[1]{Luca Bonaventura}
\author[1]{Monika Esch}
\author[2]{Helmut Frank}
\author[1]{Marco Giorgetta}
\author[2]{Thomas Heinze}
\author[1]{Peter Korn}
\author[1]{Luis Kornblueh}
\author[2]{Detlev Majewski}
\author[2]{Andreas Rhodin}
\author[2]{Pilar R\'\i{}podas}
\author[2]{Bodo Ritter}
\author[2]{Daniel Reinert}
\author[1]{Uwe Schulzweida}
\affil[1]{Max-Planck-Institut f\"ur Meteorologie\\Bundesstr. 53\\D-20146 Hamburg\\Germany}   
\affil[2]{Deutscher Wetterdienst\\Frankfurter Str. 135\\D-63067 Offenbach\\Germany}  

\date{\vspace{8cm}\today}  
  
\begin{document}  

\maketitle  
  
\tableofcontents  
  
\newpage  
\part{ICON programming standard}

% Introduction
%
\section{Introduction}  
 %
Programming standards have been developed earlier to support the development
of big software packages following some technical standards and to increase readability.  
Examples are the \textit{Rules for Interchange of Physical Parametrizations} 
by Kalnay et al. (1989) or  the \textit{European Standards For  Writing and Documenting 
Exchangeable Fortran 90 Code}  by Phillip et al. (1995) . The ICON programming standard is 
inspired by such standards. It is tailored for the development of the Fortran codes of the ICON 
models, which will be used on a wide variety of computer systems, Unix or Linux related 
operating systems, and compilers.
%
The goal of this programming standard is to contribute to a stable, technically up-to-date 
and well readable code. This concerns:
 %
\begin{itemize}  
\item writing of readable/comprehensive source code   
\item modularizing of code with well structured dependencies between \texttt{MODULE}s  
\item standardizing the look and usage of \texttt{MODULE}s  
\item avoiding semantical errors  
\item reducing the maintenance cost  
\item creating machine independent source code  
\item well defined exception/error handling  
\item version control  
\item quality management and control  
\end{itemize}  

% Programming language
%
\section{Programming language}  
%
Codes shall generally be written in Fortran 2003. Excellent textbooks are given by 
Metcalf et al. (2004) and Adams et al. (2009). However, some relevant compilers do 
not comply with the full standard, so that not all Fortran 2003 features can be used  
for ICON codes. Depreciated or obsolete  features of Fortran 2003 shall not be used. 
Vendor specific extensions are not allowed. 
%
In cases, where Fortran 2003 is insufficient, e.g. for including libraries written in
C, ANSI C with POSIX extensions can be used. This type of code should be stored in 
a separate directory.

% Rules, conventions, and recommendations
%
\section{Rules, conventions, and recommendations}
%
The general objective behind a style guide is to write portable code  
that is easily readable and has a common style that can be maintained 
by a team of developers. Many rules follow common sense and should be obvious. We note,  
that many of the formatting suggestions are easily achieved if you use  
the GNU emacs (or xemacs) editor in Fortran 90 mode.  
%
The ICON coding standard comprises three sections: 
\begin{itemize}  
\item \emph{Rules} comprise style features, which can be checked or diagnosed by 
scripts or compiler warnings. Some features may even be adjusted by scripts.
\item \emph{Conventions} comprise style features, which cannot be checked by a script but are
  highly recommended to follow.
\item \emph{Recommendations} refer to style features, which are mostly helpful but not 
necessarily are applicable in all cases.
\end{itemize}  

% Rules
%
\subsection{Rules}
%
\begin{itemize}  
\item Use free format syntax.
%
\item Characters
\begin{itemize}
\item Keywords in upper case
\item Declared names (subroutines, functions, types, variables, etc.) in lower case
\item Do not use tab characters in your code: this ensures that  the code looks as intended, 
independent of local tab definitions. 
\end{itemize}
%
\item Line continuation  
\begin{itemize}
\item Lines have a maximum length of 99 characters. This is convenient for viewing and printing.
\item Continuation lines start with the ''\&'' sign.
\item Split equation so that the operator ("+", "-", ...) follows the ''\&'' sign on the following line.
\item Align the end-of-the-line ''\&'' of a continuation block.
\item Align the beginning-of-the-line ''\&'' of a continuation block.
\end{itemize}
%
\item Variables, constants , and operands
\begin{itemize}
\item Variable declarations always with "::" syntax
\item Variable declarations always without DIMENSION attribute
\item Variables and numbers of type \texttt{REAL} are declared with an explicit kind 
specifier to be used from module \texttt{mo\_kind}, see Appendix.
\item Use the working precision \texttt{wp}.
\item No implicit casting, i.e. all operands of an equation must be of the same kind.
\end{itemize}

Example for the use of  \texttt{wp}:  

\begin{fortran}  
USE mo_kind, ONLY: wp  
  
IMPLICIT NONE  
  
! Declaration of a constant  of type REAL
  
REAL(wp), PARAMETER :: a_hour = 3600._wp  ! 1 hour in seconds   
  
! Declaration of a constant  of type INTEGER

INTEGER,  PARAMETER :: two    = 2

! Declaration of a 2d field  
  
REAL(wp), POINTER   :: z_snowcover(:,:)   
  
! Declaration of a local variable  
  
REAL(wp) :: z   
  
! make all operands of the same type as the variable z
z = 4.0_wp * REAL(two,wp)
\end{fortran}  
  

\begin{itemize}
\item Names of types start with \texttt{t\_}
\item Variables used as constants should be declared with the \texttt{PARAMETER} 
attribute and used always without copying to local
  variables.  This prevents from using different values for the same
  constant.

\end{itemize}
%
\item Indentation  
\begin{itemize}
\item Comments have to be aligned with the source code.
\item Indentation by 2 blanks has to happen when scope changes.
\item The leading ''\&'' character of continuation lines is indented by 2 characters w.r.t. the very 
beginning of the continued line.  Text following the leading ''\&'' character should be vertically 
aligned across the continuation lines belonging together.
\end{itemize}
%
\item Do not use 
\begin{itemize}
\item STOP! Instead use subroutine finish of \texttt{mo\_exception} (The only exception 
is \texttt{mo\_mpi})
\item PRINT and WRITE! Instead use subroutine message of \texttt{mo\_exception}.
\end{itemize}
%
\item Intrinsics, programs, functions and subroutines
\begin{itemize}
\item Use generic intrinsic functions only
\item \texttt{USE} statements of a Fortran module are collected in a block directly following 
the \texttt{MODULE} or \texttt{PROGRAM} statement.
\item \texttt{USE} statements are always used as \texttt{USE} @module, \texttt{ONLY:}@ with 
an explicit list of the used items.
\item Each Fortran module or program contains a single \texttt{IMPLICIT NONE} statement, 
following directly the \texttt{USE} block, if existing, or otherwise following directly the \texttt
{MODULE} statement.
\item Each Fortran module contains a single \texttt{PRIVATE} or \texttt{PUBLIC} statement,
  following directly the \texttt{IMPLICIT NONE} statement. This \texttt{PRIVATE} or \texttt{PUBLIC}
  statement defines the default external accessibility of the items (parameters, variables, 
  procedures, ...) defined in this module. (It is generally safer to use \texttt{PRIVATE} as default.)
\item The default \texttt{PRIVATE} (or \texttt{PUBLIC}) statement is followed by a \texttt{PUBLIC :: ...} 
(or \texttt{PRIVATE :: ...}) statement listing all items for which the default setting does not hold.

\item Subroutines and functions must follow a \texttt{CONTAINS} statement, i.e. they must be embedded in Fortran modules.
\item Arguments of subroutines, which are not of type \texttt{POINTER}, are declared with \\
  \texttt{INTENT(in|out|inout)}
\item Arguments of functions, which are not of type \texttt{POINTER}, are declared with 
\texttt{INTENT(in)}
\end{itemize}
 
\end{itemize}

% Conventions
%
\subsection{Conventions}  
\begin{itemize}  
\item All new Fortran codes are based on the ICON Fortran template files
\item Like the templates suggest always name program units and always use the \texttt{END  
PROGRAM}; \texttt{END SUBROUTINE}; \texttt{END INTERFACE}; \texttt{END 
MODULE}; etc constructs, again specifying the name of the program  
unit. This helps finding the end of the current program  
entity. \texttt{RETURN} is obsolete and so not necessary at the end of  
program units.
\item Programming and commenting in English. Use readable and meaningful English variable names.
\item Never use a Fortran 2003 keyword as a name of a routine or variable!
\item CPP keys are only used for:
\begin{itemize}  
\item Differentiating codes in relation to computer architecture and compiler properties
\item Controlling the access to external codes (e.g. MPI and CDI library).
\end{itemize}  
\item Documentation of almost all used compiler directives can be found in the following 
manuals. The list is not complete, because NEC refuses to publish it's compiler 
documentation to the general public.
\begin{itemize}  
\item Standard pre-defined C/C++ Compiler Macros
\item  Intel compiler documentation
\item  Sun compiler documentation
\item  PGI compiler documentation
\item  NAG compiler documentation
\item  IBM compiler documentation
\item  GCC compiler documentation
\item  OpenMP specification
\end{itemize}  
\item \texttt{CHARACTER(len=*) , PARAMETER :: ' \$ module\_subroutine\_name \$ID: n/a\$ \' }, 
needs a script for the existing code and ..., use it with message and finish. Unify current usage.
\item Separate the information to be output from the formatting
  information on how to output it on I/O statements. E.g. don't put
  text inside the brackets of the I/O statement.
\end{itemize} 

 % Recommendations
 %
\subsection{Recommendations}  
%
\begin{itemize}
  
\item In general subroutines or functions should not exceed a few hundred lines and
each programming unit should begin with a header explaining the given
sections.

\item Any date follows the ISO 8601 standard. That is: YYYY-MM-DD
  HH:MM:SS. Depending on the case, time information HH:MM:SS 
  or its SS portion can be omitted. 
  
\item Use blank space, in the horizontal and vertical, to improve
  readability.  In particular try to align related code into columns.
  For example, instead of:
  
\begin{fortran}  
           ! Initialize Variables  
           i=1  
           z_meaningfulname=3.0_wp  
           z_SillyName=2.0_wp  
\end{fortran}  
  
     write:  
  
\begin{fortran}  
           ! Initialize variables  
           i                = 1  
           z_meaningfulname = 3.0_wp  
           z_silly_name     = 2.0_wp  
\end{fortran}
    
\item Try to avoid using transcendental functions (\texttt{EXP, SIN, COS, ...}). 
If possibel use tables and interpolations, or use linearized versions.
\item Try to prevent \texttt{IF}'s in a loop, instead of \texttt{IF} use, if possible, 
\texttt{SELECT CASE}, or \texttt{MERGE}
\item Avoid passing strings in the argument list (except for passing file names - 
special treatment would be fine)

\item Array notation should be used whenever possible. This should help  
optimization regardless what machine architecture is used (at least in  
theory) and will reduce the number of lines of code required. To  
improve readability the array's shape should be shown in brackets, e.g.:  
  
\begin{fortran}  
onedarraya(:)   = onedarrayb(:) + onedarrayc(:)  
  
twodarray(:, :) = scalar * anothertwodarray(:, :)  
\end{fortran}
\end{itemize}  
  
\begin{itemize} 
\item  Use of \texttt{>}, \texttt{>=}, \texttt{==}, \texttt{<}, \texttt{<=},  
\texttt{/=} instead of \texttt{.GT.}, \texttt{.GE.}, \texttt{.EQ.},  
\texttt{.LT.}, \texttt{.LE.}, \texttt{.NE.} in logical comparisons is  
recommended. The new syntax, being closer to standard mathematical  
notation, should be clearer.

\item  
We recommend against the use of recursive routines for efficiency  
reasons for computational intensive routine.

\item When an error condition occurs inside a package, a
  message describing what went wrong will be printed. The name of the
  routine in which the error occurred must be included. It is
  acceptable to terminate execution within a package, but the
  developer may instead wish to return an error flag through the
  argument list.  If the user wishes to terminate execution within the
  package, a generic ICON coupler termination routine {\tt finish}
  must be called instead of issuing a Fortran STOP. Otherwise a
  message-passing version of the model could hang.

\end{itemize}  

% Naming convention
%
\section{Naming convention}\label{sec_gvn}    
%
In order to have a readable an easily understandable code, we decided on a 
naming convention for files as well as for variables, prefixes and so on.
%
\subsection{Pre- and suffixes}
%
\begin{itemize}
%
\item Fortran files
\begin{itemize}
\item driver programs: \texttt{atm\_master.f90}, \texttt{oce\_master.f90}, \texttt{cpl\_master.f90}, ...
\item module files: \texttt{mo\_<name>.f90}
\item utility interface files: \texttt{util\_<name>.c} $=>$ \texttt{mo\_util\_<name>.f90}
\end{itemize}
%
\item include files for namelists
\begin{itemize}
\item collect in \texttt{icon-dev/include/}
\item fortran include files are named \texttt{<atm|oce|lnd|cpl>\_<name>\_ctl.inc}
\end{itemize}
%
\item functions/subroutines
\begin{itemize}
\item \texttt{init\_}...: allocating and setting time invariant variables, done once only 
at the beginning of the model run
\item \texttt{setup\_}...: set default values for namelists, read namelists, consistency checks, ...
\item \texttt{prepare\_}...: within the time loop, set time variant fields/switches
\item \texttt{clean\_}...: deallocate variables, ...
\end{itemize}
%
\item types: \texttt{t\_<name>}
%
% \item variables 
% \begin{itemize}
% \item English, concise
% % \item Variables related to the parallelization have the prefix \texttt{p\_}.
% % \item Local scratch variables: \texttt{z\_aux, i\_aux, l\_aux}, (+ suffixes for grid position, if useful)
% \end{itemize}
%
\item optional prefixes
\begin{itemize}
\item derivatives
\begin{itemize}
\item \texttt{ddt\_} : temporal dervatives
\item \texttt{ddxn\_} : horizontal derivatives in normal direction
\item \texttt{ddxt\_} : horizontal derivatives in tangential direction
\item \texttt{ddz\_} : physical space vertical derivative (height coords.)
\item \texttt{ddp\_} : physical space vertical derivative (pressure coords.)
\end{itemize}

\item fluxes
\begin{itemize}
\item \texttt{flx\_}
\end{itemize}
\end{itemize}

\item optional suffixes: order: \texttt{<process><time>\_<vertical pos.><horizontal pos.>}
\begin{itemize}
\item processes: \texttt{<name>\_rad} etc.
\item 2m: \texttt{\_2m}
\item surface: \texttt{\_sfc}
\item snow: \texttt{\_snow}
\item ice: \texttt{\_ice}
\item horizontal position
\begin{itemize}
\item \texttt{\_c}: cell center
\item \texttt{\_e}: edge
\item \texttt{\_v}: vertex
\end{itemize}

\item vertical position
\begin{itemize}
\item \texttt{\_m}: main level/mid level/full level
\item \texttt{\_i}: interface between layers/half levels
\end{itemize}

\item time position
\begin{itemize}
\item \texttt{\_old}
\item \texttt{\_now}
\item \texttt{\_new}
\end{itemize}

\item vector representations
\begin{itemize}
\item \texttt{\_o}: orthogonal
\item \texttt{\_q}: contravariant
\item \texttt{\_p}: covariant
\end{itemize}

\item There should be as few as possible suffixes (as many as necessary). If there are 
more than one suffixes needed, their ordering follows the list sequence above.

\end{itemize}

\item constants
\begin{itemize}
\item \texttt{\_dp}, \texttt{\_sp} floating point convention
\end{itemize}

\end{itemize}

The important variables with global scope are supposed to follow the   
CF-Conventions as they are adopted as well for PRISM, ESMF, and other  
international projects. The tables can be found at:  
http://www.cgd.ucar.edu/cms/eaton/cf-metadata/index.html    
     
% Automatic documentation
%
\section{Automatic documentation}  
%
Each function, subroutine, or module based on the template files (see Appendix) includes a 
prologue instrumented for use with the Doxygen documentation tool (http://www.doxygen.org). 
Doxygen can extract information on the structure of the code and compile lists of the declared
objects, as for example types, variables, routines, modules,, and can show dependencies. In 
addition Doxygen can compile texts embedded as comments in Fortran codes if marked by
appropriate formatting instructions.
%
Running \texttt{make doc} in the ICON base directory (after the Makefile has been 
generated by \texttt{./configure}) generates an HTML formatted Doxygen documentation.

% Parallelization issues
%
\section{Parallelization issues}
%
ICON models are parallelized by MPI and/or OpenMP parallelization. MPI parallelization 
means that separate model executables are running on sub-domains and communicating 
with each other following the MPI standard. Following the MPI standard each such model 
is named a process. A MPI parallelized ICON model is hence split into a number of processes 
numprocs, or for short nprocs.
%
OpenMP parallelization means that the workload of a single model executable is shared by a 
number of processors or so-called threads. If both methods are combined, the total workload is 
shared by num\_procs\_ * num\_threads ''CPUs''.
The naming convention used below is borrowed from the terms and names used in the MPI and 
OpenMP standards. For variables in Fortran codes, the convention is to use names with prefix 
''\texttt{p\_}''.\\
%
\subsection{Fine-grained parallelism using OpenMP}
%
Any OpenMP parallelization should be based on the most recent OpenMP 3.0 standard. \\
{\color{blue}
\texttt{http://www.openmp.org/mp-documents/spec30.pdf} }\\
Here are the most important rules that we would ask you to adhere to, when 
implementing OpenMP directives.
%
Due to the fact that different components coupled in an MPMD fashion may require different total 
numbers of threads, a single environment variable \texttt{OMP\_NUM\_THREADS} needs later to 
be replaced by specific environment varaibles OMP\_ATM\_THREADS and 
\texttt{OMP\_OCE\_THREADS}.

\begin{table}[htb]
\renewcommand{\arraystretch}{1.4}
\begin{center}
% use packages: array,tabularx
\begin{tabular}{|p{5cm}|p{7 cm}|p{3cm}|}
\hline
\textbf{ }  & \textbf{Environment variables in shell scripts}  & \textbf{Name in Fortran code} \\ 
\hline\hline
number of threads per process & \texttt{OMP\_NUM\_THREADS}, later: \texttt{OMP\_ATM\_THREADS} and \texttt{OMP\_OCE\_THREADS} & \texttt{p\_nthreads} \\ 
ID of a single MPI process     & n.a. & \texttt{p\_threads} \\ 
\hline
\end{tabular}
\caption{OMP of ICON.}\label{tbl_OMP}
\end{center}
\end{table}

\begin{itemize}  
  
\item All OpenMP compiler directives must start with the directive sentinel 
\texttt{!\$OMP} when using the (recommended) free format syntax.
  
\item Do not use combined parallel worksharing constructs like \texttt{!\$OMP PARALLEL DO} 
or \texttt{!\$OMP PARALLEL WORKSHARE}. Instead use the parallel construct 
(\texttt{!\$OMP PARALLEL}) followed by the desired worksharing construct (e.g.\ 
\texttt{!\$OMP DO}). This will increase readability of the code.
  
\item Explicitly include loop control variables of a parallel \texttt{DO} loop or 
a sequential loop enclosed in a parallel construct into a \texttt{PRIVATE} clause. 
Although those variables are \texttt{PRIVATE} by default, please stick to this since it 
will increase readability.
  
{\color{red}
\item Be aware of Race-Conditions!}\\
{\color{blue}
Definition: Two threads access the same shared variable \textbf{and} at least one thread 
modifies the variable \textbf{and} the accesses are concurrent, i.e.\ unsynchronized.}\\
\textbf{Prototypical example:}\\
\begin{fortran}  
a(1) = 0._wp
!$OMP PARALLEL
!$OMP DO

DO i=2, n
  a(i) = 2.0_wp * i * (i-1)
  b(i) = a(i) - a(i-1)
ENDDO

!$OMP END DO
!$OMP END PARALLEL
\end{fortran}
Note that this can lead to unexpected results. It is not ensured that $a(i-1)$ has already 
been calculated, when accessed for the calculation of $b(i)$.\\

\textbf{ICON-specific example:}\\
\begin{fortran}  
#ifdef TEST_OPENMP
!$OMP PARALLEL PRIVATE(i_startblk,i_endblk)
#endif

i_startblk = ptr_patch%cells%start_blk(rl_start,1)
i_endblk   = ptr_patch%cells%end_blk(rl_end,i_nchdom)

#ifdef TEST_OPENMP
!$OMP DO PRIVATE(...)
#endif
.
 Do some work
.

#ifdef TEST_OPENMP
!$OMP END DO
!$OMP END PARALLEL
#endif
\end{fortran}
Even in this case it is necessary to declare \texttt{i\_startblk} and \texttt{i\_endblk} as 
\texttt{PRIVATE}, although each thread writes the same result to \texttt{i\_startblk} and 
\texttt{i\_endblk}. On the IBM-machine writing to the same shared variable simultaneously 
leads to unexpected results.

{\color{red}
\item Check for strong sequential equivalence (i.e.\ bitwise identical results) of your 
parallelized code whenever possible. If a reduction operator (\texttt{reduction(operator; list)}) 
is used, strong sequential equivalence is unlikely to occur. Then, at least, check for weak 
sequential equivalence (equivalent mathematically, but due to the quirks of floating 
point arithmetic, not bitwise identical).}

\item Note: reduction operators as well as the \texttt{WORKSHARE} directive may still be 
computationally inefficient depending on the compiler. Careful runtime testing is required if 
those constructs are used.
\end{itemize}


\subsection{Parallelism using MPI}

Scripts may need to distinguish the number of processes used by different models, like the 
atmosphere and ocean models, and the coupler. Fortran variables and subroutines/functions 
related to MPI parallelization are named with a prefix ''\texttt{p\_}''.

\begin{table}[htb]
\renewcommand{\arraystretch}{1.4}
\begin{center}
% use packages: array,tabularx
\begin{tabular}{|p{5cm}|p{7 cm}|p{3cm}|}
\hline
\textbf{ }  & \textbf{Variables in shell scripts}  & \textbf{Name in Fortran code} \\ 
\hline\hline
number of threads per process & \texttt{NPROCS}, later: \texttt{ATM\_NPROCS, OCE\_NPROCS} and \texttt{CPL\_NPROCS} & \texttt{p\_nprocs} \\ 
ID of a single process     & n.a. & \texttt{p\_rank} \\ 
\hline
\end{tabular}
\caption{MPI of ICON.}\label{tbl_MPI}
\end{center}
\end{table}

\section{Literature}  
%
Adams, J.C., W.S. Brainerd, R.A. Hendrickson, R.E. Maine, J.T. Martin, and B.T. Smith,
The Fortran 2003 Handbook, Springer, 712 p., 2009.
%
Kalnay, E. et al., Rules for Interchange of Physical Parametrizations,   
Bull. A.M.S., 70 No. 6, p 620, 1989.  
%
Metcalf, M., J. Reid and M. Cohen, Fortran 95/2003 explained, 
Oxford University Press, 412 p., 2004.
%
Phillip, A., G. Cats, D. Dent, M. Gertz, and J. L. Ricard,  
European Standards For Writing and Documenting Exchangeable Fortran 90  
Code, Version 1.1,1995.\\  
\texttt{http://www.meto.gov.uk/research/nwp/numerical/fortran90/f90\_standards.html}  
 

\newpage

\part{Appendix}  

\begin{appendix}  

\section{Template Fortran files}  
%
Template Fortran files are provided to support the programing following the programming 
style described above:
\begin{itemize}
\item Main programs: \texttt{src/templates/template\_main.f90}
\item Modules: \texttt{src/templates/template\_module.f90}
\item Subroutines: \texttt{src/templates/template\_subroutine.f90}
\item Functions: \texttt{src/templates/template\_function.f90}
\end{itemize}

\section{Kind specifiers for REALs and INTEGERs}\label{app_knd}  
%
All REAL numbers and variables must be defined or declared, respectively, with a specified 
REAL kind. For integers this is advised for loop indices only. Kind specifiers are provided in: 
\begin{itemize}
\item \texttt{src/shared/mo\_kind.f90}
\end{itemize}
%
\texttt{mo\_kind} provides the following kind specifiers:  
%
\begin{center}  
\begin{tabular}{|c|c|c|}
\hline  
REAL kind	& precission			& assumed number of bits	\\
\hline
\hline   
\texttt{sp}		& 6 digits				& 32						\\
\texttt{dp}		& 12 digits			& 64						\\
\texttt{wp=dp}   & working precission		&						\\
\hline
\hline   
INTEGER kind	& exp. range			& assumed number of bits	\\
\hline
\hline   
\texttt{i4}		& 4					& 32						\\   
\texttt{i8}		& 8					& 64						\\
\hline 
\end{tabular}    
\end{center}
%
\begin{quote}  
\textbf{Note:}  
%
The bit sizes given are not mandatory. The kind value is 
defined by the precision/range, which results on current 
systems in these bit sizes. This may change in future.
\end{quote}  

\section{Constants}\label{app_cn}  
%
Universal mathematical and physical constants are defined in modules. Such constants 
can be used elsewhere by USE association, and they must not be defined locally.
\begin{itemize}
\item Mathematical constants: See file \texttt{src/shared/mo\_math\_constants.f90}
\item Physical constants: See file \texttt{src/shared/mo\_physical\_constants.f90}
\end{itemize}

\section{Editor customization}\label{sec:editor_help}
%
Some editors support customized formatting for line indentation and adjustments of keyword 
upper-case typing etc. This can be employed to facilitate programming in the style described 
above. Find below settings for Emacs and Vim. 
%
\subsection{Emacs}
Add the following lines to the ".emacs" file in your Unix home directory to support the Fortran
programming following the standard described above. 
\begin{lisp}
(custom-set-variables
  ;; custom-set-variables was added by Custom.
  ;; If you edit it by hand, you could mess it up, so be careful.
  ;; Your init file should contain only one such instance.
  ;; If there is more than one, they won't work right.
 '(f90-mode-hook (quote (f90-add-imenu-menu)))
 '(f90-auto-keyword-case (quote upcase-word))
 '(f90-comment-region "!!$")
 '(f90-directive-comment-re "![\\$hHdDcCoOI][oOpPiIdDcCB][mMfFrRiIlL]")
 '(f90-indented-comment-re "!")
 '(f90-program-indent 2)
 '(f90-type-indent 2)
 '(f90-associate-indent 2)
 '(f90-do-indent 2)
 '(f90-if-indent 2)
 '(f90-continuation-indent 2)
 '(f90-beginning-ampersand t)
 '(f90-break-before-delimiters t)
 '(f90-break-delimiters "[-+\\*/><=,% 	]")
 '(f90-smart-end (quote blink))
 '(show-paren-mode t)
 '(line-number-mode t)
 '(column-number-mode t)
 '(size-indication-mode t)
)
(custom-set-faces
  ;; custom-set-faces was added by Custom.
  ;; If you edit it by hand, you could mess it up, so be careful.
  ;; Your init file should contain only one such instance.
  ;; If there is more than one, they won't work right.
 )
(if (window-system)
    (set-frame-width (selected-frame) 99)
)
\end{lisp}
%
\subsection{Vim}
Add the following lines to the Vim setup file in your Unix home directory::
\begin{csh}
set nocompatible
filetype plugin indent on
syntax enable
set shiftwidth=2
set smarttab
set autoindent 
set expandtab
\end{csh}

\section{Preprocessor directives}
%
Preprocessor directives are most useful for distinguishing architectures 
and operating system dependent code variants. However, they are usually not meant 
for selecting code options, with the exception of debugging. Use preprocessor directives
as seldom as possible and as often as necessary!
%
The following compiler predefined preprocessor macros are available:
%
\begin{center}
\begin{tabular}{|l|c|}\hline
IBM xlf, xlc		& \texttt{\_\_xlC\_\_}				\\\hline
NEC f90			& \texttt{\_\_SX\_\_}				\\\hline
SUN f95			& \texttt{\_\_SUNPRO\_F95}		\\\hline
GCC ($\ge$ 4.3.0) & \texttt{\_\_GFORTRAN\_\_}		\\\hline
PGI pgf95			& \texttt{\_\_PGI}				\\\hline
Intel				& \texttt{\_\_INTEL\_COMPILER}	\\\hline
\end{tabular}
\end{center}
%
If you like to set groups, make that at the beginning of  a source code
file like:
%
\begin{fortran}  
#if defined (__PGI)
#define __ASYNC_GATHER 1
#define __ASYNC_GATHER_ANY 1
#endif
!
! Reshape is not that powerful implemented in some compiler
!
#if defined (__sun) || (__SX__) || defined (__PGI) || defined (__GFORTRAN__)
#define __REPLACE_RESHAPE 1
#endif
!
! Switch on explicit buffer packing and unpacking, allowing vectorization
!
#if defined (__SX__) || defined(__PGI) || (defined __xlC__)
#define __EXPLICIT 1
#endif
!
! Select communication type, if not defined NON BLOCKING is selected
!
#if defined (__PGI)
#define __SENDRECV 1
#endif
\end{fortran}  

\end{appendix}  
  
\end{document}  
  
  
