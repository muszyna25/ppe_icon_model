\section{ICON Namelists}

\subsection{Scripts, Namelist files and Programs}

Run scripts starting the programs for the grid generation and the
models are stored in run/. These scripts write namelist files containing
the specified Fortran namelists. Programs are stored in $<$ icon home$>$/build/$<$architecture$>$/bin/.


\begin{table}[htd]
\caption{Namelist files}
\begin{center}
\begin{tabular}{llll}\hline
\textbf{Namelist file} & \textbf{Purpose} & \textbf{Made by script} & \textbf{Used by program} \\\hline
NAMELIST\_GRAPH   & Generate graphs     & create\_global\_grids.run & grid\_command \\
NAMELIST\_GRID    & Generate grids      & create\_global\_grids.run & grid\_command \\
NAMELIST\_GRIDREF & Gen. nested domains & create\_global\_grids.run & grid\_command \\
NAMELIST\_ICON    & Run ICON models     & exp.<name>.run & control\_model   \\ \hline
\end{tabular}
\end{center}
\label{table:namelistfiles}
\end{table}

\newpage

\subsection{Namelist parameters}

The following subsections tabulate all available Fortran namelist
parameters by name, type, default value, unit, description, and scope:

\begin{itemize}
\item \emph{Type} refers to the type of the Fortran variable, in which the
value is stored: I=INTEGER, L=LOGICAL, R=REAL, C=character string
\item \emph{Default} is the preset value, if defined, that is assigned to
this parameter within the programs.
\item \emph{Unit} shows the unit of the control parameter, where applicable.
\item \emph{Description} explains in a few words the purpose of the parameter.
\item \emph{Scope} explains under which conditions the namelist parameter
has any effect, if its scope is restricted to specific settings of
other namelist parameters.
\end{itemize}
Information on the file, where the namelist is defined and used, is
given at the end of each table.


\section{Namelist parameters for grid generation}

\subsection{Namelist parameters defining the atmosphere grid}

%-----------------------------------------------------------------------------
% graph_ini:
%-----------------------------------------------------------------------------
\subsubsection{graph\_ini (NAMELIST\_GRAPH)}

\begin{longtab}

%\hline
\textbf{nroot}&
I&
2&
&
root subdivision of initial edges&
\tabularnewline

%\hline
\textbf{grid\_levels}&
I&
4&
&
number of edge bisections following the root subdivision&
\tabularnewline

%\hline
\textbf{lplane}&
L&
.FALSE.&
&
switch for generating a double periodic planar grid. The root level
consists of 8 triangles.&
\tabularnewline

\end{longtab}

Defined and used in: \verb+src/grid_generator/mo_io_graph.f90+

%-----------------------------------------------------------------------------
% grid_ini:
%-----------------------------------------------------------------------------
\subsubsection{grid\_ini (NAMELIST\_GRID)}

\begin{longtab}

%\hline
\textbf{nroot}&
I&
2&
&
root subdivision of initial edges&
\tabularnewline

%\hline
\textbf{grid\_levels}&
I&
4&
&
number of edge bisections following the root subdivision&
\tabularnewline

%\hline
\textbf{lplane}&
L&
.FALSE.&
&
switch for generating planar grid. The root level consists of 8 triangles.&
\tabularnewline

\textbf{lread\_graph}&
L&
.FALSE.&
&
switch for reading graph information from precomputed file; .TRUE. implies that
the graph generator needs to be executed in advance&
\tabularnewline
\end{longtab}

Defined and used in: \verb+src/grid_generator/mo_grid_levels.f90+

%-----------------------------------------------------------------------------
% grid_options:
%-----------------------------------------------------------------------------
\subsubsection{grid\_options (NAMELIST\_GRID)}

\begin{longtab}

%\hline
\textbf{x\_rot\_angle}&
R&
0.0&
deg&
Rotation of the icosahedron about the x-axis (connecting the origin
and {[}0$^\circ$E, 0$^\circ$N])&
\tabularnewline

%\hline
\textbf{y\_rot\_angle} &
R&
0.0&
deg&
Rotation of the icosahedron about the y-axis (connecting the origin
and {[}90$^\circ$E, 0$^\circ$N), done after the rotation about the x-axis.&
\tabularnewline

%\hline
\textbf{z\_rot\_angle} &
R&
0.0&
deg&
rotation of the icosahedron about the z-axis (connecting the origin
and {[}0$^\circ$E, 90$^\circ$N), done after the rotation about the y-axis.&
\tabularnewline

%\hline
\textbf{itype\_optimize}&
I&
4&
&
Grid optimization type&
\tabularnewline
&
&
&
&
0: no optimization&
\tabularnewline
&
&
&
&
1: Heikes Randall&
\tabularnewline
&
&
&
&
2: equal area&
\tabularnewline
&
&
&
&
3: c-grid small circle&
\tabularnewline
&
&
&
&
4: spring dynamics&
\tabularnewline

%\hline
\textbf{l\_c\_grid} &
L &
.FALSE. &
&
C-grid constraint on last level&
\tabularnewline

%\hline
\textbf{maxlev\_optim} &
I &
100 &
&
Maximum grid level where the optimization is applied&
i\_type\_optimize = 1 or 4
\tabularnewline

%\hline
\textbf{beta\_spring} &
R &
0.90&
&
tuning factor for target grid length &
i\_type\_optimize = 4
\tabularnewline

\end{longtab}

Defined and used in: \verb+src/grid_generator/mo_grid_levels.f90+


%-----------------------------------------------------------------------------
% plane_options:
%-----------------------------------------------------------------------------
\subsubsection{plane\_options (NAMELIST\_GRID)}

\begin{longtab}

%\hline
tria\_arc\_km &
R &
10.0 &
km&
length of triangle edge on plane&
lplane=.TRUE.
\tabularnewline

\end{longtab}

The number of grid points is generated by root level section and further
bisections. The double periodic root level consists of 8 triangles.
The spatial coordinates are $-1<=x<=1$, and $-\sqrt{3}/2<=y<=\sqrt{3}/2$.
Currently the planar option can only be used as an $f$-plane.

\noindent Defined and used in: \verb+src/grid_generator/mo_grid_levels.f90+


%-----------------------------------------------------------------------------
% gridref_ini:
%-----------------------------------------------------------------------------
\subsubsection{gridref\_ini (NAMELIST\_GRIDREF)}

\begin{longtab}

%\hline
\textbf{grid\_root}&
I&
2&
&
root subdivision of initial edges&
\tabularnewline

%\hline
\textbf{start\_lev}&
I&
4&
&
number of edge bisections following the root subdivision&
\tabularnewline

%\hline
\textbf{n\_dom}&
I&
2&
&
number of logical model domains, including the global one&
\tabularnewline

%\hline
\textbf{n\_phys\_dom}&
I&
n\_dom&
&
number of physical model domains, may be larger than n\_dom (in this case, domain merging is applied)&
\tabularnewline

%\hline
\textbf{parent\_id}&
I(n\_phys\_ dom-1)&
i&
&
ID of parent domain (first entry refers to first nested domain; needs to be
specified only in case of more than one nested domain per grid level)&
\tabularnewline

%\hline
\textbf{logical\_id}&
I(n\_phys\_ dom-1)&
i+1&
&
logical grid ID of domain (first entry refers to first nested domain; needs to be
specified only in case of domain merging, i.e. n\_dom $<$ n\_phys\_dom) &
\tabularnewline

%\hline
\textbf{l\_plot}&
L&
.FALSE.&
&
produces GMT plots showing the locations of the nested domains&
\tabularnewline

%\hline
\textbf{l\_circ}&
L&
.FALSE.&
&
Create circular (.T.) or rectangular (.F.) refined domains &
\tabularnewline

%\hline
\textbf{l\_rotate}&
L&
.FALSE.&
&
Rotates center point into the equator in case of l\_circ = .FALSE. & lcirc=.FALSE.
\tabularnewline

%\hline
write\_hierarchy&
I&
1&
&
0: Output only computational grids \\
1: Output in addition parent grid of global model domain (required for computing physics on a reduced grid) \\
2: Output all grids back to level 0 (required for hierarchical search algorithms) &
\tabularnewline

%\hline
lsep\_gridref\_info&
L&
.FALSE.&
&
.TRUE.: write fields describing parent-child connectivities into separate grid files &
\tabularnewline

%\hline
uuid\_sourcefile&
C(n\_dom)&
'EMPTY'&
&
If specified, provides the names of existing grid files from which the uuid shall be copied.
If a radiation grid is present, the first entry refers to this grid. &
\tabularnewline

%\hline
\textbf{bdy\_indexing\_depth}&
I&
12&
&
Number of cell rows along the lateral boundary of a model domain for which the refin\_ctrl
fields contain the distance from the lateral boundary; needs to be enlarged when lateral
boundary nudging is required for one-way nesting &
\tabularnewline

%\hline
\textbf{radius}&
R(n\_dom-1)&
30.&
deg&
radius of nested domain (first entry refers to first nested domain; needs to be
specified for each nested domain separately)&
lcirc=.TRUE.
\tabularnewline

%\hline
\textbf{hwidth\_lon}&
R(n\_dom-1)&
20.&
deg&
zonal half-width of refined domain (first entry refers to first nested domain; needs to be
specified for each nested domain separately)&
lcirc=.FALSE.
\tabularnewline

%\hline
\textbf{hwidth\_lat}&
R(n\_dom-1)&
20.&
deg&
meridional half-width of refined domain (first entry refers to first nested domain; needs to be
specified for each nested domain separately)&
lcirc=.FALSE.
\tabularnewline

%\hline
\textbf{center\_lon}&
R(n\_dom-1)&
30.&
deg&
center longitude of refined domain (first entry refers to first nested domain; needs to be
specified for each nested domain separately)&
\tabularnewline

%\hline
\textbf{center\_lat}&
R(n\_dom-1)&
90.&
deg&
center latitude of refined domain (first entry refers to first nested domain; needs to be
specified for each nested domain separately)&
\tabularnewline

\end{longtab}

Defined and used in: \verb+src/grid_generator/mo_gridrefinement.f90+


%-----------------------------------------------------------------------------
% gridref_metadata:
%-----------------------------------------------------------------------------
\subsubsection{gridref\_metadata (NAMELIST\_GRIDREF)}

\begin{longtab}

%\hline
number\_of\_grid\_used &
I(n\_dom+ 1)&
0&
&
sets the number of grid used in the netcdf header; the number of entries must be n\_dom+1
since the first number refers to the radiation grid &
\tabularnewline

%\hline
centre&
I&
0&
&
centre running the grid generator \\
78: EDZW (DWD)\\
252: MPIM &
\tabularnewline

%\hline
subcentre&
I&
0&
&
subcentre to be assigned by centre, usually 0 &
\tabularnewline

%\hline
outname\_style&
I&
1&
&
Output name style\\
1: Standard: \emph{iconR\textcolor{red}{X}B\textcolor{red}{XX}\_DOM\textcolor{red}{XX}.nc}\\
2: DWD: \emph{icon\_grid\_\textcolor{red}{XXXX}\_R\textcolor{red}{XX}B\textcolor{red}{XX}\_\textcolor{red}{X}.nc} &
\tabularnewline

\end{longtab}

Defined and used in: \verb+src/grid_generator/mo_gridrefinement.f90+

%------------------------------------------------------------------------------
% coupling_model_nml:
%------------------------------------------------------------------------------
\subsection{coupling\_mode\_nml}
\begin{longtab}

%\hline
coupled\_mode &
L &
.FALSE. &  &
.TRUE.: if yac coupling routines have to be called &
\tabularnewline

\end{longtab}

Defined and used in: \verb+src/namelists/mo_coupling_nml.f90+


%MMMMMMMMMMMMMMMMMMMMMMMMMMMMMMMMMMMMMMMMMMMMMMMMMMMMMMMMMMMMMMMMMMMMMMMMMMMMMM
%MMMMMMMMMMMMMMMMMMMMMMMMMMMMMMMMMMMMMMMMMMMMMMMMMMMMMMMMMMMMMMMMMMMMMMMMMMMMMM
%MMMMMMMMMMMMMMMMMMMMMMMMMMMMMMMMMMMMMMMMMMMMMMMMMMMMMMMMMMMMMMMMMMMMMMMMMMMMMM

\section{Namelist parameters defining the atmospheric model}

Namelist parameters for the ICON models are organized in several thematic
Fortran namelists controling the experiment, and the properties of
dynamics, transport, physics etc.

%-----------------------------------------------------------------------------
% diffusion_nml: horizontal (numerical) diffusion
%-----------------------------------------------------------------------------
\subsection{diffusion\_nml}
\begin{longtab}

%\hline
\textbf{lhdiff\_temp}&
L& .TRUE. & &
Diffusion on the temperature field&
\tabularnewline

%\hline
\textbf{lhdiff\_vn}&
L& .TRUE. & &
Diffusion on the horizontal wind field&
\tabularnewline

%\hline
\textbf{lhdiff\_w}&
L& .TRUE. & &
Diffusion on the vertical wind field&
\tabularnewline

%\hline
hdiff\_order&
I&{4~(hydro)} \\ {5 (NH)}& &
Order of $\nabla$ operator for diffusion:\\
-1: no diffusion\\
2: $\nabla^{2}$ diffusion\\
3: Smagorinsky $\nabla^{2}$ diffusion \\
4: $\nabla^{4}$ diffusion \\
5: Smagorinsky $\nabla^{2}$ diffusion combined with $\nabla^{4}$
background diffusion as specified via hdiff\_efdt\_ratio \\
24 or 42: $\nabla{2}$ diffusion from model top to a certain level
(cf. k2\_pres\_max and k2\_klev\_max below);
$\nabla^{4}$ for the lower levels.  &
Options 2, 24 and 42 are allowed only in the hydrostatic atm model
(iequations = 1 or 2 in dynamics\_nml).
\tabularnewline

%\hline
lsmag\_3d&
L& .FALSE. & &
.TRUE.: Use 3D Smagorinsky formulation for computing the horizontal diffusion coefficient (recommended at mesh sizes
finer than 1 km if the LES turbulence scheme is not used)& hdiff\_order=3 or 5; itype\_vn\_diffu=1
\tabularnewline

%\hline
itype\_vn\_diffu&
I& 1 & &
Reconstruction method used for Smagorinsky diffusion: \\
1: u/v reconstruction at vertices only \\
2: u/v reconstruction at cells and vertices& iequations=3, hdiff\_order=3 or 5
\tabularnewline

%\hline
itype\_t\_diffu&
I& 2 & &
Discretization of temperature diffusion: \\
1: $K_h \nabla^2 T$ \\
2: $\nabla \cdot (K_h \nabla T)$  & iequations=3, hdiff\_order=3 or 5
\tabularnewline

%\hline
k2\_pres\_max&
R& -99.& Pa &
Pressure level above which $\nabla^2$ diffusion is applied.&
hdiff\_order = 24 or 42, and dynamics\_nml:iequations = 1 or 2.
\tabularnewline

%\hline
k2\_klev\_max&
I& 0&&
Index of the vertical level till which (from the model top)
$\nabla^2$ diffusion is applied.
If a positive value is specified for k2\_pres\_max,
k2\_klev\_max is reset accordingly during the initialization
of a model run.&
hdiff\_order = 24 or 42, and dynamics\_nml:iequations = 1 or 2.
\tabularnewline

%\hline
hdiff\_efdt\_ratio&
R& 1.0 (hydro) \\ 36.0 (NH) & &
ratio of e-folding time to time step (or 2{*} time step when using
a 3 time level time stepping scheme) (for triangular NH model, values above 30 are recommended when using hdiff\_order=5) &
\tabularnewline

%\hline
hdiff\_w\_efdt\_ratio&
R& 15.0  & &
ratio of e-folding time to time step for diffusion on vertical wind speed & iequations=3
\tabularnewline

%\hline
hdiff\_min\_efdt\_ratio&
R& 1.0 & &
minimum value of hdiff\_efdt\_ratio near model top & iequations=3  .AND. hdiff\_order=4
\tabularnewline

%\hline
hdiff\_tv\_ratio&
R& 1.0& &
Ratio of diffusion coefficients for temperature and normal wind: $T:v_{n}$&
\tabularnewline

%\hline
hdiff\_multfac&
R& 1.0& &
Multiplication factor of normalized diffusion coefficient for nested
domains&
n\_dom$>$1\tabularnewline

%\hline
\textbf{hdiff\_smag\_fac}&
R& 0.15 (hydro) \\ 0.015 (NH)& &
Scaling factor for Smagorinsky diffusion&
iequations=3
\tabularnewline

\end{longtab}

Defined and used in: \verb+src/namelists/mo_diffusion_nml.f90+


%-----------------------------------------------------------------------------
% dynamics_nml:
%-----------------------------------------------------------------------------
\subsection{dynamics\_nml}
This namelist is relevant if run\_nml:ldynamics=.TRUE.

\begin{longtab}

%\hline
\textbf{iequations}&
I& 3& &
Equations and prognostic variables. Use positive indices for the atmosphere
and negative indices for the ocean.&\tabularnewline
&&&&0: shallow water model&\tabularnewline
&&&&1: hydrostatic atmosphere, T&\tabularnewline
&&&&2: hydrostatic atm., $\theta \cdot$dp&\tabularnewline
&&&&3: non-hydrostatic atmosphere&\tabularnewline
&&&&-1: hydrostatic ocean&
\tabularnewline


%\hline
idiv\_method &
I& 1& &
Method for divergence computation:&
\tabularnewline
& & & & 1: Standard Gaussian integral. \\
Hydrostatic atm.~model: for unaveraged normal components\\
Non-hydrostatic atm.~model: for averaged normal components &
\tabularnewline
& & & & 2: bilinear averaging of divergence& \tabularnewline

%\hline
divavg\_cntrwgt&
R& 0.5& &
Weight of central cell for divergence averaging&
idiv\_method= 2
\tabularnewline

%\hline
lcoriolis &
L & .TRUE.& &
Coriolis force&
\tabularnewline

%\hline
sw\_ref\_height &
R &  0.9* 2.94e4/g & m &
Reference height of shallow water model used for
linearization in the semi-implicit time stepping scheme&
\tabularnewline

\end{longtab}

Defined and used in: \verb+src/namelists/mo_dynamics_nml.f90+


%------------------------------ switches for ECHAM physics


%------------------------------------------------------------------------------
% echam_conv_nml:
%------------------------------------------------------------------------------
\subsection{echam\_conv\_nml}

\begin{longtab}

%\hline
\textbf{iconv}&
I & 1 &&
Choice of cumulus convection scheme.\par
1: Nordeng scheme\par
2: Tiedtke scheme\par
3: hybrid scheme &
iforcing = 2 .AND. lconv = .TRUE.
\tabularnewline

%\hline
ncvmicro&
I & 0 &&
Choice of convective microphysics scheme.&

iforcing = 2 .AND. lconv = .TRUE.
\tabularnewline

%\hline
lmfpen &
L& .TRUE.&&
Switch on penetrative convection.&
iforcing = 2 .AND. lconv = .TRUE.
\tabularnewline

%\hline
lmfmid &
L& .TRUE.&&
Switch on midlevel convection.&
iforcing = 2 .AND. lconv = .TRUE.
\tabularnewline

%\hline
lmfdd &
L& .TRUE.&&
Switch on cumulus downdraft.&
iforcing = 2 .AND. lconv = .TRUE.
\tabularnewline

%\hline
lmfdudv &
L& .TRUE.&&
Switch on cumulus friction.&
iforcing = 2 .AND. lconv = .TRUE.
\tabularnewline

%\hline
cmftau &
R & 10800. &&
Characteristic convective adjustment time scale.&
iforcing = 2 .AND. lconv = .TRUE.
\tabularnewline

%\hline
cmfctop &
R & 0.3 &&
Fractional convective mass flux (valid range [0,1])
across the top of cloud &
iforcing = 2 .AND. lconv = .TRUE.
\tabularnewline

%\hline
cprcon &
R & 1.0e-4 &&
Coefficient for determining conversion from cloud water to rain.&
iforcing = 2 .AND. lconv = .TRUE.
\tabularnewline

%\hline
cminbuoy &
R & 0.025 &&
Minimum excess buoyancy.&
iforcing = 2 .AND. lconv = .TRUE.
\tabularnewline

%\hline
entrpen &
R & 1.0e-4 &&
Entrainment rate for penetrative convection.&
iforcing = 2 .AND. lconv = .TRUE.
\tabularnewline

%\hline
dlev &
R & 3.e4 & Pa &
Critical thickness necessary for the onset of convective precipitation.&
iforcing = 2 .AND. lconv = .TRUE.
\tabularnewline

\end{longtab}

Defined and used in: \verb+src/namelists/mo_echam_conv_nml.f90+


\newpage

%------------------------------------------------------------------------------
% echam_phy_nml:
%------------------------------------------------------------------------------
\subsection{echam\_phy\_nml}

\begin{longtab}

\textbf{lrad} &
L& .TRUE.&&
.TRUE. for radiation.&
run\_nml/iforcing = 2
\tabularnewline

%\hline
\textbf{dt\_rad} &
R&
3600.&
s&
time interval for radiative transfer computation&
run\_nml/iforcing = 2
\tabularnewline

%\hline
\textbf{lvdiff} &
L& .TRUE.&&
.TRUE. for vertical turbulent diffusion&
run\_nml/iforcing = 2
\tabularnewline

%\hline
\textbf{lconv} &
L& .TRUE.&&
.TRUE. for cumulus convection&
run\_nml/iforcing = 2
\tabularnewline

%\hline
\textbf{lcond} &
L& .TRUE.&&
.TRUE. for large scale condensation&
run\_nml/iforcing = 2
\tabularnewline

%\hline
\textbf{lgw\_hines} &
L& .TRUE.&&
.TRUE. for non-orographic gravity wave drag (Hines)&
run\_nml/iforcing = 2
\tabularnewline

%\hline
\textbf{lssodrag} &
L& .TRUE.&&
.TRUE. for subgrid scale orographic effects (Lott and Miller)&
run\_nml/iforcing = 2
\tabularnewline

%\hline
\textbf{lice} &
L& .FALSE.&&
.TRUE. for sea-ice temperature calculation&
run\_nml/iforcing = 2
\tabularnewline

%\hline
\textbf{lmlo} &
L& .FALSE.&&
.TRUE. for mixed layer ocean&
run\_nml/iforcing = 2
\tabularnewline

%\hline
\textbf{ljsbach} &
L& .FALSE.&&
.TRUE. for calculating land surface properties (JSBACH)&
run\_nml/iforcing = 2
\tabularnewline

%\hline
\textbf{lamip} &
L& .FALSE.&&
.TRUE. for AMIP boundary conditions&
run\_nml/iforcing = 2
\tabularnewline

%\hline
\end{longtab}

Defined and used in: \verb+src/namelists/mo_echam_phy_nml.f90+


%------------------------------------------------------------------------------
% ensemble_pert_nml:
%------------------------------------------------------------------------------
\subsection{ensemble\_pert\_nml}


\begin{longtab}


\hline
use\_ensemble\_pert&
L&
.FALSE.&
&
Main switch to activate physics parameter perturbations for ensemble forecasts / ensemble data assimilation; 
the perturbations are applied via random numbers depending on the perturbationNumber 
(ensemble member ID) specified in gribout\_nml    &
run\_nml:iforcing = inwp
\tabularnewline


\hline
range\_gkwake&
R&
0.333&
&
Variability range for low level wake drag constant&
\tabularnewline

\hline
range\_gkdrag&
R&
0.04&
&
Variability range for orographic gravity wave drag constant&
\tabularnewline


\hline
range\_gfluxlaun&
R&
0.75e-3&
&
Variability range for non-orographic gravity wave launch momentum flux &
\tabularnewline


\hline
range\_zvz0i&
R&
0.2&
m/s&
Variability range for terminal fall velocity of ice&
inwp\_gscp = 1 or 2
\tabularnewline


\hline
range\_entrorg&
R&
0.2e-3&
1/m&
Variability range for entrainment parameter in convection scheme & 
inwp\_convection = 1
\tabularnewline


\hline
range\_capdcfac\_et&
R&
0.75&
&
Maximum fraction of CAPE diurnal cycle correction applied in the extratropics &
icapdcycl = 3
\tabularnewline

\hline
range\_rhebc&
R&
0.05&
&
Variability range for RH threshold for the onset of evaporation below cloud base & 
inwp\_convection = 1
\tabularnewline

\hline
range\_texc&
R&
0.05&
K&
Variability range for temperature excess value in test parcel ascent & 
inwp\_convection = 1
\tabularnewline

\hline
range\_box\_liq&
R&
0.01&
&
Variability range for box width scale of liquid clouds in cloud cover scheme &
inwp\_cldcover = 1
\tabularnewline


\hline
range\_tkhmin&
R&
0.2&
&
Variability range for minimum vertical diffusion for heat/moisture &
inwp\_turb = 1
\tabularnewline

\hline
range\_tkmmin&
R&
0.2&
&
Variability range for minimum vertical diffusion for momentum &
inwp\_turb = 1
\tabularnewline

\hline
range\_rlam\_heat&
R&
3.0&
&
Variability range (multiplicative!) of laminar transport resistance parameter &
inwp\_turb = 1
\tabularnewline

\hline
range\_charnock&
R&
1.5&
&
Variability range (multiplicative!) of upper and lower bound of wind-speed dependent Charnock parameter &
inwp\_turb = 1
\tabularnewline

\hline
range\_minsnowfrac&
R&
0.05&
&
Variability range for minimum value to which snow cover fraction is artificially reduced in case of melting snow & 
idiag\_snowfrac = 20/30/40
\tabularnewline

\hline
range\_z0\_lcc&
R&
0.25&
&
Variability range (relative change) of roughness length attributed to each landuse class &
\tabularnewline

\hline
range\_rootdp&
R&
0.2&
&
Variability range (relative change) of root depth attributed to each landuse class &
\tabularnewline

\hline
range\_rsmin&
R&
0.2&
&
Variability range (relative change) of minimum stomata resistance attributed to each landuse class &
\tabularnewline

\hline
range\_laimax&
R&
0.15&
&
Variability range (relative change) of leaf area index (maximum of annual cycle) attributed to each landuse class &
\tabularnewline


\end{longtab}


Defined and used in: \verb+src/namelists/mo_ensemble_pert_nml.f90+

%------------------------------------------------------------------------------
% gribout_nml:
%------------------------------------------------------------------------------
\subsection{gribout\_nml}
\begin{longtab}

%\hline
preset&
C& ''determ\dots'' & &
Setting this different to ''none'' enables a couple of defaults for
the other \texttt{gribout\_nml} namelist parameters. If, additionally, the
user tries to set any of these other parameters to a conflicting
value, an error message is thrown. 
Possible values are ''none'', ''deterministic'', ''ensemble''.
&
filetype=2
\tabularnewline

%\hline
backgroundProcess&
I& 0 & &
Background process \\
- GRIB2 code table backgroundProcess.table &
filetype=2
\tabularnewline

%\hline
generatingCenter&
I& -1 & &
Output generating center. If this key is not set, center information is taken from the grid file\\
DWD: 78 \\
MPIMET: 98 \\
ECMWF: 98 &
filetype=2
\tabularnewline

%\hline
generatingSubcenter&
I& -1 & &
Output generating Subcenter. If this key is not set, subcenter information is taken from the grid file\\
DWD: 255\\
MPIMET: 232\\
ECMWF: 0 &
filetype=2
\tabularnewline

%\hline
generatingProcess\par Identifier&
I(n\_dom)& 1 & &
generating Process Identifier \\
- GRIB2 code table generatingProcessIdentifier.table &
filetype=2
\tabularnewline

%\hline
numberOfForecastsIn- Ensemble&
I& -1 & &
Local definiton for ensemble products,
(only set if value changed from default) &
filetype=2
\tabularnewline

%\hline
perturbationNumber&
I& -1 & &
Local definiton for ensemble products,
(only set if value changed from default) &
filetype=2
\tabularnewline

%\hline
productionStatusOfPro-\par cessedData&
I& 1 & &
Production status of data\\
- GRIB2 code table 1.3 &
filetype=2
\tabularnewline

%\hline
significanceOfReference- Time &
I& 1& &
Significance of reference time\\
- GRIB2 code table 1.2 &
filetype=2
\tabularnewline

%\hline
typeOfEnsembleForecast&
I& -1 & &
Local definiton for ensemble products
(only set if value changed from default) &
filetype=2
\tabularnewline

%\hline
typeOfGeneratingPro- cess&
I& -1 & &
Type of generating process \\
- GRIB2 code table 4.3 &
filetype=2
\tabularnewline

%\hline
typeOfProcessedData&
I& -1 & &
Type of data \\
- GRIB2 code table 1.4 &
filetype=2
\tabularnewline

%\hline
localDefinitionNumber&
I& -1 & &
local Definition Number\\
- GRIB2 code table grib2LocalSectionNumber.78.table &
filetype=2
\tabularnewline

%\hline
localNumberOfExperi- ment&
I& 1 & &
local Number of Experiment &
filetype=2
\tabularnewline

%\hline
localTypeOfEnsemble-\par Forecast&
I& -1 & &
Local definiton for ensemble products
(only set if value changed from default) &
filetype=2
\tabularnewline

lspecialdate\_invar&
L& .FALSE. &&
Special reference date for invariant and climatological fields\\
.TRUE.: set special reference date 0001-01-01, 00:00\\
.FASLE.: no special reference date &
filetype = 2
\tabularnewline

%\hline
ldate\_grib\_act&
L& .TRUE. & &
GRIB creation date\\
.TRUE.: add creation date\\
.FALSE.: add dummy date &
filetype=2
\tabularnewline

%\hline
lgribout\_24bit&
L& .FALSE. & &
If TRUE, write thermodynamic fields $\rho$, $\theta_{v}$, $T$, $p$ with 24bit precision instead of 16bit &
filetype=2
\tabularnewline

%\hline
\end{longtab}

Defined and used in: \verb+src/namelists/mo_gribout_nml.f90+


%-------------------------------------------------------------------
% grid_nml: horizontal grid
%-------------------------------------------------------------------
\subsection{grid\_nml}
\begin{longtab}

%\hline
cell\_type&
I & 3& &
Cell type: not used &
\tabularnewline

%\hline
lplane &
L & .FALSE.& &
planar option&
\tabularnewline

%\hline
is\_plane\_torus &
L & .FALSE.& &
f-plane approximation on triangular grid &
\tabularnewline

%\hline
corio\_lat &
R & 0.0& deg&
Center of the f-plane is located at this geographical latitude &
lplane=.TRUE. and is\_plane\_torus=.TRUE.
\tabularnewline

%\hline
grid\_angular \_velocity &
R & Earth's & rad/s &
The angular velocity in rad per sec. &
\tabularnewline

%%\hline
%start\_lev &
%I & 4& &
%coarsest bisection level&
%\tabularnewline

%\hline
\textbf{l\_limited\_area} &
L & .FALSE.& & &
\tabularnewline

%\hline
grid\_rescale\_factor &
R & 1.0   &  &
The geometry and the timestep will be multiplied by this factor.\\
The angular velocity will be divided by this factor.
&
\tabularnewline

%\hline
\textbf{lfeedback} &
L(n\_dom) & .TRUE.& &
Specifies if feedback to parent grid is performed. Setting lfeedback(1)=.false. turns off feedback
for all nested domains; to turn off feedback for selected nested domains, set lfeedback(1)=.true.
and set ``.false." for the desired model domains&
n\_dom$>$1
\tabularnewline

%\hline
ifeedback\_type &
I & 2& &
1: incremental feedback \\ 2: relaxation-based feedback \\
Note: vertical nesting requires option 2 to run numerically stable over longer time periods & n\_dom$>$1
\tabularnewline

%\hline
start\_time &
R(n\_dom) & 0.   & s &
Time when a nested domain starts to be active (namelist entry is ignored for the global domain)
& n\_dom$>$1
\tabularnewline

%\hline
end\_time &
R(n\_dom) & 1.E30  & s &
Time when a nested domain terminates (namelist entry is ignored for the global domain)
& n\_dom$>$1
\tabularnewline

%\hline
patch\_weight &
R(n\_dom) & 0.& &
If patch\_weight is set to a value $>$ 0 for any of the first level child patches,
processor splitting will be performed, i.e. every of the first level child patches
gets a subset of the total number or processors corresponding to its patch\_weight.
A value of 0. corresponds to exactly 1 processor for this patch, regardless of
the total number of processors. For the root patch and higher level childs,
patch\_weight is not used. However, patch\_weight must be set to 0 for these patches
to avoid confusion.&
n\_dom$>$1
\tabularnewline

%\hline
lredgrid\_phys &
L & .FALSE.& &
If set to .true. radiation is calculated on a reduced grid (= one grid level higher) &
\tabularnewline

%\hline
\textbf{dynamics\_grid\_ filename} &
C & & &
Array of the grid filenames to be used by the dycore.
May contain the keyword \texttt{<path>} which will be substituted by
\texttt{model\_base\_dir}. &
\tabularnewline

%\hline
\textbf{dynamics\_parent\_ grid\_id} &
I(n\_dom) & $i-1$& &
Array of the indexes of the parent grid filenames, as described by the dynamics\_grid\_filename array.
Indexes start at 1, an index of 0 indicates no parent. &
\tabularnewline

%\hline
\textbf{radiation\_grid\_ filename} &
C & & &
Array of the grid filenames to be used for the radiation model.
Filled only if the radiation grid is different from the dycore grid.
May contain the keyword \texttt{<path>} which will be substituted by
\texttt{model\_base\_dir}.
& lredgrid\_phys=.TRUE.
\tabularnewline

%\hline
\textbf{dynamics\_radiation\_g rid\_link} &
I(n\_dom) & 1 for i=1 & &
Array of the indexes linking the dycore grids, as described by the dynamics\_grid\_filename array,
and the radiation\_grid\_filename array. It provides the link index of the radiation\_grid\_filename,
for each entry of the dynamics\_grid\_filename array.
Indexes start at 1, an index of 0 indicates that the radiation grid is the same as the dycore grid.
Only needs to be filled when the radiation\_grid\_filename is defined. &
\tabularnewline

%\hline
create\_vgrid &
L & .FALSE. & &
.TRUE.: Write vertical grid files containing (\texttt{vct\_a}, \texttt{vct\_b}, \texttt{z\_ifc}, and \texttt{z\_ifv}. &
\tabularnewline

%\hline
vertical\_grid\_filename &
C(n\_dom) & & &
Array of filenames. These files contain the vertical grid definition (\texttt{vct\_a}, \texttt{vct\_b}, \texttt{z\_ifc}). 
If empty, the vertical grid is created within ICON during the setup phase.&
\tabularnewline

%\hline
use\_duplicated\_\par connectivity &
L & .TRUE. & &
if .TRUE., the zero connectivity is replaced by the last non-zero value&
\tabularnewline

%\hline
use\_dummy\_cell\_closure &
L & .FALSE. & &
if .TRUE. then create a dummy cell and connect it to cells and edges with no neighbor&
\tabularnewline

\end{longtab}

Defined and used in: \verb+src/namelists/mo_grid_nml.f90+



%-------------------------------------------------------------------
% gridref_nml: grid refinement and nesting
%-------------------------------------------------------------------
\subsection{gridref\_nml}
\begin{longtab}

%\hline
grf\_intmethod\_c&
I& 2& &
Interpolation method for grid refinement (cell-based dynamical variables):&
n\_dom$>$1\tabularnewline
& & & & 1: parent-to-child copying & \tabularnewline
& & & & 2: gradient-based interpolation & \tabularnewline

%\hline
grf\_intmethod\_ct&
I& 2& &
Interpolation method for grid refinement (cell-based tracer variables):&
n\_dom$>$1\tabularnewline
& & & & 1: parent-to-child copying & \tabularnewline
& & & & 2: gradient-based interpolation & \tabularnewline

%\hline
grf\_intmethod\_e&
I& 6& &
Interpolation method for grid refinement (edge-based variables):&
n\_dom$>$1\tabularnewline
& & & & 1: inverse-distance weighting (IDW) & \tabularnewline
& & & & 2: RBF interpolation & \tabularnewline
& & & & 3: combination gradient-based / IDW & \tabularnewline
& & & & 4: combination gradient-based / RBF & \tabularnewline
& & & & 5/6: same as 3/4, respectively, but direct interpolation of mass fluxes along nest interface edges & \tabularnewline

%\hline
grf\_velfbk&
I& 1& & Method of velocity feedback:&
n\_dom$>$1\tabularnewline
& & & & 1: average of child edges 1 and 2 & \tabularnewline
& & & & 2: 2nd-order method using RBF interpolation & \tabularnewline

%\hline
grf\_scalfbk&
I& 2& & Feedback method for dynamical scalar variables ($T, p_{sfc}$):&
n\_dom$>$1\tabularnewline
& & & & 1: area-weighted averaging & \tabularnewline
& & & & 2: bilinear interpolation & \tabularnewline

%\hline
grf\_tracfbk&
I& 2& & Feedback method for tracer variables:&
n\_dom$>$1\tabularnewline
& & & & 1: area-weighted averaging & \tabularnewline
& & & & 2: bilinear interpolation & \tabularnewline

%\hline
grf\_idw\_exp\_e12&
R& 1.2& &
exponent of generalized IDW function for child edges 1/2 &
n\_dom$>$1\tabularnewline

%\hline
grf\_idw\_exp\_e34&
R& 1.7& &
exponent of generalized IDW function for child edges 3/4 &
n\_dom$>$1\tabularnewline

%\hline
rbf\_vec\_kern\_grf\_e&
I& 1& &
RBF kernel for grid refinement (edges):&
n\_dom$>$1\tabularnewline
& & & & 1: Gaussian & \tabularnewline
& & & & 2: $1/(1+r^{2})$ & \tabularnewline
& & & & 3: inverse multiquadric & \tabularnewline

%\hline
rbf\_scale\_grf\_e&
R(n\_dom)& 0.5& &
RBF scale factor for grid refinement (lateral boundary interpolation to edges). Refers to the
respective parent domain and thus does not need to be specified for the innermost nest. Lower values
than the default of 0.5 are needed for child mesh sizes less than about 500 m.&
n\_dom$>$1\tabularnewline

%\hline
denom\_diffu\_t&
R& 135& &
Deniminator for lateral boundary diffusion of temperature&
n\_dom$>$1\tabularnewline

%\hline
denom\_diffu\_v&
R& 200& &
Deniminator for lateral boundary diffusion of velocity&
n\_dom$>$1\tabularnewline

%\hline
l\_mass\_consvcorr&
L& .FALSE.& &
.TRUE.: Apply mass conservation correction in feedback routine &
n\_dom$>$1\tabularnewline

%\hline
l\_density\_nudging &
L& .FALSE.& &
.TRUE.: Apply density nudging near lateral nest boundary if grf\_intmethod\_e $\le$ 4&
n\_dom$>$1 .AND. lfeedback = .TRUE. \tabularnewline

\hline
fbk\_relax\_timescale &
R & 10800& &
Relaxation time scale for feedback &
n\_dom>1 .AND. lfeedback = .TRUE. .AND. ifeedback\_type = 2 \tabularnewline

\end{longtab}

Defined and used in: \verb+src/namelists/mo_gridref_nml.f90+


%------------------------------------------------------------------------------
% gw_hines_nml:
%------------------------------------------------------------------------------
\subsection{gw\_hines\_nml (Scope: lgw\_hines = .TRUE. in echam\_phy\_nml)}

\begin{longtab}

%\hline
lheatcal    &
L           &
.FALSE.     &&
.TRUE.: compute drag, heating rate and diffusion coefficient from the dissipation of gravity waves&
\tabularnewline
&&&&
.FALSE.: compute drag only &
\tabularnewline

%\hline
emiss\_lev  &
I           &
10          &&
Index of model level, counted from the surface, from which the gravity wave spectra are emitted &
\tabularnewline

%\hline
rmscon      &
R           &
1.0         &
m/s         &
Root mean square gravity wave wind at the emission level &
\tabularnewline

%\hline
kstar       &
R           &
5.0e-5      &
1/m         &
Typical gravity wave horizontal wavenumber &
\tabularnewline

%\hline
m\_min      &
R           &
0.0         &
1/m         &
Minimum bound in  vertical wavenumber &
\tabularnewline

%\hline
lrmscon\_lat &
L            &
.FALSE.      &
             &
.TRUE.:  use latitude dependent rms wind &
\tabularnewline
&&&& - |latitude| $>$= lat\_rmscon: use rmscon &
\tabularnewline
&&&& - |latitude| $<$= lat\_rmscon\_eq: use rmscon\_eq &
\tabularnewline
&&&& - lat\_rmscon\_eq $<$ |latitude| $<$ lat\_rmscon: use linear interpolation between rmscon\_eq and rmscon &
\tabularnewline
&&&& .FALSE.: use globally constant rms wind rmscon &
\tabularnewline

%\hline
lat\_rmscon\_eq &
R               &
5.0             &
deg N           &
rmscon\_eq is used equatorward of this latitude &
lrmscon\_lat = .TRUE.
\tabularnewline

%\hline
lat\_rmscon     &
R               &
10.0            &
deg N           &
rmscon is used polward of this latitude &
lrmscon\_lat = .TRUE.
\tabularnewline

%\hline
rmscon\_eq      &
R               &
1.2             &
m/s             &
is used equatorward of latitude lat\_rmscon\_eq &
lrmscon\_lat = .TRUE.
\tabularnewline

\end{longtab}

Defined and used in: \verb+src/namelists/mo_gw_hines_nml.f90+


%-----------------------------------------------------------------------------
% ha_dyn_nml: hydrostatic atm dynamics
%-----------------------------------------------------------------------------
\subsection{ha\_dyn\_nml}

This namelist is relevant if
run\_nml:ldynamics=.TRUE.
and dynamics\_nml:iequations=IHS\_ATM\_TEMP or IHS\_ATM\_THETA.

\begin{longtab}

%\hline
\textbf{itime\_scheme}&
I& 14& &
Time integration scheme:& \tabularnewline
& & & & 11: pure advection (no dynamics)& \tabularnewline
& & & & 12: 2 time level semi implicit (not yet implemented)&
\tabularnewline
& & & & 13: 3 time level explicit&
\tabularnewline
& & & & 14: 3 time level with semi implicit correction&
\tabularnewline
& & & & 15: standard 4th-order Runge-Kutta method (4-stage)&
\tabularnewline
& & & & 16: SSPRK(5,4) scheme (5-stage)&
\tabularnewline

%\hline
ileapfrog\_startup&
I& 1& &
How to integrate the first time step when the leapfrog scheme
is chosen. 1 = Euler forward; 2 = a series of sub-steps. &
itime\_scheme= 13 or 14 \tabularnewline

%\hline
asselin\_coeff&
R& 0.1& &
Asselin filter coefficient&
itime\_scheme= 13 or 14 \tabularnewline

%\hline
si\_2tls &
R & 0.6 & &
weight of time step n+1. Valid range: [0,1] &
itime\_scheme=12\tabularnewline

%\hline
si\_expl\_scheme &
I & 2 & &
scheme for the explicit part used in the 2 time level
semi-implicit time stepping scheme. 1 = Euler forward;
2 = Adams-Bashforth 2nd order &
itime\_scheme=12\tabularnewline

%\hline
si\_cmin&
R & 30.0 & m/s&
semi implicit correction is done for eigenmodes with speeds larger
than si\_cmin&
itime\_scheme=14 and lsi\_3d=.FALSE.\tabularnewline

%\hline
si\_coeff&
R & 1.0 & &
weight of the semi implicit correction&
itime\_scheme=14 \tabularnewline

%\hline
si\_offctr&
R & 0.7 & &
&
itime\_scheme=14 \tabularnewline

%\hline
si\_rtol &
R & 1.0e-3 & &
relative tolerance for GMRES solver&
itime\_scheme=14 \tabularnewline

%\hline
lsi\_3d&
L& .FALSE.& &
3D GMRES solver or decomposistion into 2D problems&
lshallow\_water=.FALSE. and itime\_scheme=14\tabularnewline
%\hline

%\hline
\textbf{ldry\_dycore} &
L & .TRUE. & &
Assume dry atmosphere &
iequations$\in$\{1,2\}
\tabularnewline

%\hline
\textbf{lref\_temp} &
L & .FALSE. & &
Set a background temperature profile as base state
when computing the pressure gradient force &
iequations$\in$\{1,2\}
\tabularnewline

\end{longtab}


%------------------------------------------------------------------------------
% initicon_nml
%------------------------------------------------------------------------------
\subsection{initicon\_nml}

\begin{longtab}

%\hline
\textbf{init\_mode}&
I & 2& &
1: MODE\_DWDANA\\ \quad start from DWD analysis or FG \\
2: MODE\_IFSANA\\ \quad start from IFS analysis \\
3: MODE\_COMBINED\\ \quad IFS atm + ICON/GME soil \\
4: MODE\_COSMODE\\ \quad start from COSMO-DE forecast \\
5: MODE\_IAU\\ \quad start from DWD analysis with incremental analysis update. Extension of MODE\_IAU\_OLD including snow increments\\
6: MODE\_IAU\_OLD\\ \quad start from DWD analysis with incremental analysis update. NOTE: Extension of mode MODE\_DWDANA\_INC 
   including W\_SO increments. \\
7: MODE\_ICONVREMAP\\ \quad start from DWD first guess with subsequent vertical remapping (work in progress; so far, changing
the number of model levels does not yet work) &
\tabularnewline

dt\_iau&
R & 10800 & s &
Time interval during which an incremental analysis update (IAU) is performed &
init\_mode=5,6
\tabularnewline

dt\_shift&
R & 0 & s &
Time by which the actual model start time is shifted ahead of the nominal date.
Must be NEGATIVE, usually $- 0.5$ dt\_iau. &
init\_mode=5,6
\tabularnewline

start\_time\_avg\_fg&
R & 0 & s &
Start time for calculating temporally averaged first guess output for data assimilation. &
\tabularnewline

end\_time\_avg\_fg&
R & 0 & s &
End time for calculating temporally averaged first guess output for data assimilation. \\
Setting end\_time\_avg\_fg $>$ start\_time\_avg\_fg activates the averaging &
\tabularnewline

interval\_avg\_fg&
R & 0 & s &
Corresponding averaging interval. Note that end\_time\_avg\_fg $-$ start\_time\_avg\_fg must not be smaller than the averaging interval &
\tabularnewline

rho\_incr\_filter\_wgt &
R & 0 &  &
Vertical filtering weight on density increments &
init\_mode=5,6
\tabularnewline

%\hline
type\_iau\_wgt&
I & 1 &  &
Weighting function for performing IAU\\
1: Top-Hat\\
2: SIN2 &
init\_mode=5,6
\tabularnewline

%\hline
\textbf{nlevsoil\_in}&
I & 4 & &
number of soil levels of input data&
init\_mode=2
\tabularnewline

%\hline
zpbl1 &
R & 500.0& m&
bottom height (AGL) of layer used for gradient computation&
\tabularnewline

%\hline
zpbl2 &
R & 1000.0& m&
top height (AGL) of layer used for gradient computation&
\tabularnewline

%\hline
l\_sst\_in&
L & .TRUE.& &
Logical switch. If true, the surface temperature of the water sea points is initialized
with the SST provided in the ifs2icon file. If false, it is initialized with the skin
temperature. If the SST is not provided in the ifs2icon file, l\_sst\_in is reset to false.  &
init\_mode=2
\tabularnewline

%\hline
lread\_ana&
L & .TRUE.& &
If .FALSE., ICON is started from first guess only. Analysis field is not required, and skipped if provided. &
init\_mode=1,3
\tabularnewline

%\hline
lconsistency\_checks&
L & .TRUE.& &
If .FALSE., consistency checks for Analysis and First Guess fields are skipped. On default, checks are performed for 
\emph{uuidOfHGrid} and \emph{validity time}. &
init\_mode=1,3,4,5,6
\tabularnewline

%\hline
l\_coarse2fine\_mode&
L(n\_dom) & .FALSE.& &
If true, apply corrections for coarse-to-fine mesh interpolation to wind and temperature &
\tabularnewline

%\hline
lp2cintp\_incr&
L(n\_dom) & .FALSE.& &
If true, interpolate atmospheric data assimilation increments from parent domain. \\
Can be specified separately for each nested domain; setting the first (global) entry to true activates
the interpolation for all nested domains. & init\_mode=5,6
\tabularnewline

%\hline
lp2cintp\_sfcana&
L(n\_dom) & .FALSE.& &
If true, interpolate atmospheric surface analysis data from parent domain. \\
Can be specified separately for each nested domain; setting the first (global) entry to true activates
the interpolation for all nested domains. & init\_mode=5,6
\tabularnewline

%\hline
ltile\_init&
L & .FALSE.& &
True: initialize tiled surface fields from a first guess coming from a run without tiles. \\
Along coastlines and lake shores, a neighbor search is executed to fill the variables on previously non-existing
land or water points with reasonable values. Should be combined with ltile\_coldstart = .TRUE. & init\_mode=1,5,6
\tabularnewline

%\hline
ltile\_coldstart&
L & .FALSE.& &
If true, tiled surface fields are initialized with tile-averaged fields from a previous run with tiles. \\
A neighbor search is applied to subgrid-scale ocean points for SST and sea-ice fraction. & init\_mode=1,5,6
\tabularnewline

%\hline
lvert\_remap\_fg&
L & .FALSE.& &
If true, vertical remapping is applied to the atmospheric first-guess fields, whereas the analysis increments
remain unchanged. The number of model levels must be the same for input and output fields, and the z\_ifc 
(alias HHL) field pertaining to the input fields must be appended to the first-guess file. & init\_mode=5,6
\tabularnewline


%\hline
\textbf{ifs2icon\_filename}&
C &
&
&
Filename of IFS2ICON input file, default
''\texttt{<path>ifs2icon\_R<nroot>B<jlev>\_DOM <idom>.nc}''.
May contain the keywords \texttt{<path>} which will be substituted by
\texttt{model\_base\_dir}, as well as \texttt{nroot}, \texttt{jlev},
and \texttt{idom} defining the current patch. & init\_mode=2
\tabularnewline

%\hline
\textbf{dwdfg\_filename}&
C &
&
&
Filename of DWD first-guess input file, default
''\texttt{<path>dwdFG\_R<nroot>B<jlev>\_DOM <idom>.nc}''.
May contain the keywords \texttt{<path>} which will be substituted by
\texttt{model\_base\_dir}, as well as \texttt{nroot}, \texttt{jlev},
and \texttt{idom} defining the current patch. & init\_mode=1,3,5,6
\tabularnewline

%\hline
\textbf{dwdana\_filename}&
C &
&
&
Filename of DWD analysis input file, default
''\texttt{<path>dwdana\_R<nroot>B<jlev>\_DOM
<idom>.nc}''.
May contain the keywords \texttt{<path>} which will be substituted by
\texttt{model\_base\_dir}, as well as \texttt{nroot}, \texttt{jlev},
and \texttt{idom} defining the current patch. & init\_mode=1,3,5,6
\tabularnewline

%\hline
\textbf{filetype} &
I& -1 (undef.)& &
One of CDI's FILETYPE\_XXX constants.
Possible values: 2 (=FILETYPE\_GRB2), 4 (=FILETYPE\_NC2).
If this parameter has not been set, we try to determine the file type by its extension "*.grb*" or ".nc".
&
\tabularnewline


%\hline
ana\_varlist &
C(:)& & &
List of mandatory analysis fields for the \textbf{global domain} that must be present in the analysis file. If these fields are not found, 
the model aborts. For all other analysis fields, the FG-fields will serve as fallback position.
& init\_mode=1,5,6
\tabularnewline

%\hline
ana\_varlist\_n2 &
C(:)& & &
List of mandatory analysis fields for \textbf{domain 2} that must be present in the analysis file. If these fields are not found, 
the model aborts. For all other analysis fields, the FG-fields will serve as fallback position.
& init\_mode=5,6
\tabularnewline


%\hline
\textbf{ana\_varnames\_map\_ file} &
C& & &
Dictionary file which maps internal variable names onto
GRIB2 shortnames or NetCDF var names.
This is a text file with two columns separated by whitespace, where
left column: ICON variable name, right column: GRIB2 short name.
&
\tabularnewline

%\hline
\textbf{latbc\_varnames\_map\_ file} &
C& & &
Dictionary file which maps internal variable names onto
GRIB2 shortnames or NetCDF var names.
This is a text file with two columns separated by whitespace, where
left column: ICON variable name, right column: GRIB2 short name.
This list contains variables that are to be read asynchronously for
boundary data nudging in a HDCP2 simulation. All new boundary variables
that in the future, would be read asynchronously. Need to be added to text 
file dict.latbc in run folder.   
& num\_prefetch\_proc=1
\tabularnewline

\end{longtab}

Defined and used in: \verb+src/namelists/mo_initicon_nml.f90+


%------------------------------------------------------------------------------
% interpol_nml: horizontal interpolation/reconstruction
%------------------------------------------------------------------------------
\subsection{interpol\_nml}
\begin{longtab}

%\hline
l\_intp\_c2l&
L& .TRUE.& &
If .TRUE. directly interpolate scalar variables from cell centers to
lon-lat points, otherwise do gradient interpolation and
reconstruction.&
\tabularnewline

%\hline
l\_mono\_c2l&
L& .TRUE.& &
Monotonicity can be enforced by demanding that the interpolated
value is not higher or lower than the stencil point values.&
\tabularnewline

%\hline
llsq\_high\_consv&
L& .TRUE.& &
conservative (T) or non-conservative (F) least-squares reconstruction for high order transport&
\tabularnewline

%\hline
lsq\_high\_ord&
I& 3& &
polynomial order for high order reconstruction& \tabularnewline
& & & & 1: linear & ihadv\_tracer=4 \tabularnewline
& & & & 2: quadratic & \tabularnewline
& & & & 30: cubic (no $3^{rd}$ order cross deriv.) & \tabularnewline
& & & & 3: cubic & \tabularnewline

%\hline
llsq\_lin\_consv&
L& .FALSE.& &
conservative (T) or non-conservative (F) least-squares reconstruction for 2nd order (linear) transport&
\tabularnewline

%\hline
nudge\_efold\_width&
R& 2.0& &
e-folding width (in units of cell rows) for lateral boundary nudging coefficient &
\tabularnewline

%\hline
nudge\_max\_coeff&
R& 0.02& &
Maximum relaxation coefficient for lateral boundary nudging&
\tabularnewline

%\hline
nudge\_zone\_width&
I& 8& &
Total width (in units of cell rows) for lateral boundary nudging zone. 
If < 0 the patch boundary\_depth\_index is used. &
\tabularnewline

%\hline
rbf\_dim\_c2l&
I& 10& &
stencil size for direct lon-lat interpolation:
 4 = nearest neighbor,
13 = vertex stencil,
10 = edge stencil.&
\tabularnewline

%\hline
rbf\_scale\_mode\_ll&
I& 2& &
Specifies, how the RBF shape parameter is
determined for lon-lat interpolation.\\
1 : lookup table based on grid level\\
2 : determine automatically.\\
So far, this routine only estimates the smallest value for the shape parameter for which the Cholesky is likely to succeed in floating point arithmetic.
3 : explicitly set shape parameter in each output namelist
&
\tabularnewline

%\hline
rbf\_vec\_kern\_c&
I& 1& &
Kernel type for reconstruction at cell centres:& \tabularnewline
& & & & 1: Gaussian & \tabularnewline
& & & & 3: inverse multiquadric & \tabularnewline

%\hline
rbf\_vec\_kern\_e&
I& 3& &
Kernel type for reconstruction at edges:& \tabularnewline
& & & & 1: Gaussian & \tabularnewline
& & & & 3: inverse multiquadric & \tabularnewline

%\hline
rbf\_vec\_kern\_ll&
I& 1& &
Kernel type for reconstruction at lon-lat-points:& \tabularnewline
& & & & 1: Gaussian & \tabularnewline
& & & & 3: inverse multiquadric & \tabularnewline

%\hline
rbf\_vec\_kern\_v&
I& 1& &
Kernel type for reconstruction at vertices:& \tabularnewline
& & & & 1: Gaussian & \tabularnewline
& & & & 3: inverse multiquadric & \tabularnewline

%\hline
rbf\_vec\_scale\_c&
R(n\_dom)& resolution-dependent& &
Scale factor for RBF reconstruction at cell centres&
\tabularnewline

%\hline
rbf\_vec\_scale\_e&
R(n\_dom)& resolution-dependent& &
Scale factor for RBF reconstruction at edges&
\tabularnewline

%\hline
rbf\_vec\_scale\_v&
R(n\_dom)& resolution-dependent& &
Scale factor for RBF reconstruction at vertices&
\tabularnewline

%\hline
support\_baryctr\_intp &
L& .FALSE. & &
Flag. If .FALSE. barycentric interpolation is replaced by a
fallback interpolation.&
\tabularnewline

\end{longtab}

Defined and used in: \verb+src/namelists/mo_interpol_nml.f90+


%------------------------------------------------------------------------------
% io_nml:
%------------------------------------------------------------------------------
\subsection{io\_nml}
\begin{longtab}

%\hline
lkeep\_in\_sync&
L& .FALSE. & &
Sync output stream with file on disk after each timestep&
\tabularnewline

%\hline
dt\_diag&
R& 86400. & s&
diagnostic integral output interval &
\texttt{run\_nml:output = "totint"}
\tabularnewline


%\hline
\textbf{dt\_checkpoint}&
R& 2592000 & s&
Time interval for writing restart files.
Note that if the value of dt\_checkpoint resulting from
model default or user's specification is longer than time\_nml:dt\_restart,
it will be reset (by the model) to dt\_restart so
that at least one restart file is generated during the restart cycle.
&
\texttt{output /= "none"} (\texttt{run\_nml})
\tabularnewline

%\hline
inextra\_2d&
I &
0&&
Number of extra 2D Fields for diagnostic/debugging output. &
dynamics\_nml:iequations = 3 {\color{red}(to be done for 1, 2)}
\tabularnewline
%\hline
inextra\_3d&
I &
0&&
Number of extra 3D Fields for diagnostic/debugging output. &
dynamics\_nml:iequations = 3 {\color{red}(to be done for 1, 2)}
\tabularnewline

%\hline
lflux\_avg&
L& .TRUE. & &
if .FALSE. the output fluxes are accumulated  \\
 from the beginning of the run                \\
if .TRUE. the output fluxes are average values\\
 from the beginning of the run, except of     \\
 TOT\_PREC that would be accumulated &
iequations=3\\
iforcing=3
\tabularnewline

%\hline
itype\_pres\_msl&
I& 1 & &
Specifies method for computation of mean sea level pressure (and geopotential at
pressure levels below the surface). \\
1: GME-type extrapolation, \\
2: stepwise analytical integration, \\
3: current IFS method, \\
4: IFS method with consistency correction
&
\tabularnewline

%\hline
itype\_rh&
I& 1 & &
Specifies method for computation of relative humidity \\
1: WMO-type: water only (e\_s=e\_s\_water), \\
2: IFS-type: mixed phase (water and ice), \\
3: IFS-type with clipping ($\mathrm{rh}\leq100$)
&
\tabularnewline

%\hline
 output\_nml\_dict &
C&' '& &
 File containing the mapping of variable names to the internal ICON names.
 May contain the keyword \texttt{<path>} which will be substituted by
 \texttt{model\_base\_dir}.\\
 The format of this file: \\
 One mapping per line, first the name as given in the \texttt{ml\_varlist},
 \texttt{hl\_varlist}, \texttt{pl\_varlist} or \texttt{il\_varlist}
 of the \texttt{output\_nml} namelists, then the internal ICON name,
 separated by an arbitrary number of blanks.
 The line may also start and end with an arbitrary number of blanks.
 Empty lines or lines starting with \# are treated as comments. \\
 Names not covered by the mapping are used as they are.
&
\texttt{output\_nml} namelists
\tabularnewline

%\hline
 netcdf\_dict &
C&' '& &
 File containing the mapping from internal names to names written to NetCDF.
 May contain the keyword \texttt{<path>} which will be substituted by
 \texttt{model\_base\_dir}.\\
 The format of this file: \\
 One mapping per line, first the name written to NetCDF,
 then the internal name, separated by an arbitrary number of blanks
 (\emph{inverse to the definition of \emph{output\_nml\_dict}}).
 The line may also start and end with an arbitrary number of blanks.
 Empty lines or lines starting with \# are treated as comments. \\
 Names not covered by the mapping are output as they are. \\
 Note that the specification of output variables, e.\,g.\ in
 \texttt{ml\_varlist}, is independent from this renaming, see
 the namelist parameter \texttt{output\_nml\_dict} for this.
&
\texttt{output\_nml} namelists,
NetCDF output
\tabularnewline

%\hline
lnetcdf\_flt64\_output&
L& .FALSE. & &
If .TRUE. floating point variable output in NetCDF files is written in 64-bit instead of 32-bit accuracy. \\
This is currently implemented for the atm. dynamical core and ECHAM physics. &
\tabularnewline

%\hline
restart\_file\_type&
I&  4& &
Type of restart file. One of CDI's FILETYPE\_XXX. So far, only 4 (=FILETYPE\_NC2) is allowed&
\tabularnewline

%\hline
\end{longtab}

Defined and used in: \verb+src/namelists/mo_io_nml.f90+


%------------------------------------------------------------------------------
% les_nml:
%------------------------------------------------------------------------------
\subsection{les\_nml (parameters for LES turbulence scheme; valid for inwp\_turb=5)}

\begin{longtab}

%\hline
sst & R & 300 & K &
sea surface temperature for idealized LES simulations &
isrfc\_type=5,4
\tabularnewline

%\hline
shflx & R & 0.1 & Km/s &
Kinematic sensible heat flux at surface &
isrfc\_type = 2
\tabularnewline

%\hline
lhflx & R & 0 & m/s &
Kinematic latent heat flux at surface &
isrfc\_type = 2
\tabularnewline

%\hline
isrfc\_type & I & 1 &  &
surface type \\
0 = No fluxes and zero shear stress \\
1 = TERRA land physics \\
2 = fixed surface fluxes \\
3 = fixed buoyancy fluxes \\
4 = RICO test case \\
5 = fixed SST &
\tabularnewline

%\hline
ufric & R & -999 & m/s &
friction velocity for idealized LES simulations; if < 0 then it is
automatically diagnosed &
\tabularnewline

%\hline
psfc & R & -999 & Pa &
surface pressure for idealized LES simulations; if < 0 then it uses
the surface pressure from dynamics &
\tabularnewline

%\hline
min\_sfc\_wind & R & 1.0 & m/s &
Minimum surface wind for surface layer useful in the limit of free convection &
\tabularnewline

%\hline
is\_dry\_cbl & L & .FALSE. &  &
switch for dry convective boundary layer simulations &
\tabularnewline

%\hline
smag\_constant & R & 0.23 &  &
Smagorinsky constant &
\tabularnewline

%\hline
km\_min & R & 0.0 &  &
Minimum turbulent viscosity &
\tabularnewline

%\hline
max\_turb\_scale & R & 300.0 &  &
Asymtotic maximum turblence length scale (useful for coarse grid LES and when grid is vertically stretched) &
\tabularnewline

%\hline
turb\_prandtl & R & 0.333333 &  &
turbulent Prandtl number &
\tabularnewline

%\hline
bflux & R & 0.0007 &  m$^2$/s$^3$ &
buoyancy flux for idealized LES simulations (Stevens 2007) &
isrfc\_type=3
\tabularnewline

%\hline
tran\_coeff & R & 0.02 &  m/s &
transfer coefficient near surface for idealized LES simulation (Stevens 2007)&
isrfc\_type=3
\tabularnewline

%\hline
vert\_scheme\_type & I & 2 &   &
type of time integration scheme in vertical diffusion \\
1 = explicit \\
2 = fully implicit \\ &
\tabularnewline

%\hline
sampl\_freq\_sec & R & 60 & s  &
sampling frequency in seconds for statistical (1D and 0D) output &
\tabularnewline

%\hline
avg\_interval\_sec & R & 900 & s  &
(time) averaging interval in seconds for 1D statistical output &
\tabularnewline

%\hline
expname & C & ICOLES &   &
expname to name the statistical output file &
\tabularnewline

%\hline
ldiag\_les\_out & L & .FALSE. &   &
Control for the statistical output in LES mode &
\tabularnewline

%\hline
les\_metric & L & .FALSE. &   &
Switch to turn on Smagorinsky diffusion with 3D metric terms to account for topography &
\tabularnewline

\end{longtab}

Defined and used in: \verb+src/namelists/mo_les_nml.f90+


%------------------------------------------------------------------------------
% limarea_nml:
%------------------------------------------------------------------------------
\subsection{limarea\_nml (Scope: l\_limited\_area=1 in grid\_nml)}

\begin{longtab}

%\hline
\textbf{itype\_latbc}&
I & 0& &
Type of lateral boundary nudging. Nudge from\\
0: the initial data,\\
1: IFS data analysis/forecast (if \texttt{initicon\_nml:init\_mode}=4, we take COSMO-DE data),\\
2: ICON output data (with the identical 3d grid)&
\tabularnewline

%\hline
\textbf{dtime\_latbc}&
R &
10800.0& s
&
Time difference between two consecutive boundary data.
&
itype\_latbc $\ge$ 1
\tabularnewline

%\hline
\textbf{nlev\_latbc}&
I &
0& s
&
Number of vertical levels in boundary data.
&
itype\_latbc $\ge$ 1
\tabularnewline

%\hline
\textbf{latbc\_filename}&
C &
&
&
Filename of boundary data input file, default:\\
''\texttt{prepiconR<nroot>B<jlev>\_<y><m><d><h>.nc}''.
\texttt{<y>}, \texttt{<m>}, \texttt{<d>}, and \texttt{<h>} 
will be automatically replaced during the run-time. In case
the time span between two consecutive boundary data is less than 1 hour, 
one can use \texttt{<min>} and \texttt{<sec>}.
These files must be located in
the \texttt{latbc\_path} directory.
&
itype\_latbc $\ge$ 1
\tabularnewline

%\hline
\textbf{latbc\_path}&
C &
&
&
Absolute path to boundary data.
&
itype\_latbc $\ge$ 1
\tabularnewline


\end{longtab}

Defined and used in: \verb+src/namelists/mo_limarea_nml.f90+


%------------------------------------------------------------------------------
% nwp_lnd_nml:
%------------------------------------------------------------------------------
\subsection{lnd\_nml}

\begin{longtab}

nlev\_snow &
I&
2&
&
number of snow layers&
lmulti\_snow=.true.
\tabularnewline
%\hline
\textbf{ntiles} &
I&
1&
&
number of tiles&
\tabularnewline

%\hline
\textbf{lsnowtile} &
L&
.FALSE.&
&
.TRUE.: consider snow-covered and snow-free tiles separately &
ntiles$>$1
\tabularnewline

%\hline
frlnd\_thrhld &
R&
0.05&
&
fraction threshold for creating a land grid point &
ntiles$>$1
\tabularnewline

%\hline
frlake\_thrhld &
R&
0.05&
&
fraction threshold for creating a lake grid point &
ntiles$>$1
\tabularnewline
%\hline
frsea\_thrhld &
R&
0.05&
&
fraction threshold for creating a sea grid point &
ntiles$>$1
\tabularnewline

%\hline
frlndtile\_thrhld &
R&
0.05&
&
fraction threshold for retaining the respective tile for a grid point&
ntiles$>$1
\tabularnewline

%\hline
lmelt &
L&
.TRUE.&
&
.TRUE. soil model with melting process&
\tabularnewline

%\hline
lmelt\_var &
L&
.TRUE.&
&
.TRUE. freezing temperature dependent on water content&
\tabularnewline

%\hline
lana\_rho\_snow &
L&
.TRUE.&
&
.TRUE. take rho\_snow-values from analysis file&
init\_mode=1
\tabularnewline

%\hline
\textbf{lmulti\_snow} &
L&
.TRUE.&
&
.TRUE. for use of multi-layer snow model&
\tabularnewline

l2lay\_rho\_snow &
L&
.FALSE.&
&
.TRUE. predict additional snow density for upper part of the snowpack, having
a maximum depth of max\_toplaydepth & lmulti\_snow = .FALSE.
\tabularnewline

%\hline
max\_toplaydepth &
R &
0.25&
m &
maximum depth of uppermost snow layer & lmulti\_snow=.TRUE. or l2lay\_rho\_snow=.TRUE.
\tabularnewline

%\hline
idiag\_snowfrac &
I & 1 &  & Type of snow-fraction diagnosis:\\ 
1 = based on SWE only\\
2--4 = more advanced experimental methods \\ 
20, 30, 40 = same as 2, 3, 4, respectively, but with artificial reduction of snow fraction in case of melting snow&
\tabularnewline

%\hline
itype\_lndtbl &
I & 1 &  & Table values used for associating surface parameters to land-cover classes: \\
1 = defaults from extpar (GLC2000 and GLOBCOVER2009)\\
2 = Tuned version based on IFS values for globcover classes (GLOBCOVER2009 only)\\
3 = even more tuned version (EXPERIMENTAL!!, GLOBCOVER2009 only) &
\tabularnewline

%\hline
itype\_root &
I & 2 &  & root density distribution: \\
1 = constant\\
2 = exponential &
\tabularnewline

%\hline
itype\_evsl &
I & 2 &  & type of bare soil evaporation parameterization \\
2 = Dickinson (1984)\\
3 = Noilhan and Platon (1989) &
\tabularnewline

%\hline
itype\_heatcond &
I & 2 &  & type of soil heat conductivity \\
1 = constant soil heat conductivity\\
2 = moisture dependent soil heat conductivity &
\tabularnewline

%\hline
itype\_interception &
I & 1 &  & type of plant interception \\
1 = effectively switched off (secirity minimum of $1E-6\,\mathrm{m}$ for surface area index)\\
2 = Rain and snow interception (\textcolor{red}{under development}) &
\tabularnewline

%\hline
itype\_hydbound &
I & 1 &  & type of hydraulic lower boundary condition \\
1 = none \\
3 = ground water as lower boundary of soil column &
\tabularnewline

%\hline
lstomata &
L & .TRUE. &  & If .TRUE., use map of minimum stomatal resistance\\
If .FALSE., use constant value of $150\, \mathrm{s/m}$.
&
\tabularnewline

%\hline
l2tls &
L & .TRUE. &  & If .TRUE., forecast with 2-Time-Level integration scheme
&
\tabularnewline

%\hline
\textbf{lseaice} &
L&
.TRUE.&
&
.TRUE. for use of sea-ice model&
\tabularnewline
%\hline
\textbf{llake} &
L&
.TRUE.&
&
.TRUE. for use of lake model&
\tabularnewline
%\hline
sstice\_mode &
I&
1&
&
1: SST and sea ice fraction are read from the analysis and kept constant. The
sea ice fraction can be modified by the seaice model.\\
2: SST and sea ice fraction are updated daily, based on climatological monthly
means\\
3: SST and sea ice fraction are updated daily, based on actual monthly means\\
4: SST and sea ice fraction are updated daily, based on actual daily means,
\textcolor{red}{not yet implemented}&
iequations=3\\
iforcing=3
\tabularnewline
%\hline
sst\_td\_filename &
C&
&
&
Filename of SST input files for time dependent SST.
Default is "$<$path$>$SST\_$<$year$>$\_<month>\_$<$gridfile$>$". May contain the
keyword $<$path$>$ which will be substituted by model\_base\_dir&
sstice\_mode=2,3
\tabularnewline
%\hline
ci\_td\_filename &
C&
&
&
Filename of sea ice fraction input files for time dependent sea ice fraction.
Default is "$<$path$>$CI\_$<$year$>$\_$<$month$>$\_$<$gridfile$>$". May contain
the keyword $<$path$>$ which will be substituted by model\_base\_dir&
sstice\_mode=2,3
\tabularnewline
\end{longtab}

Defined and used in: \verb+src/namelists/mo_lnd_nwp_nml.f90+


%------------------------------------------------------------------------------
% ls_forcing_nml:
%------------------------------------------------------------------------------
\subsection{ls\_forcing\_nml (parameters for large-scale forcing; valid for torus geometry)}

\begin{longtab}

%\hline
is\_subsidence\_moment & L & .FALSE. &  &
switch for enabling LS vertical advection due to subsidence for momentum equations&
is\_plane\_torus=.TRUE.
\tabularnewline

%\hline
is\_subsidence\_heat & L & .FALSE. &  &
switch for enabling LS vertical advection due to subsidence for thermal equations &
is\_plane\_torus=.TRUE.
\tabularnewline


%\hline
is\_advection & L & .FALSE. &  &
switch for enabling LS horizontal advection (currently only for thermal equations)&
is\_plane\_torus=.TRUE.
\tabularnewline

%\hline
is\_geowind & L & .FALSE. &  &
switch for enabling geostrophic wind &
is\_plane\_torus=.TRUE.
\tabularnewline

%\hline
is\_rad\_forcing & L & .FALSE. &  &
switch for enabling radiative forcing &
is\_plane\_torus=.TRUE. \\
inwp\_rad=.FALSE.
\tabularnewline

%\hline
is\_theta & L & .FALSE. &  &
switch to indicate that the prescribed radiative forcing is for potential temperature &
is\_plane\_torus=.TRUE. \\
is\_rad\_forcing=.TRUE.
\tabularnewline

\end{longtab}

Defined and used in: \verb+src/namelists/mo_ls_forcing_nml.f90+


\subsection{master\_model\_nml (repeated for each model)}
\begin{longtab}

%\hline
\textbf{model\_name} &
C & & &
Character string for naming this component.&
\tabularnewline

%\hline
\textbf{model\_namelist\_ filename} &
C & & &
File name containing the model namelists.&
\tabularnewline

%\hline
\textbf{model\_type} &
I & -1 & &
Identifies which component to run.\\
1=atmosphere\\
2=ocean\\
3=radiation\\
99=dummy\_model &
\tabularnewline

%\hline
model\_min\_rank &
I & 0 & &
Start MPI rank for this model.&
\tabularnewline

%\hline
model\_max\_rank &
I & -1 & &
End MPI rank for this model.&
\tabularnewline

%\hline
model\_inc\_rank &
I & 1 & &
Stride of MPI ranks.&
\tabularnewline

\end{longtab}


%------------------------------------------------------------------------------
% master_nml: the minimum one needs to specify about an integration
%------------------------------------------------------------------------------
\subsection{master\_nml}
\begin{longtab}

%\hline
\textbf{lrestart}&
L & .FALSE. & &
If .TRUE.: Current experiment is started from a restart.&
\tabularnewline

%\hline
\textbf{model\_base\_dir} &
C & ' ' & &
General path which may be used in file names of other name lists:
If a file name contains the keyword "\texttt{<path>}", then this
\texttt{model\_base\_dir} will be substituted.
 &
\tabularnewline

\end{longtab}


%------------------------------------------------------------------------------
% meteogram_output_nml:
%------------------------------------------------------------------------------
\subsection{meteogram\_output\_nml}
Nearest neighbour 'interpolation' is used for all variables.
\begin{longtab}

%\hline
lmeteogram\_enabled&
L(n\_dom) &
.FALSE.&&
Flag. True, if meteogram of output variables is desired.&
\tabularnewline

%\hline
zprefix&
C(n\_dom) &
``METEO GRAM\_''&&
string with file name prefix for output file&
\tabularnewline

%\hline
ldistributed&
L(n\_dom) &
.TRUE.&&
Flag. Separate files for each PE.&
\tabularnewline

%\hline
loutput\_tiles&
L &
.FALSE.&&
Write tile-specific output for some selected surface/soil fields&
\tabularnewline

%\hline
n0\_mtgrm&
I(n\_dom) &
0&&
initial time step for meteogram output.&
\tabularnewline

%\hline
ninc\_mtgrm&
I(n\_dom) &
1&&
output interval (in time steps)&
\tabularnewline

%\hline
stationlist\_tot&
&
53.633,  9.983, 'Hamburg' &&
list of meteogram stations (triples with lat, lon, name string)&
\tabularnewline

%\hline
var\_list&
C(:)
&
" " &&
Positive-list of variables (optional). Only variables contained in
this list are included in the meteogram. If the default list is not
changed by user input, then all available variables are added to the
meteogram
&
\tabularnewline

%\hline
\end{longtab}

Defined and used in: \verb+src/namelists/mo_mtgrm_nml.f90+



%-----------------------------------------------------------------------------
% nonhydrostatic_nml:
%-----------------------------------------------------------------------------
\subsection{nonhydrostatic\_nml (relevant if run\_nml:iequations=3)}

\begin{longtab}

%\hline
\textbf{itime\_scheme}&
I& 4& &
Options for predictor-corrector time-stepping scheme:& \tabularnewline
& & & &
4: Contravariant vertical velocity is computed in the predictor step only,
   velocity tendencies are computed in the corrector step only (most efficient option) \\
5: Contravariant vertical velocity is computed in both substeps (beneficial for numerical
   stability in very-high resolution setups with extremely steep slops, otherwise no significant impact)\\
6: As 5, but velocity tendencies are also computed in both substeps (no apparent benefit, but more expensive) &
iequations=3
\tabularnewline

%\hline
rayleigh\_type&
I& 2& &
Type of Rayleigh damping\\
1: CLASSICAL (requires velocity reference state!)\\
2: Klemp (2008) type &
\tabularnewline

%\hline
\textbf{rayleigh\_coeff}&
R(n\_dom)& 0.05 for i=1& &
Rayleigh damping coefficient $1/\tau_{0}$ (Klemp, Dudhia, Hassiotis: MWR136, pp.3987-4004);
higher values are recommended for R2B6 or finer resolution&
\tabularnewline

%\hline
\textbf{damp\_height}&
R(n\_dom)& 45000 for i=1& m&
Height at which Rayleigh damping of vertical wind starts (needs to be adjusted to model top height; the damping
layer should have a depth of at least 20 km when the model top is above the stratopause)&
\tabularnewline

%\hline
htop\_moist\_proc&
R& 22500.0& m&
Height above which moist physics and advection of cloud and precipitation variables are turned off&
\tabularnewline

%\hline
hbot\_qvsubstep&
R& 22500.0& m&
Height above which QV is advected with substepping scheme (must be at least as large as htop\_moist\_proc)&
ihadv\_tracer=22, 32, 42 or 52
\tabularnewline


%\hline
vwind\_offctr&
R& 0.15& &
Off-centering in vertical wind solver. Higher values may be needed for R2B5 or coarser grids when the model top is above 50 km.&
\tabularnewline

%\hline
rhotheta\_offctr&
R& -0.1& &
Off-centering of density and potential temperature at interface level (may be set to 0.0 for R2B6 or finer grids) &
\tabularnewline

%\hline
veladv\_offctr&
R& 0.25& &
Off-centering of velocity advection in corrector step &
\tabularnewline

%\hline
ivctype&
I& 2& &
Type of vertical coordinate:\\
1: Gal-Chen hybrid \\
2: SLEVE (uses sleve\_nml)&
\tabularnewline

%\hline
ndyn\_substeps&
I& 5& &
number of dynamics substeps per fast-physics / transport step &
\tabularnewline

%\hline
lhdiff\_rcf&
L& .TRUE. & &
.TRUE.: Compute diffusion only at advection time steps (in this case,
divergence damping is applied in the dynamical core) &
\tabularnewline

%\hline
lextra\_diffu&
L& .TRUE. & &
.TRUE.: Apply additional momentum diffusion at grid points close to the stability limit for vertical advection (becomes effective
extremely rarely in practice; this is mostly an emergency fix for pathological cases with very large orographic gravity waves)
& 
\tabularnewline

%\hline
divdamp\_fac&
R& 0.0025& &
Scaling factor for divergence damping&
lhdiff\_rcf = .TRUE.
\tabularnewline

%\hline
divdamp\_order&
I& 4& &
Order of divergence damping: \\
2 = second-order divergence damping \\
4 = fourth-order divergence damping \\
24 = combined second-order and fourth-order divergence damping and enhanced vertical wind off-centering during the initial spinup phase (does not allow checkpointing/restarting earlier than 2.5 hours of integration) &
lhdiff\_rcf = .TRUE.
\tabularnewline

%\hline
divdamp\_type&
I& 3 & &
Type of divergence damping: \\
2 = divergence damping acting on 2D divergence \\
3 = divergence damping acting on 3D divergence \\
32 = combination of 3D div. damping in the troposphere with transition to 2D div. damping in the stratosphere &
lhdiff\_rcf = .TRUE.
\tabularnewline

divdamp\_trans\_start&
R & 12500. & &
Lower bound of transition zone between 2D and 3D divergence damping  &
divdamp\_type = 32
\tabularnewline

divdamp\_trans\_end&
R & 17500. & &
Upper bound of transition zone between 2D and 3D divergence damping  &
divdamp\_type = 32
\tabularnewline

%\hline
nest\_substeps&
I& 2& &
Number of dynamics substeps for the child patches.\\
\textcolor{red}{DO NOT CHANGE!!! The code will not work correctly with other values}&
\tabularnewline

%\hline
l\_masscorr\_nest&
L& .FALSE.& &
.TRUE.: Apply mass conservation correction also in nested domain &
\tabularnewline

%\hline
iadv\_rhotheta&
I& 2& &
Advection method for rho and rhotheta:\\
1: simple second-order upwind-biased scheme \\
2: 2nd order Miura horizontal &
\tabularnewline
& & & & 3: 3rd order Miura horizontal (not recommended)&
\tabularnewline

%\hline
igradp\_method&
I& 3& &
Discretization of horizontal pressure gradient:\\
1: conventional discretization with metric correction term\\
2: Taylor-expansion-based reconstruction of pressure (advantageous at very high resolution)\\
3: Similar discretization as option 2, but uses hydrostatic approximation
for downward extrapolation over steep slopes \\
4: Cubic/quadratic polynomial interpolation for pressure reconstruction \\
5: Same as 4, but hydrostatic approximation for downward extrapolation over steep slopes &
\tabularnewline

%\hline
l\_zdiffu\_t&
L& .TRUE.& &
.TRUE.: Compute Smagorinsky temperature diffusion truly horizontally over steep slopes&
 hdiff\_order=3/5 .AND. lhdiff\_temp = .true.
\tabularnewline

%\hline
thslp\_zdiffu&
R& 0.025& &
Slope threshold above which truly horizontal temperature diffusion is activated&
hdiff\_order=3/5 .AND. lhdiff\_temp=.true. .AND. l\_zdiffu\_t=.true.
\tabularnewline

%\hline
thhgtd\_zdiffu&
R& 200& m&
Threshold of height difference between neighboring grid points above which
truly horizontal temperature diffusion is activated (alternative criterion to thslp\_zdiffu)&
 hdiff\_order=3/5 .AND. lhdiff\_temp=.true. .AND. l\_zdiffu\_t=.true.
\tabularnewline

%\hline
exner\_expol&
R& 1./3.& &
Temporal extrapolation (fraction of dt) of Exner function for computation of horizontal pressure gradient.
This damps horizontally propagating sound waves. For R2B5 or coarser grids, values between 1/2 and 2/3 are recommended.&
\tabularnewline

%\hline
l\_open\_ubc&
L& .FALSE.& &
.TRUE.: Use open upper boundary condition (rather than w=0) to allow vertical motions related to diabatic heating to
extend beyond the model top&
\tabularnewline


%\hline
\end{longtab}

Defined and used in: \verb+src/namelists/mo_nonhydrostatic_nml.f90+


%------------------------------------------------------------------------------
% nwp_phy_nml:
%------------------------------------------------------------------------------
\subsection{nwp\_phy\_nml}

The switches for the physics schemes and the time steps can be set for each model domain individually.
If only one value is specified, it is copied to all child domains, implying that the same set
of parameterizations and time steps is used in all domains. If the number of values given
in the namelist is larger than 1 but less than the number of model domains, then the settings
from the highest domain ID are used for the remaining model domains. If the time steps are not
an integer multiple of the advective time step (dtime), then the time step of the
respective physics parameterization is automatically rounded to the next higher integer multiple
of the advective time step.

\begin{longtab}

%\hline
\textbf{inwp\_gscp}&
I (max\_ dom)&
1&
&
cloud microphysics and precipitation\\
0: none\\
1: hydci (COSMO-EU microphysics, 2-cat ice: cloud ice, snow)\\
2: hydci\_gr (COSMO-DE microphysics, 3-cat ice: cloud ice, snow, graupel)\\
3: as 1, but with improved ice nucleation scheme by C. Koehler\\
4: Two-moment microphysics by A. Seifert\\

9: Kessler scheme &
run\_nml:iforcing = inwp
\tabularnewline

%\hline
qi0&
R&
0.0&
kg/kg &
cloud ice threshold for autoconversion &
inwp\_gscp=1
\tabularnewline

%\hline
qc0&
R&
0.0&
kg/kg &
cloud water threshold for autoconversion &
inwp\_gscp=1
\tabularnewline

mu\_rain&
R&
0.0&
 &
shape parameter in gamma distribution for rain&
inwp\_gscp$>$0
\tabularnewline

mu\_snow&
R&
0.0&
 &
shape parameter in gamma distribution for snow&
inwp\_gscp$>$0
\tabularnewline

icpl\_aero\_gscp&
I&
0&
 &
0: off \\
1: simple coupling between autoconversion and Tegen aerosol climatology; requires irad\_aero=6 \\
More advanced options are in preparation&
currently only for inwp\_gscp = 1
\tabularnewline

%\hline
\textbf{inwp\_convection}&
I (max\_ dom)&
1&
&
convection\\
0: none\\
1: Tiedtke/Bechtold convection
&run\_nml:iforcing = inwp
\tabularnewline

%\hline
lshallowconv\_only&
L (max\_ dom)&
.FALSE.&
&
.TRUE.: use shallow convection only\\
& inwp\_convection = 1
\tabularnewline

%\hline
icapdcycl&
I & 0 &  & Type of CAPE correction to improve diurnal cycle for convection: \\
0 = none (IFS default prior to autumn 2013) \\
1 = intermediate testing option \\
2 = correctoins over land and water now operational at ECMWF \\
3 = correction over land as in 2 restricted to the tropics, no correction over water (this choice optimizes the NWP skill scores) &
inwp\_convection = 1
\tabularnewline

icpl\_aero\_conv&
I&
0&
 &
0: off \\
1: simple coupling between autoconversion and Tegen aerosol climatology; requires irad\_aero=6 &
\tabularnewline

iprog\_aero &
I&
0&
 &
0: off \\
1: simple prognostic aerosol scheme, based on 2D aerosol optical depth fields of Tegen climatology; requires irad\_aero=6 &
\tabularnewline

icpl\_o3\_tp&
I&
1&
 &
0: off \\
1: simple coupling between the ozone mixing ratio and the thermal tropopause, restricted to the extratropics & irad\_o3 = 7 or 9
\tabularnewline

%\hline
\textbf{inwp\_cldcover}&
I (max\_ dom)&
1&
&
cloud cover scheme for radiation\\
0: no clouds (only QV)\\
1: diagnostic cloud cover (by Martin Koehler)\\
2: prognostic total water variance (not yet started)\\
3: clouds from COSMO SGS cloud scheme\\
4: clouds as in turbulence (turbdiff)\\
5: grid scale clouds
&run\_nml:iforcing = inwp
\tabularnewline

%\hline
\textbf{inwp\_radiation}&
I (max\_ dom)&
1&
&
radiation\\
0: none\\
1: RRTM radiation\\
2: Ritter-Geleyn radiation\\
3: PSRAD radiation
&run\_nml:iforcing = inwp
\tabularnewline

%\hline
\textbf{inwp\_satad}&
I&
1&
&
saturation adjustment\\
0: none\\
1: saturation adjustment at constant density
&run\_nml:iforcing = inwp
\tabularnewline

%\hline
\textbf{inwp\_turb}&
I (max\_ dom)&
1&
&
vertical diffusion and transfer\\
0: none\\
1: COSMO diffusion and transfer\\
2: GME turbulence scheme\\
3: EDMF-DUALM (work in progress)\\
5: Classical Smagorinsky diffusion
&run\_nml:iforcing = inwp
\tabularnewline
%\hline
\textbf{inwp\_sso}&
I (max\_ dom)&
1&
&
subgrid scale orographic drag\\
0: none\\
1: Lott and Miller scheme (COSMO)
&run\_nml:iforcing = inwp
\tabularnewline

%\hline
\textbf{inwp\_gwd}&
I (max\_ dom)&
1&
&
non-orographic gravity wave drag\\
0: none\\
1: Orr-Ern-Bechtold-scheme (IFS)
&run\_nml:iforcing = inwp
\tabularnewline


%\hline
\textbf{inwp\_surface}&
I (max\_ dom)&
1&
&
surface scheme\\
0: none\\
1: TERRA
&run\_nml:iforcing = inwp
\tabularnewline


%\hline
ustart\_raylfric&
R& 160.0& m/s& wind speed at which extra Rayleigh friction starts &
inwp\_gwd $>$ 0
\tabularnewline


%\hline
efdt\_min\_raylfric&
R& 10800.& s & minimum e-folding time of Rayleigh friction (effective for u $>$ ustart\_raylfric + 90 m/s) &
inwp\_gwd $>$ 0
\tabularnewline

%\hline
latm\_above\_top&
L (max\_ dom)& .FALSE.&  & .TRUE.: take into account atmosphere above model top for radiation computation &
inwp\_radiation $>$ 0
\tabularnewline

%\hline
itype\_z0&
I & 2 &  & Type of roughness length data used for turbulence scheme: \\
1 = land-cover-related roughness including contribution from sub-scale orography \\
2 = land-cover-related roughness only &
inwp\_turb $>$ 0
\tabularnewline

%\hline
\textbf{dt\_conv} &
R (max\_ dom)&
600.&
s&
time interval of convection call\\
currently each subdomain has the same value
&run\_nml:iforcing = inwp
\tabularnewline

%\hline
\textbf{dt\_rad} &
R (max\_ dom)&
1800.&
s&
time interval of radiation call\\
currently each subdomain has the same value
&run\_nml:iforcing = inwp
\tabularnewline

%\hline
\textbf{dt\_sso}&
R (max\_ dom)&
1200.&
s&
time interval of sso call\\
currently each subdomain has the same value
&run\_nml:iforcing = inwp
\tabularnewline

%\hline
\textbf{dt\_gwd} &
R (max\_ dom)&
1200.&
s&
time interval of gwd call\\
currently each subdomain has the same value
&run\_nml:iforcing = inwp
\tabularnewline

%\hline
lrtm\_filename &
C(:)&
``rrtmg\_ lw.nc''&
&
NetCDF file containing longwave absorption coefficients and other data
for RRTMG\_LW k-distribution model. &
\tabularnewline

%\hline
cldopt\_filename &
C(:)&
``ECHAM 6\_CldOpt Props.nc''&
&
NetCDF file with RRTM Cloud Optical Properties for ECHAM6. &
\tabularnewline

\end{longtab}


Defined and used in: \verb+src/namelists/mo_nwp_phy_nml.f90+



%------------------------------------------------------------------------------
% nwp_tuning_nml:
%------------------------------------------------------------------------------
\subsection{nwp\_tuning\_nml}

Please note: These tuning parameters are NOT domain specific.

\begin{longtab}

\hline
\hline
\multicolumn{6}{|l|}{\textbf{SSO} (Lott and Miller)} 
\tabularnewline

\hline
tune\_gkwake&
R&
1.5&
&
low level wake drag constant&
run\_nml:iforcing = inwp
\tabularnewline

\hline
tune\_gkdrag&
R&
0.075&
&
gravity wave drag constant&
run\_nml:iforcing = inwp
\tabularnewline

\hline
\hline
\multicolumn{6}{|l|}{\textbf{GWD} (Warner McIntyre)} 
\tabularnewline

\hline
tune\_gfluxlaun&
R&
2.50e-3&
&
total launch momentum flux in each azimuth (rho\_o x F\_o)&
run\_nml:iforcing = inwp
\tabularnewline


\hline
\hline
\multicolumn{6}{|l|}{\textbf{Grid scale microphysics} (one moment)} 
\tabularnewline

\hline
tune\_zceff\_min&
R&
0.075&
&
Minimum value for sticking efficiency&
run\_nml:iforcing = inwp
\tabularnewline

\hline
tune\_v0snow&
R&
25.0&
&
factor in the terminal velocity for snow&
run\_nml:iforcing = inwp
\tabularnewline

\hline
tune\_zvz0i&
R&
1.25&
m/s&
Terminal fall velocity of ice&
run\_nml:iforcing = inwp
\tabularnewline


\hline
\hline
\multicolumn{6}{|l|}{\textbf{Convection scheme}} 
\tabularnewline


\hline
tune\_entrorg&
R&
1.85e-3&
1/m&
Entrainment parameter valid for dx=20 km (depends on model resolution)&
run\_nml:iforcing = inwp
\tabularnewline

\hline
tune\_capdcfac\_et&
R&
0&
&
Fraction of CAPE diurnal cycle correction applied in the extratropics&
icapdcycl = 3
\tabularnewline

\hline
tune\_rhebc\_land&
R&
0.75&
&
RH threshold for onset of evaporation below cloud base over land&
run\_nml:iforcing = inwp
\tabularnewline

\hline
tune\_rhebc\_land\_trop&
R&
0.70&
&
RH threshold for onset of evaporation below cloud base over land in the tropics&
run\_nml:iforcing = inwp
\tabularnewline

\hline
tune\_rhebc\_ocean &
R&
0.85&
&
RH threshold for onset of evaporation below cloud base over sea&
run\_nml:iforcing = inwp
\tabularnewline

\hline
tune\_rhebc\_ocean\_trop &
R&
0.80&
&
RH threshold for onset of evaporation below cloud base over sea in the tropics&
run\_nml:iforcing = inwp
\tabularnewline

\hline
tune\_rcucov&
R&
0.05&
&
Convective area fraction used for computing evaporation below cloud base&
run\_nml:iforcing = inwp
\tabularnewline

\hline
tune\_rcucov\_trop&
R&
0.05&
&
Convective area fraction used for computing evaporation below cloud base in the tropics&
run\_nml:iforcing = inwp
\tabularnewline

\hline
tune\_texc&
R&
0.125&
K&
Excess value for temperature used in test parcel ascent&
run\_nml:iforcing = inwp
\tabularnewline

\hline
tune\_qexc&
R&
0.0125&
&
Excess fraction of grid-scale QV used in test parcel ascent&
run\_nml:iforcing = inwp
\tabularnewline




\hline
\hline
\multicolumn{6}{|l|}{\textbf{Misc}} 
\tabularnewline



\hline
itune\_albedo&
I&
0&
&
MODIS albedo tuning\\
0: None\\
1: dimmed sahara&
run\_nml:iforcing = inwp\\
albedo\_type=2
\tabularnewline


\hline
tune\_minsnowfrac&
R&
0.125&
&
Minimum value to which the snow cover fraction is artificially reduced in case of melting show &
lnd\_nml:idiag\_snowfrac = 20/30/40
\tabularnewline


\hline
\hline
\multicolumn{6}{|l|}{\textbf{IAU}} 
\tabularnewline

\hline
max\_freshsnow\_inc&
R&
0.025&
&
Maximum allowed freshsnow increment per analysis cycle (positive or negative)&
init\_mode=5 (MODE\_IAU)
\tabularnewline
\end{longtab}


Defined and used in: \verb+src/namelists/mo_nwp_tuning_nml.f90+




%------------------------------------------------------------------------------
% output_nml:
%------------------------------------------------------------------------------
\subsection{output\_nml (relevant if run\_nml/output='nml')}

Please note: There may be several instances of
output\_nml in the namelist file, every one defining a list of variables with
separate attributes for output.

\begin{longtab}

%\hline
\textbf{dom}&
I(:)&-1& &
 Array of domains for which this name-list is used.
 If not specified (or specified as -1 as the first array member),
 this name-list will be used for all domains. \\
 Attention: Depending on the setting of the parameter l\_output\_phys\_patch
 these are either logical or physical domain numbers!
&
\tabularnewline

%\hline
\textbf{file\_interval} &
C&\'' \''& &
Defines the length of a file in terms of an ISO-8601 duration string. An example for this time stamp format is given below. This namelist parameter can be set instead of \texttt{steps\_per\_file}.
&
\tabularnewline

%\hline
\textbf{filename\_format} &
C&see description.& &
 Output filename format. Includes keywords \texttt{path}, \texttt{output\_filename}, \texttt{physdom}, etc. (see below).
 Default is \texttt{<output\_filename>\_DOM<physdom>\_<levtype>\_ <jfile>}
&
\tabularnewline

%\hline
filename\_extn &
C&"default"& &
User-specified filename extension (empty string also possible).
If this namelist parameter is chosen as "default", then we have
".nc" for NetCDF output files, and ".grb" for GRIB1/2.
&
\tabularnewline

%\hline
filetype &
I&4& &
One of CDI's FILETYPE\_XXX constants.
Possible values:\\
2=FILETYPE\_GRB2,\\
4=FILETYPE\_NC2, \\
5=FILETYPE\_NC4
&
\tabularnewline

%\hline
m\_levels &
C&None&  &
 Model level indices (optional). \\
 Allowed is a comma- (or semicolon-) separated list of integers,
 and of integer ranges like "10...20".  One may also use the
 keyword "nlev" to denote the maximum integer (or, equivalently,
 "n" or "N").
 Furthermore, arithmetic expressions like "(nlev~-~2)" are
 possible.\\
 Basic example:\\\texttt{m\_levels = "1,3,5...10,20...(nlev-2)"}
 \\
&
\tabularnewline

%\hline
h\_levels &
R(:)&None& m &
 height levels \\
&
\tabularnewline

%\hline
p\_levels &
R(:)&None& Pa &
 pressure levels \\
&
\tabularnewline

%\hline
i\_levels &
R(:)&None& K &
 isentropic levels \\
&
\tabularnewline

%\hline
\textbf{ml\_varlist} &
C(:)&None& &
 Name of model level fields to be output.
&
\tabularnewline

%\hline
\textbf{hl\_varlist }&
C(:)&None& &
 Name of height level fields to be output.
&
\tabularnewline

%\hline
\textbf{pl\_varlist} &
C(:)&None& &
 Name of pressure level fields to be output.
&
\tabularnewline

%\hline
\textbf{il\_varlist }&
C(:)&None& &
 Name of isentropic level fields to be output.
&
\tabularnewline

%\hline
\textbf{include\_last} &
L&.TRUE.& &
 Flag whether to include the last time step
&
\tabularnewline

%\hline
 \textbf{mode }&
I&2& &
 1 = forecast mode, 2 = climate mode \\
 In climate mode the time axis of the output file
 is set to TAXIS\_ABSOLUTE. In forecast mode it is set
 to TAXIS\_RELATIVE. Till now the forecast mode only
 works if the output is at multiples of 1 hour
&
\tabularnewline

%\hline
 taxis\_tunit &
I&2& &
 Time unit of the TAXIS\_RELATIVE time axis.\\
 1 = TUNIT\_SECOND\\ 
 2 = TUNIT\_MINUTE\\
 5 = TUNIT\_HOUR\\
 9 = TUNIT\_DAY\\
 For a complete list of possible values see cdilib.c
& mode=1
\tabularnewline

%\hline
 \textbf{output\_bounds} &
R($k \ast$ 3)&None& &
 Post-processing times: start, end, increment.
 We choose  the advection time step matching or following the 
 requested output time, therefore we require \texttt{output\_bounds(3) > dtime}.
 Multiple triples are possible in order to define multiple starts/ends/intervals.
 See namelist parameters \texttt{output\_start}, \texttt{output\_end}, \texttt{output\_interval}
 for an alternative specification of output events.
&
\tabularnewline

%\hline
 \textbf{output\_time\_unit} &
I& 1 & &
 Units of output bounds specification.\\
 1 = second\\
 2 = minute\\
 3 = hour\\
 4 = day\\
 5 = month\\
 6 = year
&
\tabularnewline

%\hline
\textbf{output\_filename }&
C&None& &
 Output filename prefix (which may include path).
 Domain number, level type, file number and extension will be added,
 according to the format given in namelist parameter ``filename\_format''.
&
\tabularnewline

%\hline
\textbf{output\_grid} &
L&.FALSE.& &
 Flag whether grid information is added to output.
&
\tabularnewline

%\hline
 \textbf{output\_start }&
C(:)&\'' \''& &
 ISO8601 time stamp for begin of output.
 An example for this time stamp format is given below.
 More than one value is possible in order to define multiple start/end/interval triples.
 See namelist parameter \texttt{output\_bounds} for an alternative specification of output events.
&
\tabularnewline

%\hline
\textbf{output\_end }&
C(:)&\'' \''& &
 ISO8601 time stamp for end of output.
 An example for this time stamp format is given below.
 More than one value is possible in order to define multiple start/end/interval triples.
 See namelist parameter \texttt{output\_bounds} for an alternative specification of output events.
&
\tabularnewline

%\hline
\textbf{output\_interval} &
C(:)&\'' \''& &
 ISO8601 time stamp for repeating output intervals.
 We choose  the advection time step matching or following the 
 requested output time, therefore we require \texttt{output\_bounds(3) > dtime}.
 An example for this time stamp format is given below.
 More than one value is possible in order to define multiple start/end/interval triples.
 See namelist parameter \texttt{output\_bounds} for an alternative specification of output events.
&
\tabularnewline

%\hline
 pe\_placement\_il &
I(:)& -1 & &
Advanced output option:
Explicit assignment of output MPI ranks to the isentropic level output file.
At most \texttt{stream\_partitions\_il} different ranks can be specified.
See namelist parameter \texttt{pe\_placement\_ml} for further details.
&
\tabularnewline

%\hline
 pe\_placement\_hl &
I(:)& -1 & &
Advanced output option:
Explicit assignment of output MPI ranks to the height level output file.
At most \texttt{stream\_partitions\_hl} different ranks can be specified.
See namelist parameter \texttt{pe\_placement\_ml} for further details.
&
\tabularnewline

%\hline
 pe\_placement\_ml &
I(:)& -1 & &
Advanced output option:
Explicit assignment of output MPI ranks to the model level output file.
At most \texttt{stream\_partitions\_ml} different ranks can be specified,
out of the following list: \texttt{0} $\ldots$ (\texttt{num\_io\_procs} - 1).
If this namelist parameters is not provided, then the output ranks are chosen
in a Round-Robin fashion among those ranks that are not occupied by explicitly
placed output files.
&
\tabularnewline

%\hline
 pe\_placement\_pl &
I(:)& -1 & &
Advanced output option:
Explicit assignment of output MPI ranks to the pressure level output file.
At most \texttt{stream\_partitions\_pl} different ranks can be specified.
See namelist parameter \texttt{pe\_placement\_ml} for further details.
&
\tabularnewline

%\hline
 ready\_file &
 C & \''default\'' & &
 A \emph{ready file} is a technique for handling dependencies between the NWP processes.
 The completion of the write process is signalled by creating a small file 
 with name \texttt{ready\_file}.
 Different \texttt{output\_nml}'s may be joined together to form a single ready file event.
 The setting of \texttt{ready\_file = "default"} does not create a ready file.
 The ready file name may contain string tokens \texttt{<path>}, \texttt{<datetime>}, \texttt{<ddhhmmss>}
 which are substituted as described for the namelist parameter \texttt{filename\_format}.
&
\tabularnewline

%\hline
 reg\_def\_mode &
I&0& &
Specify if the "delta" value prescribes an interval size or
the total *number* of intervals:
0: switch automatically between increment and no. of grid points,
1: \texttt{reg\_lon/lat\_def(2)} specifies increment,
2: \texttt{reg\_lon/lat\_def(2)} specifies no. of grid points.
&
remap=1
\tabularnewline

%\hline
 \textbf{remap }&
I&0& &
 interpolate horizontally\\ 
0: none\\
1: to regular lat-lon grid
&
\tabularnewline

%\hline
 north\_pole &
R(2)&0,90& &
 definition of north pole for rotated lon-lat grids (\texttt{[longitude, latitude]}.
&
\tabularnewline

%\hline
 \textbf{reg\_lat\_def} &
R(3)&None& &
 start, increment, end latitude in degrees.
 Alternatively, the user may set the number of grid points instead of an increment.
 Details for the setting of regular grids is given below together with an example. 
&
remap=1
\tabularnewline

%\hline
 \textbf{reg\_lon\_def} &
R(3)&None& &
 The regular grid points are specified by three values: start, increment, end given in degrees.
 Alternatively, the user may set the number of grid points instead of an increment.
 Details for the setting of regular grids is given below together with an example. 
&
remap=1
\tabularnewline

%\hline
\textbf{steps\_per\_file }&
I&-1& &
 Max number of output steps in one output file. If this number is reached, a new output
 file will be opened.
&
\tabularnewline

%\hline
 steps\_per\_file\_inclfirst &
L&see descr.& &
 Defines if first step is counted wrt. \texttt{steps\_per\_file} files count.
 The default is \texttt{.FALSE.} for GRIB2 output, and \texttt{.TRUE.} otherwise.
&
\tabularnewline

%\hline
 stream\_partitions\_hl &
I&1& &
Splits height level output of this namelist into several concurrent alternating files.
See namelist parameter \texttt{stream\_partitions\_ml} for details.
&
\tabularnewline

%\hline
 stream\_partitions\_il &
I&1& &
Splits isentropic level output of this namelist into several concurrent alternating files.
See namelist parameter \texttt{stream\_partitions\_ml} for details.
&
\tabularnewline

%\hline
 stream\_partitions\_ml &
I&1& &
Splits model level output of this namelist into several concurrent alternating files.
The output is split into $N$ files, where the start date of part $i$ gets an offset
of $(i-1)*\texttt{output\_interval}$. 
The output interval is then replaced by $N*\texttt{output\_interval}$,
the \texttt{include\_last} flag is set to \texttt{.FALSE.}, 
the \texttt{steps\_per\_file\_inclfirst} flag is set to \texttt{.FALSE.}, 
and
the \texttt{steps\_per\_file} counter is set to \texttt{1}.
&
\tabularnewline

%\hline
 stream\_partitions\_pl &
I&1& &
Splits pressure level output of this namelist into several concurrent alternating files.
See namelist parameter \texttt{stream\_partitions\_ml} for details.
&
\tabularnewline


%\hline
 rbf\_scale &
R&-1.& &
Explicit setting of RBF shape parameter for interpolated lon-lat output. 
This namelist parameter is only active in combination with interpol\_nml:rbf\_scale\_mode\_ll=3.
&
interpol\_nml:rbf\_scale\_mode\_ll=3
\tabularnewline

%\hline
\end{longtab}

Defined and used in: \verb+src/io/shared/mo_name_list_output_init.f90+

% -------------------------------------------------------------
\paragraph{Interpolation onto regular grids:}
% -------------------------------------------------------------

Horizontal interpolation onto regular grids is possible through the namelist setting  \texttt{remap=1}, where
the mesh is defined by the parameters
\begin{itemize}
  \item \texttt{reg\_lon\_def}: mesh latitudes in degrees,
  \item \texttt{reg\_lat\_def}: mesh longitudes in degrees,
  \item \texttt{north\_pole}: definition of north pole for rotated lon-lat grids.
\end{itemize}
The regular grid points in \texttt{reg\_lon\_def}, \texttt{reg\_lat\_def} are each specified by three values, given in degrees:
\emph{start}, \emph{increment}, \emph{end}.
The mesh then contains all grid points $start + k * increment <= end$, where $k$ is an integer.
Instead of defining an increment it is also possible to prescribe the number of grid points.
\begin{itemize}
  \item Setting the namelist parameter \texttt{reg\_def\_mode=0}: 
        Switch automatically from increment specification to no. of grid points,
        when the  \texttt{reg\_lon/lat\_def(2)} value is larger than 5.0.
  \item 1: \texttt{reg\_lon/lat\_def(2)} specifies increment
  \item 2: \texttt{reg\_lon/lat\_def(2)} specifies no. of grid points
\end{itemize}
For longitude values the last grid point is omitted if the end point matches the start point, e.g. for 0 and 360 degrees.

\begin{tabbing}
  \parbox{0.7\textwidth}{\textbf{Examples}} \= \\
  local grid with 0.5 degree increment: \>
  \texttt{reg\_lon\_def = -30.,0.5,30.}
  \\ \>
  \texttt{reg\_lat\_def = 90.,-0.5, -90.}
  %
  \\[0.5em]
  global grid with 720x361 grid points: \>
  \texttt{reg\_lon\_def = 0.,720,360.}
  \\ \>
  \texttt{reg\_lat\_def = -90.,360,90.}
\end{tabbing}


% -------------------------------------------------------------
\paragraph{Time stamp format:}
% -------------------------------------------------------------

The namelist parameters \texttt{output\_start}, \texttt{output\_end}, \texttt{output\_interval} allow
the specification of time stamps according to ISO 8601.
The general format for time stamps is \texttt{YYYY-MM-DDThh:mm:ss}
where \texttt{Y}: year, \texttt{M}: month, \texttt{D}: day for dates, 
and   \texttt{hh}: hour, \texttt{mm}: minute, \texttt{ss}: second for time strings.  
The general format for durations is \texttt{PnYnMnDTnHnMnS}.
See, for example, \texttt{http://en.wikipedia.org/wiki/ISO\_8601} for details and further specifications.

\color{red}NOTE: as the mtime library underlaying the output driver
  currently has some restrictions concerning the specification of durations:\begin{enumerate}
\item Any number \texttt{n} in \texttt{PnYnMnDTnHnMnS} must have two digits. For instance use \texttt{"PT06H"} instead of \texttt{"PT6H"}
\item In a duration string \texttt{PnyearYnmonMndayDTnhrHnminMnsecS} the numbers \texttt{nxyz} must not pass the carry over number to the next larger time unit: 0<=nmon<=12, 0<=nhr<=23, 0<=nmin<=59, 0<=nsec<=59.999. For instance use \texttt{"P01D"} instead of \texttt{"PT24H"}, or \texttt{"PT01M"} instead of \texttt{"PT60S"}.
\end{enumerate}

Soon the formatting problem will be resolved and the valid number ranges will be enlarged.
(2013-12-16).\color{black}

\begin{tabbing}
  \parbox{0.7\textwidth}{\textbf{Examples}} \= \\
  date and time representation (\texttt{output\_start}, \texttt{output\_end}) \>
  \texttt{2013-10-27T13:41:00Z} \\
  duration (\texttt{output\_interval}) \>
  \texttt{P00DT06H00M00S} \\
\end{tabbing}



% -------------------------------------------------------------
\subsubsection*{Variable Groups}
% -------------------------------------------------------------
\paragraph{Keyword \texttt{"group:"}:}
Using the \texttt{"group:"} keyword for the namelist parameters \texttt{ml\_varlist}, \texttt{hl\_varlist}, \texttt{pl\_varlist},
sets of common variables can be added to the output:
\begin{tabbing}
\hspace*{0.4\textwidth} \= \kill
\texttt{group:all}                     \>      output of all variables (caution: do not combine with \underline{mixed} vertical interpolation) \\
\texttt{group:atmo\_ml\_vars}          \>      basic atmospheric variables on model levels                          \\
\texttt{group:atmo\_pl\_vars}          \>      same set as atmo\_ml\_vars, but except pres\\
\texttt{group:atmo\_zl\_vars}          \>      same set as atmo\_ml\_vars, but expect height\\
\texttt{group:nh\_prog\_vars}          \>      additional prognostic variables of the nonhydrostatic model          \\
\texttt{group:atmo\_derived\_vars}     \>      derived atmospheric variables                                        \\
\texttt{group:rad\_vars}               \>                                                                           \\
\texttt{group:precip\_vars}            \>                                                                           \\
\texttt{group:cloud\_diag}             \>                                                                           \\
\texttt{group:pbl\_vars}               \>                                                                           \\
\texttt{group:phys\_tendencies}        \>                                                                           \\
\texttt{group:land\_vars}              \>                                                                           \\
\texttt{group:snow\_vars}              \>     snow variables                                                        \\
\texttt{group:multisnow\_vars}         \>     multi-layer snow variables                                            \\
\texttt{group:additional\_precip\_vars}\>                                                                           \\
\texttt{group:dwd\_fg\_atm\_vars}      \>     DWD first guess fields (atmosphere)                                   \\
\texttt{group:dwd\_fg\_sfc\_vars}      \>     DWD first guess fields (surface/soil)                                 \\
\texttt{group:ART\_AERO\_VOLC}         \>     ART volcanic ash fields                                               \\
\texttt{group:ART\_AERO\_RADIO}        \>     ART radioactive tracer fields                                         \\
\texttt{group:ART\_AERO\_DUST}         \>     ART mineral dust aerosol fields                                       \\
\texttt{group:ART\_AERO\_SEAS}         \>     ART sea salt aerosol fields                                           \\
\texttt{group:prog\_timemean}          \>     time mean output: temp, u, v, rho                                     \\
\texttt{group:tracer\_timemean}        \>     time mean output: qv, qc, qi                                          \\
\texttt{group:echam\_timemean}         \>     time mean output: most echam surface variables                        \\
\texttt{group:atmo\_timemean}          \>     time mean variables from \texttt{prog\_timemean},\texttt{tracer\_timemean}, \texttt{echam\_timemean}\\                                                  \\
\end{tabbing}

\paragraph{Keyword \texttt{"tiles:"}:}
The \texttt{"tiles:"} keyword allows to add all tiles of a specific variable to the output, without the need to specify all 
tile fields separately. E.g.\ \texttt{"tiles:t\_g"} (read: ``tiles of \texttt{t\_g}'') automatically adds all \texttt{t\_g\_t\_X} 
fields to the output. Here, \texttt{X} is a placeholder for the tile number. Make sure to specify the name of the aggregated 
variable rather than the name of the corresponding tile container (i.e.\ in the given example it must be \texttt{t\_g}, and not 
\texttt{t\_g\_t}!).

\begin{note}
      There exists a special syntax which allows to remove variables from the output list, e.\,g.\ if
      these undesired variables were contained in a previously selected group.\\
      Typing \texttt{"-\textnormal{<varname>}"} (for example \texttt{"-temp"}) removes the
      variable from the union set of group variables and other selected variables.
      Note that typos are not detected but that the corresponding variable is
      simply not removed!
\end{note}

\paragraph{Keyword substitution in output filename (\texttt{filename\_format}):}
\begin{tabbing}
\hspace*{0.4\textwidth} \= \kill
\texttt{path}              \>  substituted by \texttt{model\_base\_dir}                 \\
\texttt{output\_filename}  \>  substituted by \texttt{output\_filename}                 \\
\texttt{physdom}           \>  substituted by physical patch ID                         \\
\texttt{levtype}           \>  substituted by level type ``ML'', ``PL'', ``HL'', ``IL'' \\
\texttt{levtype\_l}        \>  like \texttt{levtype}, but in lower case                 \\
\texttt{jfile}             \>  substituted by output file counter                       \\
\texttt{datetime}          \>  substituted by ISO-8601 date-time stamp in format \texttt{YYYY-MM-DDThh:mm:ss.sssZ} \\
\texttt{datetime2}         \>  substituted by ISO-8601 date-time stamp in format \texttt{YYYYMMDDThhmmssZ}         \\
\texttt{datetime3}         \>  substituted by ISO-8601 date-time stamp in format \texttt{YYYYMMDDThhmmss.sssZ}     \\
\texttt{ddhhmmss}          \>  substituted by \emph{relative} day-hour-minute-second string \\
\texttt{hhhmmss}           \>  substituted by \emph{relative} hour-minute-second string     \\
\texttt{npartitions}       \>  If namelist is split into concurrent files: number of stream partitions.           \\
\texttt{ifile\_partition}  \>  If namelist is split into concurrent files: stream partition index of this file.   \\
\texttt{total\_index}      \>  If namelist is split into concurrent files: substituted by the file counter \\ 
                           \>  (like in \texttt{jfile}), which an "unsplit" namelist would have produced
\end{tabbing}

\newpage

%------------------------------------------------------------------------------
% parallel_nml:
%------------------------------------------------------------------------------
\subsection{parallel\_nml}
\begin{longtab}

%\hline
\textbf{nproma} &
I & 1& &
chunk length&
\tabularnewline

%\hline
n\_ghost\_rows &
I & 1& &
number of halo cell rows&
\tabularnewline

%\hline
division\_method &
I & 1& &
method of domain decomposition\\
0: read in from file \\
1: use built-in geometric subdivision&
\tabularnewline


%\hline
division\_file\_name &
C &  & &
Name of division file &
division\_method = 0
\tabularnewline

%\hline
ldiv\_phys\_dom &
L & .TRUE. & &
.TRUE.: split into physical domains before computing domain decomposition (in case of merged domains)\\
(This reduces load imbalance; turning off this option is not recommended except for very small processor numbers) &
division\_method = 1
\tabularnewline

%\hline
p\_test\_run &
L & .FALSE.& &
.TRUE. means verification run for MPI parallelization (PE 0
processes full domain) &
\tabularnewline

%\hline
l\_test\_openmp &
L & .FALSE.& &
if .TRUE. is combined with p\_test\_run=.TRUE. and OpenMP parallelization,
the test PE gets only 1 thread in order to verify the OpenMP parallelization&
p\_test\_run = .TRUE.
\tabularnewline

%\hline
l\_log\_checks &
L & .FALSE.& &
if .TRUE. messages are generated during each synchonization step
(use for debugging only)&
\tabularnewline

%\hline
l\_fast\_sum &
L & .FALSE.& &
if .TRUE., use fast (not processor-configuration-invariant) global summation&
\tabularnewline

%\hline
use\_dycore\_barrier &
L & .FALSE.& &
if .TRUE., set an MPI barrier at the beginning of the nonhydrostatic solver (do not use for production runs!)&
\tabularnewline

%\hline
itype\_exch\_barrier &
I & 0 & &
1: set an MPI barrier at the beginning of each MPI exchange call\\
2: set an MPI barrier after each MPI WAIT call \\
3: 1+2 (do not use for production runs!) &
\tabularnewline

%\hline
iorder\_sendrecv &
I & 1& &
Sequence of send/receive calls: \\
 1 = irecv/send \\
 2 = isend/recv  \\
 3 = isend/irecv
&
\tabularnewline

%\hline
itype\_comm &
I & 1& &
1: use local memory for exchange buffers \\
3: asynchronous halo communication for dynamical core (currently deactivated)&
\tabularnewline

%\hline
\textbf{num\_io\_procs} &
I & 0& &
Number of I/O processors (running exclusively for doing I/O)&
\tabularnewline

%\hline 
\textbf{num\_restart\_procs} &
I & 0& &
Number of restart processors (running exclusively for doing restart)&
\tabularnewline

%\hline
\textbf{num\_prefetch\_proc} &
I & 0& &
Number of processors for prefetching of boundary data asynchronously for
a limited area run (running exclusively for reading Input boundary
data. Maximum no of processors used for it is limited to 1). &
itype\_latbc $\ge$ 1 
\tabularnewline  

%\hline
pio\_type &
I & 1& &
Type of parallel I/O. Only used if number of I/O processors greater than number of domains.
Experimental!&
\tabularnewline

% % %\hline
% % nh\_stepping\_threads &
% % I & 1& &
% % The number of OpenMP threads to be used by the non-hydrostatic dycore. Only used if the \_\_OMP\_RADIATION\_\_
% % flag is set during compilation. Experimental! &
% % \tabularnewline
% %
% % %\hline
% % radiation\_threads &
% % I & 1& &
% % The number of OpenMP threads to be used by the radiation. Only used if the \_\_OMP\_RADIATION\_\_
% % flag is set during compilation. Experimental! &
% % \tabularnewline

%\hline
use\_icon\_comm &
L & .FALSE. & &
Enable the use of MPI bulk communication through the icon\_comm\_lib &
\tabularnewline

%\hline
icon\_comm\_debug &
L & .FALSE. & &
Enable debug mode for the icon\_comm\_lib &
\tabularnewline

%\hline
max\_send\_recv-\\
 \_buffer\_size &
I & 131072 & &
Size of the send/receive buffers for the icon\_comm\_lib. &
\tabularnewline

%\hline
use\_dp\_mpi2io &
L & .FALSE. & &
 Enable this flag if output fields shall be gathered by the output processes in DOUBLE PRECISION. &
\tabularnewline

%\hline
restart\_chunk\_size &
I & 1 & &
\emph{(Advanced namelist parameter:)}
Number of levels to be buffered by the asynchronous restart process.
The (asynchronous) restart is capable of writing and communicating
more than one 2D slice at once. &
\tabularnewline

\end{longtab}

Defined and used in: \verb+src/namelists/mo_parallel_nml.f90+

%------------------------------------------------------------------------------
% psrad_nml:
%------------------------------------------------------------------------------
\subsection{psrad\_nml}

\begin{longtab}

%\hline
lradforcing &
L(2)&
.FALSE.&
&
switch for diagnostics of aerosol forcing in the solar spectral range
(lradforcing(1)) and the thermal spectral range (lradforcing(2)). 
\tabularnewline

%hline
lw\_gpts\_ts &
I&
1&
&
number of g--points in Monte--Carlo
   spectral integration for thermal radiation, see lw\_spec\_samp
\tabularnewline

%hline
lw\_spec\_samp &
I&
1&
&
sampling of spectral bands in radiation
calculation for thermal radiation\newline
lw\_spec\_samp = 1: standard broad
band sampling\newline
lw\_spec\_samp = 2: Monte--Carlo spec-
 tral integration (MSCI); lw\_gpts\_ts
randomly chosen g--points per column
 and radiation call\newline
lw\_spec\_samp = 3: choose g--points not
 completely randomly in order to reduce
errors in the surface radiative fluxes
\tabularnewline

%hline
rad\_perm &
I&
0&
&
integer number that influences the perturba- 
  tion of the random seed from column 
  to column
\tabularnewline

%hline
sw\_gpts\_ts &
I&
1&
&
number of g--points in Monte--Carlo 
    spectral integration for solar radiation, 
    see sw\_spec\_samp 
\tabularnewline


%hline
sw\_spec\_samp&
I&
1&
&
sampling of spectral bands in radiation 
calculation for solar radiation \newline
sw\_spec\_samp = 1: standard broad 
band sampling\newline 
sw\_spec\_samp = 2: Monte--Carlo spectral integration (MSCI); lw\_gpts\_ts 
  randomly chosen g--points per column 
    and radiation call\newline 
    sw\_spec\_samp = 3: choose g--points not 
      completely randomly in order to reduce 
      errors in the surface radiative fluxes 
\tabularnewline

%hline
\end{longtab}

Defined and used in: \verb+src/echam_phy_psrad/mo_psrad_radiation.f90+
%------------------------------------------------------------------------------
% radiation_nml:
%------------------------------------------------------------------------------
\subsection{radiation\_nml}

\begin{longtab}

%\hline
ldiur &
L&
.TRUE.&
&
switch for solar irradiation: \\.TRUE.:diurnal cycle, \\.FALSE.:zonally averaged irradiation&
\tabularnewline

%\hline
nmonth &
I&
0&
&
0: Earth circles on orbit\\1-12: Earth orbit position fixed for specified month&
\tabularnewline

%\hline
lyr\_perp &
L&
.FALSE.&
&
.FALSE.: transient Earth orbit following VSOP87 \\ .TRUE.: Earth orbit of year yr\_perp of the VSOP87 orbit is perpertuated &
\tabularnewline

%\hline
yr\_perp &
L&
-99999&
&
year used for lyr\_perp = .TRUE.&
\tabularnewline

%\hline
isolrad &
I&
0&
&
Insolation scheme\\
0: Use original SRTM insolation.\\
1: Use insolation from external file containing the spectrally
resolved insolation (monthly means)\\
2: Use preindustrial insolation as in CMIP5 (average from 1844--1856)\\
3: Use insolation for AMIP--type CMIP5 simulation (average from 1979--1988)
 &
\tabularnewline

%\hline
izenith&
I&
4 &
&
Choice of zenith angle formula for the radiative transfer computation.\par
0: Sun in zenith everywhere\par
1: Zenith angle depends only on latitude\par
2: Zenith angle depends only on latitude. Local time of day fixed at 07:14:15 for radiative transfer computation (sin(time of day) = 1/pi\par
3: Zenith angle changing with latitude and time of day\par
4: Zenith angle and irradiance changing with season, latitude, and time of day (iforcing=inwp only)
&
\tabularnewline

%\hline
\textbf{albedo\_type}&
I&
1&
&
Type of surface albedo\\
1: based on soil type specific tabulated values (dry soil)\\
2: MODIS albedo
&
iforcing=inwp
\tabularnewline

%\hline
\textbf{direct\_albedo}&
I&
4&
&
Direct beam surface albedo. Options mainly differ in terms of their solar zenith angle (SZA) dependency)\\
1: SZA dependency following Ritter-Geleyn; applied to unconditionally all grid points\\
2: SZA dependency following Zaengl (pers.\ comm.). Same as 1 for water, but for 'rough surfaces' over land the direct albedo 
   is not allowed to exceed the corresponding broadband diffuse albedo.\\
3: SZA dependency following Yang (2008) for snow-free land points. Same as 1 for water/ice and 2 for snow.\\
4: SZA dependency following Briegleb (1992) for snow-free land points. Same as 1 for water/ice and 2 for snow.
&
iforcing=inwp\\
albedo\_type=2
\tabularnewline

%\hline
icld\_overlap &
I&
2&
&
Method for cloud overlap calculation in shortwave part of RRTM\\
1: maximum-random overlap \\
2: generalized overlap (Hogan, Illingworth, 2000) \\
3: maximum overlap \\
4: random overlap
&
iforcing=inwp\\
inwp\_radiation=1
\tabularnewline

%\hline
irad\_h2o\par
irad\_co2\par
irad\_ch4\par
irad\_n2o\par
irad\_o3\par
irad\_o2\par
irad\_cfc11\par
irad\_cfc12\par
&
I&
1\par
2\par
3\par
3\par
0\par
2\par
2\par
2\par
&
&
Switches for the concentration of radiative agents\par
0: 0.\par
1: prognostic variable\par
2: global constant\par
3: externally specified\par
irad\_o3 = 2: ozone climatology from MPI \par
irad\_o3 = 4: ozone clim for Aqua Planet Exp \par
irad\_o3 = 6: ozone climatology with T5 geographical distribution and Fourier series for seasonal cycle {\color{red}for run\_nml/iforcing = 3 (NWP)} \par
irad\_o3 = 7: GEMS ozone climatology (from IFS) {\color{red}for run\_nml/iforcing = 3 (NWP)} \par
irad\_o3 = 8: ozone climatology for AMIP \par
irad\_o3 = 9: MACC ozone climatology (from IFS) {\color{red}for run\_nml/iforcing = 3 (NWP)} \par
irad\_o3 = 10: Linearized ozone chemistry (ART extension necessary) {\color{red}for run\_nml/iforcing = 3 (NWP)}
&
Note: until further notice, please use \par
irad\_h2o = 1\par
irad\_co2 = 2\par
and 0 for all the other agents for run\_nml/iforcing = 2 (ECHAM).\par
\tabularnewline
%\hline
vmr\_co2\par
vmr\_ch4\par
vmr\_n2o\par
vmr\_o2\par
vmr\_cfc11\par
vmr\_cfc12\par
&
R&
348.0e-6\par
\mbox{1650.0e-9}\par
306.0e-9\par
0.20946\par
\mbox{214.5e-12}\par
\mbox{371.1e-12}
&
&
Volume mixing ratio of the radiative agents&
\tabularnewline

%\hline
irad\_aero &
I&
2&
&
Aerosols\\
1: prognostic variable\\
2: global constant\\
3: externally specified\\
5: Tanre aerosol climatology {\color{red}for run\_nml/iforcing = 3 (NWP) }\\
6: Tegen aerosol climatology {\color{red}for run\_nml/iforcing = 3 (NWP) .AND. itopo =1 }\\
9: ART online aerosol radiation interaction, uses Tegen for aerosols not chosen to be represented in ART {\color{red}for run\_nml/iforcing = 3 (NWP) .AND. itopo =1 .AND. lart=TRUE .AND. iart\_ari=1}
&
\tabularnewline

%\hline
lrad\_aero\_diag &
L&
.FALSE.&
&
writes actual aerosol optical properties to output &
\tabularnewline

%\hline
ighg &
I&
0&
&
Select dynamic greenhouse gases scenario (read from file)\\
0 : select default gas volume mixing ratios - 1990 values (CMIP5)\\
1 : transient CMIP5 scenario from file & run\_nml/iforcing=2 (ECHAM)
\tabularnewline

%\hline
\end{longtab}

Defined and used in: \verb+src/namelists/mo_radiation_nml.f90+


%------------------------------------------------------------------------------
% run_nml
%------------------------------------------------------------------------------
\subsection{run\_nml}

\begin{longtab}

%\hline
\textbf{nsteps} &
I & -999 & &
Number of time steps of this run. Allowed range is $\ge0$; setting a value
of 0 allows writing initial output (including internal remapping) without 
calculating time steps.&
\tabularnewline

%\hline
\textbf{dtime} &
R & 600.0& s&
time step.\\
For real case runs the maximum allowable time step can be estimated as\\
 $1.8 \cdot \mathrm{ndyn\_substeps} \cdot \overline{\Delta x}\,\mathrm{s\, km^{-1}}$,\\
where $\overline{\Delta x}$ is the average resolution in $\mathrm{km}$ and
$\mathrm{ndyn\_substeps}$ is the number of dynamics substeps set in
nonhydrostatic\_nml.  ndyn\_substeps should not be increased beyond the
default value $5$.&
\tabularnewline


%\hline
\textbf{ltestcase} &
L & .TRUE.& &
Idealized testcase runs&
\tabularnewline

%\hline
\textbf{ldynamics}&
L& .TRUE.& &
Compute adiabatic dynamic tendencies&
\tabularnewline

%\hline
\textbf{iforcing}&
I&
0&
&
Forcing of dynamics and transport by parameterized processes. Use positive indices for the
atmosphere and negative indices for the ocean.\\
0: no forcing\\
1: Held-Suarez forcing\\
2: ECHAM forcing\\
3: NWP forcing\\
4: local diabatic forcing without physics\\
5: local diabatic forcing with physics\\
-1: MPIOM forcing (to be done)&
\tabularnewline

%\hline
\textbf{ltransport}&
L& .FALSE.& &
Compute large-scale tracer transport&
\tabularnewline

%\hline
\textbf{ntracer}&
I& 0& &
Number of advected tracers handled by the large-scale transport scheme&
\tabularnewline

%\hline
\textbf{lvert\_nest} &
L & .FALSE.& &
If set to .true. vertical nesting is switched on (i.e.\ variable number of vertical levels) &
\tabularnewline

%\hline
\textbf{num\_lev}&
I(max\_ dom)& 31& &
Number of full levels (atm.) for each domain&
lvert\_nest=.TRUE.
\tabularnewline

%\hline
nshift&
I(max\_ dom)& 0& &
vertical half level of parent domain which coincides with
upper boundary of the current domain \textcolor{red}{required for vertical refinement, which is not yet implemented}&
lvert\_nest=.TRUE.
\tabularnewline

%\hline
ltimer&
L & .TRUE.& &
TRUE: Timer for monitoring the runtime of specific routines is on (FALSE = off)&
\tabularnewline

%\hline
timers\_level&
I & 1& &
&
\tabularnewline

%\hline
activate\_sync\_timers&
L & F& &
TRUE: Timer for monitoring runtime of communication routines (FALSE = off)&
\tabularnewline

%\hline
\textbf{msg\_level}&
I & 10& &
controls how much printout is written during runtime. \\
For values less than 5, only the time step is written.&
\tabularnewline

%\hline
msg\_timestamp&
L & .FALSE.& &
If .TRUE., precede output messages by time stamp.&
\tabularnewline

%\hline
test\_mode&
I & 0& &
Setting a value larger than 0 activates a dummy mode in which time stepping is changed
into just doing iterations, and MPI communication is replaced by copying some value from
the send buffer into the receive buffer (does not work with nesting and reduced radiation
grid because the send buffer may then be empty on some PEs) & iequations = 3
\tabularnewline

%\hline
debug\_check\_level&
I & 0& &
Setting a value larger than 0 activates debug checks.
& 
\tabularnewline

%\hline
\textbf{output} &
C(:)& ``nml'', ''totint'' & &
Main switch for enabling/disabling components of the model output. One or more choices can be set (as an array of string constants). Possible choices are:
\begin{itemize}
\item ``none'': switch off all output;
\item``nml'': new output mode (cf.\ \texttt{output\_nml});
\item``totint'': computation of total integrals.
\item``maxwinds'': write max. winds to separate ASCII file ''maxwinds.log''.
\end{itemize}
If the \texttt{output} namelist parameter is not set explicitly, the default setting ``nml'',''totint'' is assumed.
 &
\tabularnewline

\hline
restart\_filename&
C &
&
&
File name for restart/checkpoint files (containing keyword
substitution patterns \texttt{<gridfile>}, \texttt{<idom>}, \texttt{<rsttime>}, \texttt{<mtype>}).
default: ''\texttt{<gridfile>\_restart\_<mtype>\_<rsttime>.nc}''.
&
\tabularnewline

\hline
profiling\_output&
I & 1& &
controls how profiling printout is written: 
TIMER\_MODE\_AGGREGATED=1, 
TIMER\_MODE\_DETAILED=2,
TIMER\_MODE\_WRITE\_FILES=3.
&
\tabularnewline

\hline
lart&
L & .FALSE.& &
Main switch which enables the treatment of atmospheric aerosol and trace gases (The ART package of KIT is needed for this purpose)
&
\tabularnewline

\hline
check\_uuid\_gracefully&
L & .FALSE.& &
If this flag is set to \texttt{.TRUE.} we give only warnings for non-matching UUIDs.
&
\tabularnewline

\end{longtab}

Defined and used in: \verb+src/namelists/mo_run_nml.f90+



%-----------------------------------------------------------------------------
% sleve_nml:
%-----------------------------------------------------------------------------
\subsection{sleve\_nml (relevant if nonhydrostatic\_nml:ivctype=2)}
\begin{longtab}

%\hline
\textbf{min\_lay\_thckn}&
R& 50& m&
Layer thickness of lowermost layer; specifying zero or a negative value leads to constant layer thicknesses
determined by top\_height and nlev&
\tabularnewline

%\hline
max\_lay\_thckn&
R& 25000& m&
Maximum layer thickness below the height given by htop\_thcknlimit (NWP recommendation: 400 m) \\
{\it Use with caution! Too ambitious settings may result in numerically unstable layer configurations.}&
\tabularnewline

%\hline
htop\_thcknlimit&
R& 15000& m&
Height below which the layer thickness does not exceed max\_lay\_thckn &
\tabularnewline

%\hline
\textbf{top\_height}&
R& 23500.0& m&
Height of model top&
\tabularnewline

%\hline
stretch\_fac&
R& 1.0& &
Stretching factor to vary distribution of model levels;
values $<$1 increase the layer thickness near the model top&
\tabularnewline

%\hline
decay\_scale\_1&
R& 4000& m&
Decay scale of large-scale topography component&
\tabularnewline

%\hline
decay\_scale\_2&
R& 2500& m&
Decay scale of small-scale topography component&
\tabularnewline

%\hline
decay\_exp&
R& 1.2& &
Exponent of decay function&
\tabularnewline

%\hline
\textbf{flat\_height} &
R& 16000& m&
Height above which the coordinate surfaces are flat&
\tabularnewline

%\hline
lread\_smt &
L& .FALSE.& &
read smoothed topography from file (TRUE) or compute internally (FALSE)&
\tabularnewline

%\hline
\end{longtab}

Defined and used in: \verb+src/namelists/mo_sleve_nml.f90+


%-----------------------------------------------------------------------------
% synsat_nml:
%-----------------------------------------------------------------------------
\subsection[synsat\_nml]{synsat\_nml%
\footnote{Important note: This feature is currently active for configuration \texttt{dwd+cray} only.}}

This namelist enables the RTTOV library incorporated into ICON for
simulating satellite radiance and brightness temperatures.
%
RTTOV is a radiative transfer model for nadir-viewing passive visible, infrared
and microwave satellite radiometers, spectrometers and
interferometers, see
\begin{center}
\verb+https://nwpsaf.eu/deliverables/rtm+
\end{center}
for detailed information.

\begin{longtab}

%\hline
{lsynsat}&
L (max\_dom) & .FALSE.& 
& Main switch: Enables/disables computation of synthetic satellite imagery for each model domain.
&
\tabularnewline

%\hline
{nlev\_rttov}&
I& 51&
& Number of RTTOV levels.
&
\tabularnewline

%\hline
\end{longtab}

Enabling the synsat module makes the following 32 two-dimensional output fields available:
%
{\renewcommand{\baselinestretch}{1.25}\normalsize%
\begin{tabbing}
\hspace*{14em} \= \hspace*{14em} \= \hspace*{14em} \= \hspace*{14em} \kill
\verb+SYNMSG_RAD_CL_IR3.9+   \>  \verb+SYNMSG_BT_CL_IR3.9+    \>
\verb+SYNMSG_RAD_CL_WV6.2+   \>   \verb+SYNMSG_BT_CL_WV6.2+   \\
\verb+SYNMSG_RAD_CL_WV7.3+   \>   \verb+SYNMSG_BT_CL_WV7.3+   \>
\verb+SYNMSG_RAD_CL_IR8.7+   \>   \verb+SYNMSG_BT_CL_IR8.7+   \\
\verb+SYNMSG_RAD_CL_IR9.7+   \>   \verb+SYNMSG_BT_CL_IR9.7+   \>
\verb+SYNMSG_RAD_CL_IR10.8+  \>   \verb+SYNMSG_BT_CL_IR10.8+  \\
\verb+SYNMSG_RAD_CL_IR12.1+  \>   \verb+SYNMSG_BT_CL_IR12.1+  \>
\verb+SYNMSG_RAD_CL_IR13.4+  \>   \verb+SYNMSG_BT_CL_IR13.4+  \\
\verb+SYNMSG_RAD_CS_IR3.9+   \>   \verb+SYNMSG_BT_CS_IR3.9+   \>
\verb+SYNMSG_RAD_CS_WV6.2+   \>   \verb+SYNMSG_BT_CS_WV6.2+   \\
\verb+SYNMSG_RAD_CS_WV7.3+   \>   \verb+SYNMSG_BT_CS_WV7.3+   \>
\verb+SYNMSG_RAD_CS_IR8.7+   \>   \verb+SYNMSG_BT_CS_IR8.7+   \\
\verb+SYNMSG_RAD_CS_IR9.7+   \>   \verb+SYNMSG_BT_CS_IR9.7+   \>
\verb+SYNMSG_RAD_CS_IR10.8+  \>   \verb+SYNMSG_BT_CS_IR10.8+  \\
\verb+SYNMSG_RAD_CS_IR12.1+  \>   \verb+SYNMSG_BT_CS_IR12.1+  \>
\verb+SYNMSG_RAD_CS_IR13.4+  \>   \verb+SYNMSG_BT_CS_IR13.4+
\end{tabbing}
}
Here, 
\verb+RAD+ denotes radiance, 
\verb+BT+ brightness temperature,
\verb+CL+ cloudy, and
\verb+CS+ clear sky, 
supplemented by the channel name.

Defined and used in: \verb+src/namelists/mo_synsat_nml.f90+


%------------------------------------------------------------------------------
% time_nml
%------------------------------------------------------------------------------
\subsection{time\_nml}
\begin{longtab}

%\hline
calendar&
I& 1& &
Calendar type: \\
0=Julian/Gregorian \\
1=proleptic Gregorian\\
2=30day/month, 360day/year &
\tabularnewline

%\hline
dt\_restart &
R & 86400.*30.& s &
Length of restart cycle in seconds.
This namelist parameter specifies how long the model runs until it saves its
state to a file and stops.
Later, the model run can be resumed, s.\,t.\ a simulation over
a long period of time can be split into a chain of restarted model runs.

Note that the frequency of writing restart files is controlled by
\texttt{io\_nml:dt\_checkpoint}. 
Only if the value of \texttt{dt\_checkpoint} resulting from
model default or user's specification is longer than \texttt{dt\_restart},
it will be reset (by the model) to \texttt{dt\_restart} so
that at least one restart file is generated during the restart cycle.
If \texttt{dt\_restart} is larger than but not a multiple of \texttt{dt\_checkpoint},
restart file will \emph{not} be generated at the end of the restart cycle.
&
\tabularnewline

%\hline
ini\_datetime\_string &
C & '2008- 09-01T 00:00:00Z' &&
Initial date and time of the simulation&
\tabularnewline

%\hline
end\_datetime\_string &
C & '2008- 09-01T 01:40:00Z' &&
End date and time of the simulation&
\tabularnewline

%\hline
is\_relative\_time &
L & .FALSE. &&
.TRUE., if time loop shall start with
step 0 regardless whether we are in a standard run or in a
restarted run (which means re-initialized run).
&
\tabularnewline

\end{longtab}

% -------------------------------------------------------------
\paragraph{Length of the run}
% -------------------------------------------------------------

If "nsteps" in run\_nml is positive, then nsteps*dtime
is used to compute the end date and time of the run.

Else the initial date and time, the end date and time, dt\_restart,
as well as the time step are used to compute "nsteps".

\newpage

%------------------------------------------------------------------------------
% transport_nml:
%------------------------------------------------------------------------------
\subsection{transport\_nml (used if run\_nml/ltransport=.TRUE.)}

\begin{longtab}

%\hline
\textbf{lvadv\_tracer}&
L& .TRUE.& & TRUE : compute vertical tracer advection& \tabularnewline
& &       & & FALSE: do not compute vertical tracer advection &
\tabularnewline

%\hline
\textbf{ihadv\_tracer}&
I(ntracer)&
2& & Tracer specific method to compute horizontal advection:& \tabularnewline
& & & & 0: no horiz. transport (note that the specific tracer quantity $q$ is kept constant and not tracer mass $\rho q$) & \tabularnewline
& & & & 1: upwind (1st order) & \tabularnewline
& & & & 2: Miura (2nd order, linear reconstr.)&  \tabularnewline
& & & & 3: Miura3 (quadr. or cubic reconstr.)& lsq\_high\_ord $\in$ [2,3] \tabularnewline
& & & & 4: FFSL (quadr. or cubic reconstr.)& lsq\_high\_ord $\in$ [2,3] \tabularnewline
& & & & 5: hybrid Miura3/FFSL (quadr. or cubic reconstr.)& lsq\_high\_ord $\in$ [2,3] \tabularnewline
& & & & 20: miura (2nd order, lin. reconstr.) with subcycling&  \tabularnewline
& & & & 22: combination of miura and miura with subcycling&  \tabularnewline
& & & & 32: combination of miura3 and miura with subcycling&  \tabularnewline
& & & & 42: combination of FFSL and miura with subcycling& \tabularnewline
& & & & 52: combination of hybrid FFSL/Miura3 with subcycling \\

Subcycling means that the integration from time step n to n+1 is splitted into substeps to meet the stability requirements. 
For NWP runs, substepping is generally applied above $z=22\,\mathrm{km}$ (see nonhydrostatic\_nml/hbot\_qvsubstep).
& \tabularnewline


%\hline
ivadv\_tracer&
I(ntracer)&
3& & Tracer specific method to compute vertical advection:& lvadv\_tracer=TRUE \tabularnewline
& & & & 0: no vert. transport (note that tracer mass $\rho q$ instead of the specific tracer quantity $q$ is kept constant. This differs from the behaviour in horizontal direction!) & \tabularnewline
& & & & 1: upwind (1st order)& \tabularnewline
& & & & 3: ppm\_cfl ($3^{\mathrm{rd}}$ order, handles $\mathrm{CFL}>1$)& \tabularnewline
& & & & 30: ppm (3rd order, CFL$<=$1)& \tabularnewline

%\hline
iadv\_tke& 
I& 
0& & Type of TKE advection & inwp\_turb=1 \tabularnewline
&  & & & 0: no TKE advection & \tabularnewline
&  & & & 1: vertical advection only & \tabularnewline
&  & & & 2: vertical and horizontal advection &\tabularnewline


%\hline
lstrang &
L& .FALSE.& & Time splitting method & \tabularnewline
& & & & TRUE: second order Strang splitting & \tabularnewline
& & & & FALSE: first order Godunov splitting & \tabularnewline

%\hline
ctracer\_list&
C& ''& & Two purposes: & \tabularnewline
 & &   &  & 
 - used for selecting those tracers which should be initialized for idealized testcases. & [nh/ha]\_test\_name=\\'jabw','PA','DF'
\tabularnewline
 & &   &  & 
 - used for tracer output names. In some idealized cases tracers are named 'Q\emph{x}', with \emph{x} being a 1-digit integer taken from \texttt{ctracer\_list}. & iforcing$\ne$ inwp, iecham
\tabularnewline

%\hline
npassive\_tracer&
I& 0 & & number of additional passive tracers which have no sources and are transparent to any physical process (no effect).\\ 
Passive tracers are named Qpassive\_ID, where ID is a number between \texttt{ntracer} and \texttt{ntracer}$+$\texttt{npassive\_tracer}.\\ 
\textbf{NOTE:} By default, limiters are switched of for passive tracers and the scheme $52$ is selected for horizontal advection. & 
\tabularnewline

%\hline
init\_formula&
C& ' ' & & Comma-separated list of initialization formulas for additional passive tracers. & npassive\_tracer > 0
\tabularnewline

%\hline
\textbf{itype\_hlimit}&
I(ntracer)&
4& & Type of limiter for horizontal transport:& \tabularnewline
& & & & 0: no limiter& \tabularnewline
& & & & 3: monotonous flux limiter& \tabularnewline
& & & & 4: positive definite flux limiter& \tabularnewline

%\hline
\textbf{itype\_vlimit}&
I(ntracer)&
1& & Type of limiter for vertical transport:& \tabularnewline
& & & & 0: no limiter& \tabularnewline
& & & & 1: semi-monotone slope limiter& \tabularnewline
& & & & 2: monotonous slope limiter& \tabularnewline
& & & & 4: positive definite flux limiter& \tabularnewline

%\hline
niter\_fct&
I& 1& & number of iterations  of monotone flux correction procedure (experimental!)& itype\_hlimit = 3
\tabularnewline

%\hline
beta\_fct&
R& 1.005& & factor of allowed over-/undershooting in monotonous limiter & 
itype\_hlimit = 3
\tabularnewline

%\hline
iord\_backtraj&
I& 1& & order of backward trajectory calculation:& \tabularnewline
& & & & 1: first order& \tabularnewline
& & & & 2: second order (iterative; currently 1 iteration hardcoded; experimental!)& ihadv\_tracer='miura'
\tabularnewline

%\hline
igrad\_c\_miura&
I& 1& & Method for gradient reconstruction at cell center for 2nd order miura scheme& \tabularnewline
& & & & 1: Least-squares (linear, non-consv)& ihadv\_tracer=2\tabularnewline
& & & & 2: Green-Gauss&
\tabularnewline

%\hline
ivcfl\_max&
I& 5& &
determines stability range of vertical PPM-scheme in terms of the maximum allowable CFL-number&
ivadv\_tracer=3
\tabularnewline

%\hline
llsq\_svd&
L&
.TRUE.&
&
use QR decomposition (FALSE) or SV decomposition (TRUE) for least squares design matrix A&
\tabularnewline

%\hline
lclip\_tracer&
L& .FALSE.& & Clipping of negative values&
\tabularnewline

%\hline
\end{longtab}

Defined and used in: \verb+src/namelists/mo_advection_nml.f90+


%------------------------------------------------------------------------------
% turbdiff_nml:
%------------------------------------------------------------------------------
\subsection{turbdiff\_nml}

\begin{longtab}

%\hline
imode\_turb &
I                &      1      & &
Mode of solving the TKE equation for atmosph. layers:\\
0: diagnostic equation\\
1: prognostic equation (current version)\\
2: prognostic equation (intrinsically positive definite) & 
\tabularnewline

%\hline
imode\_tran &
I                &      0      & &
Same as \emph{imode\_turb} but only for the transfer layer &
\tabularnewline

%\hline
icldm\_turb &
I                &      2      & &
Mode of water cloud representation in turbulence for atmosph. layers:\\
-1: ignoring cloud water completely (pure dry scheme)\\
 0: no clouds considered (all cloud water is evaporated)\\
 1: only grid scale condensation possible\\
 2: also sub grid (turbulent) condensation considered  &
\tabularnewline

%\hline
icldm\_tran &
I                &      2      & &
Same as \emph{icldm\_turb} but only for the transfer layer &
\tabularnewline

%\hline
itype\_wcld &
I                &     2      & &
type of water cloud diagnosis within the turbulence scheme:\\
1: employing a scheme based on relative humitidy\\
2: employing a statistical saturation adjustment & icldm\_turb=2 or icldm\_tran=2
\tabularnewline

%\hline
itype\_sher &
I                &      0      & &
Type of shear forcing used in turbulence:\\
0: only vertical shear of horizontal wind\\
1: previous plus horizontal shear correction \\
2: previous plus shear from vertical velocity \\
3: same as option 1, but (when combined with ltkeshs=.TRUE.) scaling of
coarse-grid horizontal shear production term with $\frac{1}{\sqrt{Ri}}$ &
\tabularnewline

%\hline
ltkeshs &
L                &     .FALSE.      & &
Include correction term for coarse grids in horizontal shear production term (needed 
at non-convection-resolving model resolutions in order to get a non-negligible impact) & itype\_sher $\ge$ 1
\tabularnewline

%\hline
ltkesso &
L                &     .FALSE.      & &
Consider TKE-production by sub grid SSO wakes & inwp\_sso = 1
\tabularnewline

%\hline
ltkecon &
L                &     .FALSE.      & &
Consider TKE-production by sub grid convective plumes (inactive) & inwp\_conv = 1
\tabularnewline

%\hline
ltkeshs &
L                &     .FALSE.      & &
Consider TKE-production by separated horizontal shear eddies (inactive) &
\tabularnewline

%\hline
ltmpcor &
L                &     .FALSE.      & &
Consider thermal TKE sources in enthalpy equation &
\tabularnewline

%\hline
lsflcnd &
L                &     .TRUE.      & &
Use lower flux condition for vertical diffusion calculation (TRUE) instead of a lower concentration condition (FALSE) &
\tabularnewline

%\hline
lexpcor &
L                &     .FALSE.      & &
Explicit corrections of implicitly calculated vertical diffusion of non-conservative scalars that are involved in sub grid condensation processes &
\tabularnewline

%\hline
tur\_len &
R                &     500.0      & m &
Asymptotic maximal turbulent distance ($\kappa * tur\_len$ is the integral turbulent master length scale) &
\tabularnewline

%\hline
pat\_len &
R                &     100.0      & m &
Effective length scale of thermal surface patterns controlling TKE-production by sub grid kata/ana-batic circulations. In case of $pat\_len=0$, 
this production is switched off.&
\tabularnewline

%\hline
c\_diff &
R                &     0.2      & 1 &
Length scale factor for vertical diffusion of TKE. In case of $c\_diff=0$, TKE is not diffused vertically.&
\tabularnewline

%\hline
a\_stab &
R                &     0.0      & 1 &
Factor for stability correction of turbulent length scale. In case of $a\_stab=0$, the turbulent length scale is not reduced for stable stratification.&
\tabularnewline

%\hline
a\_hshr &
R                &     0.20      & 1 &
Length scale factor for the separated horizontal shear mode. In case of $a\_hshr=0$, this shear mode has no effect.& ltkeshs=.TRUE.
\tabularnewline

%\hline
alpha0 &
R                &     0.0123      & 1 &
Lower bound of velocity-dependent Charnock parameter & 
\tabularnewline

%\hline
alpha0\_max &
R                &     0.0335      & 1 &
Upper bound of velocity-dependent Charnock parameter. Setting this parameter to 0.0335 or higher values implies unconstrained 
velocity dependence & 
\tabularnewline

%\hline
tkhmin &
R                &     0.75      & $\mathrm{m^2/s}$ &
Scaling factor for minimum vertical diffusion coefficient (proportional to $1/\sqrt{Ri}$)
for heat and moisture &
\tabularnewline

%\hline
tkmmin &
R                &     0.75      & $\mathrm{m^2/s}$ &
Scaling factor for minimum vertical diffusion coefficient (proportional to $1/\sqrt{Ri}$)
 for momentum & 
\tabularnewline


tkmmin\_strat &
R                &     5      & $\mathrm{m^2/s}$ &
Enhanced scaling factor for minimum vertical diffusion coefficient (proportional to $1/\sqrt{Ri}$)
 for momentum, valid in the stratosphere above 30 km & 
\tabularnewline

tkhmin\_strat &
R                &     5      & $\mathrm{m^2/s}$ &
Enhanced scaling factor for minimum vertical diffusion coefficient (proportional to $1/\sqrt{Ri}$)
for heat and moisture, valid in the stratosphere above 30 km  &
\tabularnewline


%\hline
itype\_synd &
I                &     2      & &
Type of diagnostics of synoptic near surface variables:\\
1: Considering the mean surface roughness of a grid box\\
2: Considering a fictive surface roughness of a SYNOP lawn &
\tabularnewline

%\hline
rlam\_heat &
R                &     1.0     & 1 &
Scaling factor of the laminar boundary layer for heat (scalars). The larger rlam\_heat, the larger is the laminar resistance. &
\tabularnewline

%\hline
rat\_sea &
R                &     10.0     & 1 &
Ratio of laminar scaling factors for scalars over sea and land. The larger rat\_sea, the larger is the laminar resistance 
for a sea surface compared to a land surface. &
\tabularnewline

%\hline
tkesmot &
R                &     0.15     & 1 &
Time smoothing factor within $[0, 1]$ for TKE. In case of $tkesmot=0$, no smoothing is active. &
\tabularnewline

%\hline
frcsmot &
R                &     0.0     & 1 &
Vertical smoothing factor within $[0, 1]$ for TKE forcing terms. In case of $frcmot=0$, no smoothing is active. &
\tabularnewline

%\hline
imode\_frcsmot &
I                &     1     &  &
1 = apply vertical smoothing (if frcsmot$>$0) uniformly over the globe \\
2 = restrict vertical smoothing to the tropics (reduces the moist bias in the tropics while avoiding 
    adverse effects on NWP skill scores in the extratropics) &
\tabularnewline

%\hline
impl\_s &
R                &     1.20     & 1 &
Implicit weight near the surface (maximal value) &
\tabularnewline

%\hline
impl\_t &
R                &     0.75     & 1 &
Implicit weight near top of the atmosphere (minimal value) &
\tabularnewline

%\hline
lconst\_z0 &
L                &     .FALSE.      & &
TRUE: horizontally homogeneous roughness length z0 & 
\tabularnewline

%\hline
const\_z0 &
R                &     0.001      & m &
value for horizontally homogeneous roughness length z0 & lconst\_z0=.TRUE.
\tabularnewline


%\hline
ldiff\_qi &
L                &     .FALSE.     &  &
Turbulent diffusion of cloud ice, if .TRUE.
\tabularnewline




%\hline
itype\_tran &
I            & 2      &&
type of surface-atmosphere transfer& 
\tabularnewline

%\hline
lprfcor &
L                &     .FALSE.      & &
using the profile values of the lowest main level instead of the mean value of the lowest layer for surface flux calculations & 
\tabularnewline

%\hline
lnonloc &
L                &     .FALSE.      & &
nonlocal calculation of vertical gradients used for turbul. diff. & 
\tabularnewline

%\hline
lfreeslip &
L                &     .FALSE.      & &
.TRUE.: use a free-slip lower boundary condition, i.e. neither momentum nor heat/moisture fluxes (use for idealized runs only!) & 
\tabularnewline

%\hline
lcpfluc &
L                &     .FALSE.      & &
consideration of fluctuations of the heat capacity of air & 
\tabularnewline


\end{longtab}

Defined and used in: \verb+src/namelists/mo_turbdiff_nml.f90+


%------------------------------------------------------------------------------
% vdiff_nml:
%------------------------------------------------------------------------------
\subsection{vdiff\_nml}

\begin{longtab}

%\hline
lsfc\_mon\_flux &
L& .TRUE.&&
Switch on surface momentum flux.& lvdiff = .TRUE.
\tabularnewline

%\hline
lsfc\_heat\_flux &
L                & .TRUE.           & &
Switch on surface sensible and latent heat flux.& lvdiff = .TRUE.
\tabularnewline

\end{longtab}

Defined and used in: \verb+src/namelists/mo_vdiff_nml.f90+

% =============================================================================
%
\input{\basepath icon_ocean_nml_tables.tex}
%
% This LaTeX file defines ocean (and sea-ice) specific namelist parameters.
% The file is included by ICON_NML_TABLES.TEX. For atmosphere-only releases
% the ocean namelists can also easily be excluded.
%
% =============================================================================


%MMMMMMMMMMMMMMMMMMMMMMMMMMMMMMMMMMMMMMMMMMMMMMMMMMMMMMMMMMMMMMMMMMMMMMMMMMMMMM
%MMMMMMMMMMMMMMMMMMMMMMMMMMMMMMMMMMMMMMMMMMMMMMMMMMMMMMMMMMMMMMMMMMMMMMMMMMMMMM
%MMMMMMMMMMMMMMMMMMMMMMMMMMMMMMMMMMMMMMMMMMMMMMMMMMMMMMMMMMMMMMMMMMMMMMMMMMMMMM

\section{Namelist parameters for testcases (NAMELIST\_ICON)}

The ICON model code includes several experiments, so-called test cases,
for the shallow water model as well as the 3-dimensional atmosphere.
Depending on the specified experiment, initial conditions and boundary
conditions are computed internally.

%-------------------------------------------------------------------
% ha_testcase_nml:
%-------------------------------------------------------------------
\subsection{ha\_testcase\_nml (Scope: ltestcase=.TRUE. and iequations=[0,1,2] in run\_nml)}
\begin{longtab}

%\hline
ctest\_name&
C& 'JWw'& &
Name of test case: &
\tabularnewline
& & & &
'SW\_GW': gravity wave&
lshallow\_water=.TRUE.
\tabularnewline
& & & &
'USBR': unsteady solid body rotation &
lshallow\_water=.TRUE.
\tabularnewline
& & & &
'Will\_2': Williamson test 2&
lshallow\_water=.TRUE.
\tabularnewline
& & & &
'Will\_3': Williamson test 3&
lshallow\_water=.TRUE.
\tabularnewline
& & & &
'Will\_5': Williamson test 5&
lshallow\_water=.TRUE.
\tabularnewline
& & & &
'Will\_6': Williamson test 6&
lshallow\_water=.TRUE.
\tabularnewline
& & & &
'GW': gravity wave (nlev=20 only!) &
lshallow\_water=.FALSE.
\tabularnewline
& & & &
'LDF': local diabatic forcing test without physics&
lshallow\_water=.FALSE.\\and iforcing=4
\tabularnewline
& & & &
'LDF-Moist': local diabatic forcing test with physics initalised with zonal wind field &
lshallow\_water=.FALSE.,\\and iforcing=5
\tabularnewline
& & & &
'HS': Held-Suarez test &
lshallow\_water=.FALSE.
\tabularnewline
& & & &
'JWs': Jablonowski-Will. steady state&
lshallow\_water=.FALSE.
\tabularnewline
& & & &
'JWw': Jablonowski-Will. wave test&
lshallow\_water=.FALSE.
\tabularnewline
& & & &
'JWw-Moist': Jablonowski-Will. wave test including moisture&
lshallow\_water=.FALSE.
\tabularnewline
& & & &
'APE': aqua planet experiment&
lshallow\_water=.FALSE.
\tabularnewline
& & & &
'MRW': mountain induced Rossby wave&
lshallow\_water=.FALSE.
\tabularnewline
& & & &
'MRW2': modified mountain induced Rossby wave&
lshallow\_water=.FALSE.
\tabularnewline
& & & &
'PA': pure advection&
lshallow\_water=.FALSE.
\tabularnewline
& & & &
'SV': stationary vortex&
lshallow\_water=.FALSE.,
ntracer = 2
\tabularnewline
& & & &
'DF1': deformational flow test 1&
\tabularnewline
& & & &
'DF2': deformational flow test 2&
\tabularnewline
& & & &
'DF3': deformational flow test 3&
\tabularnewline
& & & &
'DF4': deformational flow test 4&
\tabularnewline
& & & &
'RH': Rossby-Haurwitz wave test&
lshallow\_water=.FALSE.
\tabularnewline

%\hline
rotate\_axis\_deg&
R& 0.0& deg&
Earth's rotation axis pitch angle&
ctest\_name= 'Will\_2', 'Will\_3', 'JWs', 'JWw', 'PA', 'DF1234'
\tabularnewline

%\hline
gw\_brunt\_vais&
R& 0.01& 1/s&
Brunt Vaisala frequency&
ctest\_name= 'GW'
\tabularnewline

%\hline
gw\_u0&
R& 0.0& m/s&
zonal wind parameter&
ctest\_name= 'GW'
\tabularnewline

%\hline
gw\_lon\_deg&
R& 180.0& deg&
longitude of initial perturbation&
ctest\_name= 'GW'
\tabularnewline

%\hline
gw\_lat\_deg&
R& 0.0& deg&
latitude of initial perturbation&
ctest\_name= 'GW'
\tabularnewline

%\hline
jw\_uptb&
R& 1.0& m/s (?)&
amplitude of the wave pertubation&
ctest\_name= 'JWw'
\tabularnewline

%\hline
mountctr\_lon\_deg&
R& 90.0& deg&
longitude of mountain peak&
ctest\_name= 'MRW(2)'
\tabularnewline

%\hline
mountctr\_lat\_deg&
R& 30.0& deg&
latitude of mountain peak&
ctest\_name= 'MRW(2)'
\tabularnewline

%\hline
mountctr\_height&
R& 2000.0& m&
mountain height&
ctest\_name= 'MRW(2)'
\tabularnewline

%\hline
mountctr\_half\_width&
R& 1500000.0& m&
mountain half width&
ctest\_name= 'MRW(2)'
\tabularnewline

%\hline
mount\_u0&
R& 20.0& m/s&
wind speed for MRW cases&
ctest\_name= 'MRW(2)'
\tabularnewline

%\hline
rh\_wavenum&
I& 4& &
wave number&
ctest\_name= 'RH'
\tabularnewline

%\hline
rh\_init\_shift\_deg&
R& 0.0& deg&
pattern shift&
ctest\_name= 'RH'
\tabularnewline

%\hline
ihs\_init\_type&
I& 1& &
Choice of initial condition for the Held-Suarez test. 1: the zonal
state defined in the JWs test case; other integers: isothermal state
(T=300~K, ps=1000 hPa, u=v=0.)&
ctest\_name= 'HS'
\tabularnewline

%\hline
lhs\_vn\_ptb&
L& .TRUE.& &
Add random noise to the initial wind field in the Held-Suarez test. &
ctest\_name= 'HS'
\tabularnewline

%\hline
hs\_vn\_ptb\_scale&
R& 1.& m/s &
Magnitude of the random noise added to the initial wind field in the
Held-Suarez test. &
ctest\_name= 'HS'
\tabularnewline

%\hline
lrh\_linear\_pres&
L& .FALSE.& &
Initialize the relative humidity using a linear function of pressure. &
ctest\_name= 'JWw-Moist','APE', 'LDF-Moist'
\tabularnewline

%\hline
rh\_at\_1000hpa&
R& 0.75& &
relative humidity \[0,1\] at 1000 hPa &
ctest\_name= 'JWw-Moist','APE', 'LDF-Moist'
\tabularnewline

%\hline
linit\_tracer\_fv&
L& .TRUE.& &
Finite volume initialization for tracer fields &
ctest\_name='PA'
\tabularnewline

%\hline
ape\_sst\_case&
C& 'sst1'& &
SST distribution selection\\
'sst1': Control experiment\\
'sst2': Peaked experiment\\
'sst3': Flat experiment\\
'sst4': Control-5N experiment\\
'sst\_qobs': Qobs SST distribution exp\\
'sst\_ice': Control SST distribution with -1.8 C above 64 N/S.
&ctest\_name='APE'
\tabularnewline

%\hline
ildf\_init\_type&
I& 0& &
Choice of initial condition for the Local diabatic forcing test. 1: the zonal
state defined in the JWs test case; other: isothermal state
(T=300 K, ps=1000 hPa, u=v=0.)&
ctest\_name= 'LDF'
\tabularnewline

%\hline
ldf\_symm&
L& .TRUE.& &
Shape of local diabatic forcing:\\
.TRUE.: local diabatic forcing symmetric about the equator (at 0 N)\\
.FALSE.: local diabatic forcing asym. about the equator (at 30 N)
& ctest\_name= 'LDF','LDF-Moist'
\tabularnewline

\end{longtab}

Defined and used in: \verb+src/testcases/mo_ha_testcases.f90+



%-------------------------------------------------------------------------------
% nh_testcase_nml:
%-------------------------------------------------------------------------------
\subsection{nh\_testcase\_nml (Scope: ltestcase=.TRUE. and iequations=3 in run\_nml)}

\begin{longtab}

%\hline
nh\_test\_name&
C& 'jabw'& &
testcase selection&
\tabularnewline
&&&& '\textbf{zero}': no orography&
\tabularnewline
&&&& '\textbf{bell}': bell shaped mountain at 0E,0N&
\tabularnewline
&&&& '\textbf{schaer}': hilly mountain at 0E,0N&
\tabularnewline
&&&& '\textbf{jabw}': Initializes the full Jablonowski Williamson test case.&
\tabularnewline
&&&&'\textbf{jabw\_s}': Initializes the Jablonowski Williamson steady state test case.&
\tabularnewline
&&&&'\textbf{jabw\_m}': Initializes the Jablonowski Williamson test case with a mountain instead of the wind perturbation (specify mount\_height).&
\tabularnewline
&&&&'\textbf{mrw\_nh}': Initializes the full Mountain-induced Rossby wave test case.&
\tabularnewline
&&&&'\textbf{mrw2\_nh}': Initializes the modified mountain-induced Rossby wave test case.&
\tabularnewline
&&&&'\textbf{mwbr\_const}': Initializes the mountain wave with two layers test case.
The lower layer is isothermal and the upper layer has constant brunt
vaisala frequency. The interface has constant pressure.&
\tabularnewline
&&&&'\textbf{PA}': Initializes the pure advection test case.&
\tabularnewline
&&&&'\textbf{HS\_nh}': Initializes the Held-Suarez test case. At the moment
 with an isothermal atmosphere at rest (T=300K, ps=1000hPa,
u=v=0, topography=0.0).&
\tabularnewline
&&&&'\textbf{HS\_jw}': Initializes the Held-Suarez test case
with Jablonowski Williamson initial conditions and zero topography.&
\tabularnewline
&&&&'\textbf{APE\_nwp, APE\_echam, APE\_nh, APEc\_nh, }': Initializes the APE experiments. With the
jabw test case, including moisture.&
\tabularnewline
&&&&'\textbf{wk82}': Initializes the Weisman Klemp test case&
l\_limited\_area =.TRUE.
\tabularnewline
&&&&'\textbf{g\_lim\_area}': Initializes a series of general limited area test cases:
 itype\_atmos\_ana determines the atmospheric profile, itype\_anaprof\_uv
determines the wind profile and itype\_topo\_ana determines the topography&
\tabularnewline
&&&&'\textbf{dcmip\_pa\_12}': Initializes Hadley-like meridional circulation pure advection test case.&
\tabularnewline
&&&&'\textbf{dcmip\_rest\_200}': atmosphere at rest test (Schaer-type mountain)&
lcoriolis = .FALSE.
\tabularnewline
&&&&'\textbf{dcmip\_mw\_2x}': nonhydrostatic mountain waves triggered by Schaer-type mountain&
lcoriolis = .FALSE.
\tabularnewline
&&&&'\textbf{dcmip\_gw\_31}': nonhydrostatic gravity waves triggered by a localized perturbation (nonlinear)&
\tabularnewline
&&&&'\textbf{dcmip\_gw\_32}': nonhydrostatic gravity waves triggered by a localized perturbation (linear)&
l\_limited\_area =.TRUE. \\
and lcoriolis = .FALSE.
\tabularnewline
&&&&'\textbf{dcmip\_tc\_51}': tropical cyclone test case with 'simple physics' parameterizations (\textbf{not yet implemented})&
lcoriolis = .TRUE.
\tabularnewline
&&&&'\textbf{dcmip\_tc\_52}': tropical cyclone test case with with full physics in Aqua-planet mode&
lcoriolis = .TRUE.
\tabularnewline
&&&&'\textbf{CBL}': convective boundary layer simulations for LES package on torus (doubly periodic) grid&
is\_plane\_torus= .TRUE.
\tabularnewline
%\hline
jw\_up&
R& 1.0& m/s&
amplitude of the u-perturbation in jabw test case&
nh\_test\_name='jabw'
\tabularnewline

%\hline
u0\_mrw&
R& 20.0& m/s&
wind speed for mrw(2) and mwbr\_const cases&
nh\_test\_name= 'mrw(2)\_nh' and 'mwbr\_const'
\tabularnewline

%\hline
mount\_height\_mrw&
R& 2000.0& m&
maximum mount height in mrw(2) and mwbr\_const&
nh\_test\_name= 'mrw(2)\_nh' and 'mwbr\_const'
\tabularnewline

%\hline
mount\_half\_width&
R& 1500000.0& m&
half width of mountain in mrw(2), mwbr\_const and bell &
nh\_test\_name= 'mrw(2)\_nh', 'mwbr\_const' and 'bell'
\tabularnewline

%\hline
mount\_lonctr\_mrw\_deg&
R& 90.& deg&
lon of mountain center in mrw(2) and mwbr\_const&
nh\_test\_name= 'mrw(2)\_nh' and 'mwbr\_const'
\tabularnewline

%\hline
mount\_latctr\_mrw\_deg&
R& 30.& deg&
lat of mountain center in mrw(2) and mwbr\_const&
nh\_test\_name= 'mrw(2)\_nh' and 'mwbr\_const'
\tabularnewline


%\hline
temp\_i\_mwbr\_const&
R& 288.0& K&
temp at isothermal lower layer for mwbr\_const case&
nh\_test\_name= 'mwbr\_const'
\tabularnewline

%\hline
p\_int\_mwbr\_const&
R& 70000.& Pa&
pres at the interface of the two layers for mwbr\_const case&
nh\_test\_name= 'mwbr\_const'
\tabularnewline

%\hline
bruntvais\_u\_mwbr\_const&
R& 0.025& 1/s&
constant brunt vaissala frequency at upper layer for mwbr\_const case&
nh\_test\_name= 'mwbr\_const'
\tabularnewline

%\hline
mount\_height&
R& 100.0& m&
peak height of mountain&
nh\_test\_name=  'bell'
\tabularnewline

%\hline
layer\_thickness&
R& -999.0& m&
thickness of vertical layers&
If layer\_thickness $<$ 0, the vertical level distribution is read in from externally given HYB\_PARAMS\_XX.
\tabularnewline

%\hline
n\_flat\_level&
I& 2& &
level number for which the layer is still flat and not terrain-following&
layer\_thickness $>$ 0
\tabularnewline

%\hline
nh\_u0&
R& 0.0& m/s&
initial constant zonal wind speed &
nh\_test\_name = 'bell'
\tabularnewline

%\hline
nh\_t0&
R& 300.0& K&
initial temperature at lowest level &
nh\_test\_name = 'bell'
\tabularnewline

%\hline
nh\_brunt\_vais&
R& 0.01& 1/s&
initial Brunt-Vaisala frequency &
nh\_test\_name = 'bell'
\tabularnewline

%\hline
torus\_domain\_length&
R& 100000.0& m&
length of slice domain &
nh\_test\_name = 'bell', lplane=.TRUE.
\tabularnewline

%\hline
rotate\_axis\_deg&
R& 0.0& deg&
Earth's rotation axis pitch angle&
nh\_test\_name= 'PA'
\tabularnewline

%\hline
lhs\_nh\_vn\_ptb&
L& .TRUE.& &
Add random noise to the initial wind field in the Held-Suarez test. &
nh\_test\_name= 'HS\_nh'
\tabularnewline

%\hline
lhs\_fric\_heat&
L& .FALSE.& &
add frictional heating from Rayleigh friction in the Held-Suarez test.&
nh\_test\_name= 'HS\_nh'
\tabularnewline

%\hline
hs\_nh\_vn\_ptb\_scale&
R& 1.& m/s &
Magnitude of the random noise added to the initial wind field in the
Held-Suarez test. &
nh\_test\_name= 'HS\_nh'
\tabularnewline

%\hline
rh\_at\_1000hpa&
R& 0.7& 1 &
relative humidity at 1000 hPa &
nh\_test\_name= 'jabw', nh\_test\_name= 'mrw'
\tabularnewline

%\hline
qv\_max&
R& 20.e-3& kg/kg &
specific humidity in the tropics &
nh\_test\_name= 'jabw', nh\_test\_name= 'mrw'
\tabularnewline

%\hline
ape\_sst\_case&
C& 'sst1'& &
SST distribution selection\\
'sst1': Control experiment\\
'sst2': Peaked experiment\\
'sst3': Flat experiment\\
'sst4': Control-5N experiment\\
'sst\_qobs': Qobs SST distribution exp.\\
'sst\_const': constant SST&
nh\_test\_name='APE\_nwp', 'APE\_echam'
\tabularnewline

%\hline
ape\_sst\_val&
R& 29.0& degC &
aqua planet SST  for ape\_sst\_case='sst\_const'&
nh\_test\_name= 'APE\_nwp', 'APE\_echam'
\tabularnewline

%\hline
linit\_tracer\_fv&
L& .TRUE.& &
Finite volume initialization for tracer fields &
pure advection tests, only
\tabularnewline

%\hline
lcoupled\_rho&
L& .FALSE.& &
Integrate density equation 'offline'&
pure advection tests, only
\tabularnewline

%\hline
qv\_max\_wk&
R& 0.014 & Kg/kg &
maximum specific humidity near \\
the surface, range  0.012 - 0.016\\
used to vary the buoyancy&
nh\_test\_name='wk82'
\tabularnewline

%\hline
u\_infty\_wk&
R& 20. & m/s &
zonal wind at infinity height\\
range 0. - 45.               \\
used to vary the wind shear&
nh\_test\_name='wk82'
\tabularnewline

%\hline
bub\_amp&
R& 2.& K&
maximum amplitud of the thermal perturbation&
nh\_test\_name='wk82'
\tabularnewline

%\hline
bubctr\_lat&
R& 0.& deg&
latitude of the center of the thermal perturbation&
nh\_test\_name='wk82'
\tabularnewline

%\hline
bubctr\_lon&
R& 90.& deg&
longitude of the center of the thermal perturbation&
nh\_test\_name='wk82'
\tabularnewline

%\hline
bubctr\_z&
R& 1400.& m&
height of the center of the thermal perturbation&
nh\_test\_name='wk82'
\tabularnewline

%\hline
bub\_hor\_width&
R& 10000.& m&
horizontal radius of the thermal perturbation&
nh\_test\_name='wk82'
\tabularnewline

%\hline
bub\_ver\_width&
R& 1400.& m&
vertical radius of the thermal perturbation&
nh\_test\_name='wk82'
\tabularnewline

%\hline
itype\_atmo\_ana&
I& 1 & &
kind of atmospheric profile:\\
1 piecewise N constant layers\\
2 piecewise polytropic layers&
nh\_test\_name=\\'g\_lim\_area'
\tabularnewline
%\hline
itype\_anaprof\_uv&
I& 1 & &
kind of wind profile:\\
1 piecewise linear wind layers\\
2 constant zonal wind\\
3 constant meridional wind&
nh\_test\_name=\\'g\_lim\_area'
\tabularnewline
%\hline
itype\_topo\_ana&
I& 1 & &
kind of orography:\\
1 schaer test case mountain\\
2 gaussian\_2d mountain\\
3 gaussian\_3d mountain\\
any other no orography&
nh\_test\_name=\\'g\_lim\_area'
\tabularnewline
%\hline
nlayers\_nconst&
I& 1 & &
Number of the desired layers with a constant Brunt-Vaisala-frequency&
nh\_test\_name=\\'g\_lim\_area' and
itype\_atmo\_ana=1
\tabularnewline
%\hline
p\_base\_nconst&
R& 100000. & Pa &
pressure at the base of the first N constant layer&
nh\_test\_name=\\'g\_lim\_area' and
itype\_atmo\_ana=1
\tabularnewline
%\hline
theta0\_base\_nconst&
R& 288. & K &
potential temperature at the base of the first N constant layer&
nh\_test\_name=\\'g\_lim\_area' and
itype\_atmo\_ana=1
\tabularnewline
%\hline
h\_nconst&
R(nlayers \_nconst)& 0., 1500., 12000.  & m &
height of the base of each of the N constant layers&
nh\_test\_name=\\'g\_lim\_area' and
itype\_atmo\_ana=1
\tabularnewline
%\hline
N\_nconst&
R(nlayers \_nconst)& 0.01  & 1/s &
Brunt-Vaisala-frequency at each of the N constant layers&
nh\_test\_name=\\'g\_lim\_area' and
itype\_atmo\_ana=1
\tabularnewline
%\hline
rh\_nconst&
R(nlayers \_nconst)& 0.5  & \% &
relative humidity at the base of each N constant layers&
nh\_test\_name=\\'g\_lim\_area' and
itype\_atmo\_ana=1
\tabularnewline
%\hline
rhgr\_nconst&
R(nlayers \_nconst)& 0.  & \% &
relative humidity gradient at each of the N constant layers&
nh\_test\_name=\\'g\_lim\_area' and
itype\_atmo\_ana=1
\tabularnewline
%\hline
nlayers\_poly&
I& 2 & &
Number of the desired layers with constant gradient temperature&
nh\_test\_name=\\'g\_lim\_area' and
itype\_atmo\_ana=2
\tabularnewline
%\hline
p\_base\_poly&
R& 100000. & Pa &
pressure at the base of the first polytropic layer&
nh\_test\_name=\\'g\_lim\_area' and
itype\_atmo\_ana=2
\tabularnewline
%\hline
h\_poly&
R(nlayers \_poly)& 0., 12000.  & m &
height of the base of each of the polytropic layers&
nh\_test\_name=\\'g\_lim\_area' and
itype\_atmo\_ana=2
\tabularnewline
%\hline
t\_poly&
R(nlayers \_poly)& 288., 213.  & K &
temperature at the base of each of the polytropic layers&
nh\_test\_name=\\'g\_lim\_area' and
itype\_atmo\_ana=2
\tabularnewline
%\hline
rh\_poly&
R(nlayers \_poly)& 0.8, 0.2  & \% &
relative humidity at the base of each of the polytropic layers&
nh\_test\_name=\\'g\_lim\_area' and
itype\_atmo\_ana=2
\tabularnewline
%\hline
rhgr\_poly&
R(nlayers \_poly)& 5.e-5, 0. & \% &
relative humidity gradient at each of the polytropic layers&
nh\_test\_name=\\'g\_lim\_area' and
itype\_atmo\_ana=2
\tabularnewline
%\hline
nlayers\_linwind&
I& 2 & &
Number of the desired layers with constant U gradient &
nh\_test\_name=\\'g\_lim\_area' and
itype\_anaprof\_uv=1
\tabularnewline
%\hline
h\_linwind&
R(nlayers \_linwind)& 0., 2500.  & m &
height of the base of each of the linear wind layers&
nh\_test\_name=\\'g\_lim\_area' and
itype\_anaprof\_uv=1
\tabularnewline
%\hline
u\_linwind&
R(nlayers \_linwind)& 5,  10.  & m/s &
zonal wind at the base of each of the linear wind layers&
nh\_test\_name=\\'g\_lim\_area' and
itype\_anaprof\_uv=1
\tabularnewline
%\hline
ugr\_linwind&
R(nlayers \_linwind)& 0., 0. & 1/s &
zonal wind gradient at each of the linear wind layers&
nh\_test\_name=\\'g\_lim\_area' and
itype\_anaprof\_uv=1
\tabularnewline
%\hline
vel\_const&
R& 20. & m/s &
constant zonal/meridional wind (itype\_anaprof\_uv=2,3)&
nh\_test\_name=\\'g\_lim\_area' and
itype\_anaprof\_uv=2,3
\tabularnewline
%\hline
mount\_lonc\_deg&
R& 90. & deg &
longitud of the center of the  mountain&
nh\_test\_name=\\'g\_lim\_area'
\tabularnewline
%\hline
mount\_latc\_deg&
R& 0. & deg &
latitud of the center of the  mountain&
nh\_test\_name=\\'g\_lim\_area'
\tabularnewline
%\hline
schaer\_h0&
R& 250. & m &
h0 parameter for the schaer mountain&
nh\_test\_name=\\'g\_lim\_area' and
itype\_topo\_ana=1
\tabularnewline
%\hline
schaer\_a&
R& 5000. & m &
-a- parameter for the schaer mountain, \\
also half width in the north and south side of the
finite ridge to round the sharp edges&
nh\_test\_name=\\'g\_lim\_area' and
itype\_topo\_ana=1,2
\tabularnewline
%\hline
schaer\_lambda&
R& 4000. & m &
lambda parameter for the schaer mountain&
nh\_test\_name=\\'g\_lim\_area' and
itype\_topo\_ana=1
\tabularnewline
%\hline
lshear\_dcmip&
L& FALSE &  &
run dcmip\_mw\_2x with/without vertical wind shear\\
FALSE: dcmip\_mw\_21: non-sheared \\
TRUE : dcmip\_mw\_22: sheared &
nh\_test\_name=\\'dcmip\_mw\_2x'
\tabularnewline
%\hline
halfwidth\_2d&
R& 10000. & m &
half lenght of the finite ridge in the north-south
direction&
nh\_test\_name=\\'g\_lim\_area' and
itype\_topo\_ana=1,2
\tabularnewline
%\hline
m\_height&
R& 1000. & m &
height of the mountain&
nh\_test\_name=\\'g\_lim\_area' and
itype\_topo\_ana=2,3
\tabularnewline
%\hline
m\_width\_x&
R& 5000. & m &
half width of the gaussian mountain in the east-west direction \\
half width in the north-south direction in the rounding of the
finite ridge (gaussian\_2d)&
nh\_test\_name=\\'g\_lim\_area' and
itype\_topo\_ana=2,3
\tabularnewline
%\hline
m\_width\_y&
R& 5000. & m &
half width of the gaussian mountain in the north-south direction&
nh\_test\_name=\\'g\_lim\_area' and
itype\_topo\_ana=2,3
\tabularnewline
%\hline
gw\_u0&
R& 0. & m/s &
maximum amplitude of the zonal wind&
nh\_test\_name=\\'dcmip\_gw\_3X'
\tabularnewline
%\hline
gw\_clat&
R& 90. & deg &
Lat of perturbation center&
nh\_test\_name=\\'dcmip\_gw\_3X'
\tabularnewline
%\hline
gw\_delta\_temp&
R& 0.01 & K &
maximum temperature perturbation&
nh\_test\_name=\\'dcmip\_gw\_32'
\tabularnewline

%\hline
u\_cbl(2)&
R& 0:0 & m/s and 1/s &
to prescribe initial zonal velocity profile for convective boundary layer simulations where u\_cbl(1)
sets the constant and u\_cbl(2) sets the vertical gradient &
nh\_test\_name=CBL
\tabularnewline

%\hline
v\_cbl(2)&
R& 0:0 & m/s and 1/s &
to prescribe initial meridional velocity profile for convective boundary layer simulations where v\_cbl(1)
sets the constant and v\_cbl(2) sets the vertical gradient &
nh\_test\_name=CBL
\tabularnewline

%\hline
th\_cbl(2)&
R& 290:0.006 & K and K/m &
to prescribe initial potential temperature profile for convective boundary layer simulations where th\_cbl(1)
sets the constant and th\_cbl(2) sets the gradient &
nh\_test\_name=CBL
\tabularnewline

\end{longtab}

Defined and used in: \verb+src/testcases/mo_nh_testcases.f90+


\section{External data}
%------------------------------------------------------------------------------
% ext_par_nml:
%------------------------------------------------------------------------------
\subsection{extpar\_nml (Scope: itopo=1 in run\_nml)}

\begin{longtab}

%\hline
\textbf{itopo}&
I & 0& &
0: analytical topography/ext. data \\
1: topography/ext. data read from file&
\tabularnewline

%\hline
\textbf{n\_iter\_smooth\_topo} &
I(n\_dom) &
0&
&
iterations of topography smoother
&
itopo = 1
\tabularnewline

%\hline
fac\_smooth\_topo&
R &
0.015625&
&
pre-factor of topography smoother
&
n\_iter\_smooth\_topo $>$ 0
\tabularnewline


%\hline
hgtdiff\_max\_smooth\_topo&
R &
0. &
m &
RMS height difference to neighbor grid points at which the smoothing pre-factor fac\_smooth\_topo
reaches its maximum value (linear proportionality for weaker slopes)
&
n\_iter\_smooth\_topo $>$ 0
\tabularnewline


%\hline
heightdiff\_threshold&
R(n\_dom) &
3000.&
m &
height difference between neighboring grid points above which additional local nabla2 diffusion is applied
&
\tabularnewline

%\hline
l\_emiss&
L &
.TRUE.&
&
read and use  external surface emissivity map
&
itopo = 1
\tabularnewline

%\hline
\textbf{extpar\_filename}&
C &
&
&
Filename of external parameter input file,
default: ''\texttt{<path>extpar\_<gridfile>}''.
May contain the keyword \texttt{<path>} which will be substituted by
\texttt{model\_base\_dir}. &
\tabularnewline

%\hline
extpar\_varnames\_map\_ file&
C & ' '
&
&
Filename of external parameter dictionary,
This is a text file with two columns separated by whitespace, where
left column: NetCDF name, right column: GRIB2 short name. It is required, if external parameter are read from a file in GRIB2 format.
 &
\tabularnewline

\end{longtab}

Defined and used in: \verb+src/namelists/mo_extpar_nml.f90+


\section{External packages}



\section{Information on vertical level distribution}

If no vertical sleve coodinate is chosen (ivctype $/=$2), the hydrostatic and nonhydrostatic models need hybrid vertical level information to generate the
terrain following coordinates. The hybrid level specification is stored in
$<$icon home$>$/hyb\_params/HYB\_PARAMS\_$<$nlev$>$.
The {\bf hydrostatic} model assumes to get {\bf pressure based} coordinates, the {\bf nonhydrostatic}
model expects {\bf height based} coordinates. For further information see $<$icon home$>$/hyb\_params/README.
