% =============================================================================
%
% ICON_OCEAN_NML_TABLES.TEX
%
% This LaTeX file defines ocean (and sea-ice) specific namelist parameters.
% The file is included by ICON_NML_TABLES.TEX. For atmosphere-only releases
% the ocean namelists can also easily be excluded.
%
% =============================================================================


%MMMMMMMMMMMMMMMMMMMMMMMMMMMMMMMMMMMMMMMMMMMMMMMMMMMMMMMMMMMMMMMMMMMMMMMMMMMMMM
%MMMMMMMMMMMMMMMMMMMMMMMMMMMMMMMMMMMMMMMMMMMMMMMMMMMMMMMMMMMMMMMMMMMMMMMMMMMMMM
%MMMMMMMMMMMMMMMMMMMMMMMMMMMMMMMMMMMMMMMMMMMMMMMMMMMMMMMMMMMMMMMMMMMMMMMMMMMMMM

\section{Ocean-specific namelist parameters}


\subsection{ocean\_physics\_nml}

\begin{longtab}

%\hline
i\_sea\_ice    & I & 1  && 0: No sea ice, 1: Include sea ice& \tabularnewline
&&&& .FALSE.: compute drag only & \tabularnewline
%\hline
richardson\_factor\_tracer  & I & 0.5e-5 & m/s&  & \tabularnewline
%\hline
richardson\_factor\_veloc  & I & 0.5e-5 & m/s&  & \tabularnewline
%\hline
l\_constant\_mixing  & L & .FALSE. & &  & \tabularnewline

\end{longtab}


\subsection{sea\_ice\_nml (relevant if run\_nml/iforcing=2 (ECHAM))}

\begin{longtab}

  %\hline
  i\_ice\_therm &
  I             &
  2             &&
  Switch for thermodynamic model: \\
  1: Zero-layer model \\
  2: Two layer Winton (2000) model \\
  3: Zero-layer model with analytical forcing (for diagnostics) \\
  4: Zero-layer model for atmosphere-only runs (for diagnostics) &
  In an ocean run i\_sea\_ice must be >=1. In an atmospheric run the ice surface type must be defined.
  \tabularnewline

  i\_ice\_dyn   &
  I             &
  0             &&
  Switch for sea-ice dynamics: \\
  0: No dynamics \\
  1: FEM dynamics (from AWI) &
  \tabularnewline

  i\_ice\_albedo        &
  I                     &
  1                     &&
  Switch for albedo model. Only one is implemented so far. &
  \tabularnewline

  i\_Qio\_type          &
  I                     &
  2                     &&
  Switch for ice-ocean heat-flux calculation method: \\
  1: Proportional to ocean cell thickness (like MPI-OM) \\
  2: Proportional to speed difference between ice and ocean &
  Defaults to 1 when i\_ice\_dyn=0 and 2 otherwise.
  \tabularnewline

  kice  &
  I     &
  1     &&
  Number of ice classes (must be one for now) &
  \tabularnewline

  hnull &
  R     &
  0.5   &
  m     &
  Hibler's $h_0$ parameter for new-ice growth. &
  \tabularnewline

  hmin  &
  R     &
  0.05  &
  m     &
  Minimum sea-ice thickness allowed. &
  \tabularnewline

  ramp\_wind    &
  R             &
  10            &
  days          &
  Number of days it takes the wind to reach correct strength. Only used at the start of an OMIP/NCEP simulation (not after restart). &
  \tabularnewline

\end{longtab}
