%% LyX 1.4.5.1 created this file.  For more info, see http://www.lyx.org/.
%% Do not edit unless you really know what you are doing.

\documentclass[10pt,english,a4paper,landscape]{article}

\usepackage[T1]{fontenc}

\usepackage[latin1]{inputenc}

\usepackage{array}

\usepackage[colorlinks=true,bookmarks=true,linkcolor=blue,citecolor=blue]{hyperref}
\usepackage{longtable}
%%%%%%%%%%%%%%%%%%%%%%%%%%%%%% LyX specific LaTeX commands.
%% Bold symbol macro for standard LaTeX users
\providecommand{\boldsymbol}[1]{\mbox{\boldmath $#1$}}
%% Because html converters don't know tabularnewline
\providecommand{\tabularnewline}{\\}
%%%%%%%%%%%%%%%%%%%%%%%%%%%%%% User specified LaTeX commands.
\makeatletter

\usepackage[landscape]{geometry}
\AtBeginDvi{\special{landscape}} 
\makeatother

\usepackage{babel}
\makeatother

\newenvironment{longtab}
{
\begin{longtable}
{|>{\raggedright}p{4cm}
 |>{\raggedright}p{1.5cm}
 |>{\raggedright}p{1.5cm}
 |>{\raggedright}p{1cm}
 |>{\raggedright}p{8cm}
 |>{\raggedright}p{4cm}|}
\hline
Parameter& Type& Default& Unit& Description & Scope
\tabularnewline
\hline
\endhead
}
{
\hline
\end{longtable}
}

% ------------------------------------------------------
% new environment for documentation of namelist changes
% that are incompatible with former versions.
% ------------------------------------------------------

\newenvironment{changeitem}[3]
{
 \rule[-1em]{0.2ex}{4em}\hspace*{1em}%
 \begin{minipage}{9.95\textwidth}
 \begin{tabbing} 
 \hspace*{3cm} \= ~ \kill
 \textcolor{red}{\emph{Change:}}         \> \textbf{#1} \\
 \textcolor{red}{\emph{Date of Change:}} \> \textbf{#2} \\
 \textcolor{red}{\emph{Revision:}}       \> \textbf{#3} \\
 \end{tabbing}
 \end{minipage}
}{\vspace*{1em}}



\begin{document}

\title{ICON Namelist Overview}

\maketitle

\tableofcontents{}

\newpage{}

\section{ICON Namelists}

\subsection{Scripts, Namelist files and Programs}

Run scripts starting the programs for the grid generation and the
models are stored in run/. These scripts write namelist files containing
the specified Fortran namelists. Programs are stored in <icon home>/build/<architecture>/bin/.


\begin{table}[htd]
\caption{Namelist files}
\begin{center}
\begin{tabular}{llll}
Namelist file     & Purpose             & Made by script & Used by program  \\
\hline 
NAMELIST\_GRAPH   & Generate graphs     & create\_global\_grids.run & grid\_command \\
NAMELIST\_GRID    & Generate grids      & create\_global\_grids.run & grid\_command \\
NAMELIST\_GRIDREF & Gen. nested domains & create\_global\_grids.run & grid\_command \\
NAMELIST\_OCEAN\_GRID & Gen. ocean grid & create\_ocean\_grid.run & grid\_command \\
NAMELIST\_TORUS\_GRID & Gen. torus grid & create\_torus\_grid.run & grid\_command \\
NAMELIST\_ICON    & Run ICON models     & exp.<name>.run & control\_model   \\
\end{tabular}
\end{center}
\label{table:namelistfiles}
\end{table}%

\subsection{Namelist parameters}

The following subsections tabulate all available Fortran namelist
parameters by name, type, default value, unit, description, and scope:

\begin{itemize}
\item \emph{Type} refers to the type of the Fortran variable, in which the
value is stored: I=INTEGER, L=LOGICAL, R=REAL, C=character string 
\item \emph{Default} is the preset value, if defined, that is assigned to
this parameter within the programs. 
\item \emph{Unit} shows the unit of the control parameter, where applicable. 
\item \emph{Description} explains in a few words the purpose of the parameter. 
\item \emph{Scope} explains under which conditions the namelist parameter
has any effect, if its scope is restricted to specific settings of
other namelist parameters. 
\end{itemize}
Information on the file, where the namelist is defined and used, is
given at the end of each table.


\section{Namelist parameters for grid generation}

\subsection{Namelist parameters defining the atmosphere grid}

%-----------------------------------------------------------------------------
% graph_ini:
%-----------------------------------------------------------------------------
\subsubsection{graph\_ini (NAMELIST\_GRAPH)}

\begin{longtab}

\hline 
nroot&
I&
2&
&
root subdivision of initial edges&
\tabularnewline

\hline 
grid\_levels&
I&
4&
&
number of edge bisections following the root subdivision&
\tabularnewline

\hline 
lplane&
L&
.FALSE.&
&
switch for generating a double periodic planar grid. The root level
consists of 8 triangles.&
\tabularnewline

\end{longtab}

Defined and used in: src/grid\_generator/mo\_io\_graph.f90

%-----------------------------------------------------------------------------
% grid_ini:
%-----------------------------------------------------------------------------
\subsubsection{grid\_ini (NAMELIST\_GRID)}

\begin{longtab}

\hline
nroot&
I&
2&
&
root subdivision of initial edges&
\tabularnewline

\hline
grid\_levels&
I&
4&
&
number of edge bisections following the root subdivision&
\tabularnewline

\hline
lplane&
L&
.FALSE.&
&
switch for generating planar grid. The root level consists of 8 triangles.&
\tabularnewline

\end{longtab}

Defined and used in: src/grid\_generator/mo\_io\_grid.f90

%-----------------------------------------------------------------------------
% grid_options:
%-----------------------------------------------------------------------------
\subsubsection{grid\_options (NAMELIST\_GRID)}

\begin{longtab}

\hline
x\_rot\_angle&
R&
0.0&
deg&
Rotation of the icosahedron about the x-axis (connecting the origin
and {[}0$^\circ$E, 0$^\circ$N])&
\tabularnewline

\hline
y\_rot\_angle &
R&
0.0&
deg&
Rotation of the icosahedron about the y-axis (connecting the origin
and {[}90$^\circ$E, 0$^\circ$N), done after the rotation about the x-axis.&
\tabularnewline

\hline
z\_rot\_angle &
R&
0.0&
deg&
rotation of the icosahedron about the z-axis (connecting the origin
and {[}0$^\circ$E, 90$^\circ$N), done after the rotation about the y-axis.&
\tabularnewline

\hline
itype\_optimize&
I&
4&
&
Grid optimization type&
\tabularnewline
&
&
&
&
0: no optimization&
\tabularnewline
&
&
&
&
1: Heikes Randall&
\tabularnewline
&
&
&
&
2: equal area&
\tabularnewline
&
&
&
&
3: c-grid small circle&
\tabularnewline
&
&
&
&
4: spring dynamics&
\tabularnewline

\hline
l\_c\_grid &
L &
.FALSE. &
&
C-grid constraint on last level&
\tabularnewline

\hline
maxlev\_optim &
I &
100 &
&
Maximum grid level where the optimization is applied&
i\_type\_optimize = 1 or 4
\tabularnewline

\hline
beta\_spring &
R &
0.90&
&
tuning factor for target grid length &
i\_type\_optimize = 4
\tabularnewline

\end{longtab}

Defined and used in: src/grid\_generator/mo\_io\_grid.f90


%-----------------------------------------------------------------------------
% plane_options:
%-----------------------------------------------------------------------------
\subsubsection{plane\_options (NAMELIST\_GRID)}

\begin{longtab}

\hline
tria\_arc\_km &
R &
10.0 &
km&
length of triangle edge on plane&
lplane=.TRUE.
\tabularnewline

\end{longtab}

The number of grid points is generated by root level section and further
bisections. The double periodic root level consists of 8 triangles.
The spatial coordinates are $-1<=x<=1$, and $-\sqrt{3}/2<=y<=\sqrt{3}/2$.
Currently the planar option can only be used as an $f$-plane.

\noindent Defined and used in: src/grid\_generator/mo\_io\_grid.f90


%-----------------------------------------------------------------------------
% gridref_ini:
%-----------------------------------------------------------------------------
\subsubsection{gridref\_ini (NAMELIST\_GRIDREF)}

\begin{longtab}

\hline
grid\_root&
I&
2&
&
root subdivision of initial edges&
\tabularnewline

\hline
start\_lev&
I&
4&
&
number of edge bisections following the root subdivision&
\tabularnewline

\hline
n\_dom&
I&
2&
&
number of logical model domains, including the global one&
\tabularnewline

\hline
n\_phys\_dom&
I&
n\_dom&
&
number of physical model domains, may be larger than n\_dom (in this case, domain merging is applied)&
\tabularnewline

\hline
parent\_id&
I(n\_phys\_ dom-1)&
i&
&
ID of parent domain (first entry refers to first nested domain; needs to be
specified only in case of more than one nested domain per grid level)&
\tabularnewline

\hline
logical\_id&
I(n\_phys\_ dom-1)&
i+1&
&
logical grid ID of domain (first entry refers to first nested domain; needs to be
specified only in case of domain merging, i.e. n\_dom $<$ n\_phys\_dom) &
\tabularnewline

\hline
l\_plot&
L&
.FALSE.&
&
produces GMT plots showing the locations of the nested domains&
\tabularnewline

\hline
l\_circ&
L&
.TRUE.&
&
Create circular (.T.) or rectangular (.F.) refined domains &
\tabularnewline

\hline
l\_rotate&
L&
.FALSE.&
&
Rotates center point into the equator in case of l\_circ = .FALSE. & lcirc=.FALSE.
\tabularnewline

\hline
write\_hierarchy&
I&
1&
&
0: Output only computational grids \\
1: Output in addition parent grid of global model domain (required for computing physics on a reduced grid) \\
2: Output all grids back to level 0 (required for hierarchical search algorithms) &
\tabularnewline

\hline
bdy\_indexing\_depth&
I&
max\_rlcell (=8)&
&
Number of cell rows along the lateral boundary of a model domain for which the refin\_ctrl
fields contain the distance from the lateral boundary; needs to be enlarged when lateral
boundary nudging is required for one-way nesting &
\tabularnewline

\hline
radius&
R(n\_dom-1)&
30.&
deg&
radius of nested domain (first entry refers to first nested domain; needs to be
specified for each nested domain separately)&
lcirc=.TRUE.
\tabularnewline

\hline
hwidth\_lon&
R(n\_dom-1)&
20.&
deg&
zonal half-width of refined domain (first entry refers to first nested domain; needs to be
specified for each nested domain separately)&
lcirc=.FALSE.
\tabularnewline

\hline
hwidth\_lat&
R(n\_dom-1)&
20.&
deg&
meridional half-width of refined domain (first entry refers to first nested domain; needs to be
specified for each nested domain separately)&
lcirc=.FALSE.
\tabularnewline

\hline
center\_lon&
R(n\_dom-1)&
90.&
deg&
center longitude of refined domain (first entry refers to first nested domain; needs to be
specified for each nested domain separately)&
\tabularnewline

\hline
center\_lat&
R(n\_dom-1)&
30.&
deg&
center latitude of refined domain (first entry refers to first nested domain; needs to be
specified for each nested domain separately)&
\tabularnewline

\end{longtab}

Defined and used in: src/grid\_generator/mo\_gridrefinement.f90


%-----------------------------------------------------------------------------
% gridref_metadata:
%-----------------------------------------------------------------------------
\subsubsection{gridref\_metadata (NAMELIST\_GRIDREF)}

\begin{longtab}

\hline
number\_of\_grid\_used&
I(n\_dom+1)&
0&
&
sets the number of grid used in the netcdf header; the number of entries must be n\_dom+1
because the first number refers to the radiation grid &
\tabularnewline

\hline
centre&
I&
0&
&
centre running the grid generator: 78 - edzw (DWD), 252 - MPIM &
\tabularnewline

\hline
subcentre&
I&
0&
&
subcentre to be assigned by centre, usually 0 &
\tabularnewline

\hline
outname\_style&
I&
1&
&
Output name style\\
1: Standard: \emph{iconR\textcolor{red}{X}B\textcolor{red}{XX}\_DOM\textcolor{red}{XX}.nc}\\
2: DWD: \emph{icon\_grid\_\textcolor{red}{XXXX}\_R\textcolor{red}{XX}B\textcolor{red}{XX}\_\textcolor{red}{X}.nc} &
\tabularnewline

\end{longtab}

\subsection{Namelist parameters defining the local grid generation}

The ocean grids are created by the script \verb+run/create_ocen_grid.run+


%-----------------------------------------------------------------------------
% grid_geometry_conditions:
%-----------------------------------------------------------------------------
\subsubsection{grid\_geometry\_conditions}

\begin{longtab}

\hline
no\_of\_conditions &
I &
0 &
&
Number of geometric conditions &
\tabularnewline

\hline
patch\_shape &
I(no\_of\_ conditions) &
0 &
&
1=rectangle; 2=circle &
\tabularnewline

\hline
patch\_center\_x &
R(no\_of \_ conditions) &
0.0 &
degrees &
longitude of patch center &
\tabularnewline

\hline
patch\_center\_y &
R(no\_of \_ conditions) &
0.0 &
degrees &
latitude of patch center &
\tabularnewline

\hline
rectangle\_xradious &
R(no\_of\_ conditions) &
0.0 &
degrees &
half meridional extension of a rectangular patch &
patch\_shape=1
\tabularnewline

\hline
rectangle\_yradious &
R(no\_of\_ conditions) &
0.0 &
degrees &
half zonal extension of a rectangular patch &
patch\_shape=1
\tabularnewline

\hline
circle\_radious &
R(no\_of\_ conditions) &
0.0 &
degrees &
radius of a circular patch &
patch\_shape=2
\tabularnewline

\end{longtab}

Defined in \verb+mo_grid_conditions.f90+


%-----------------------------------------------------------------------------
% grid_geometry_conditions:
%-----------------------------------------------------------------------------
\subsubsection{local\_grid\_optimization}

\begin{longtab}

\hline
use\_optimization &
L &
.FALSE. &
&
Apply, or not, optimization &
\tabularnewline

\hline
use\_edge\_springs &
L &
.FALSE. &
&
Use spring dynamics &
\tabularnewline

\hline
prime\_ref\_length \_coeff &
R &
1.0 &
&
Spring length coefficient &
\tabularnewline

\hline
use\_adaptive\_ spring\_length &
L &
.FALSE. &
&
Use adaptive spring length &
\tabularnewline

\hline
use\_local\_reference \_length &
L &
.FALSE. &
&
Use locally adaptive spring length &
\tabularnewline

\hline
local\_reference\_ length\_coeff &
R &
0.0 &
&
Coefficient of local vs global spring length &
\tabularnewline

\hline
use\_isotropy\_force &
L &
.FALSE. &
&
Use isotropy force, tends to create symmetric triangles &
\tabularnewline

\hline
isotropy\_rotation \_coeff &
R &
0.0 &
&
Coefficient of the rotational isotropy force &
\tabularnewline

\hline
isotropy\_stretch \_coeff &
R &
0.0 &
&
Coefficient of the stretch isotropy force &
\tabularnewline


\hline
optimize\_vertex \_depth &
I &
1 &
&
For patches the min depth of the vertices that will be optimized.
The boundary vertices have depth 0, the next level 1, etc.
 &
\tabularnewline


\end{longtab}


Defined in \verb+mo_local_grid_optimization.f90+


%-----------------------------------------------------------------------------
% create_ocean_grid:
%-----------------------------------------------------------------------------
\subsubsection{create\_ocean\_grid}

\begin{longtab}

\hline
only\_get\_sea\_ land\_mask &
L &
.false. &
&
.true.:returns the whole grid with a sea-land mask; .false.:returns only the ocean grid &
\tabularnewline

\hline
smooth\_ocean\_ boundary &
L &
.true. &
&
.true.:smooths the ocean boundaries so no triabgle has two boundary edges; .false.:no smoothing&
\tabularnewline

\hline
input\_file &
C &
&
&
name of the input grid file &
\tabularnewline

\hline
elevation\_file &
C &
&
&
name of the file containing cell elevation values for the input\_file  &
no\_of\_conditions=0
\tabularnewline

\hline
elevation\_field &
C &
&
&
name of the field containing the cell elevation values  &
no\_of\_conditions=0
\tabularnewline

\hline
min\_sea\_depth &
R &
0.0 &
m (negative)&
if cell elevation < min\_sea\_depth then the cell is consider sea   &
\tabularnewline

\hline
set\_sea\_depth &
R &
0.0 &
m (negative)&
if not 0, then sea cells are of set\_sea\_depth elevation  &
\tabularnewline

\hline
set\_min\_sea\_depth &
R &
0.0 &
m (negative)&
if not 0, then sea cells have a maximum of set\_min\_sea\_depth elevation  &
\tabularnewline

\hline
edge\_elev\_ interp\_method &
I &
2 &
&
compute edge elevation from cells using: linear interpolation=1; min value   = 2  &
\tabularnewline

\hline
output\_refined\_ ocean\_file &
C &
&
&
name of the output refined ocean grid file &
\tabularnewline

\end{longtab}

Defined in \verb+mo_create_ocean_grid.f90+


%-----------------------------------------------------------------------------
% torus_grid_parameters:
%-----------------------------------------------------------------------------
\subsubsection{torus\_grid\_parameters}

\begin{longtab}

\hline
y\_no\_of\_rows &
I &
 &
4 &
number of triangle rows of the torus grid, >=2 &
\tabularnewline

\hline
x\_no\_of\_columns &
I &
 &
8 &
number of triangle columns of the torus grid, >=2 &
\tabularnewline

\hline
edge\_length &
R &
m &
1000.0 &
the triangle edge length&
\tabularnewline

\hline
x\_center &
R &
m &
0.0 &
the x coordinate of the torus center&
\tabularnewline

\hline
y\_center &
R &
m &
0.0 &
the y coordinate of the torus center&
\tabularnewline

\hline
out\_file\_name &
C &
 &
&
the torus grid file name &
\tabularnewline

\hline
unfolded\_torus\_ file\_name &
C &
 &
&
the unfolded torus grid file name (for plotting) &
\tabularnewline

\hline
ascii\_filename &
C &
 &
&
the unfolded torus grid ascci file name (for plotting) &
\tabularnewline

\end{longtab}

Defined in \verb+mo_create_torus_grid.f90+. See the run script \verb+run/create_torus_grid.run+.





%MMMMMMMMMMMMMMMMMMMMMMMMMMMMMMMMMMMMMMMMMMMMMMMMMMMMMMMMMMMMMMMMMMMMMMMMMMMMMM
%MMMMMMMMMMMMMMMMMMMMMMMMMMMMMMMMMMMMMMMMMMMMMMMMMMMMMMMMMMMMMMMMMMMMMMMMMMMMMM
%MMMMMMMMMMMMMMMMMMMMMMMMMMMMMMMMMMMMMMMMMMMMMMMMMMMMMMMMMMMMMMMMMMMMMMMMMMMMMM

\section{Namelist parameters defining the ICON model}

Namelist parameters for the ICON models are organized in several thematic
Fortran namelists controling the experiment, and the properties of
dynamics, transport, physics etc.


%------------------------------------------------------------------------------
% master_nml: the minimum one needs to specify about an integration
%------------------------------------------------------------------------------
\subsection{master\_nml}
\begin{longtab}

\hline
l\_restart&
L & .FALSE. & &
If .TRUE.: Current experiment is started from a restart.&
\tabularnewline

\hline
model\_base\_dir &
C & ' ' & &
General path which may be used in file names of other name lists:
If a file name contains the keyword "\texttt{<path>}", then this
\texttt{model\_base\_dir} will be substituted.
 &
\tabularnewline

\end{longtab}

\subsection{master\_model\_nml (repeated for each model)}
\begin{longtab}

\hline
model\_name &
C & & &
Character string for naming this component.&
\tabularnewline

\hline
model\_namelist\_ filename &
C & & &
File name containing the model namelists.&
\tabularnewline

\hline
model\_type &
I & 0 & &
Identifies which component to run.
atmosphere=1, ocean=2, radiation=3,
dummy\_model=99 &
\tabularnewline

\hline
model\_min\_rank &
I & 0 & &
Start MPI rank for this model.&
\tabularnewline

\hline
model\_max\_rank &
I & -1 & &
End MPI rank for this model.&
\tabularnewline

\hline
model\_inc\_rank &
I & 0 & &
Stride of MPI ranks.&
\tabularnewline

\hline
model\_restart\_info \_filename &
C & restart.info & &
Name (including full path) of the restart info file for this model&
\tabularnewline

\end{longtab}

%------------------------------------------------------------------------------
% time_nml
%------------------------------------------------------------------------------
\subsection{time\_nml}
\begin{longtab}

\hline
dt\_restart &
R & 86400.*30.& s &
Length of restart cycle in seconds. 
Note that the frequency of writing restart files is controlled by
io\_nml:dt\_checkpoint. If the value of dt\_checkpoint resulting from 
model default or user's specification is longer than dt\_restart, 
it will be reset (by the model) to dt\_restart so 
that at least one restart file is generated during the restart cycle. 
If dt\_restart is larger than but not a multiple of dt\_checkpoint, 
restart file will NOT be generated at the end of the restart cycle. 
&
\tabularnewline

\hline 
calendar&
I& 1& &
Calendar type: \\ 
0=Julian/Gregorian \\
1=proleptic Gregorian\\
2=30day/month,360day/year & 
\tabularnewline

\hline 
ini\_datetime\_string &
C & '2008-09-01T00:00:00Z' &&
Initial date and time of the simulation&
\tabularnewline

\hline 
end\_datetime\_string &
C & 2008-09-01T01:40:00Z' &&
End date and time of the simulation&
\tabularnewline

\hline 
&&&& Length of the run&
\tabularnewline
&&&& If "nsteps" in run\_nml (see below) is positive, then nsteps*dtime 
is used to compute the end date and time of the run.&
\tabularnewline
&&&& Else the initial date and time, the end date and time, dt\_restart,
as well as the time step are used to compute "nsteps".&
\tabularnewline

\end{longtab}


%------------------------------------------------------------------------------
% parallel_nml:
%------------------------------------------------------------------------------
\subsection{parallel\_nml}
\begin{longtab}

\hline 
nproma &
I & 1& &
chunk length&
\tabularnewline

\hline 
n\_ghost\_rows &
I & 1& &
number of halo cell rows&
\tabularnewline

\hline 
division\_method &
I & 1& &
method of domain decomposition\\
0: read in from file \\
1: use built-in geometric subdivision \\
2: use METIS&
\tabularnewline


\hline 
division\_file\_name &
C &  & &
Name of division file &
division\_method = 0
\tabularnewline

\hline 
ldiv\_phys\_dom &
L & .TRUE. & &
.TRUE.: split into physical domains before computing domain decomposition (in case of merged domains)\\
(This reduces load imbalance; turning off this option is not recommended except for very small processor numbers) &
division\_method = 1
\tabularnewline

\hline 
p\_test\_run &
L & .FALSE.& &
.TRUE. means verification run for MPI parallelization (PE 0
processes full domain) &
\tabularnewline

\hline 
l\_test\_openmp &
L & .FALSE.& &
if .TRUE. is combined with p\_test\_run=.TRUE. and OpenMP paralllelization,
the test PE gets only 1 thread in order to verify the OpenMP paralllelization&
p\_test\_run = .TRUE.
\tabularnewline

\hline 
l\_log\_checks &
L & .FALSE.& &
if .TRUE. messages are generated during each synchonization step
(use for debugging only)&
\tabularnewline

\hline 
l\_fast\_sum &
L & .FALSE.& &
if .TRUE., use fast (not processor-configuration-invariant) global summation&
\tabularnewline

\hline 
use\_dycore\_barrier &
L & .FALSE.& &
if .TRUE., set an MPI barrier at the beginning of the nonhydrostatic solver (do not use for production runs!)&
\tabularnewline

\hline 
itype\_exch\_barrier &
I & 0 & &
1: set an MPI barrier at the beginning of each MPI exchange call\\
2: set an MPI barrier after each MPI WAIT call \\
3: 1+2 (do not use for production runs!) &
\tabularnewline

\hline 
iorder\_sendrecv &
I & 1& &
Sequence of send/receive calls: \\
 1 = irecv/send \\
 2 = isend/recv  \\
 3 = isend/irecv \\
\tabularnewline

\hline 
itype\_comm &
I & 1& &
1: use local memory for exchange buffers \\
3: asynchronous halo communication for dynamical core (currently deactivated)&
\tabularnewline

\hline 
num\_io\_procs &
I & 0& &
Number of I/O processors (running exclusively for doing I/O)&
\tabularnewline

\hline 
pio\_type &
I & 1& &
Type of parallel I/O. Only used if number of I/O processors greater number of domains.
Experimental!&
\tabularnewline

% % \hline 
% % nh\_stepping\_threads &
% % I & 1& &
% % The number of OpenMP threads to be used by the non-hydrostatic dycore. Only used if the \_\_OMP\_RADIATION\_\_
% % flag is set during compilation. Experimental! &
% % \tabularnewline
% % 
% % \hline 
% % radiation\_threads &
% % I & 1& &
% % The number of OpenMP threads to be used by the radiation. Only used if the \_\_OMP\_RADIATION\_\_
% % flag is set during compilation. Experimental! &
% % \tabularnewline

\hline 
use\_icon\_comm &
L & .FALSE. & &
Enable the use of MPI bulk communication through the icon\_comm\_lib &
\tabularnewline

\hline 
icon\_comm\_debug &
L & .FALSE. & &
Enable debug mode for the icon\_comm\_lib &
\tabularnewline

\hline 
max\_send\_recv \_buffer\_size &
I & 131072 & &
Size of the send/receive buffers for the icon\_comm\_lib. &
\tabularnewline

\hline 
use\_dp\_mpi2io &
L & .FALSE. & &
 Enable this flag if output fields shall be gathered and written in
 single-precision. &
\tabularnewline

\end{longtab}

Defined and used in: src/namelists/mo\_parallel\_nml.f90

%------------------------------------------------------------------------------
% coupling_nml:
%------------------------------------------------------------------------------
\subsection{coupling\_nml}
\begin{longtab}

\hline 
name & 
C &
blank &  &
short name of the coupling field &
\tabularnewline

\hline 
dt\_coupling &
I &
0 &
s &
coupling time step / coupling interval &
\tabularnewline

\hline 
dt\_model &
I &
0 &
s &
model time step &
\tabularnewline

\hline 
lag &
I &
0 &   &
offset to coupling event in number of model time steps &
\tabularnewline

\hline 
l\_time\_average &
L &
.FALSE. &  &
.TRUE.: time averaging between two coupling events &
\tabularnewline

\hline 
l\_time\_accumulation &
L &
.FALSE. &  &
.TRUE.: accumulation of coupling fields in time between two coupling events &
\tabularnewline

\hline 
l\_diagnostic &
L &
.FALSE. &  &
.TRUE.: simple diagnostics (min, max, avg) for coupling fields is switched on &
\tabularnewline

\hline 
l\_activated &
L &
.FALSE. &  &
.TRUE.: activate the coupling of the respective coupling field &
\tabularnewline

\end{longtab}

Defined and used in: src/namelists/mo\_coupling\_nml.f90

%------------------------------------------------------------------------------
% run_nml
%------------------------------------------------------------------------------
\subsection{run\_nml}

\begin{longtab}

\hline 
ldump\_states&
L & .FALSE.& &
Dump patch/interpolation/grid refinement state of every
patch (after subdivision in case of a parallel run)
to a Netcdf file and exit program.&
\tabularnewline

\hline 
lrestore\_states&
L & .FALSE.& &
Restore patch/interpolation/grid refinement states
from NetCDF dump files instead of calculating them.&
\tabularnewline

\hline 
dump\_filename&
C &
&
&
Filename of dump/restore files,
default: ''\texttt{<path>dump\_<proc><gridfile>}''.
May contain the keyword \texttt{<path>} which will be substituted by
\texttt{model\_base\_dir}, \texttt{<proc>} substituted by ``procXofY\_'',
and the grid filename \texttt{<gridfile>}. &
\tabularnewline

\hline 
dd\_filename&
C &
&
&
Filename of NetCDF domain decomposition dump files,
default: ''\texttt{<path>dd\_<gridfile>}''.
May contain the keyword \texttt{<path>} which will be substituted by
\texttt{model\_base\_dir},
and the grid filename \texttt{<gridfile>}. &
\tabularnewline

\hline 
l\_one\_file\_per\_patch&
L & .FALSE.& &
Use one file per patch for all processors.\\
This will decrease the amount of files used for dump/restore
considerably, especially for massively parallel runs on hundreds
or thousands of processors.\\
Time for dumping will increase since the file has to be written sequentially,
the time for restore should stay roughly the same, however.&
ldump\_states=.TRUE. or lrestore\_states=.TRUE.
\tabularnewline

\hline 
ldump\_dd&
L & .FALSE.& &
Dump the domain decomposition (and a few related fields).
This can be done either in a parallel run or in a single-CPU run.
When done in a parallel run, the domain decoposition
is for the number of parallel processes in use.
When done in a single-CPU run, nproc\_dd (see below) determines
the number of processes for the decomposition.
Uses always only one file per patch,&
\tabularnewline

\hline 
lread\_dd&
L & .FALSE.& &
Read the domain decomposition when dumped with ldump\_dd.
&
\tabularnewline

\hline 
nproc\_dd&
I & 1 & &
Number of processors for the target domain decomposition
(only relevant when running on a single processor).&
ldump\_dd = TRUE and a single processor run
\tabularnewline

\hline 
nsteps &
I & 0 & &
number of time steps of this run.&
\tabularnewline

\hline 
dtime &
R & 600.0& s&
time step&
\tabularnewline


\hline 
ltestcase &
L & .TRUE.& &
Idealized testcase runs&
\tabularnewline

\hline 
ldynamics&
L& .TRUE.& &
Compute adiabatic dynamic tendencies&
\tabularnewline

\hline 
iforcing&
I&
0&
&
Forcing of dynamics and transport by parameterized processes. Use positive indices for the
atmosphere and negative indices for the ocean.\\
0: no forcing\\
1: Held-Suarez forcing\\
2: ECHAM forcing\\
3: NWP forcing\\
4: local diabatic forcing without physics\\
5: local diabatic forcing with physics\\
-1: MPIOM forcing (to be done)&
\tabularnewline

\hline 
ltransport&
L& .FALSE.& &
Compute large-scale tracer transport&
\tabularnewline

\hline 
ntracer&
I& 0& &
Number of advected tracers handled by the large-scale transport scheme&
\tabularnewline

\hline
lvert\_nest &
L & .FALSE.& &
If set to .true. vertical nesting is switched on (i.e.\ variable number of vertical levels) &
\tabularnewline

\hline 
num\_lev&
I(max\_dom)& 31& &
Number of full levels (atm.) for each domain&
lvert\_nest=.TRUE.
\tabularnewline

\hline 
nshift&
I(max\_dom)& 0& & 
vertical half level of parent domain which coincides with 
upper boundary of the current domain&
lvert\_nest=.TRUE.
\tabularnewline

\hline 
ltimer&
L & .TRUE.& &
TRUE: Timer for monitoring thr runtime of specific routines is on (FALSE = off)&
\tabularnewline

\hline 
timers\_level&
I & 1& &
&
\tabularnewline

\hline 
activate\_sync\_timers&
L & F& &
TRUE: Timer for monitoring runtime of communication routines (FALSE = off)&
\tabularnewline

\hline 
msg\_level&
I & 10& &
controls how much printout is written during runtime. \\
For values less than 5, only the time step is written.&
\tabularnewline

\hline 
msg\_timestamp&
L & .FALSE.& &
If .TRUE., precede output messages by time stamp.&
\tabularnewline

\hline 
test\_mode&
I & 0& &
Setting a value larger than 0 activates a dummy mode in which time stepping is changed
into just doing iterations, and MPI communication is replaced by copying some value from
the send buffer into the receive buffer (does not work with nesting and reduced radiation
grid because the send buffer may then be empty on some PEs) & iequations = 3
\tabularnewline

\hline 
output &
C(:)& ``nml'',''totint'' & &
Main switch for enabling/disabling components of the model output. One or more choices can be set (as an array of string constants). Possible choices are:
\begin{itemize}
\item ``none'': switch off all output;
\item ``vlist'' : old, vlist-based output mode;
\item``nml'': new output mode (cf.\ \texttt{output\_nml});
\item``totint'': computation of total integrals.
\end{itemize}
If the \texttt{output} namelist parameter is not set explicitly, the default setting ``nml'',''totint'' is assumed.
 &
\tabularnewline

\end{longtab}

Defined and used in: src/namelists/mo\_run\_nml.f90


%-------------------------------------------------------------------
% grid_nml: horizontal grid
%-------------------------------------------------------------------
\subsection{grid\_nml}
\begin{longtab}

\hline 
cell\_type&
I & 3& &
Cell type\\
3: triangular cells\\
4: quadrilateral cells (to be done)\\
6: pentagonal/hexagonal cells&
\tabularnewline

\hline
lplane &
L & .FALSE.& &
planar option&
\tabularnewline

\hline
is\_plane\_torus &
L & .FALSE.& &
f-plane approximation on triangular grid &
\tabularnewline

\hline 
corio\_lat &
R & 0.0& deg&
Center of the f-plane is located at this geographical latitude &
lplane=.TRUE. and is\_plane\_torus=.TRUE.
\tabularnewline

\hline 
grid\_angular \_velocity &
R & Earth's & rad/sec &
The angular velocity in rad per sec. &
\tabularnewline

%\hline 
%start\_lev &
%I & 4& &
%coarsest bisection level&
%\tabularnewline

\hline 
l\_limited\_area &
L & .FALSE.& & &
\tabularnewline

\hline 
grid\_rescale\_factor &
R & 1.0   &  &
The geometry and the timestep will be multiplied by this factor.\\
The angular velocity will be divided by this factor.
&
\tabularnewline

\hline 
lfeedback &
L(n\_dom) & .TRUE.& &
Specifies if feedback to parent grid is performed. Setting lfeedback(1)=.false. turns off feedback 
for all nested domains; to turn off feedback for selected nested domains, set lfeedback(1)=.true.
and set ``.false." for the desired model domains&
n\_dom>1
\tabularnewline

\hline 
ifeedback\_type &
I & 2& &
1: incremental feedback \\ 2: relaxation-based feedback \\
Note: vertical nesting requires option 2 to run numerically stable over longer time periods & n\_dom>1
\tabularnewline

\hline 
start\_time &
R(n\_dom) & 0.   & s &
Time when a nested domain starts to be active (namelist entry is ignored for the global domain)
& n\_dom>1
\tabularnewline

\hline 
end\_time &
R(n\_dom) & 1.E30  & s &
Time when a nested domain terminates (namelist entry is ignored for the global domain)
& n\_dom>1
\tabularnewline

\hline 
patch\_weight &
R(n\_dom) & 0.& &
If patch\_weight is set to a value > 0 for any of the first level child patches,
processor splitting will be performed, i.e. every of the first level child patches
gets a subset of the total number or processors corresponding to its patch\_weight.
A value of 0. corresponds to exactly 1 processor for this patch, regardless of
the total number of processors. For the root patch and higher level childs,
patch\_weight is not used. However, patch\_weight must be set to 0 for these patches
to avoid confusion.&
n\_dom>1
\tabularnewline

\hline 
lredgrid\_phys &
L & .FALSE.& &
If set to .true.  is calculated on a reduced grid (= one grid level higher) &
\tabularnewline

\hline 
dynamics\_grid\_ filename &
C & & &
Array of the grid filenames to be used by the dycore.
May contain the keyword \texttt{<path>} which will be substituted by
\texttt{model\_base\_dir}. &
\tabularnewline

\hline 
dynamics\_parent\_ grid\_id &
I & & &
Array of the indexes of the parent grid filenames, as described by the dynamics\_grid\_filename array.
Indexes start at 1, an index of 0 indicates no parent. &
\tabularnewline

\hline 
radiation\_grid\_ filename &
C & & &
Array of the grid filenames to be used for the radiation model.
Filled only if the radiation grid is different from the dycore grid.
May contain the keyword \texttt{<path>} which will be substituted by
\texttt{model\_base\_dir}.
&
\tabularnewline

\hline 
dynamics\_radiation \_grid\_link &
I & & &
Array of the indexes linking the dycore grids, as described by the dynamics\_grid\_filename array,
and the radiation\_grid\_filename array. It provides the link index of the radiation\_grid\_filename,
for each entry of the dynamics\_grid\_filename array.
Indexes start at 1, an index of 0 indicates that the radiation grid is the same as the dycore grid.
Only needs to be filled when the radiation\_grid\_filename is defined. &
\tabularnewline

\end{longtab}

Defined and used in: src/namelists/mo\_grid\_nml.f90



%-------------------------------------------------------------------
% gridref_nml: grid refinement and nesting
%-------------------------------------------------------------------
\subsection{gridref\_nml}
\begin{longtab}

\hline 
grf\_intmethod\_c&
I& 2& &
Interpolation method for grid refinement (cell-based dynamical variables):&
n\_dom>1\tabularnewline
& & & & 1: parent-to-child copying & \tabularnewline
& & & & 2: gradient-based interpolation & \tabularnewline

\hline 
grf\_intmethod\_ct&
I& 2& &
Interpolation method for grid refinement (cell-based tracer variables):&
n\_dom>1\tabularnewline
& & & & 1: parent-to-child copying & \tabularnewline
& & & & 2: gradient-based interpolation & \tabularnewline

\hline 
grf\_intmethod\_e&
I& 4& &
Interpolation method for grid refinement (edge-based variables):&
n\_dom>1\tabularnewline
& & & & 1: inverse-distance weighting (IDW) & \tabularnewline
& & & & 2: RBF interpolation & \tabularnewline
& & & & 3: combination gradient-based / IDW & \tabularnewline
& & & & 4: combination gradient-based / RBF & \tabularnewline
& & & & 5/6: same as 3/4, respectively, but direct interpolation of mass fluxes along nest interface edges & \tabularnewline

\hline 
grf\_velfbk&
I& 1& & Method of velocity feedback:&
n\_dom>1\tabularnewline
& & & & 1: average of child edges 1 and 2 & \tabularnewline
& & & & 2: 2nd-order method using RBF interpolation & \tabularnewline

\hline 
grf\_scalfbk&
I& 2& & Feedback method for dynamical scalar variables ($T, p_{sfc}$):&
n\_dom>1\tabularnewline
& & & & 1: area-weighted averaging & \tabularnewline
& & & & 2: bilinear interpolation & \tabularnewline

\hline 
grf\_tracfbk&
I& 2& & Feedback method for tracer variables:&
n\_dom>1\tabularnewline
& & & & 1: area-weighted averaging & \tabularnewline
& & & & 2: bilinear interpolation & \tabularnewline

\hline 
grf\_idw\_exp\_e12&
R& 1.2& &
exponent of generalized IDW function for child edges 1/2 &
n\_dom>1\tabularnewline

\hline 
grf\_idw\_exp\_e34&
R& 1.7& &
exponent of generalized IDW function for child edges 3/4 &
n\_dom>1\tabularnewline

\hline 
rbf\_vec\_kern\_grf\_e&
I& 1& &
RBF kernel for grid refinement (edges):&
n\_dom>1\tabularnewline
& & & & 1: Gaussian & \tabularnewline
& & & & 2: $1/(1+r^{2})$ & \tabularnewline
& & & & 3: inverse multiquadric & \tabularnewline

\hline 
rbf\_scale\_grf\_e&
R& 0.5& &
RBF scale factor for grid refinement (edges)&
n\_dom>1\tabularnewline

\hline 
denom\_diffu\_t&
R& 135& &
Deniminator for lateral boundary diffusion of temperature&
n\_dom>1\tabularnewline

\hline 
denom\_diffu\_v&
R& 200& &
Deniminator for lateral boundary diffusion of velocity&
n\_dom>1\tabularnewline

\hline 
l\_mass\_consvcorr&
L& .TRUE.& &
.TRUE.: Apply mass conservation correction in feedback routine &
n\_dom>1\tabularnewline

\hline 
l\_density\_nudging &
L& .TRUE.& &
.TRUE.: Apply density nudging near lateral nest boundary &
n\_dom>1 .AND. lfeedback = .TRUE. \tabularnewline

\end{longtab}

Defined and used in: src/namelists/mo\_gridref\_nml.f90



%------------------------------------------------------------------------------
% initicon_nml
%------------------------------------------------------------------------------
\subsection{initicon\_nml}

\begin{longtab}

\hline 
init\_mode&
I & 1& &
1: start from DWD analysis \\
2: start from IFS analysis \\
3: combined mode: IFS atm + GME soil &
\tabularnewline

\hline 
nlevsoil\_in&
I & 4 & &
number of soil levels of input data&
init\_mode=2
\tabularnewline

\hline 
zpbl1 &
R & 500.0& m&
bottom height (AGL) of layer used for gradient computation&
\tabularnewline

\hline 
zpbl2 &
R & 1000.0& m&
top height (AGL) of layer used for gradient computation&
\tabularnewline

\hline 
l\_sst\_in&
L & .TRUE.& &
Logical switch. If true, the surface temperature of the water sea points is initialized 
with the SST provided in the ifs2icon file. If false, it is initialized with the skin 
temperature. If the SST is not provided in the ifs2icon file,l\_sst\_in is reset to false.  &
init\_mode=2
\tabularnewline

\hline 
l\_ana\_sfc&
L & .TRUE.& &
Logical switch. If true, soil/surface analysis fields are read from the analysis file dwdfg\_filename
If false, soil/surface analysis is not read. First guess is used, instead. &
init\_mode=1
\tabularnewline


\hline 
l\_coarse2fine\_mode&
L(max\_dom) & .FALSE.& &
If true, apply corrections for coarse-to-fine mesh interpolation to wind and temperature & 
\tabularnewline

\hline 
ifs2icon\_filename&
C &
&
&
Filename of IFS2ICON input file, default
''\texttt{<path>ifs2icon\_R<nroot>B<jlev>\_DOM<idom>.nc}''.
May contain the keywords \texttt{<path>} which will be substituted by
\texttt{model\_base\_dir}, as well as \texttt{nroot}, \texttt{jlev},
and \texttt{idom} defining the current patch. & init\_mode=2
\tabularnewline

\hline 
dwdfg\_filename&
C &
&
&
Filename of DWD first-guess input file, default
''\texttt{<path>dwdFG\_R<nroot>B<jlev>\_DOM<idom>.nc}''.
May contain the keywords \texttt{<path>} which will be substituted by
\texttt{model\_base\_dir}, as well as \texttt{nroot}, \texttt{jlev},
and \texttt{idom} defining the current patch. & init\_mode=1
\tabularnewline

\hline 
dwdana\_filename&
C &
&
&
Filename of DWD analysis input file, default
''\texttt{<path>dwdana\_R<nroot>B<jlev>\_DOM<idom>.nc}''.
May contain the keywords \texttt{<path>} which will be substituted by
\texttt{model\_base\_dir}, as well as \texttt{nroot}, \texttt{jlev},
and \texttt{idom} defining the current patch. & init\_mode=1
\tabularnewline

\hline 
filetype &
I& -1 (undef.)& &
One of CDI's FILETYPE\_XXX constants.
Possible values: 2 (=FILETYPE\_GRB2), 4 (=FILETYPE\_NC2).
If this parameter has not been set, we try to determine the file type by its extension "*.grb*" or ".nc".
&
\tabularnewline


\hline 
ana\_varlist &
C& & &
List of mandatory analysis fields that must be present in the analysis file. If these fields cannot be found in the analysis 
file, the model aborts. For all other analysis fields, the FG-fields will serve as fallback position.
& init\_mode=1
\tabularnewline


\hline 
ana\_varnames\_map\_file &
C& & &
Dictionary file which maps internal variable names onto
GRIB2 shortnames or NetCDF var names.
This is a text file with two columns separated by whitespace, where
left column: ICON variable name, right column: GRIB short name.
&
\tabularnewline

\end{longtab}

Defined and used in: src/namelists/mo\_initicon\_nml.f90


%------------------------------------------------------------------------------
% interpol_nml: horizontal interpolation/reconstruction
%------------------------------------------------------------------------------
\subsection{interpol\_nml}
\begin{longtab}

\hline 
llsq\_lin\_consv&
L& .FALSE.& &
conservative (T) or non-conservative (F) least-squares reconstruction for 2nd order (linear) transport&
\tabularnewline

\hline 
llsq\_high\_consv&
L& .TRUE.& &
conservative (T) or non-conservative (F) least-squares reconstruction for high order transport&
\tabularnewline

\hline
lsq\_high\_ord&
I& 3& &
polynomial order for high order reconstruction& \tabularnewline
& & & & 1: linear & ihadv\_tracer=4 \tabularnewline
& & & & 2: quadratic & \tabularnewline
& & & & 30: cubic (no $3^{rd}$ order cross deriv.) & \tabularnewline
& & & & 3: cubic & \tabularnewline

\hline 
rbf\_vec\_kern\_c&
I& 1& &
Kernel type for reconstruction at cell centres:& \tabularnewline
& & & & 1: Gaussian & \tabularnewline
& & & & 3: inverse multiquadric & \tabularnewline

\hline 
rbf\_vec\_kern\_e&
I& 3& &
Kernel type for reconstruction at edges:& \tabularnewline
& & & & 1: Gaussian & \tabularnewline
& & & & 3: inverse multiquadric & \tabularnewline

\hline 
rbf\_vec\_kern\_v&
I& 1& &
Kernel type for reconstruction at vertices:& \tabularnewline
& & & & 1: Gaussian & \tabularnewline
& & & & 3: inverse multiquadric & \tabularnewline

\hline 
rbf\_vec\_kern\_ll&
I& 1& &
Kernel type for reconstruction at lon-lat-points:& \tabularnewline
& & & & 1: Gaussian & \tabularnewline
& & & & 3: inverse multiquadric & \tabularnewline

\hline 
rbf\_vec\_scale\_c&
R(n\_dom)& resolution-dependent& &
Scale factor for RBF reconstruction at cell centres&
\tabularnewline

\hline 
rbf\_vec\_scale\_e&
R(n\_dom)& resolution-dependent& &
Scale factor for RBF reconstruction at edges&
\tabularnewline

\hline 
rbf\_vec\_scale\_v&
R(n\_dom)& resolution-dependent& &
Scale factor for RBF reconstruction at vertices&
\tabularnewline

\hline 
rbf\_vec\_scale\_ll&
R(n\_dom)& resolution-dependent& &
Scale factor for RBF reconstruction at lon-lat-points&
\tabularnewline

\hline 
nudge\_max\_coeff&
R& 0.02& &
Maximum relaxation coefficient for lateral boundary nudging&
\tabularnewline

\hline 
nudge\_efold\_width&
R& 2.5& &
e-folding width (in units of cell rows) for lateral boundary nudging coefficient &
\tabularnewline

\hline 
nudge\_zone\_width&
I& 8& &
Total width (in units of cell rows) for lateral boundary nudging zone&
\tabularnewline

\hline
i\_cori\_method&
I& 3& &
Selector for tangential wind reconstruction method
& currently only for cell\_type=6 \tabularnewline
& & & & 1: Almut's method for tangential wind, but PV usage as in TRSK &\tabularnewline
& & & & 2: method of Thuburn, Ringler, Skamarock and Klemp (TRSK) &\tabularnewline
& & & & 3: Almut's method for tangential wind and PV usage &\tabularnewline

\hline
l\_corner\_vort&
L& .TRUE.& &
switch whether the rhombus averaged corner vorticity is averaged 
to the hexagon (.TRUE.) or the rhombi are directly averaged to the hexagon (.FALSE.)&
i\_cori\_method=3
\tabularnewline

\hline 
l\_intp\_c2l&
L& .TRUE.& &
If .TRUE. directly interpolate scalar variables from cell centers to
lon-lat points, otherwise do gradient interpolation and
reconstruction.&
\tabularnewline

\hline 
rbf\_dim\_c2l&
I& 10& &
stencil size for direct lon-lat interpolation:
 4 = nearest neighbor, 
13 = vertex stencil, 
10 = edge stencil.&
\tabularnewline

\hline 
l\_mono\_c2l&
L& .TRUE.& &
Monotonicity can be enforced by demanding that the interpolated 
value is not higher or lower than the stencil point values.&
\tabularnewline

\end{longtab}

Defined and used in: src/namelists/mo\_interpol\_nml.f90



%-----------------------------------------------------------------------------
% dynamics_nml:
%-----------------------------------------------------------------------------
\subsection{dynamics\_nml}
This namelist is relevant if run\_nml:ldynamics=.TRUE.

\begin{longtab}

\hline 
iequations&
I& 1& &
Equations and prognostic variables. Use positive indices for the atmosphere 
and negative indices for the ocean.&\tabularnewline
&&&&0: shallow water model&\tabularnewline
&&&&1: hydrostatic atmosphere, T&\tabularnewline
&&&&2: hydrostatic atm., $\theta$$\cdot$dp&\tabularnewline
&&&&3: non-hydrostatic atmosphere&\tabularnewline
&&&&-1: hydrostatic ocean&
\tabularnewline


\hline 
idiv\_method&
I& 1& &
Method for divergence computation:&
grid\_nml:cell\_type=3\tabularnewline
& & & & 1: Standard Gaussian integral. Hydrostatic atm.~model: 
for unaveraged normal components,
Non-hydrostatic atm.~model: for averaged normal components &
\tabularnewline
& & & & 2: bilinear averaging of divergence& \tabularnewline

\hline 
divavg\_cntrwgt&
R& 0.5& &
Weight of central cell for divergence averaging&
idiv\_method= 2 
\tabularnewline

\hline 
sw\_ref\_height &
R &  0.9*2.94e4/g& m &
Reference height of shallow water model used for 
linearization in the semi-implicit time stepping scheme&
\tabularnewline

\hline 
lcoriolis &
L & .TRUE.& &
Coriolis force&
\tabularnewline

\end{longtab}

Defined and used in: src/namelists/mo\_dynamics\_nml.f90


%------------------------------------------------------------------------------
% limarea_nml:
%------------------------------------------------------------------------------
\subsection{limarea\_nml (Scope: l\_limited\_area=1 in grid\_nml)}

\begin{longtab}

\hline 
itype\_latbc&
I & 0& &
Type of lateral boundary nudging. Nudge from\\
0: the initial date,\\
1: IFS data analysis/forecast,\\
2: ICON output data (with the identical 3d grid)&
\tabularnewline

\hline 
dtime\_latbc&
R &
43200.0& s
&
Time step size of boundary data
&
itype\_latbc $\ge$ 1
\tabularnewline

\hline 
nlev\_latbc&
I &
0& s
&
Number of vertical levels in boundary data
&
itype\_latbc $\ge$ 1
\tabularnewline

\hline 
latbc\_filename&
C &
&
&
Filename of boundary data input file, default:\\
''\texttt{<path>prepicon<gridfile>\_<timestamp>}''. May contain the
keyword ''\texttt{<path>}'' which will be substituted by
\texttt{latbc\_path}.
&
itype\_latbc $\ge$ 1
\tabularnewline

\hline 
latbc\_path&
C &
&
&
Absolute path to boundary data.
&
itype\_latbc $\ge$ 1
\tabularnewline

\hline 
lupdate\_qvqc&
L &
.FALSE.
&
&
Switch to update $q_v$, $q_c$ from available boundary data. Only first ''\texttt{grf\_bdywidth\_c}'' cells and ''\texttt{grf\_bdywidth\_e}'' edges are updated.
&
itype\_latbc $\ge$ 1
\tabularnewline

\end{longtab}

Defined and used in: src/namelists/mo\_limarea\_nml.f90



%-----------------------------------------------------------------------------
% ha_dyn_nml: hydrostatic atm dynamics
%-----------------------------------------------------------------------------
\subsection{ha\_dyn\_nml}

This namelist is relevant if 
run\_nml:ldynamics=.TRUE. 
and dynamics\_nml:iequations=IHS\_ATM\_TEMP or IHS\_ATM\_THETA.

\begin{longtab}

\hline
itime\_scheme&
I& 4& &
Time integration scheme:& \tabularnewline
& & & & 11: pure advection (no dynamics)& \tabularnewline
& & & & 12: 2 time level semi implicit (not yet implemented)&
\tabularnewline
& & & & 13: 3 time level explicit&
\tabularnewline
& & & & 14: 3 time level with semi implicit correction&
\tabularnewline
& & & & 15: standard 4th-order Runge-Kutta method (4-stage)&
\tabularnewline
& & & & 16: SSPRK(5,4) scheme (5-stage)&
\tabularnewline

\hline 
ileapfrog\_startup&
I& 1& &
How to integrate the first time step when the leapfrog scheme
is chosen. 1 = Euler forward; 2 = a series of sub-steps. &
itime\_scheme= 13 or 14 \tabularnewline

\hline 
asselin\_coeff&
R& 0.1& &
Asselin filter coefficient&
itime\_scheme= 13 or 14 \tabularnewline

\hline 
si\_2tls &
R & 0.6 & &
weight of time step n+1. Valid range: [0,1] &
itime\_scheme=12\tabularnewline

\hline
si\_expl\_scheme &
I & 2 & &
scheme for the explicit part used in the 2 time level
semi-implicit time stepping scheme. 1 = Euler forward;
2 = Adams-Bashforth 2nd order &
itime\_scheme=12\tabularnewline

\hline 
si\_cmin&
R & 30.0 & m/s&
semi implicit correction is done for eigenmodes with speeds larger
than si\_cmin&
itime\_scheme=14 and lsi\_3d=.FALSE.\tabularnewline

\hline 
si\_coeff&
R & 1.0 & &
weight of the semi implicit correction&
itime\_scheme=14 \tabularnewline

\hline 
si\_offctr&
R & 0.7 & &
&
itime\_scheme=14 \tabularnewline

\hline 
si\_rtol &
R & 1.0e-3 & &
relative tolerance for GMRES solver&
itime\_scheme=14 \tabularnewline

\hline 
lsi\_3d&
L& .FALSE.& &
3D GMRES solver or decomposistion into 2D problems&
lshallow\_water=.FALSE. and itime\_scheme=14\tabularnewline
\hline 

\hline
ldry\_dycore &
L & .TRUE. & &
Assume dry atmosphere &
iequations$\in$\{1,2\}
\tabularnewline

\hline 
lref\_temp &
L & .FALSE. & &
Set a background temperature profile as base state 
when computing the pressure graident force &
iequations$\in$\{1,2\} 
\tabularnewline

\end{longtab}



%-----------------------------------------------------------------------------
% nonhydrostatic_nml:
%-----------------------------------------------------------------------------
\subsection{nonhydrostatic\_nml (relevant if run\_nml:iequations=3)}

\begin{longtab}

\hline
itime\_scheme&
I& 4& &
Options for predictor-corrector time-stepping scheme:& \tabularnewline 
& & & &
4: Contravariant vertical velocity is computed in the predictor step only, 
   velocity tendencies are computed in the corrector step only (most efficient option) \\
5: Contravariant vertical velocity is computed in both substeps (beneficial for numerical
   stability in very-high resolution setups with extremely steep slops, otherwise no significant impact)\\
6: As 5, but velocity tendencies are also computed in both substeps (no apparent benefit, but more expensive) &
iequations=3 and cell\_type=3
\tabularnewline

\hline 
rayleigh\_type&
I& 2& &
Type of Rayleigh damping\\
1: CLASSICAL (requires velocity reference state!)\\
2: Klemp (2008) type &
cell\_type=3
\tabularnewline

\hline 
rayleigh\_coeff&
R(n\_dom)& 0.05& &
Rayleigh damping coefficient $1/\tau_{0}$ (Klemp, Dudhia, Hassiotis: MWR136, pp.3987-4004)&
cell\_type=3
\tabularnewline

\hline 
damp\_height&
R(n\_dom)& 45000& m&
Height at which Rayleigh damping of vertical wind starts&
\tabularnewline

\hline 
htop\_moist\_proc&
R& 22500.0& m&
Height above which moist physics and advection of cloud and precipitation variables are turned off&
\tabularnewline

\hline 
hbot\_qvsubstep&
R& 24000.0& m&
Height above which QV is advected with substepping scheme (must be larger than htop\_moist\_proc)&
cell\_type=3 and \\
ihadv\_tracer=22 or 32
\tabularnewline

\hline 
k2\_updamp\_coeff&
R& 2.0e6& &
enhanced 2nd order diffusion coefficient in upper damping layer&
cell\_type=6, hdiff\_order=3 (Smagorinski)
\tabularnewline

\hline 
vwind\_offctr&
R& 0.15& &
Off-centering in vertical wind solver&
cell\_type=3
\tabularnewline

\hline 
rhotheta\_offctr&
R& -0.1& &
Off-centering of density and potential temperature at interface level &
cell\_type=3
\tabularnewline

\hline 
veladv\_offctr&
R& 0.25& &
Off-centering of velocita advection in corrector step &
cell\_type=3
\tabularnewline

\hline 
ivctype&
I& 2& &
Type of vertical coordinate:\\
1: Gal-Chen hybrid \\
2: SLEVE (uses sleve\_nml)&
\tabularnewline

\hline 
iadv\_rcf&
I& 4& &
reduced calling frequency (rcf) for transport\\
1: no rcf (every dynamics-step)\\
2: transport every 2. step \\
4: $\dots$  \\
Setting odd values (besides 1) requires l\_nest\_rcf = .TRUE.&
\tabularnewline

\hline 
lhdiff\_rcf&
L& .TRUE. & &
.TRUE.: Compute diffusion only at advection time steps (in this case, 
divergence damping is applied in the dynamical core)
& cell\_type=3
\tabularnewline

\hline 
lextra\_diffu&
L& .TRUE. & &
.TRUE.: Apply additional momentum diffusion at grid points close to the stability limit for vertical advection (becomes effective 
extremely rarely in practice; this is mostly an emergency fix for pathological cases with very large orographic gravity waves)
& cell\_type=3
\tabularnewline

\hline 
lbackward\_integr&
L& .FALSE. & &
.TRUE.: Integrate backward in time (preparation for testing a digital filter initialization)
& cell\_type=3
\tabularnewline

\hline 
divdamp\_fac&
R& 0.004& &
Scaling factor for divergence damping&
lhdiff\_rcf = .TRUE.
\tabularnewline

\hline 
divdamp\_order&
I& 4& &
Order of divergence damping: \\ 
2 = second-order divergence damping \\
4 = fourth-order divergence damping \\
24 = combined second-order and fourth-order divergence damping (use for data assimilation cycle only!) &
lhdiff\_rcf = .TRUE.
\tabularnewline

\hline 
divdamp\_type&
I& 2 & &
Type of divergence damping: \\ 
2 = divergence damping acting on 2D divergence \\
3 = divergence damping acting on 3D divergence &
lhdiff\_rcf = .TRUE.
\tabularnewline

\hline 
l\_nest\_rcf&
L& .TRUE.& &
Synchronize interpolation/feedback calls with advection (transport) time steps.
l\_nest\_rcf is automatically reset to .FALSE. if iadv\_rcf=1 &
cell\_type=3
\tabularnewline

\hline 
l\_masscorr\_nest&
L& .FALSE.& &
.TRUE.: Apply mass conservation correction also in nested domain &
cell\_type=3
\tabularnewline

\hline 
iadv\_rhotheta&
I& 2& &
Advection method for rho and rhotheta:\\
1: simple second-order upwind-biased scheme \\
2: 2nd order Miura horizontal &
cell\_type=3 \tabularnewline
& & & & 3: 3rd order Miura horizontal (not recommended)&
\tabularnewline

\hline 
igradp\_method&
I& 3& &
Discretization of horizontal pressure gradient:\\
1: conventional discretization with metric correction term\\
2: Taylor-expansion-based reconstruction of pressure (advantageous at very high resolution)\\
3: Similar discretization as option 2, but uses hydrostatic approximation
for downward extrapolation over steep slopes \\
4: Cubic/quadratic polynomial interpolation for pressure reconstruction \\
5: Same as 4, but hydrostatic approximation for downward extrapolation over steep slopes &
cell\_type=3
\tabularnewline

\hline 
l\_zdiffu\_t&
L& .TRUE.& &
.TRUE.: Compute Smagorinsky temperature diffusion truly horizontally over steep slopes&
cell\_type=3 .AND. hdiff\_order=3/5 .AND. lhdiff\_temp = .true.
\tabularnewline

\hline 
thslp\_zdiffu&
R& 0.025& &
Slope threshold above which truly horizontal temperature diffusion is activated&
cell\_type=3 .AND. hdiff\_order=3/5 .AND. lhdiff\_temp=.true. .AND. l\_zdiffu\_t=.true.
\tabularnewline

\hline
thhgtd\_zdiffu&
R& 200& m&
Threshold of height difference between neighboring grid points above which 
truly horizontal temperature diffusion is activated (alternative criterion to thslp\_zdiffu)&
cell\_type=3 .AND. hdiff\_order=3/5 .AND. lhdiff\_temp=.true. .AND. l\_zdiffu\_t=.true.
\tabularnewline

\hline 
exner\_expol&
R& 0.5& &
Temporal extrapolation (fraction of dt) of Exner function for computation of horizontal pressure gradient&
cell\_type=3
\tabularnewline

\hline 
l\_open\_ubc&
L& .FALSE.& &
.TRUE.: Use open upper boundary condition (rather than w=0) to better conserve sea-level pressure in the presence of diabatic heating&
cell\_type=3
\tabularnewline

\hline 
ltheta\_up\_hori &
L& .FALSE.& &
upstream biased horizontal advection for theta (see also upstr\_beta)&
cell\_type=6
\tabularnewline

\hline 
upstr\_beta &
R& 1.0& &
Selection of order for horiz.~theta advection: 3rd order=1.0, 4th order=0.0  &
cell\_type=6
\tabularnewline

\hline
gmres\_rtol\_nh &
R& 1.0e-6& &
relative tolerance for convergence in gmres solver &
cell\_type=6
\tabularnewline

\hline
\end{longtab}

Defined and used in: src/namelists/mo\_nonhydrostatic\_nml.f90

%-----------------------------------------------------------------------------
% sleve_nml:
%-----------------------------------------------------------------------------
\subsection{sleve\_nml (relevant if nonhydrostatic\_nml:ivctype=2)}
\begin{longtab}

\hline 
min\_lay\_thckn&
R& 50& m&
Layer thickness of lowermost layer; specifying zero or a negative value leads to constant layer thicknesses
determined by top\_height and nlev&
\tabularnewline

\hline 
top\_height&
R& 23500.0& m&
Height of model top&
\tabularnewline

\hline 
stretch\_fac&
R& 1.0& &
Stretching factor to vary distribution of model levels; 
values $<$1 increase the layer thickness near the model top&
\tabularnewline

\hline 
decay\_scale\_1&
R& 4000& m&
Decay scale of large-scale topography component&
\tabularnewline

\hline 
decay\_scale\_2&
R& 2500& m&
Decay scale of small-scale topography component&
\tabularnewline

\hline 
decay\_exp&
R& 1.2& &
Exponent of decay function&
\tabularnewline

\hline 
flat\_height &
R& 16000& m&
Height above which the coordinate surfaces are flat&
\tabularnewline

\hline 
lread\_smt &
L& .FALSE.& &
read smoothed topography from file (TRUE) or compute internally (FALSE)&
\tabularnewline

\hline
\end{longtab}

Defined and used in: src/namelists/mo\_sleve\_nml.f90


%-----------------------------------------------------------------------------
% diffusion_nml: horizontal (numerical) diffusion
%-----------------------------------------------------------------------------
\subsection{diffusion\_nml}
\begin{longtab}

\hline 
lhdiff\_temp&
L& .TRUE. & &
Diffusion on the temperature field&
\tabularnewline

\hline 
lhdiff\_vn&
L& .TRUE. & &
Diffusion on the horizontal wind field&
\tabularnewline

\hline 
lhdiff\_w&
L& .TRUE. & &
Diffusion on the vertical wind field&
\tabularnewline

\hline 
hdiff\_order&
I& 4 (hydro) \\ 5 (NH)& &
Order of $\nabla$ operator for diffusion:& \tabularnewline
& & & & -1: no diffusion& \tabularnewline
& & & & 2: $\nabla^{2}$ diffusion (not available for NH model on triangles!)& \tabularnewline
& & & & 3: Smagorinsky $\nabla^{2}$ diffusion 
(includes frictional heating for the hexagonal model if lhdiff\_temp=.TRUE.) &
\tabularnewline
& & & & 4: $\nabla^{4}$ diffusion&
\tabularnewline
& & & &
5: Smagorinsky $\nabla^{2}$ diffusion combined with $\nabla^{4}$
background diffusion as specified via hdiff\_efdt\_ratio&
\tabularnewline
& & & &
defaults: 2 for hexagonal model, 4 for triangular model; for triangular NH model, 5 is strongly recommended!&
\tabularnewline
& & & & 24 or 42: $\nabla{2}$ diffusion from model top to a certain level
(cf. k2\_pres\_max and k2\_klev\_max below); 
$\nabla^{4}$ for the lower levels.  &
24 and 42 currently allowed only in the hydrostatic atm model
(run\_nml:iequation = 1 or 2).
\tabularnewline

\hline 
itype\_vn\_diffu&
I& 1 & &
Reconstruction method used for Smagorinsky diffusion: \\
1: u/v reconstruction at vertices only \\
2: u/v reconstruction at cells and vertices& iequations=3, hdiff\_order=3 or 5
\tabularnewline

\hline 
itype\_t\_diffu&
I& 1 & &
Discretization of temperature diffusion: \\
1: $K_h \nabla^2 T$ \\
2: $\nabla \cdot (K_h \nabla T)$  & iequations=3, hdiff\_order=3 or 5
\tabularnewline

\hline 
k2\_pres\_max&
R& -99.& Pa &
Pressure level above which $\nabla^2$ diffusion is applied.&
hdiff\_order = 24 or 42, and run\_nml:iequation = 1 or 2. 
\tabularnewline

\hline 
k2\_klev\_max&
I& 0&&
Index of the vertical level till which (from the model top) 
$\nabla^2$ diffusion is applied.
If a positive value is specified for k2\_pres\_max, 
k2\_klev\_max is reset accordingly during the initialization 
of a model run.&
hdiff\_order = 24 or 42, and run\_nml:iequation = 1 or 2. 
\tabularnewline

\hline 
hdiff\_efdt\_ratio&
R& 1.0 (hydro) \\ 15.0 (NH) & &
ratio of e-folding time to time step (or 2{*} time step when using
a 3 time level time stepping scheme) (only for triangles currently; 
for triangular NH model, values between 10 and 20 are recommended when using hdiff\_order=5) &
\tabularnewline

\hline 
hdiff\_w\_efdt\_ratio&
R& 15.0  & &
ratio of e-folding time to time step for diffusion on vertical wind speed & iequations=3
\tabularnewline

\hline 
hdiff\_min\_efdt\_ratio&
R& 1.0 & &
minimum value of hdiff\_efdt\_ratio near model top & iequations=3 .AND. cell\_type=3 .AND. hdiff\_order=4
\tabularnewline

\hline 
hdiff\_tv\_ratio&
R& 1.0& &
Ratio of diffusion coefficients for temperature and normal wind: $T:v_{n}$&
\tabularnewline

\hline 
hdiff\_multfac&
R& 1.0& &
Multiplication factor of normalized diffusion coefficient for nested
domains&
n\_dom>1\tabularnewline

\hline 
hdiff\_smag\_fac&
R& 0.15 (hydro) \\ 0.025 (NH)& &
Scaling factor for Smagorinsky diffusion&
for triangles only with iequations=3, for hexagons with hdiff\_order=3
\tabularnewline

\end{longtab}

Defined and used in: src/namelists/mo\_diffusion\_nml.f90


%------------------------------------------------------------------------------
% io_nml:
%------------------------------------------------------------------------------
\subsection{io\_nml}
\begin{longtab}

\hline 
out\_expname &
C& 'IIIEEEETTTT'& &
Outfile basename&
\tabularnewline

\hline 
out\_filetype&
I& 2 & &
Type of output format:\\
1: GRIB1 (not yet implemented)\\
2: netCDF&
\tabularnewline

\hline 
lkeep\_in\_sync&
L& .FALSE. & &
Sync output stream with file on disk after each timestep&
\tabularnewline

\hline 
dt\_data&
R& 21600.0 & s&
Output time interval&
\tabularnewline

\hline 
dt\_diag&
R& 86400. & &
diagnostic integral output interval &
\tabularnewline

\hline 
dt\_file&
R& 2592000 & s&
Time interval of triggering new output file&
\tabularnewline

\hline
dt\_checkpoint&
R& 2592000 & s&
Time interval for writing restart files.
Note that if the value of dt\_checkpoint resulting from 
model default or user's specification is longer than time\_nml:dt\_restart, 
it will be reset (by the model) to dt\_restart so 
that at least one restart file is generated during the restart cycle. 
&
\tabularnewline

\hline 
lwrite\_vorticity &
L& .TRUE.& &
write out averaged vorticity at vertices &
\tabularnewline

\hline 
lwrite\_initial &
L& .TRUE.& &
write out initial state &
\tabularnewline

\hline 
lwrite\_dblprec &
L& .FALSE.& &
write out double precision &
\tabularnewline

\hline 
lwrite\_oce\_timestepping &
L& .FALSE.& &
write out intermediate ocean vars&
\tabularnewline

\hline 
lwrite\_divergence &
L& .TRUE.& &
write out divergence at cells &
\tabularnewline

\hline 
lwrite\_omega &
L& .TRUE.& &
write out vertical velocity in pressure coords.&
Always .FALSE. for nonhydrostatic and shallow water models
\tabularnewline

\hline 
lwrite\_pres &
L& .TRUE.& &
write out full level pressure&
lshallow\_water=.FALSE.\tabularnewline

\hline 
lwrite\_z3 &
L& .TRUE.& &
write out geopotential on full levels&
lshallow\_water=.FALSE.\tabularnewline

\hline 
lwrite\_tracer &
L(ntracer)& .TRUE.& &
write out tracer at cells &
\tabularnewline

\hline
lwrite\_tend\_phy &
L& .TRUE. \\ .FALSE. \\ (Scope) & &
Physics induced tendencies.&
.TRUE. if iforcing=iecham\\
.FALSE. else
\tabularnewline

\hline
lwrite\_radiation &
L&  .FALSE. & &
Radiation related fields.&
Always .FALSE. if iforcing=inoforcing, iheldsuarez, ildf\_dry 
\tabularnewline

\hline
lwrite\_precip &
L&  .FALSE. & &
Precipitation &
Always .FALSE. if iforcing=inoforcing, iheldsuarez, ildf\_dry 
\tabularnewline

\hline
lwrite\_cloud&
L& .FALSE.  & &
Cloud variables &
Always .FALSE. if iforcing=inoforcing, iheldsuarez, ildf\_dry 
\tabularnewline

\hline
lwrite\_tke& L& .TRUE. &
& TKE & .FALSE.  \\
Always .FALSE. if iforcing=inoforcing, iheldsuarez, ildf\_dry 
\tabularnewline

\hline
lwrite\_surface&
L& .FALSE. & &
surface variables &
Always .FALSE. if iforcing=inoforcing, iheldsuarez, ildf\_dry 
\tabularnewline

\hline
lwrite\_extra&
L& .FALSE. & &
debug fields &
.TRUE. if inextra\_2d /\_3d $> 0$ \\
.FALSE. else
\tabularnewline

\hline 
inextra\_2d&
I &
0&&
Number of 2D Fields for diagnostic/debugging output. &
iequations = 3 {\color{red}(to be done for 1, 2)}
\tabularnewline
\hline 
inextra\_3d&
I &
0&&
Number of 3D Fields for diagnostic/debugging output. &
iequations = 3 {\color{red}(to be done for 1, 2)}
\tabularnewline

\hline 
lflux\_avg&
L& .TRUE. & &
if .FALSE. the output fluxes are accumulated  \\
 from the beginning of the run                \\
if .TRUE. the output fluxes are average values\\ 
 from the beginning of the run, except of     \\
 TOT\_PREC that would be accumulated &
iequations=3\\
iforcing=3
\tabularnewline

\hline 
itype\_pres\_msl&
I& 1 & &
Specifies method for computation of mean sea level pressure (and geopotential at
pressure levels below the surface). \\
1: GME-type extrapolation, \\
2: stepwise analytical integration, \\
3: current IFS method, \\
4: IFS method with consistency correction
&
\tabularnewline

\hline 
itype\_rh&
I& 1 & &
Specifies method for computation of relative humidity \\
1: WMO-type: water only (e\_s=e\_s\_water), \\
2: IFS-type: mixed phase (water and ice), \\
3: IFS-type with clipping ($\mathrm{rh}\leq100$)
&
\tabularnewline

\hline
 output\_nml\_dict &
C&' '& &
 File containing the mapping of variable names to the internal ICON names. 
 May contain the keyword \texttt{<path>} which will be substituted by
 \texttt{model\_base\_dir}.\\
 The format of this file: \\
 One mapping per line, first the name as given in the \texttt{ml\_varlist},
 \texttt{hl\_varlist}, \texttt{pl\_varlist} or \texttt{il\_varlist}
 of the \texttt{output\_nml} namelists, then the internal ICON name,
 separated by an arbitrary number of blanks.
 The line may also start and end with an arbitrary number of blanks.
 Empty lines or lines starting with \# are treated as comments. \\
 Names not covered by the mapping are used as they are.
&
\texttt{output\_nml} namelists
\tabularnewline

\hline
 netcdf\_dict &
C&' '& &
 File containing the mapping from internal names to names written to NetCDF. 
 May contain the keyword \texttt{<path>} which will be substituted by
 \texttt{model\_base\_dir}.\\
 The format of this file: \\
 One mapping per line, first the name written to NetCDF,
 then the internal name, separated by an arbitrary number of blanks
 (\emph{inverse to the definition of \emph{output\_nml\_dict}}).
 The line may also start and end with an arbitrary number of blanks.
 Empty lines or lines starting with \# are treated as comments. \\
 Names not covered by the mapping are output as they are. \\
 Note that the specification of output variables, e.\,g.\ in
 \texttt{ml\_varlist}, is independent from this renaming, see
 the namelist parameter \texttt{varnames\_map\_file} for this.
& 
\texttt{output\_nml} namelists, 
NetCDF output
\tabularnewline

\hline
lzaxis\_reference&
L& .TRUE. & &
FALSE: use vertical axis ZAXIS\_HYBRID for 3D atmospheric fields\\
TRUE: use vertical axis ZAXIS\_REFERENCE for 3D atmospheric fields &
will be removed after some testing phase
\tabularnewline

\hline 
\end{longtab}

Defined and used in: src/namelists/mo\_io\_nml.f90



%------------------------------------------------------------------------------
% output_nml:
%------------------------------------------------------------------------------
\subsection{output\_nml}

Please note: There may be several instances of
output\_nml in the namelist file, every one defining a list of variables with
separate attributes for output.

\begin{longtab}

\hline 
filetype &
I&4& &
One of CDI's FILETYPE\_XXX constants.
Possible values: 2 (=FILETYPE\_GRB2), 4 (=FILETYPE\_NC2), 5 (=FILETYPE\_NC4)
&
\tabularnewline

\hline
 mode &
I&2& &
 1 = forecast mode, 2 = climate mode \\
 In climate mode the time axis of the output file 
 is set to TAXIS\_ABSOLUTE. In forecast mode it is set 
 to TAXIS\_RELATIVE. Till now the forecast mode only 
 works if the output is at multiples of 1 hour 
&
\tabularnewline

\hline
 taxis\_tunit &
I&3& &
 3 = TUNIT\_HOUR , 2 = TUNIT\_MINUTE \\
 Time unit of the TAXIS\_RELATIVE time axis. 
 For a complete list of possible values see cdi.inc
 Till now it only works for taxis\_tunit=3 
& mode=1
\tabularnewline

\hline
 dom(:) &
I&-1& &
 Array of domains for which this name-list is used.
 If not specified (or specified as -1 as the first array member),
 this name-list will be used for all domains. \\
 Attention: Depending on the setting of the parameter l\_output\_phys\_patch
 these are either logical or physical domain numbers!
&
\tabularnewline

\hline
 output\_time\_unit &
I&1& &
 1 = second, 2=minute, 3=hour, 4=day, 5=month, 6=year
&
\tabularnewline

\hline
 output\_bounds(3,:) &
R&None& &
 post-processing times in units defined by output\_time\_unit: start, end, increment.
 There may be specified several triples (up to 100) which must be in increasing order.
&
\tabularnewline

\hline
 steps\_per\_file &
I&100& &
 Max number of output steps in one output file. If this number is reached, a new output
 file will be opened.
&
\tabularnewline

\hline
 include\_last &
L&.TRUE.& &
 Flag whether to include the last time step
&
\tabularnewline

\hline
 output\_grid &
L&.FALSE.& &
 Flag whether grid information is output (in NetCDF output)
&
\tabularnewline

\hline
 output\_filename &
C&None& &
 Output filename prefix (which may include path).
 Domain number, level type, file number and extension will be added,
 according to the format given in namelist parameter ``filename\_format''.
&
\tabularnewline

\hline
 filename\_format &
C&see description.& &
 Output filename format. Includes keywords \texttt{path}, \texttt{output\_filename}, \texttt{physdom}, \texttt{levtype}, \texttt{levtype\_l}, \texttt{jfile}, \texttt{ddhhmmss}, see below.
 Default is \texttt{<output\_filename>\_DOM<physdom>\_<levtype>\_<jfile>}
&
\tabularnewline

\hline
 lwrite\_ready &
L&.FALSE.& &
 Flag if a "ready file" (sentinel file) should be written at the end of each output stage.
&
\tabularnewline

\hline
 ready\_directory &
C&None& &
 Output directory for ready files.
&
\tabularnewline

\hline
 ml\_varlist(:) &
C&None& &
 Name of model level fields to be output.
&
\tabularnewline

\hline
 pl\_varlist(:) &
C&None& &
 Name of pressure level fields to be output.
&
\tabularnewline

\hline
 hl\_varlist(:) &
C&None& &
 Name of height level fields to be output.
&
\tabularnewline


\hline
 il\_varlist(:) &
C&None& &
 Name of isentropic level fields to be output.
&
\tabularnewline

\hline
 p\_levels(:) &
R&None& hPa &
 pressure levels \\
 {\color{red} Not yet implemented.} \\
 {\color{red} The pressure levels are currently always taken from array plevels in namelist nh\_pzlev\_nml. }

&
\tabularnewline

\hline
 h\_levels(:) &
R&None& m &
 height levels \\
 {\color{red} Not yet implemented.} \\
 {\color{red} The height levels are currently always taken from array zlevels in namelist nh\_pzlev\_nml. }
&
\tabularnewline

\hline
 i\_levels(:) &
R&None& K &
 isentropic levels \\
 {\color{red} Not yet implemented.} \\
 {\color{red} The isentropic levels are currently always taken from array ilevels in namelist nh\_pzlev\_nml. }
&
\tabularnewline

\hline
 remap &
I&0& &
 interpolate horizontally, 0: none, 1: to regular lat-lon grid
&
\tabularnewline

\hline
 reg\_lon\_def(3) &
R&None& &
 The regular grid points are specified by three values: start, increment, end given in degrees.
It contains all grid points $start + k * increment <= end$, where $k$ is an integer.
For longitude values the last grid point is omitted if the end point matches the start point, e.g. for 0 and 360 degrees.
Instead of defining an increment it is also possible to prescribe the number of grid points, where it is expected that this
value is larger than 5.0.
&
remap=1
\tabularnewline

\hline
 reg\_lat\_def(3) &
R&None& &
 start, increment, end latitude in degrees.
 See reg\_lon\_def for details.
&
remap=1
\tabularnewline

\hline
 north\_pole(2) &
R&0,90& &
 definition of north pole for rotated lon-lat grids.
&
\tabularnewline

\hline 
\end{longtab}

\paragraph{Variable Groups:} 
Using the \texttt{"group:"} keyword for the namelist parameters \texttt{ml\_varlist}, \texttt{hl\_varlist}, \texttt{pl\_varlist},
sets of common variables can be added to the output:
\begin{tabbing}
\hspace*{0.4\textwidth} \= \kill
\texttt{group:all}                     \>      output of all variables (caution: do not combine with \underline{mixed} vertical interpolation) \\
\texttt{group:atmo\_ml\_vars}          \>      basic atmospheric variables on model levels                          \\
\texttt{group:atmo\_pl\_vars}, 
\texttt{group:atmo\_zl\_vars}          \>      same set as atmo\_ml\_vars, but except pres and height, respectively \\
\texttt{group:nh\_prog\_vars}          \>      additional prognostic variables of the nonhydrostatic model          \\
\texttt{group:atmo\_derived\_vars}     \>      derived atmospheric variables                                        \\
\texttt{group:rad\_vars}               \>                                                                           \\
\texttt{group:precip\_vars}            \>                                                                           \\
\texttt{group:cloud\_diag}             \>                                                                           \\
\texttt{group:pbl\_vars}               \>                                                                           \\
\texttt{group:phys\_tendencies}        \>                                                                           \\
\texttt{group:land\_vars}              \>                                                                           \\
\texttt{group:multisnow\_vars}         \>     tile-averaged variables                                               
\end{tabbing}

\paragraph{Keyword substitution in output filename (\texttt{filename\_format}):} 
\begin{tabbing}
\hspace*{0.4\textwidth} \= \kill
\texttt{path}              \>  substituted by \texttt{model\_base\_dir}                 \\ 
\texttt{output\_filename}  \>  substituted by \texttt{output\_filename}                 \\
\texttt{physdom}           \>  substituted by physical patch ID                         \\
\texttt{levtype}           \>  substituted by level type ``ML'', ``PL'', ``HL'', ``IL'' \\
\texttt{levtype\_l}        \>  like \texttt{levtype}, but in lower case                 \\
\texttt{jfile}             \>  substituted by output file counter                       \\
\texttt{ddhhmmss}          \>  substituted by day-hour-minute-second string
\end{tabbing}


Defined and used in: src/namelists/mo\_name\_list\_output.f90


%------------------------------------------------------------------------------
% gribout_nml:
%------------------------------------------------------------------------------
\subsection{gribout\_nml}
\begin{longtab}

\hline 
significanceOfReferenceTime &
I& 1& &
Significance of reference time\\
- GRIB2 code table 1.2 &
filetype=2
\tabularnewline

\hline 
productionStatusOfProcessedData&
I& 1 & &
Production status of data\\
- GRIB2 code table 1.3 &
filetype=2
\tabularnewline

\hline 
typeOfProcessedData&
I& 1 & &
Type of data \\
- GRIB2 code table 1.4 &
filetype=2
\tabularnewline

\hline 
typeOfGeneratingProcess&
I& 2 & &
Type of generating process \\
- GRIB2 code table 4.3 &
filetype=2
\tabularnewline

\hline 
backgroundProcess&
I& 0 & &
Background process \\
- GRIB2 code table backgroundProcess.table &
filetype=2
\tabularnewline

\hline 
generatingProcessIdentifier&
I(n\_dom)& 1 & &
generating Process Identifier \\
- GRIB2 code table generatingProcessIdentifier.table &
filetype=2
\tabularnewline

\hline 
localDefinitionNumber&
I& 254 & &
local Definition Number\\
- GRIB2 code table grib2LocalSectionNumber.78.table &
filetype=2
\tabularnewline

\hline 
localNumberOfExperiment&
I& 1 & &
local Number of Experiment &
filetype=2
\tabularnewline

\hline 
generatingCenter&
I& -1 & &
Output generating center. If this key is not set, center information is taken from the grid file\\
DWD: 78 \\
MPIMET: 98 \\
ECMWF: 98 &
filetype=2
\tabularnewline

\hline 
generatingSubcenter&
I& -1 & &
Output generating Subcenter. If this key is not set, subcenter information is taken from the grid file\\
DWD: 255\\
MPIMET: 232\\ 
ECMWF: 0 &
filetype=2
\tabularnewline

\hline 
ldate\_grib\_act&
L& .TRUE. & &
GRIB creation date\\
.TRUE.: add creation date\\
.FALSE.: add dummy date &
filetype=2
\tabularnewline

\hline 
productDefinitionTemplateNumber&
I& -1 & &
Local definiton for ensemble products
(only set if value changed from default) &
filetype=2
\tabularnewline

\hline 
typeOfEnsembleForecast&
I& -1 & &
Local definiton for ensemble products
(only set if value changed from default) &
filetype=2
\tabularnewline

\hline 
numberOfForecastsInEnsemble&
I& -1 & &
Local definiton for ensemble products,
(only set if value changed from default) &
filetype=2
\tabularnewline

\hline 
perturbationNumber&
I& -1 & &
Local definiton for ensemble products,
(only set if value changed from default) &
filetype=2
\tabularnewline

\hline 
\end{longtab}

Defined and used in: src/namelists/mo\_gribout\_nml.f90


%------------------------------------------------------------------------------
% meteogram_output_nml:
%------------------------------------------------------------------------------
\subsection{meteogram\_output\_nml}
\begin{longtab}

\hline 
lmeteogram\_enabled&
L(n\_dom) &
.FALSE.&&
Flag. True, if meteogram of output variables is desired.&
\tabularnewline

\hline 
zprefix&
C(n\_dom) &
``METEOGRAM\_''&&
string with file name prefix for output file&
\tabularnewline

\hline 
ldistributed&
L(n\_dom) &
.TRUE.&&
Flag. Separate files for each PE.&
\tabularnewline

\hline 
n0\_mtgrm&
I(n\_dom) &
1&&
initial time step for meteogram output&
\tabularnewline

\hline 
ninc\_mtgrm&
I(n\_dom) &
1&&
output interval (in time steps)&
\tabularnewline

\hline 
stationlist\_tot&
&
53.633,  9.983, 'Hamburg' &&
list of meteogram stations (triples with lat, lon, name string)&
\tabularnewline

\hline
\end{longtab}

Defined and used in: src/namelists/mo\_mtgrm\_nml.f90



%------------------------------------------------------------------------------
% nh_pzlev_nml:
%------------------------------------------------------------------------------
\subsection{nh\_pzlev\_nml}
\begin{longtab}
 
\hline 
nzlev &
I& 10 & &
number of height levels&
iequations=3\\
\tabularnewline

\hline 
nplev &
I& 10 & &
number of pressure levels&
iequations=3\\
\tabularnewline

\hline 
nilev &
I& 3 & &
number of isentropes&
iequations=3\\
\tabularnewline

\hline 
zlevels &
R& 10000,\\ 9000,\\...,\\1000,\\0 & m&
array of height levels&
iequations=3\\
\textcolor{red}{level ordering from TOA to bottom}
\tabularnewline

\hline 
plevels &
R&  100000,\\ 90000,\\ 80000,\\...,\\10000& Pa &
array of pressure levels&
iequations=3\\
\textcolor{red}{level ordering from TOA to bottom}
\tabularnewline

\hline 
ilevels &
R&  340,\\ 320,\\ 300 & K &
array of isentropic levels&
iequations=3\\
\textcolor{red}{level ordering from TOA to bottom}
\tabularnewline

\end{longtab}

Defined and used in: src/namelists/mo\_nh\_pzlev\_nml.f90



%------------------------------------------------------------------------------
% transport_nml:
%------------------------------------------------------------------------------
\subsection{transport\_nml (used if run\_nml/ltransport=.TRUE.)}

\begin{longtab}

lvadv\_tracer&
L& .TRUE.& & TRUE : compute vertical tracer advection& \tabularnewline
& &       & & FALSE: do not compute vertical tracer advection &
\tabularnewline

\hline 
ihadv\_tracer&
I(ntracer)&
2& & Tracer specific method to compute horizontal advection:& \tabularnewline
& & 5& & 0: no horiz. transport & \tabularnewline
& & & & 1: upwind (1st order) & \tabularnewline
& & & & 2: miura (2nd order, lin. reconstr.)& if cell\_type=3 \tabularnewline
& & & & 3: miura3 (quadr. or cubic reconstr.)& lsq\_high\_ord $\in$ [2,3] \tabularnewline
& & & & 4: FFSL (quadr. or cubic reconstr.)& lsq\_high\_ord $\in$ [2,3] \tabularnewline
& & & & 20: miura (2nd order, lin. reconstr.) with subcycling& if cell\_type=3 \tabularnewline
& & & & 22: combination of miura and miura with subcycling& if cell\_type=3 \tabularnewline
& & & & 32: combination of miura3 and miura with subcycling& if cell\_type=3 \tabularnewline
& & & & 42: combination of FFSL and miura with subcycling& if cell\_type=3 \tabularnewline
& & & & 5: up3 (3rd or 4th order upstream)& if cell\_type=6 \tabularnewline

\hline 
ivadv\_tracer&
I(ntracer)&
3& & Tracer specific method to compute vertical advection:& lvadv\_tracer=TRUE \tabularnewline
& & & & 0: no vert. transport& \tabularnewline
& & & & 1: upwind (1st order)& \tabularnewline
& & & & 3: ppm\_cfl ($3^{\mathrm{rd}}$ order, handles $\mathrm{CFL}>1$)& \tabularnewline
& & & & 30: ppm (3rd order)& \tabularnewline

\hline 
lstrang &
L& .FALSE.& & splitting into fractional steps & \tabularnewline
& & & & - second order Strang splitting (.TRUE.) & \tabularnewline
& & & & - first order Godunov splitting (.FALSE.)& \tabularnewline

\hline 
ctracer\_list&
C& ''& & list of tracer names &
\tabularnewline

\hline 
itype\_hlimit&
I(ntracer)&
3& & Type of limiter for horizontal transport:& \tabularnewline
& & 4& & 0: no limiter& \tabularnewline
& & & & 3: monotonous flux limiter& ihadv\_tracer$\,\ne\,$'iup3[4]'\tabularnewline
& & & & 4: positive definite flux limiter& \tabularnewline

\hline 
itype\_vlimit&
I(ntracer)&
1& & Type of limiter for vertical transport:& \tabularnewline
& & & & 0: no limiter& \tabularnewline
& & & & 1: semi-monotone slope limiter& \tabularnewline
& & & & 2: monotonous slope limiter& \tabularnewline
& & & & 4: positive definite flux limiter& \tabularnewline

\hline 
niter\_fct&
I& 1& & number of iterations  of monotone flux correction procedure& ihadv\_tracer = 3, 32, 4 \\ itype\_hlimit = 3 
\tabularnewline

\hline 
beta\_fct&
R& 1.0& & factor for multiplicative spreading of range of permissible values (limiter) \\
\textcolor{red}{Tentative suggestion: beta\_fct=1.0015}& ihadv\_tracer = 3, 32, 4 \\ itype\_hlimit = 3 
\tabularnewline

\hline 
iord\_backtraj&
I& 1& & order of backward trajectory calculation:& \tabularnewline
& & & & 1: first order& \tabularnewline
& & & & 2: second order (iterative; currently 1 iteration hardcoded)& ihadv\_tracer='miura'
\tabularnewline

\hline 
igrad\_c\_miura&
I& 1& & Method for gradient reconstruction at cell center for 2nd order miura& \tabularnewline
& & & & 1: Least-squares (linear, non-consv)& ihadv\_tracer=2\tabularnewline
& & & & 2: Green-Gauss& 
\tabularnewline

\hline
ivcfl\_max&
I& 5& &
determines stability range of vertical PPM-scheme in terms of the maximum allowable CFL-number&
ivadv\_tracer=3
\tabularnewline

\hline
llsq\_svd&
L&
.FALSE.&
&
use QR decomposition (FALSE) or SV decomposition (TRUE) for least squares design matrix A&
\tabularnewline

\hline
lclip\_tracer&
L& .FALSE.& & Clipping of negative values&
\tabularnewline

\hline 
upstr\_beta\_adv&
R& 1.0& & parameter to select 3rd order (=1) or 4th order (=0) advection, or something inbetween (0..1)&
ihadv\_tracer=iup3
\tabularnewline
\hline
\end{longtab}

Defined and used in: src/namelists/mo\_advection\_nml.f90


%------------------------------------------------------------------------------
% nwp_phy_nml:
%------------------------------------------------------------------------------
\subsection{nwp\_phy\_nml}

The switches for the physics schemes and the time steps can be set for each model domain individually.
If only one value is specified, it is copied to all child domains, implying that the same set
of parameterizations and time steps is used in all domains. If the number of values given
in the namelist is larger than 1 but less than the number of model domains, then the settings
from the highest domain ID are used for the remaining model domains. If the time steps are not
an integer multiple of the advective time step (dtime*iadv\_rcf), then the time step of the 
respective physics parameterization is automatically rounded to the next higher integer multiple
of the advective time step.

\begin{longtab}

inwp\_gscp&
I (max\_dom)&
1&
&
cloud microphysics and precipitation &
run\_nml/iforcing = inwp
\tabularnewline
&
&
&
&
0: none
&
\tabularnewline
&
&
&
&
1: hydci (COSMO-EU microphysics)
&
\tabularnewline
&
&
&
&
9: Kessler scheme
&
\tabularnewline

\hline
$qi0$&
R&
0.0&
kg/kg &
cloud ice threshold for autoconversion &
inwp\_gscp=1 
\tabularnewline
&
&
&
&
&
\tabularnewline

\hline
$qc0$&
R&
0.0&
kg/kg &
cloud water threshold for autoconversion &
inwp\_gscp=1 
\tabularnewline
&
&
&
&
&
\tabularnewline


\hline 
inwp\_convection&
I&
1 (max\_dom)&
&
convection
&run\_nml/iforcing = inwp
\tabularnewline
&
&
&
&
0: none
&
\tabularnewline
&
&
&
&
1: Tiedtke/Bechtold convection
&
\tabularnewline

\hline 
inwp\_cldcover&
I (max\_dom)&
3&
&
cloud cover scheme for radiation
&run\_nml/iforcing = inwp
\tabularnewline
&
&
&
&
0: no clouds (only QV)
&
\tabularnewline
&
&
&
&
1: grid-scale clouds and QV
&
\tabularnewline
&
&
&
&
2: clouds from COSMO turbulence scheme
&
\tabularnewline
&
&
&
&
3: clouds from COSMO SGS cloud scheme
&
\tabularnewline

\hline 
inwp\_radiation&
I (max\_dom)&
1&
&
radiation
&run\_nml/iforcing = inwp
\tabularnewline
&
&
&
&
0: none
&
\tabularnewline
&
&
&
&
1: RRTM radiation
&
\tabularnewline
&
&
&
&
2: Ritter-Geleyn radiation
&
\tabularnewline

\hline 
inwp\_satad&
I&
1&
&
saturation adjustment
&run\_nml/iforcing = inwp
\tabularnewline
&
&
&
&
0: none
&
\tabularnewline
&
&
&
&
1: 
&
\tabularnewline

\hline 
inwp\_turb&
I (max\_dom)&
1&
&
vertical diffusion and transfer 
&run\_nml/iforcing = inwp
\tabularnewline
&
&
&
&
0: none
&
\tabularnewline
&
&
&
&
1: COSMO diffusion and transfer
&
\tabularnewline
&
&
&
&
2: GME turbulence scheme (to be implemented)
&
\tabularnewline
&
&
&
&
3: EDMF-DUALM (work in progress)
&
\tabularnewline
&
&
&
&
4: ECHAM diffusion (currently for water only)
&
\tabularnewline
&
&
&
&
5: Classical Smagorinsky diffusion 
&
\tabularnewline
\hline 
inwp\_sso&
I (max\_dom)&
1&
&
subgrid scale orographic drag
&run\_nml/iforcing = inwp
\tabularnewline
&
&
&
&
0: none
&
\tabularnewline
&
&
&
&
1: (COSMO) Lott and Miller scheme 
&
\tabularnewline

\hline 
inwp\_gwd&
I (max\_dom)&
1&
&
non-orographic gravity wave drag
&run\_nml/iforcing = inwp
\tabularnewline
&
&
&
&
0: none
&
\tabularnewline
&
&
&
&
1:Orr-Ern-Bechtold-scheme(IFS)  
&
\tabularnewline


\hline 
inwp\_surface&
I (max\_dom)&
1&
&
surface scheme
&run\_nml/iforcing = inwp
\tabularnewline
&
&
&
&
0: none
&
\tabularnewline
&
&
&
&
1: TERRA
&
\tabularnewline


\hline 
ustart\_raylfric&
R& 160.0& m/s& wind speed at which extra Rayleigh friction starts &
inwp\_gwd $>$ 0
\tabularnewline


\hline 
efdt\_min\_raylfric&
R& 10800.& s & minimum e-folding time of Rayleigh friction (effective for u $>$ ustart\_raylfric + 90 m/s) &
inwp\_gwd $>$ 0
\tabularnewline

\hline 
latm\_above\_top&
L (max\_dom)& .FALSE.&  & .TRUE.: take into account atmosphere above model top for radiation computation &
inwp\_radiation $>$ 0
\tabularnewline

\hline 
itype\_z0&
I & 1 &  & Type of roughness length data used for turbulence scheme: 1 = including contribution from
sub-scale orography, 2 = land-cover-related roughness only &
inwp\_turb $>$ 0
\tabularnewline

\hline 
 dt\_conv &
R&
600.&
seconds&
time interval of convection call
&run\_nml/iforcing = inwp
\tabularnewline
&
(max\_dom)&
&
&
currently each subdomain has
&
\tabularnewline
&
&
&
&
the same value
&
\tabularnewline

\hline 
 dt\_ccov &
R&
dt\_conv&
seconds&
time interval of cloud cover call
&run\_nml/iforcing = inwp
\tabularnewline
&
(max\_dom)&
&
&
currently each subdomain has
&
\tabularnewline
&
&
&
&
the same value
&
\tabularnewline

\hline 
 dt\_rad &
R&
1800.&
seconds&
time interval of radiation call
&run\_nml/iforcing = inwp
\tabularnewline
&
(max\_dom)&
&
&
currently each subdomain has
&
\tabularnewline
&
&
&
&
the same value
&
\tabularnewline

\hline 
 dt\_sso &
R&
1200.&
seconds&
time interval of sso call
&run\_nml/iforcing = inwp
\tabularnewline
&
(max\_dom)&
&
&
currently each subdomain has
&
\tabularnewline
&
&
&
&
the same value
&
\tabularnewline

\hline 
 dt\_gwd &
R&
1200.&
seconds&
time interval of gwd call
&run\_nml/iforcing = inwp
\tabularnewline
&
(max\_dom)&
&
&
currently each subdomain has
&
\tabularnewline
&
&
&
&
the same value
&
\tabularnewline

\hline
lrtm\_filename &
C(:)&
``rrtmg\_lw.nc''&
&
NetCDF file containing longwave absorption coefficients and other data
for RRTMG\_LW k-distribution model. &
\tabularnewline

\hline
cldopt\_filename &
C(:)&
``ECHAM6\_CldOptProps.nc''&
&
NetCDF file with RRTM Cloud Optical Properties for ECHAM6. &
\tabularnewline

\end{longtab}


Defined and used in: src/namelists/mo\_atm\_phy\_nwp\_nml.f90


%------------------------------------------------------------------------------
% radiation_nml:
%------------------------------------------------------------------------------
\subsection{radiation\_nml}

\begin{longtab}

\hline
ldiur &
L&
.TRUE.&
&
switch for solar irradiation: \\.TRUE.:diurnal cycle, \\.FALSE.:zonally averaged irradiation&
\tabularnewline

\hline
nmonth &
I&
0&
&
0: Earth circles on orbit\\1-12: Earth orbit position fixed for specified month&
\tabularnewline

\hline
lyr\_perp &
L&
.FALSE.&
&
.FALSE.: transient Earth orbit following VSOP87 \\ .TRUE.: Earth orbit of year yr\_perp of the VSOP87 orbit is perpertuated &
\tabularnewline

\hline
yr\_perp &
L&
-99999&
&
year used for lyr\_perp = .TRUE.&
\tabularnewline

\hline
isolrad &
I&
0&
&
Insolation scheme\\
0: Use insolation defined in code.\\
1: Use insolation from external file containing the spectrally resolved insolation averaged over a year (not yet implemented) &
\tabularnewline

\hline
izenith&
I&
3\par
4 (for iforcing = inwp) &
&
Choice of zenith angle formula for the radiative transfer computation.\par
0: Sun in zenith everywhere\par
1: Zenith angle depends only on latitude\par
2: Zenith angle depends only on latitude. Local time of day fixed at 07:14:15 for radiative transfer computation (sin(time of day) = 1/pi\par
3: Zenith angle changing with latitude and time of day\par
4: Zenith angle and irradiance changing with season, latitude, and time of day (iforcing=inwp only)
&
\tabularnewline

\hline
albedo\_type&
I&
1&
&
Type of surface albedo\\
1: based on soil type specific tabulated values (dry soil)\\
2: MODIS albedo
&
iforcing=inwp
\tabularnewline

\hline 
irad\_h2o\par 
irad\_co2\par 
irad\_ch4\par 
irad\_n2o\par 
irad\_o3\par 
irad\_o2\par 
irad\_cfc11\par 
irad\_cfc12\par 
irad\_aero\par 
&
I&
1\par
2\par
3\par
3\par
3\par
2\par
2\par
2\par
2\par
&
&
Switches for the concentration of radiative agents\par
0: 0.\par
1: prognostic variable\par
2: global constant\par
3: externally specified\par
irad\_aero = 5: Tanre aerosol climatology {\color{red}for run\_nml/iforcing = 3 (NWP) }\par
irad\_aero = 6: Tegen aerosol climatology {\color{red}for run\_nml/iforcing = 3 (NWP) .AND. itopo =1 } \par
irad\_o3 = 2: ozone climatology from MPI \par
irad\_o3 = 4: ozone clim for Aqua Planet Exp \par
irad\_o3 = 6: ozone climatology with T5 geographical distribution and Fourier series for seasonal cycle {\color{red}for run\_nml/iforcing = 3 (NWP)} \par
irad\_o3 = 7: GEMS ozone climatology (from IFS) {\color{red}for run\_nml/iforcing = 3 (NWP)}
&
Note: until further notice, please use \par
irad\_h2o = 1\par
irad\_co2 = 2\par
and 0 for all the other agents for run\_nml/iforcing = 2 (ECHAM).\par
\tabularnewline
\hline
vmr\_co2\par
vmr\_ch4\par
vmr\_n2o\par
vmr\_o2\par
vmr\_cfc11\par
vmr\_cfc12\par
&
R&
353.9e-6\par
1693.6e-9\par
309.5e-9\par
0.20946\par
252.8e-12\par
466.2e-12
&
&
Volume mixing ratio of the radiative agents&
\tabularnewline
\hline
\end{longtab}

Defined and used in: src/namelists/mo\_radiation\_nml.f90


%------------------------------------------------------------------------------
% nwp_lnd_nml:
%------------------------------------------------------------------------------
\subsection{lnd\_nml}

\begin{longtab}

nlev\_snow &
I&
2&
&
number of snow layers\\
for {\tt lmulti\_snow=.true.}&
lmulti\_snow=.true.
\tabularnewline
\hline
ntiles &
I&
1&
&
number of tiles&
\tabularnewline

\hline
lsnowtile &
L&
.FALSE.&
&
.TRUE.: consider snow-covered and snow-free tiles separately &
ntiles>1
\tabularnewline

\hline
frlnd\_thrhld &
R&
0.05&
&
fraction threshold for creating a land grid point &
ntiles>1
\tabularnewline

\hline
frlake\_thrhld &
R&
0.05&
&
fraction threshold for creating a lake grid point &
ntiles>1
\tabularnewline
\hline
frsea\_thrhld &
R&
0.05&
&
fraction threshold for creating a sea grid point &
ntiles>1
\tabularnewline

\hline
frlndtile\_thrhld &
R&
0.05&
&
fraction threshold for retaining the respective tile for a grid point&
ntiles>1
\tabularnewline

\hline
nztlev &
I&
2&
&
used time integration scheme&
\tabularnewline
\hline
lmulti\_snow &
L&
.TRUE.&
&
.TRUE. for use of multi-layer snow model&
\tabularnewline

\hline
max\_toplaydepth &
R &
0.25&
m &
maximum depth of uppermost snow layer & lmulti\_snow=.TRUE.
\tabularnewline

\hline 
idiag\_snowfrac &
I & 1 &  & Type of snow-fraction diagnosis: 1 = based on SWE only, 2--4 = more advanced experimental methods &
\tabularnewline

\hline 
itype\_lndtbl &
I & 1 &  & Table values used for associating surface parameters to land-cover classes: 1 = defaults from extpar, 
2 = IFS values for globcover classes (currently no effect in case of glc2000 data) &
\tabularnewline

\hline
lseaice &
L&
.TRUE.&
&
.TRUE. for use of sea-ice model&
\tabularnewline
\hline
llake &
L&
.FALSE.&
&
.TRUE. for use of lake model&
\tabularnewline
\hline
sstice\_mode &
I&
1&
&
1: SST and sea ice fraction are read from the analysis and kept constant. The 
sea ice fraction can be modified by the seaice model.\\
2: SST and sea ice fraction are updated daily, based on climatological monthly 
means\\
3: SST and sea ice fraction are updated daily, based on actual monthly means\\
4: SST and sea ice fraction are updated daily, based on actual daily means, 
not yet implemented&
iequations=3\\
iforcing=3
\tabularnewline
\hline
sst\_td\_filename &
C&
&
&
Filename of SST input files for time dependent SST. 
Default is "<path>SST\_<year>\_<month>\_<gridfile>". May contain the 
keyword <path> which will be substituted by model\_base\_dir&
sstice\_mode=2,3
\tabularnewline
\hline
ci\_td\_filename &
C&
&
&
Filename of sea ice fraction input files for time dependent sea ice fraction. 
Default is "<path>CI\_<year>\_<month>\_<gridfile>". May contain 
the keyword <path> which will be substituted by model\_base\_dir&
sstice\_mode=2,3
\tabularnewline
\end{longtab}

Defined and used in: src/namelists/mo\_lnd\_nwp\_nml.f90


%------------------------------ switches for ECHAM physics 

%------------------------------------------------------------------------------
% echam_phy_nml:
%------------------------------------------------------------------------------
\subsection{echam\_phy\_nml}

\begin{longtab}

lrad &
L& .TRUE.&&
Switch on radiation.& iforcing = 2
\tabularnewline

\hline
lvdiff &
L& .TRUE.&&
Switch on turbulent mixing (i.e. vertical diffusion).&
iforcing = 2
\tabularnewline

\hline
lconv &
L& .TRUE.&&
Switch on cumulus convection.& iforcing = 2
\tabularnewline

\hline
lcond &
L& .TRUE.&&
Switch on large scale condensation.& iforcing = 2
\tabularnewline

\hline
lcover &
L& .FALSE.&&
.TRUE. for prognostic cloud cover scheme, .FALSE. for diagnostic scheme.&
iforcing = 2\par
Note: lcover = .TRUE. runs, but has not been evaluated (yet) in ICON.
\tabularnewline

\hline
lgw\_hines &
L& .FALSE.&&
.TRUE. for atmospheric gravity wave drag by the Hines scheme&
iforcing = 2\par
\tabularnewline

\hline
lssodrag &
L& .FALSE.&&
.TRUE. for subgrid scale orographic drag& iforcing = 2\par {\color{red}Not implemeted yet}
\tabularnewline

\hline
llandsurf &
L& .FALSE.&&
.TRUE. for surface exchanges& iforcing = 2\par {\color{red}Not implemeted yet}
\tabularnewline

\hline
lice &
L& .FALSE.&&
.TRUE. for sea-ice temperature calculation& iforcing = 2\par {\color{red}Not implemeted yet}
\tabularnewline

\hline
lmeltpond &
L& .FALSE.&&
.TRUE. for calculation of meltponds& iforcing = 2\par {\color{red}Not implemeted yet}
\tabularnewline

\hline
lhd &
L& .FALSE.&&
.TRUE. for hydrologic discharge model& iforcing = 2\par
{\color{red}Not implemeted yet}
\tabularnewline

\hline
lmlo &
L& .FALSE.&&
.TRUE. for mixed layer ocean& iforcing = 2\par {\color{red}Not implemeted yet}
\tabularnewline

\hline
dt\_rad &
R&
3600.&
second&
time interval of full radiation computation&
run\_nml/iforcing = iecham
\tabularnewline

\hline
\end{longtab}

Defined and used in: src/namelists/mo\_echam\_phy\_nml.f90


%------------------------------------------------------------------------------
% echam_conv_nml:
%------------------------------------------------------------------------------
\subsection{echam\_conv\_nml}

\begin{longtab}

\hline
iconv&
I & 1 &&
Choice of cumulus convection scheme.\par
1: Nordeng scheme\par
2: Tiedtke scheme\par
3: hybrid scheme &
iforcing = 2 .AND. lconv = .TRUE.
\tabularnewline

\hline
ncvmicro&
I & 0 &&
Choice of convective microphysics scheme.&

iforcing = 2 .AND. lconv = .TRUE.
\tabularnewline

\hline
lmfpen &
L& .TRUE.&&
Switch on penetrative convection.&
iforcing = 2 .AND. lconv = .TRUE.
\tabularnewline

\hline
lmfmid &
L& .TRUE.&&
Switch on midlevel convection.&
iforcing = 2 .AND. lconv = .TRUE.
\tabularnewline

\hline
lmfdd &
L& .TRUE.&&
Switch on cumulus downdraft.&
iforcing = 2 .AND. lconv = .TRUE.
\tabularnewline

\hline
lmfdudv &
L& .TRUE.&&
Switch on cumulus friction.&
iforcing = 2 .AND. lconv = .TRUE.
\tabularnewline

\hline
cmftau &
R & 10800. &&
Characteristic convective adjustment time scale.&
iforcing = 2 .AND. lconv = .TRUE.
\tabularnewline

\hline
cmfctop &
R & 0.3 &&
Fractional convective mass flux (valid range [0,1])
across the top of cloud &
iforcing = 2 .AND. lconv = .TRUE.
\tabularnewline

\hline
cprcon &
R & 1.0e-4 &&
Coefficient for determining conversion from cloud water to rain.&
iforcing = 2 .AND. lconv = .TRUE.
\tabularnewline

\hline
cminbuoy &
R & 0.025 &&
Minimum excess buoyancy.&
iforcing = 2 .AND. lconv = .TRUE.
\tabularnewline

\hline
entrpen &
R & 1.0e-4 &&
Entrainment rate for penetrative convection.&
iforcing = 2 .AND. lconv = .TRUE.
\tabularnewline

\hline
dlev &
R & 3.e4 & Pa &
Critical thickness necessary for the onset of convective precipitation.&
iforcing = 2 .AND. lconv = .TRUE.
\tabularnewline

\end{longtab}

Defined and used in: src/namelists/mo\_echam\_conv\_nml.f90


%------------------------------------------------------------------------------
% vdiff_nml:
%------------------------------------------------------------------------------
\subsection{vdiff\_nml}

\begin{longtab}

\hline
lsfc\_mon\_flux &
L& .TRUE.&&
Switch on surface momentum flux.& lvdiff = .TRUE.
\tabularnewline

\hline
lsfc\_heat\_flux &
L                & .TRUE.           & &
Switch on surface sensible and latent heat flux.& lvdiff = .TRUE.
\tabularnewline

\end{longtab}

Defined and used in: src/namelists/mo\_vdiff\_nml.f90


%------------------------------------------------------------------------------
% turbdiff_nml:
%------------------------------------------------------------------------------
\subsection{turbdiff\_nml}

\begin{longtab}

\hline
itype\_tran &
I            & 2      &&
type of surface-atmosphere transfer& inwp\_turb = 1
\tabularnewline

\hline
imode\_tran &
I                &      1      & &
mode of surface-atmosphere transfer & inwp\_turb = 1
\tabularnewline

\hline
icldm\_tran &
I                &      0      & &
mode of cloud representation in transfer parametr & inwp\_turb = 1
\tabularnewline

\hline
imode\_turb &
I                &      3      & &
mode of turbulent diffusion parametrization & inwp\_turb = 1
\tabularnewline

\hline
icldm\_turb &
I                &      2      & &
mode of cloud representation in turbulence parametr & inwp\_turb = 1
\tabularnewline

\hline
itype\_sher &
I                &      1      & &
type of shear production for TKE & inwp\_turb = 1
\tabularnewline

\hline
ltkesso &
L                &     .FALSE.      & &
calculation SSO-wake turbulence production for TKE & inwp\_turb = 1
\tabularnewline

\hline
ltkecon &
L                &     .FALSE.      & &
consider convective buoyancy production for TKE & inwp\_turb = 1
\tabularnewline

\hline
lexpcor &
L                &     .FALSE.      & &
explicit corrections of the implicit calculated turbul. diff. & inwp\_turb = 1
\tabularnewline

\hline
ltmpcor &
L                &     .FALSE.      & &
consideration of thermal TKE-sources in the enthalpy budget & inwp\_turb = 1
\tabularnewline

\hline
lprfcor &
L                &     .FALSE.      & &
using the profile values of the lowest main level instead of the mean value of the lowest layer for surface flux calulations & inwp\_turb = 1
\tabularnewline

\hline
lnonloc &
L                &     .FALSE.      & &
nonlocal calculation of vertical gradients used for turbul. diff. & inwp\_turb = 1
\tabularnewline

\hline
lcpfluc &
L                &     .FALSE.      & &
consideration of fluctuations of the heat capacity of air & inwp\_turb = 1
\tabularnewline

\hline
limpltkediff &
L                &     .TRUE.      & &
consideration of fluctuations of the heat capacity of air & inwp\_turb = 1
\tabularnewline

\hline
itype\_wcld &
I                &     2      & &
type of water cloud diagnosis & inwp\_turb = 1
\tabularnewline

\hline
itype\_synd &
I                &     2      & &
type of diagnostics of synoptical near surface variables & inwp\_turb = 1
\tabularnewline

\hline
lconst\_z0 &
L                &     .FALSE.      & &
TRUE: horizontally homogeneous roughness lenght z0 & inwp\_turb = 1
\tabularnewline

\hline
const\_z0 &
R                &     0.001      & m &
value for horizontally homogeneous roughness lenght z0 & inwp\_turb = 1 \\ 
lconst\_z0=.TRUE.
\tabularnewline

\end{longtab}

Defined and used in: src/namelists/mo\_turbdiff\_nml.f90

%------------------------------------------------------------------------------
% les_nml:
%------------------------------------------------------------------------------
\subsection{les\_nml (parameters for LES turbulence scheme; valid for inwp\_turb=5)}

\begin{longtab}

\hline 
sst & R & 300 & K &
sea surface temperature for idealized LES simulations &
nh\_test\_name=CBL, RICO\\
isrfc\_type=5,4 
\tabularnewline

\hline 
shflx & R & -999 & Km/s &
Kinematic sensible heat flux at surface &
isrfc\_type = 2
\tabularnewline

\hline 
lhflx & R & -999 & m/s &
Kinematic latent heat flux at surface &
isrfc\_type = 2
\tabularnewline

\hline 
isrfc\_type & I & 1 &  &
surface type \\
1 = TERRA land physics \\
2 = fixed surface fluxes \\ 
3 = fixed buoyancy fluxes \\
4 = RICO test case \\
5 = fixed SST &
\tabularnewline

\hline 
ufric & R & -999 & m/s &
friction velocity for idealized LES simulations &
\tabularnewline

\hline 
is\_dry\_cbl & L & .FALSE. &  &
switch for dry convective boundary layer simulations &
\tabularnewline

\hline 
karman\_constant & R & 0.4 &  &
von Karman constant &
\tabularnewline

\hline 
smag\_constant & R & 0.12 &  &
Smagorinsky constant &
\tabularnewline

\hline 
turb\_prandtl & R & 0.333333 &  &
turbulent Prandtl number &
\tabularnewline

\hline 
bflux & R & -999 &  m$^2$/s$^3$ &
buoyancy flux for idealized LES simulations (Stevens 2007) &
isrfc\_type=3
\tabularnewline

\hline 
tran\_coeff & R & -999 &  m/s &
transfer coefficient near surface for idealized LES simulation (Stevens 2007)&
isrfc\_type=3
\tabularnewline

\end{longtab}

Defined and used in: src/namelists/mo\_les\_nml.f90

%------------------------------------------------------------------------------
% ls_forcing_nml:
%------------------------------------------------------------------------------
\subsection{ls\_forcing\_nml (parameters for large-scale forcing; valid for torus geometry)}

\begin{longtab}

\hline 
is\_ls\_forcing& L & .FALSE. &  &
switch for enabling large-scale (LS) forcing on torus grid &
is\_plane\_torus=.TRUE.
\tabularnewline

\hline 
is\_subsidence\_moment & L & .FALSE. &  &
switch for enabling LS vertical advection due to subsidence for momentum equations&
is\_plane\_torus=.TRUE.
\tabularnewline

\hline 
is\_subsidence\_heat & L & .FALSE. &  &
switch for enabling LS vertical advection due to subsidence for thermal equations &
is\_plane\_torus=.TRUE.
\tabularnewline


\hline 
is\_advection & L & .FALSE. &  &
switch for enabling LS horizontal advection (currently only for thermal equations)&
is\_plane\_torus=.TRUE.
\tabularnewline

\hline 
is\_geowind & L & .FALSE. &  &
switch for enabling geostrophic wind &
is\_plane\_torus=.TRUE.
\tabularnewline

\hline 
is\_rad\_forcing & L & .FALSE. &  &
switch for enabling radiative forcing &
is\_plane\_torus=.TRUE. \\
inwp\_rad=.FALSE.
\tabularnewline

\hline 
is\_geowind & L & .FALSE. &  &
switch for enabling geostrophic wind &
is\_plane\_torus=.TRUE.
\tabularnewline

\hline 
is\_theta & L & .FALSE. &  &
switch to indicate that the prescribed radiative forcing is for potential temperature &
is\_plane\_torus=.TRUE. \\
is\_rad\_forcing=.TRUE.
\tabularnewline

\end{longtab}

Defined and used in: src/namelists/mo\_ls\_forcing\_nml.f90


%------------------------------------------------------------------------------
% gw_hines_nml:
%------------------------------------------------------------------------------
\subsection{gw\_hines\_nml (Scope: lgw\_hines = .TRUE. in echam\_phy\_nml)}

\begin{longtab}

\hline
lheatcal    &
L           &
.FALSE.     &&
.TRUE.: compute drag, heating rate and diffusion coefficient from the dissipation of gravity waves&
\tabularnewline
&&&&
.FALSE.: compute drag only &
\tabularnewline

\hline
emiss\_lev  &
I           &
10          &&
Index of model level, counted from the surface, from which the gravity wave spectra are emitted &
\tabularnewline

\hline
rmscon      &
R           &
1.0         &
m/s         &
Root mean square gravity wave wind at the emission level &
\tabularnewline

\hline
kstar       &
R           &
5.0e-5      &
1/m         &
Typical gravity wave horizontal wavenumber &
\tabularnewline

\hline
m\_min      &
R           &
0.0         &
1/m         &
Minimum bound in  vertical wavenumber &
\tabularnewline

\hline
lrmscon\_lat &
L            &
.FALSE.      &
             &
.TRUE.:  use latitude dependent rms wind &
\tabularnewline
&&&& - |latitude| >= lat\_rmscon: use rmscon &
\tabularnewline
&&&& - |latitude| <= lat\_rmscon\_eq: use rmscon\_eq &
\tabularnewline
&&&& - lat\_rmscon\_eq < |latitude| < lat\_rmscon: use linear interpolation between rmscon\_eq and rmscon &
\tabularnewline
&&&& .FALSE.: use globally constant rms wind rmscon &
\tabularnewline

\hline
lat\_rmscon\_eq &
R               &
5.0             &
deg N           &
rmscon\_eq is used equatorward of this latitude &
lrmscon\_lat = .TRUE.
\tabularnewline

\hline
lat\_rmscon     &
R               &
10.0            &
deg N           &
rmscon is used polward of this latitude &
lrmscon\_lat = .TRUE.
\tabularnewline

\hline
rmscon\_eq      &
R               &
1.2             &
m/s             &
is used equatorward of latitude lat\_rmscon\_eq &
lrmscon\_lat = .TRUE.
\tabularnewline

\end{longtab}

Defined and used in: src/namelists/mo\_gw\_hines\_nml.f90

\subsection{ocean\_physics\_nml}

\begin{longtab}

\hline
i\_sea\_ice    & I & 1  && 0: No sea ice, 1: Include sea ice& \tabularnewline 
&&&& .FALSE.: compute drag only & \tabularnewline
\hline 
richardson\_factor\_tracer  & I & 0.5e-5 & m/s&  & \tabularnewline
\hline
richardson\_factor\_veloc  & I & 0.5e-5 & m/s&  & \tabularnewline
\hline
l\_constant\_mixing  & L & .FALSE. & &  & \tabularnewline

\end{longtab}

\subsection{sea\_ice\_nml}

\begin{longtab}

  \hline
  i\_ice\_therm &
  I             &
  2             &&
  Switch for thermodynamic model: \\
  1: Zero-layer model \\
  2: Two layer Winton (2000) model \\
  3: Zero-layer model with analytical forcing (for diagnostics) \\
  4: Zero-layer model for atmosphere-only runs (for diagnostics) &
  In an ocean run i\_sea\_ice must be >=1. In an atmospheric run the ice surface type must be defined.
  \tabularnewline

  i\_ice\_dyn   &
  I             &
  0             &&
  Switch for sea-ice dynamics: \\
  0: No dynamics \\
  1: FEM dynamics (from AWI) &
  \tabularnewline

  i\_ice\_albedo        &
  I                     &
  1                     &&
  Switch for albedo model. Only one is implemented so far. &
  \tabularnewline

  i\_Qio\_type          &
  I                     &
  2                     &&
  Switch for ice-ocean heat-flux calculation method: \\
  1: Proportional to ocean cell thickness (like MPI-OM) \\
  2: Proportional to speed difference between ice and ocean &
  Defaults to 1 when i\_ice\_dyn=0 and 2 otherwise.
  \tabularnewline

  kice  &
  I     &
  1     &&
  Number of ice classes (must be one for now) &
  \tabularnewline

  hnull &
  R     &
  0.5   &
  m     &
  Hibler's $h_0$ parameter for new-ice growth. &
  \tabularnewline

  hmin  &
  R     &
  0.05  &
  m     &
  Minimum sea-ice thickness allowed. &
  \tabularnewline

  ramp\_wind    &
  R             &
  10            &
  days          &
  Number of days it takes the wind to reach correct strength. Only used at the start of an OMIP/NCEP simulation (not after restart). &
  \tabularnewline

\end{longtab}

\section{Namelist parameters for testcases (NAMELIST\_ICON)}

The ICON model code includes several experiments, so-called test cases,
for the shallow water model as well as the 3-dimensional atmosphere.
Depending on the specified experiment, initial conditions and boundary
conditions are computed internally.

%-------------------------------------------------------------------
% ha_testcase_nml:
%-------------------------------------------------------------------
\subsection{ha\_testcase\_nml (Scope: ltestcase=.TRUE. and iequations=[0,1,2] in run\_nml)}
\begin{longtab}

\hline 
ctest\_name&
C& 'JWw'& &
Name of test case: &
\tabularnewline
&&&&&
\tabularnewline
& & & &
'SW\_GW': gravity wave&
lshallow\_water=.TRUE.
\tabularnewline
& & & &
'USBR': unsteady solid body rotation &
lshallow\_water=.TRUE.
\tabularnewline
& & & &
'Will\_2': Williamson test 2&
lshallow\_water=.TRUE.
\tabularnewline
& & & &
'Will\_3': Williamson test 3&
lshallow\_water=.TRUE.
\tabularnewline
& & & &
'Will\_5': Williamson test 5&
lshallow\_water=.TRUE.
\tabularnewline
& & & &
'Will\_6': Williamson test 6&
lshallow\_water=.TRUE.
\tabularnewline
& & & &
'GW': gravity wave (nlev=20 only!) &
lshallow\_water=.FALSE.
\tabularnewline
& & & &
'LDF': local diabatic forcing test without physics&
lshallow\_water=.FALSE.\\and iforcing=4
\tabularnewline
& & & &
'LDF-Moist': local diabatic forcing test with physics initalised with zonal wind field &
lshallow\_water=.FALSE.,\\and iforcing=5
\tabularnewline
& & & &
'HS': Held-Suarez test &
lshallow\_water=.FALSE.
\tabularnewline
& & & &
'JWs': Jablonowski-Will. steady state&
lshallow\_water=.FALSE.
\tabularnewline
& & & &
'JWw': Jablonowski-Will. wave test&
lshallow\_water=.FALSE.
\tabularnewline
& & & &
'JWw-Moist': Jablonowski-Will. wave test including moisture&
lshallow\_water=.FALSE.
\tabularnewline
& & & &
'APE': aqua planet experiment&
lshallow\_water=.FALSE.
\tabularnewline
& & & &
'MRW': mountain induced Rossby wave&
lshallow\_water=.FALSE.
\tabularnewline
& & & &
'MRW2': modified mountain induced Rossby wave&
lshallow\_water=.FALSE.
\tabularnewline
& & & &
'PA': pure advection&
lshallow\_water=.FALSE.
\tabularnewline
& & & &
'SV': stationary vortex&
lshallow\_water=.FALSE., 
ntracer = 2
\tabularnewline
& & & &
'DF1': deformational flow test 1&
\tabularnewline
& & & &
'DF2': deformational flow test 2&
\tabularnewline
& & & &
'DF3': deformational flow test 3&
\tabularnewline
& & & &
'DF4': deformational flow test 4&
\tabularnewline
& & & &
'RH': Rossby-Haurwitz wave test&
lshallow\_water=.FALSE.
\tabularnewline

\hline 
rotate\_axis\_deg&
R& 0.0& deg&
Earth's rotation axis pitch angle&
ctest\_name= 'Will\_2', 'Will\_3', 'JWs', 'JWw', 'PA', 'DF1234'
\tabularnewline

\hline 
gw\_brunt\_vais&
R& 0.01& 1/s&
Brunt Vaisala frequency&
ctest\_name= 'GW'
\tabularnewline

\hline 
gw\_u0&
R& 0.0& m/s&
zonal wind parameter&
ctest\_name= 'GW'
\tabularnewline

\hline 
gw\_lon\_deg&
R& 180.0& deg&
longitude of initial perturbation&
ctest\_name= 'GW'
\tabularnewline

\hline 
gw\_lat\_deg&
R& 0.0& deg&
latitude of initial perturbation&
ctest\_name= 'GW'
\tabularnewline

\hline 
jw\_uptb&
R& 1.0& m/s (?)&
amplitude of the wave pertubation&
ctest\_name= 'JWw'
\tabularnewline

\hline 
mountctr\_lon\_deg&
R& 90.0& deg&
longitude of mountain peak&
ctest\_name= 'MRW(2)'
\tabularnewline

\hline 
mountctr\_lat\_deg&
R& 30.0& deg&
latitude of mountain peak&
ctest\_name= 'MRW(2)'
\tabularnewline

\hline 
mountctr\_height&
R& 2000.0& m&
mountain height&
ctest\_name= 'MRW(2)'
\tabularnewline

\hline 
mountctr\_half\_width&
R& 1500000.0& m&
mountain half width&
ctest\_name= 'MRW(2)'
\tabularnewline

\hline 
mount\_u0&
R& 20.0& m/s&
wind speed for MRW cases&
ctest\_name= 'MRW(2)'
\tabularnewline

\hline 
rh\_wavenum&
I& 4& &
wave number&
ctest\_name= 'RH'
\tabularnewline

\hline 
rh\_init\_shift\_deg&
R& 0.0& deg&
pattern shift&
ctest\_name= 'RH'
\tabularnewline

\hline 
ihs\_init\_type&
I& 1& &
Choice of initial condition for the Held-Suarez test. 1: the zonal
state defined in the JWs test case; other integers: isothermal state
(T=300 K, ps=1000 hPa, u=v=0.)&
ctest\_name= 'HS'
\tabularnewline

\hline 
lhs\_vn\_ptb&
L& .TRUE.& &
Add random noise to the initial wind field in the Held-Suarez test. &
ctest\_name= 'HS'
\tabularnewline

\hline 
hs\_vn\_ptb\_scale&
R& 1.& m/s &
Magnitude of the random noise added to the initial wind field in the
Held-Suarez test. &
ctest\_name= 'HS'
\tabularnewline

\hline 
lrh\_linear\_pres&
L& .FALSE.& &
Initialize the relative humidity using a linear function of pressure. &
ctest\_name= 'JWw-Moist','APE', 'LDF-Moist'
\tabularnewline

\hline 
rh\_at\_1000hpa&
R& 0.75& &
relative humidity \[0,1\] at 1000 hPa &
ctest\_name= 'JWw-Moist','APE', 'LDF-Moist'
\tabularnewline

\hline 
linit\_tracer\_fv&
L& .TRUE.& &
Finite volume initialization for tracer fields &
ctest\_name='PA'
\tabularnewline

\hline
ape\_sst\_case&
C& 'sst1'& &
SST distribution selection\\
'sst1': Control experiment\\
'sst2': Peaked experiment\\
'sst3': Flat experiment\\
'sst4': Control-5N experiment\\
'sst\_qobs': Qobs SST distribution exp\\
'sst\_ice': Control SST distribution with -1.8 C above 64 N/S.
&ctest\_name='APE'
\tabularnewline

\hline 
ildf\_init\_type&
I& 0& &
Choice of initial condition for the Local diabatic forcing test. 1: the zonal
state defined in the JWs test case; other: isothermal state
(T=300 K, ps=1000 hPa, u=v=0.)&
ctest\_name= 'LDF'
\tabularnewline

\hline
ldf\_symm&
L& .TRUE.& &
Shape of local diabatic forcing:\\
.TRUE.: local diabatic forcing symmetric about the equator (at 0 N)\\
.FALSE.: local diabatic forcing asym. about the equator (at 30 N)
& ctest\_name= 'LDF','LDF-Moist'
\tabularnewline

\end{longtab}

Defined and used in: src/testcases/mo\_ha\_testcases.f90



%-------------------------------------------------------------------------------
% nh_test_nml:
%-------------------------------------------------------------------------------
\subsection{nh\_testcase\_nml (Scope: ltestcase=.TRUE. and iequations=3 in run\_nml)}

\begin{longtab}

\hline 
nh\_test\_name&
C& 'jabw'& &
testcase selection&
\tabularnewline
&&&& '\textbf{zero}': no orography&
\tabularnewline
&&&& '\textbf{bell}': bell shaped mountain at 0E,0N&
\tabularnewline
&&&& '\textbf{schaer}': hilly mountain at 0E,0N&
\tabularnewline
&&&& '\textbf{jabw}': Initializes the full Jablonowski Williamson test case.&
\tabularnewline
&&&&'\textbf{jabw\_s}': Initializes the Jablonowski Williamson steady state test case.&
\tabularnewline
&&&&'\textbf{jabw\_m}': Initializes the Jablonowski Williamson test case with a mountain instead of the wind perturbation (specify mount\_height).&
\tabularnewline
&&&&'\textbf{mrw\_nh}': Initializes the full Mountain-induced Rossby wave test case.&
\tabularnewline
&&&&'\textbf{mrw2\_nh}': Initializes the modified mountain-induced Rossby wave test case.&
\tabularnewline
&&&&'\textbf{mwbr\_const}': Initializes the mountain wave with two layers test case. 
The lower layer is isothermal and the upper layer has constant brunt 
vaisala frequency. The interface has constant pressure.&
\tabularnewline
&&&&'\textbf{PA}': Initializes the pure advection test case.&
\tabularnewline
&&&&'\textbf{HS\_nh}': Initializes the Held-Suarez test case. At the moment 
 with an isothermal atmosphere at rest (T=300K, ps=1000hPa, 
u=v=0, topography=0.0).&
\tabularnewline
&&&&'\textbf{HS\_jw}': Initializes the Held-Suarez test case
with Jablonowski Williamson initial conditions and zero topography.&
\tabularnewline
&&&&'\textbf{APE\_nh}': Initializes the APE experiments. With the 
jabw test case, including moisture.&
\tabularnewline
&&&&'\textbf{wk82}': Initializes the Weisman Klemp test case&
l\_limited\_area =.TRUE.
\tabularnewline
&&&&'\textbf{g\_lim\_area}': Initializes a series of general limited area test cases:
 itype\_atmos\_ana determines the atmospheric profile, itype\_anaprof\_uv 
determines the wind profile and itype\_topo\_ana determines the topography&
\tabularnewline
&&&&'\textbf{dcmip\_rest\_200}': atmosphere at rest test (Schaer-type mountain)&
lcoriolis = .FALSE.
\tabularnewline
&&&&'\textbf{dcmip\_mw\_2x}': nonhydrostatic mountain waves triggered by Schaer-type mountain&
lcoriolis = .FALSE.
\tabularnewline
&&&&'\textbf{dcmip\_gw\_31}': nonhydrostatic gravity waves triggered by a localized perturbation (nonlinear)&
\tabularnewline
&&&&'\textbf{dcmip\_gw\_32}': nonhydrostatic gravity waves triggered by a localized perturbation (linear)&
l\_limited\_area =.TRUE. \\
and lcoriolis = .FALSE.
\tabularnewline
&&&&'\textbf{dcmip\_tc\_51}': tropical cyclone test case with 'simple physics' parameterizations (\textbf{not yet implemented})&
lcoriolis = .TRUE.
\tabularnewline
&&&&'\textbf{dcmip\_tc\_52}': tropical cyclone test case with with full physics in Aqua-planet mode&
lcoriolis = .TRUE.
\tabularnewline
&&&&'\textbf{CBL}': convective boundary layer simulations for LES package on torus (doubly periodic) grid&
is\_plane\_torus= .TRUE.
\tabularnewline
\hline
jw\_up&
R& 1.0& m/s&
amplitude of the u-perturbation in jabw test case&
nh\_test\_name='jabw'
\tabularnewline

\hline
u0\_mrw&
R& 20.0& m/s&
wind speed for mrw(2) and mwbr\_const cases&
nh\_test\_name= 'mrw(2)\_nh' and 'mwbr\_const'
\tabularnewline

\hline
mount\_height\_mrw&
R& 2000.0& m&
maximum mount height in mrw(2) and mwbr\_const&
nh\_test\_name= 'mrw(2)\_nh' and 'mwbr\_const'
\tabularnewline

\hline
mount\_half\_width&
R& 1500000.0& m&
half width of mountain in mrw(2), mwbr\_const and bell &
nh\_test\_name= 'mrw(2)\_nh', 'mwbr\_const' and 'bell'
\tabularnewline

\hline
mount\_lonctr\_mrw\_deg&
R& 90.& degrees&
lon of mountain center in mrw(2) and mwbr\_const&
nh\_test\_name= 'mrw(2)\_nh' and 'mwbr\_const'
\tabularnewline

\hline
mount\_latctr\_mrw\_deg&
R& 30.& degrees&
lat of mountain center in mrw(2) and mwbr\_const&
nh\_test\_name= 'mrw(2)\_nh' and 'mwbr\_const'
\tabularnewline


\hline
temp\_i\_mwbr\_const&
R& 288.0& K&
temp at isothermal lower layer for mwbr\_const case&
nh\_test\_name= 'mwbr\_const'
\tabularnewline

\hline
p\_int\_mwbr\_const&
R& 70000.& Pa&
pres at the interface of the two layers for mwbr\_const case&
nh\_test\_name= 'mwbr\_const'
\tabularnewline

\hline
bruntvais\_u\_mwbr\_const&
R& 0.025& 1/s&
constant brunt vaissala frequency at upper layer for mwbr\_const case&
nh\_test\_name= 'mwbr\_const'
\tabularnewline

\hline
mount\_height&
R& 100.0& m&
peak height of mountain&
nh\_test\_name=  'bell'
\tabularnewline

\hline
layer\_thickness&
R& -999.0& m&
thickness of vertical layers&
If layer\_thickness < 0, the vertical level distribution is read in from externally given HYB\_PARAMS\_XX.
\tabularnewline

\hline
n\_flat\_level&
I& 2& &
level number for which the layer is still flat and not terrain-following&
layer\_thickness > 0
\tabularnewline

\hline
nh\_u0&
R& 0.0& m/s&
initial constant zonal wind speed &
nh\_test\_name = 'bell'
\tabularnewline

\hline
nh\_t0&
R& 300.0& K&
initial temperature at lowest level &
nh\_test\_name = 'bell'
\tabularnewline

\hline
nh\_brunt\_vais&
R& 0.01& 1/s&
initial Brunt-Vaisala frequency &
nh\_test\_name = 'bell'
\tabularnewline

\hline
torus\_domain\_length&
R& 100000.0& m&
length of slice domain &
nh\_test\_name = 'bell', lplane=.TRUE.
\tabularnewline

\hline
rotate\_axis\_deg&
R& 0.0& deg&
Earth's rotation axis pitch angle&
nh\_test\_name= 'PA'
\tabularnewline

\hline
lhs\_nh\_vn\_ptb&
L& .TRUE.& &
Add random noise to the initial wind field in the Held-Suarez test. &
nh\_test\_name= 'HS\_nh'
\tabularnewline

\hline 
lhs\_fric\_heat&
L& .FALSE.& &
add frictional heating from Rayleigh friction in the Held-Suarez test.&
nh\_test\_name= 'HS\_nh'
\tabularnewline

\hline 
hs\_nh\_vn\_ptb\_scale&
R& 1.& m/s &
Magnitude of the random noise added to the initial wind field in the
Held-Suarez test. &
nh\_test\_name= 'HS\_nh'
\tabularnewline

\hline
rh\_at\_1000hpa&
R& 0.7& 1 &
relative humidity at 1000 hPa &
nh\_test\_name= 'jabw', nh\_test\_name= 'mrw'
\tabularnewline

\hline
qv\_max&
R& 20.e-3& kg/kg &
specific humidity in the tropics &
nh\_test\_name= 'jabw', nh\_test\_name= 'mrw'
\tabularnewline

\hline
ape\_sst\_case&
C& 'sst1'& &
SST distribution selection\\
'sst1': Control experiment\\
'sst2': Peaked experiment\\
'sst3': Flat experiment\\
'sst4': Control-5N experiment\\
'sst\_qobs': Qobs SST distribution exp.&
 nh\_test\_name='APE\_nh'
\tabularnewline

\hline 
linit\_tracer\_fv&
L& .TRUE.& &
Finite volume initialization for tracer fields &
pure advection tests, only
\tabularnewline

\hline 
lcoupled\_rho&
L& .FALSE.& &
Integrate density equation 'offline'&
pure advection tests, only
\tabularnewline

\hline 
qv\_max\_wk&
R& 0.014 & Kg/kg &
maximum specific humidity near \\
the surface, range  0.012 - 0.016\\
used to vary the buoyancy&
nh\_test\_name='wk82'
\tabularnewline

\hline 
u\_infty\_wk&
R& 20. & m/s &
zonal wind at infinity height\\
range 0. - 45.               \\
used to vary the wind shear&
nh\_test\_name='wk82'
\tabularnewline

\hline 
bub\_amp&
R& 2.& K&
maximum amplitud of the thermal perturbation&
nh\_test\_name='wk82'
\tabularnewline

\hline 
bubctr\_lat&
R& 0.& deg&
latitude of the center of the thermal perturbation&
nh\_test\_name='wk82'
\tabularnewline

\hline 
bubctr\_lon&
R& 90.& deg&
longitude of the center of the thermal perturbation&
nh\_test\_name='wk82'
\tabularnewline

\hline 
bubctr\_z&
R& 1400.& m&
height of the center of the thermal perturbation&
nh\_test\_name='wk82'
\tabularnewline

\hline 
bub\_hor\_width&
R& 10000.& m&
horizontal radius of the thermal perturbation&
nh\_test\_name='wk82'
\tabularnewline

\hline 
bub\_ver\_width&
R& 1400.& m&
vertical radius of the thermal perturbation&
nh\_test\_name='wk82'
\tabularnewline

\hline 
itype\_atmo\_ana&
I& 1 & &
kind of atmospheric profile:\\
1 piecewise N constant layers\\
2 piecewise polytropic layers&
nh\_test\_name=\\'g\_lim\_area'
\tabularnewline
\hline 
itype\_anaprof\_uv&
I& 1 & &
kind of wind profile:\\
1 piecewise linear wind layers\\
2 constant zonal wind\\
3 constant meridional wind&
nh\_test\_name=\\'g\_lim\_area'
\tabularnewline
\hline 
itype\_topo\_ana&
I& 1 & &
kind of orography:\\
1 schaer test case mountain\\
2 gaussian\_2d mountain\\
3 gaussian\_3d mountain\\
any other no orography&
nh\_test\_name=\\'g\_lim\_area'
\tabularnewline
\hline 
nlayers\_nconst&
I& 1 & &
Number of the desired layers with a constant Brunt-Vaisala-frequency&
nh\_test\_name=\\'g\_lim\_area' and 
itype\_atmo\_ana=1
\tabularnewline
\hline 
p\_base\_nconst&
R& 100000. & Pa &
pressure at the base of the first N constant layer&
nh\_test\_name=\\'g\_lim\_area' and 
itype\_atmo\_ana=1
\tabularnewline
\hline 
theta0\_base\_nconst&
R& 288. & K &
potential temperature at the base of the first N constant layer&
nh\_test\_name=\\'g\_lim\_area' and 
itype\_atmo\_ana=1
\tabularnewline
\hline 
h\_nconst&
R(nlayers\_nconst)& 0., 1500., 12000.  & m &
height of the base of each of the N constant layers&
nh\_test\_name=\\'g\_lim\_area' and 
itype\_atmo\_ana=1
\tabularnewline
\hline 
N\_nconst&
R(nlayers\_nconst)& 0.01  & 1/s &
Brunt-Vaisala-frequency at each of the N constant layers&
nh\_test\_name=\\'g\_lim\_area' and 
itype\_atmo\_ana=1
\tabularnewline
\hline 
rh\_nconst&
R(nlayers\_nconst)& 0.5  & \% &
relative humidity at the base of each N constant layers&
nh\_test\_name=\\'g\_lim\_area' and 
itype\_atmo\_ana=1
\tabularnewline
\hline 
rhgr\_nconst&
R(nlayers\_nconst)& 0.  & \% &
relative humidity gradient at each of the N constant layers&
nh\_test\_name=\\'g\_lim\_area' and 
itype\_atmo\_ana=1
\tabularnewline
\hline 
nlayers\_poly&
I& 2 & &
Number of the desired layers with constant gradient temperature&
nh\_test\_name=\\'g\_lim\_area' and 
itype\_atmo\_ana=2
\tabularnewline
\hline 
p\_base\_poly&
R& 100000. & Pa &
pressure at the base of the first polytropic layer&
nh\_test\_name=\\'g\_lim\_area' and 
itype\_atmo\_ana=2
\tabularnewline
\hline 
h\_poly&
R(nlayers\_poly)& 0., 12000.  & m &
height of the base of each of the polytropic layers&
nh\_test\_name=\\'g\_lim\_area' and 
itype\_atmo\_ana=2
\tabularnewline
\hline 
t\_poly&
R(nlayers\_poly)& 288., 213.  & K &
temperature at the base of each of the polytropic layers&
nh\_test\_name=\\'g\_lim\_area' and 
itype\_atmo\_ana=2
\tabularnewline
\hline 
rh\_poly&
R(nlayers\_poly)& 0.8, 0.2  & \% &
relative humidity at the base of each of the polytropic layers&
nh\_test\_name=\\'g\_lim\_area' and 
itype\_atmo\_ana=2
\tabularnewline
\hline 
rhgr\_poly&
R(nlayers\_poly)& 5.e-5, 0. & \% &
relative humidity gradient at each of the polytropic layers&
nh\_test\_name=\\'g\_lim\_area' and 
itype\_atmo\_ana=2
\tabularnewline
\hline 
nlayers\_linwind&
I& 2 & &
Number of the desired layers with constant U gradient &
nh\_test\_name=\\'g\_lim\_area' and 
itype\_anaprof\_uv=1
\tabularnewline
\hline 
h\_linwind&
R(nlayers\_linwind)& 0., 2500.  & m &
height of the base of each of the linear wind layers&
nh\_test\_name=\\'g\_lim\_area' and 
itype\_anaprof\_uv=1
\tabularnewline
\hline 
u\_linwind&
R(nlayers\_linwind)& 5,  10.  & m/s &
zonal wind at the base of each of the linear wind layers&
nh\_test\_name=\\'g\_lim\_area' and
itype\_anaprof\_uv=1
\tabularnewline
\hline 
ugr\_linwind&
R(nlayers\_linwind)& 0., 0. & 1/s &
zonal wind gradient at each of the linear wind layers&
nh\_test\_name=\\'g\_lim\_area' and 
itype\_anaprof\_uv=1
\tabularnewline
\hline 
vel\_const&
R& 20. & m/s &
constant zonal/meridional wind (itype\_anaprof\_uv=2,3)&
nh\_test\_name=\\'g\_lim\_area' and 
itype\_anaprof\_uv=2,3
\tabularnewline
\hline 
mount\_lonc\_deg&
R& 90. & deg &
longitud of the center of the  mountain&
nh\_test\_name=\\'g\_lim\_area' 
\tabularnewline
\hline 
mount\_latc\_deg&
R& 0. & deg &
latitud of the center of the  mountain&
nh\_test\_name=\\'g\_lim\_area' 
\tabularnewline
\hline 
schaer\_h0&
R& 250. & m &
h0 parameter for the schaer mountain&
nh\_test\_name=\\'g\_lim\_area' and 
itype\_topo\_ana=1
\tabularnewline
\hline 
schaer\_a&
R& 5000. & m &
-a- parameter for the schaer mountain, \\
also half width in the north and south side of the 
finite ridge to round the sharp edges&
nh\_test\_name=\\'g\_lim\_area' and 
itype\_topo\_ana=1,2
\tabularnewline
\hline 
schaer\_lambda&
R& 4000. & m &
lambda parameter for the schaer mountain&
nh\_test\_name=\\'g\_lim\_area' and 
itype\_topo\_ana=1
\tabularnewline
\hline 
lshear\_dcmip&
L& FALSE &  &
run dcmip\_mw\_2x with/without vertical wind shear\\
FALSE: dcmip\_mw\_21: non-sheared \\
TRUE : dcmip\_mw\_22: sheared &
nh\_test\_name=\\'dcmip\_mw\_2x'
\tabularnewline
\hline 
halfwidth\_2d&
R& 10000. & m &
half lenght of the finite ridge in the north-south 
direction&
nh\_test\_name=\\'g\_lim\_area' and 
itype\_topo\_ana=1,2
\tabularnewline
\hline 
m\_height&
R& 1000. & m &
height of the mountain&
nh\_test\_name=\\'g\_lim\_area' and 
itype\_topo\_ana=2,3
\tabularnewline
\hline 
m\_width\_x&
R& 5000. & m &
half width of the gaussian mountain in the east-west direction \\
half width in the north-south direction in the rounding of the 
finite ridge (gaussian\_2d)&
nh\_test\_name=\\'g\_lim\_area' and 
itype\_topo\_ana=2,3
\tabularnewline
\hline 
m\_width\_y&
R& 5000. & m &
half width of the gaussian mountain in the north-south direction&
nh\_test\_name=\\'g\_lim\_area' and 
itype\_topo\_ana=2,3
\tabularnewline
\hline 
gw\_u0&
R& 0. & m/s &
maximum amplitude of the zonal wind&
nh\_test\_name=\\'dcmip\_gw\_3X'
\tabularnewline
\hline 
gw\_clat&
R& 90. & deg &
Lat of perturbation center&
nh\_test\_name=\\'dcmip\_gw\_3X'
\tabularnewline
\hline 
gw\_delta\_temp&
R& 0.01 & K &
maximum temperature perturbation&
nh\_test\_name=\\'dcmip\_gw\_32'
\tabularnewline

\hline 
u\_cbl(2)&
R& 0:0 & m/s and 1/s &
to prescribe initial zonal velocity profile for convective boundary layer simulations where u\_cbl(1)
sets the constant and u\_cbl(2) sets the vertical gradient &
nh\_test\_name=CBL
\tabularnewline

\hline 
v\_cbl(2)&
R& 0:0 & m/s and 1/s &
to prescribe initial meridional velocity profile for convective boundary layer simulations where v\_cbl(1)
sets the constant and v\_cbl(2) sets the vertical gradient &
nh\_test\_name=CBL
\tabularnewline

\hline 
th\_cbl(2)&
R& 290:0.006 & K and K/m &
to prescribe initial potential temperature profile for convective boundary layer simulations where th\_cbl(1)
sets the constant and th\_cbl(2) sets the gradient &
nh\_test\_name=CBL
\tabularnewline

\end{longtab}

Defined and used in: src/testcases/mo\_nh\_testcases.f90


\section{External data}
%------------------------------------------------------------------------------
% ext_par_nml:
%------------------------------------------------------------------------------
\subsection{extpar\_nml (Scope: itopo=1 in run\_nml)}

\begin{longtab}

\hline 
itopo&
I & 0& &
0: analytical topography/ext. data \\
1: topography/ext. data read from file&
\tabularnewline

\hline 
n\_iter\_smooth\_topo &
I(n\_dom) &
0&
&
iterations of topography smoother
&
itopo = 1
\tabularnewline

\hline 
fac\_smooth\_topo&
R &
0.015625&
&
pre-factor of topography smoother
&
n\_iter\_smooth\_topo > 0
\tabularnewline

\hline 
heightdiff\_threshold&
R(n\_dom) &
3000.&
m &
height difference between neighboring grid points above which additional local nabla2 diffusion is applied
&
\tabularnewline

\hline 
l\_emiss&
L &
.TRUE.&
&
read and use  external surface emissivity map
&
itopo = 1
\tabularnewline

\hline 
extpar\_filename&
C &
&
&
Filename of external parameter input file,
default: ''\texttt{<path>extpar\_<gridfile>}''.
May contain the keyword \texttt{<path>} which will be substituted by
\texttt{model\_base\_dir}. &
\tabularnewline

\end{longtab}

Defined and used in: src/namelists/mo\_extpar\_nml.f90


\section{External packages}
%------------------------------------------------------------------------------
% art_nml:
%------------------------------------------------------------------------------
\subsection{art\_nml}

\begin{longtab}

\hline 
lart&
L & .FALSE.& &
main switch for ART-package&
\tabularnewline
lemi\_volc&
L & .FALSE.& &
Emission of volcanic ash&
\tabularnewline
lconv\_tracer&
L & .FALSE.& &
Convection of tracers&
\tabularnewline
lwash\_tracer&
L & .FALSE.& &
Washout of tracers&
\tabularnewline
lrad\_volc&
L & .FALSE.& &
Radiative impact of volcanic ash&
\tabularnewline
lcld\_tracer&
L & .FALSE.& &
Impact on clouds&
\tabularnewline
\end{longtab}

Defined and used in: src/namelists/mo\_art\_nml.f90


\section{Information on vertical level distribution}

The hydrostatic and nonhydrostatic models need hybrid vertical level information to generate the
terrain following coorindates. The hybrid level specification is stored in
<icon home>/hyb\_params/HYB\_PARAMS\_<nlev>.
The {\bf hydrostatic} model assumes to get {\bf pressure based} coordinates, the {\bf nonhydrostatic}
model expects {\bf height based} coordinates. For further information see <icon home>/hyb\_params/README.


% -------------------------------------------------------------------
\section{Changes incompatible with former versions of the model code}
% -------------------------------------------------------------------
\noindent
% In this section we propose to document all namelist changes that are incompatible with former versions of the model code.
% Such changes are
%
%    renaming namelist parameters or changing the data type
%    removing existing namelist parameters
%    changing default settings
%    changing the scope of the namelist parameter
%    introducing new cross-check rules.

% --------------------------------------------------------------------------------------------
%                   <namelist parameter>                               <date>        <revision>
\begin{changeitem}{ var\_names\_map\_file, out\_varnames\_map\_file }{ 2013-04-25 }{ 12016 }
  \begin{itemize}
   \item Renamed \textbf{var\_names\_map\_file} $\rightarrow$ \textbf{output\_nml\_dict}.
   \item Renamed \textbf{out\_varnames\_map\_file} $\rightarrow$ \textbf{netcdf\_dict}.
   \item The dictionary in \emph{netcdf\_dict} is now reversed, s.t.\ the same map file 
         as in output\_nml\_dict can be used to translate variable names to the ICON internal
         names and back.
  \end{itemize}
\end{changeitem}


\begin{changeitem}{output\_nml: namespace}{ 2013-04-26 }{ 12051 }
  \begin{itemize}
   \item Removed obsolete namelist variable \textbf{namespace} from \textbf{output\_nml}.
  \end{itemize}
\end{changeitem}

\begin{changeitem}{gribout\_nml: generatingCenter, generatingSubcenter}{ 2013-04-26 }{ 12051 }
  \begin{itemize}
   \item Introduced new namelist variables \textbf{generatingCenter} and \textbf{generatingSubcenter}.
   \item If not set explicitly, center and subcenter information is copied from the input grid file
  \end{itemize}
\end{changeitem}

\begin{changeitem}{radiation\_nml: albedo\_type}{ 2013-05-03 }{ 12118 }
  \begin{itemize}
   \item Introduced new namelist variable \textbf{albedo\_type}
   \item If set to $2$, the surface albedo will be based on the MODIS data set.
  \end{itemize}
\end{changeitem}

\begin{changeitem}{initicon\_nml: dwdinc\_filename}{ 2013-05-24 }{ 12266 }
  \begin{itemize}
   \item Renamed dwdinc\_filename to dwdana\_filename
  \end{itemize}
\end{changeitem}

\begin{changeitem}{initicon\_nml: l\_ana\_sfc}{ 2013-06-25 }{ 12582 }
  \begin{itemize}
   \item Introduced new namelist flag \textbf{l\_ana\_sfc}
   \item If true, soil/surface analysis fields are read from the analysis fiel dwdfg\_filename. 
         If false, surface analyis fields are not read. Soil and surface are initialized with the first guess instead.
  \end{itemize}
\end{changeitem}

\begin{changeitem}{new\_nwp\_phy\_tend\_list: output names consistent with variable names}{ 2013-06-25 }{ 12590 }
  \begin{itemize}
   \item temp\_tend\_radlw $\rightarrow$ ddt\_temp\_radlw
   \item temp\_tend\_turb $\rightarrow$ ddt\_temp\_turb
   \item temp\_tend\_drag $\rightarrow$ ddt\_temp\_drag
  \end{itemize}
\end{changeitem}

\begin{changeitem}{prepicon\_nml, remap\_nml, input\_field\_nml}{ 2013-06-25 }{ 12597 }
  \begin{itemize}
   \item Removed the sources for the "prepicon" binary!
   \item The "prepicon" functionality (and most of its code) has become part of the ICON tools.
  \end{itemize}
\end{changeitem}

\begin{changeitem}{initicon\_nml}{ 2013-08-19 }{ 13311 }
  \begin{itemize}
   \item The number of vertical input levels is now read from file.
         The namelist parameter \textbf{nlev\_in} has become obsolete in r12700 and has been removed.
  \end{itemize}
\end{changeitem}

\begin{changeitem}{parallel\_nml}{ 2013-08-14 }{ 14164 }
  \begin{itemize}
   \item The namelist parameter use\_sp\_output has been renamed into use\_dp\_mpi2io.
  \end{itemize}
\end{changeitem}


% --------------------------------------------------------------------------------------------


\end{document}
